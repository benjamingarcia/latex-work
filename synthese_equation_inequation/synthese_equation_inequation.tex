\documentclass[12pt, twoside]{article}
\usepackage[francais]{babel}
\usepackage[T1]{fontenc}
\usepackage[latin1]{inputenc}
\usepackage[left=8mm, right=8mm, top=8mm, bottom=8mm]{geometry}
\usepackage{float}
\usepackage{graphicx}
\usepackage{array}
\usepackage{multirow}
\usepackage{amsmath,amssymb,mathrsfs}
\pagestyle{empty}
\begin{document}

\section*{\center{Bilan �quations et in�quations}}


\bigskip
\begin{center}
\begin{tabular}{|m{9cm}|m{10cm}|}
\hline
\textbf{Ce que je dois savoir} & \textbf{Ce que je dois savoir faire} \\
\hline
\begin{itemize}
  \item[]
  \item[$\bullet$] Je connais le vocabulaire : somme, produit, d�velopper,
  r�duire et factoriser.
  \item[$\bullet$] Je connais les diff�rents types d'�quation et les m�thodes
  de r�solution : \begin{enumerate}
                     \item[]�quation du type ''$ax+b$''
                     \item[]�quation produit
                     \item[]�quation quotient
                   \end{enumerate}
  \item[$\bullet$] Je connais les diff�rentes m�thodes de r�solutions
  d'in�quations : par le calcul ou par un tableau de signes.
  \item[$\bullet$] Je sais faire le tableau de signes de ''$ax+b$''.
  \item[$\bullet$] Je sais interpr�ter un tableau de signes.
  \end{itemize}
&

\enskip
\begin{itemize}
  \item[$\bullet$] Je sais reconna�tre le type d'une expression alg�brique.
  \item[$\bullet$] Je sais d�velopper et factoriser.
  \item[$\bullet$] Je sais transformer une expression alg�brique selon mes
  besoins.
  \item[$\bullet$] Je sais r�soudre des �quations.
  \item[$\bullet$] Je sais r�soudre des in�quations � l'aide d'un tableau de
  signes.
  \end{itemize} \\
\hline

\end{tabular}
\end{center}
\section*{\center{Bilan �quations et in�quations}}


\bigskip
\begin{center}
\begin{tabular}{|m{9cm}|m{10cm}|}
\hline
\textbf{Ce que je dois savoir} & \textbf{Ce que je dois savoir faire} \\
\hline
\begin{itemize}
  \item[]
  \item[$\bullet$] Je connais le vocabulaire : somme, produit, d�velopper,
  r�duire et factoriser.
  \item[$\bullet$] Je connais les diff�rents types d'�quation et les m�thodes
  de r�solution : \begin{enumerate}
                     \item[]�quation du type ''$ax+b$''
                     \item[]�quation produit
                     \item[]�quation quotient
                   \end{enumerate}
  \item[$\bullet$] Je connais les diff�rentes m�thodes de r�solutions
  d'in�quations : par le calcul ou par un tableau de signes.
  \item[$\bullet$] Je sais faire le tableau de signes de ''$ax+b$''.
  \item[$\bullet$] Je sais interpr�ter un tableau de signes.
  \end{itemize}
&

\enskip
\begin{itemize}
  \item[$\bullet$] Je sais reconna�tre le type d'une expression alg�brique.
  \item[$\bullet$] Je sais d�velopper et factoriser.
  \item[$\bullet$] Je sais transformer une expression alg�brique selon mes
  besoins.
  \item[$\bullet$] Je sais r�soudre des �quations.
  \item[$\bullet$] Je sais r�soudre des in�quations � l'aide d'un tableau de
  signes.
  \end{itemize} \\
\hline

\end{tabular}
\end{center}


\end{document}
