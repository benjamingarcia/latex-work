
\documentclass[12pt, twoside]{article}

\usepackage[francais]{babel}
\usepackage[T1]{fontenc}
\usepackage[latin1]{inputenc}
\usepackage[left=1cm, right=1cm, top=1cm, bottom=1cm]{geometry}
\usepackage{float}
\usepackage{graphicx}
\usepackage{array}
\usepackage{multirow}
\usepackage{amsmath,amssymb,mathrsfs}

\begin{document}


\section*{\center{Encore des puissances}}

\subsection*{Exercice 8}

On pose: $A_2=\frac{(8^{3}+8^{2})^{2}}{(4^{2}-4)^{3}}$, \enskip
$A_3=\frac{(8^{4}+8^{3})^{2}}{(4^{3}-4^{2})^{3}}$, \enskip
$A_4=\frac{(8^{5}+8^{4})^{2}}{(4^{4}-4^{3})^{3}}$.


Sans utiliser la calculatrice, calculer $A_2,A_3$ et $A_4$. Que remarque t'on?\\
Quelle conjecture peut-on faire sur les nombres $A_n$? Justifier cette
conjecture.

\subsection*{Exercice 9}
Ranger les nombres suivants dans l'ordre croissant sans calculatrice:\\
$A=999 \thinspace 999 \thinspace 999 \thinspace 999 * 999 \thinspace 999
\thinspace 999 \thinspace 999$\\
$B= 999 \thinspace 999* 999 \thinspace 999 *999 \thinspace 999 * 999 \thinspace
999$\\
$C=999 \thinspace 999 \thinspace 999 \thinspace 999 \thinspace 999 \thinspace
999 * 999 \thinspace 999$

\section*{\center{Encore des puissances}}

\subsection*{Exercice 8}

On pose: $A_2=\frac{(8^{3}+8^{2})^{2}}{(4^{2}-4)^{3}}$, \enskip
$A_3=\frac{(8^{4}+8^{3})^{2}}{(4^{3}-4^{2})^{3}}$, \enskip
$A_4=\frac{(8^{5}+8^{4})^{2}}{(4^{4}-4^{3})^{3}}$.


Sans utiliser la calculatrice, calculer $A_2,A_3$ et $A_4$. Que remarque t'on?\\
Quelle conjecture peut-on faire sur les nombres $A_n$? Justifier cette
conjecture.

\subsection*{Exercice 9}
Ranger les nombres suivants dans l'ordre croissant sans calculatrice:\\
$A=999 \thinspace 999 \thinspace 999 \thinspace 999 * 999 \thinspace 999
\thinspace 999 \thinspace 999$\\
$B= 999 \thinspace 999* 999 \thinspace 999 *999 \thinspace 999 * 999 \thinspace
999$\\
$C=999 \thinspace 999 \thinspace 999 \thinspace 999 \thinspace 999 \thinspace
999 * 999 \thinspace 999$

\section*{\center{Encore des puissances}}

\subsection*{Exercice 8}

On pose: $A_2=\frac{(8^{3}+8^{2})^{2}}{(4^{2}-4)^{3}}$, \enskip
$A_3=\frac{(8^{4}+8^{3})^{2}}{(4^{3}-4^{2})^{3}}$, \enskip
$A_4=\frac{(8^{5}+8^{4})^{2}}{(4^{4}-4^{3})^{3}}$.


Sans utiliser la calculatrice, calculer $A_2,A_3$ et $A_4$. Que remarque t'on?\\
Quelle conjecture peut-on faire sur les nombres $A_n$? Justifier cette
conjecture.

\subsection*{Exercice 9}
Ranger les nombres suivants dans l'ordre croissant sans calculatrice:\\
$A=999 \thinspace 999 \thinspace 999 \thinspace 999 * 999 \thinspace 999
\thinspace 999 \thinspace 999$\\
$B= 999 \thinspace 999* 999 \thinspace 999 *999 \thinspace 999 * 999 \thinspace
999$\\
$C=999 \thinspace 999 \thinspace 999 \thinspace 999 \thinspace 999 \thinspace
999 * 999 \thinspace 999$

\end{document}
