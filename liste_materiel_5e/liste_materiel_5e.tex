\documentclass[12pt, twoside]{article}
\usepackage[francais]{babel}
\usepackage[T1]{fontenc}
\usepackage[latin1]{inputenc}
\usepackage[left=7mm, right=7mm, top=7mm, bottom=7mm]{geometry}
\usepackage{float}
\usepackage{graphicx}
\usepackage{array}
\usepackage{multirow}
\usepackage{amsmath,amssymb,mathrsfs} 
\usepackage{soul}
\usepackage{textcomp}
\usepackage{eurosym}
 \usepackage{variations}
\usepackage{tabvar}
 
\begin{document}

\section*{\center{Mat�riel pour le cours de math�matiques}}

\subsection*{Pour tous les cours}

Chaque �l�ve devra amener au cours de math�matiques:

\begin{itemize}
  \item [$\bullet$] son cahier de cours (grand format \ul{24 $\times$ 32},
  petits carreaux, 100 pages)
  \item [$\bullet$] son cahier d'exercices (grand format \ul{24 $\times$ 32},
  petits carreaux, 100 pages)  
  \item [$\bullet$] le livre de math�matiques
  \item [$\bullet$] une r�gle gradu�e (20 cm).
  
\end{itemize}

\medskip

Il devra �galement avoir dans sa trousse:

\begin{itemize}
  \item [$\bullet$] stylos bleus, rouges et verts
  \item [$\bullet$] crayon gris \ul{affut�} ou crit�rium et gomme.
\end{itemize}

 
\subsection*{Pour des s�ances sp�cifiques}

Les �l�ves doivent �galement acheter:
\begin{itemize}
  \item [$\bullet$] 1 rapporteur transparent en degr�
  \item [$\bullet$] 1 �querre
  \item [$\bullet$] 1 compas
  \item [$\bullet$] une calculatrice scientifique (casio FX92 par exemple)
\end{itemize}

\medskip

\textbf{Pour ces quatre derniers outils, le professeur signalera aux �l�ves �
quel moment les apporter. Il est cependant indispensable de les avoir d�s le
d�but de l'ann�e.}

\bigskip

\section*{\center{Mat�riel pour le cours de math�matiques}}

\subsection*{Pour tous les cours}

Chaque �l�ve devra amener au cours de math�matiques:

\begin{itemize}
  \item [$\bullet$] son cahier de cours (grand format \ul{24 $\times$ 32},
  petits carreaux, 100 pages)
  \item [$\bullet$] son cahier d'exercices (grand format \ul{24 $\times$ 32},
  petits carreaux, 100 pages)  
  \item [$\bullet$] le livre de math�matiques
  \item [$\bullet$] une r�gle gradu�e (20 cm).
  
\end{itemize}

\medskip

Il devra �galement avoir dans sa trousse:

\begin{itemize}
  \item [$\bullet$] stylos bleus, rouges et verts
  \item [$\bullet$] crayon gris \ul{affut�} ou crit�rium et gomme.
\end{itemize}

 
\subsection*{Pour des s�ances sp�cifiques}

Les �l�ves doivent �galement acheter:
\begin{itemize}
  \item [$\bullet$] 1 rapporteur transparent en degr�
  \item [$\bullet$] 1 �querre
  \item [$\bullet$] 1 compas
  \item [$\bullet$] une calculatrice scientifique (casio FX92 par exemple)
\end{itemize}

\medskip

\textbf{Pour ces quatre derniers outils, le professeur signalera aux �l�ves �
quel moment les apporter. Il est cependant indispensable de les avoir d�s le
d�but de l'ann�e.}





\end{document}
