\documentclass[12pt, twoside]{article}
\usepackage[francais]{babel}
\usepackage[T1]{fontenc}
\usepackage[latin1]{inputenc}
\usepackage[left=8mm, right=8mm, top=8mm, bottom=8mm]{geometry}
\usepackage{float}
\usepackage{graphicx}
\usepackage{array}
\usepackage{multirow}
\usepackage{amsmath,amssymb,mathrsfs}
\pagestyle{empty}
\begin{document}

\section*{\center{Bilan fonctions}}




\bigskip
\begin{center}
\begin{tabular}{|m{9cm}|m{10cm}|}
\hline
\textbf{Ce que je dois savoir} & \textbf{Ce que je dois savoir faire} \\
\hline
\begin{itemize}
  \item[$\bullet$] Je connais le vocabulaire.
  \item [$\bullet$] Je sais utiliser la calculatrice pour les calculs et les
  graphiques.
  \item [$\bullet$] Je comprends les diff�rents aspects de la notion de
  fonction: graphiques, calculs, tableaux (de valeurs, de signe et de
  variation).
\end{itemize}
&

\enskip
\begin{itemize}
  \item[$\bullet$] Je sais reconna�tre le graphe d'une fonction.
  \item[$\bullet$] J'identifie la variable et le domaine de d�finition de la
  fonction.
  \item[$\bullet$] Je sais remplir et je comprends un tableau de valeurs.
 \item[$\bullet$] Je sais lire un tableau de variation, je sais le compl�ter
 et le d�crire avec le vocabulaire adapt�.
  \item[$\bullet$] Je sais calculer des images et des ant�c�dents.
  \item[$\bullet$] Je sais interpr�ter la repr�sentation graphique d'une
  fonction: image, ant�c�dent, minimum, maximum, monotonie
  (croissante/d�croissante), signe de la fonction.
  \item[$\bullet$] Je sais r�soudre graphiquement des �quations et  in�quations.
 
 \end{itemize} \\
\hline

\end{tabular}
\end{center}

\bigskip

\bigskip

\section*{\center{Bilan triangles isom�triques et semblables}}




\bigskip
\begin{center}
\begin{tabular}{|m{9cm}|m{10cm}|}
\hline
\textbf{Ce que je dois savoir} & \textbf{Ce que je dois savoir faire} \\
\hline
\begin{itemize}
  \item[$\bullet$] Je connais les d�finitions et propri�t�s des transformations
  usuelles.
  \item [$\bullet$] Je connais la d�finition de triangles isom�triques. 
  \item [$\bullet$] Je connais les cas d'isom�tries de triangles.
  \item [$\bullet$] Je connais la propri�t� d'isom�trie dans le cas d'un
  triangle rectangle.
  \item [$\bullet$] Je connais la d�finition de triangles semblables (de m�mes
  formes).
  \item [$\bullet$] Je connais les cas de similitude de triangles.
\end{itemize}
&

\enskip
\begin{itemize}
  \item[$\bullet$] Je sais identifier les sommets homologues de triangles
  isom�triques.
  \item[$\bullet$] Je suis capable de d�terminer des �galit�s de longueur et
  d'angle sur des triangles isom�triques.
  \item[$\bullet$] Je sais identifier les sommets homologues de triangles
  semblables.
 \item[$\bullet$] Je sais �crire des �galit�s de rapport dans deux triangles
 semblables et des �galit�s d'angle.
  \item[$\bullet$] Je sais calculer le rapport des aires de deux triangles
  semblables.
 \end{itemize} \\
\hline

\end{tabular}
\end{center}


\end{document}
