\documentclass[12pt, twoside]{article}
\usepackage[francais]{babel}
\usepackage[T1]{fontenc}
\usepackage[latin1]{inputenc}
\usepackage[left=7mm, right=7mm, top=7mm, bottom=7mm]{geometry}
\usepackage{float}
\usepackage{graphicx}
\usepackage{array}
\usepackage{multirow}
\usepackage{amsmath,amssymb,mathrsfs}
\usepackage{soul}
\usepackage{textcomp}
\usepackage{eurosym}
 \usepackage{variations}
\usepackage{tabvar}
\begin{document}


\ul{\textbf{Exercice}} Benjamin a achet� 4 bouteilles de jus de fruits �
1,55 euro l'une. Avant cet achat, il disposait de 12 euros.

\begin{enumerate}
  \item Quelle somme lui reste-t-il apr�s son achat?
  \item Par quelle op�ration a-t-il fallu commencer le calcul?
  \item Ecrire l'expression qui permet de calculer la somme restant � Benjamin.
\end{enumerate}

\bigskip


\ul{\textbf{Exercice}} Benjamin a achet� 4 bouteilles de jus de fruits �
1,55 euro l'une. Avant cet achat, il disposait de 12 euros.

\begin{enumerate}
  \item Quelle somme lui reste-t-il apr�s son achat?
  \item Par quelle op�ration a-t-il fallu commencer le calcul?
  \item Ecrire l'expression qui permet de calculer la somme restant � Benjamin.
\end{enumerate}

\bigskip

\ul{\textbf{Exercice}} Benjamin a achet� 4 bouteilles de jus de fruits �
1,55 euro l'une. Avant cet achat, il disposait de 12 euros.

\begin{enumerate}
  \item Quelle somme lui reste-t-il apr�s son achat?
  \item Par quelle op�ration a-t-il fallu commencer le calcul?
  \item Ecrire l'expression qui permet de calculer la somme restant � Benjamin.
\end{enumerate}

\bigskip

\ul{\textbf{Exercice}} Benjamin a achet� 4 bouteilles de jus de fruits �
1,55 euro l'une. Avant cet achat, il disposait de 12 euros.

\begin{enumerate}
  \item Quelle somme lui reste-t-il apr�s son achat?
  \item Par quelle op�ration a-t-il fallu commencer le calcul?
  \item Ecrire l'expression qui permet de calculer la somme restant � Benjamin.
\end{enumerate}

\bigskip

\ul{\textbf{Exercice}} Benjamin a achet� 4 bouteilles de jus de fruits �
1,55 euro l'une. Avant cet achat, il disposait de 12 euros.

\begin{enumerate}
  \item Quelle somme lui reste-t-il apr�s son achat?
  \item Par quelle op�ration a-t-il fallu commencer le calcul?
  \item Ecrire l'expression qui permet de calculer la somme restant � Benjamin.
\end{enumerate}

\bigskip

\ul{\textbf{Exercice}} Benjamin a achet� 4 bouteilles de jus de fruits �
1,55 euro l'une. Avant cet achat, il disposait de 12 euros.

\begin{enumerate}
  \item Quelle somme lui reste-t-il apr�s son achat?
  \item Par quelle op�ration a-t-il fallu commencer le calcul?
  \item Ecrire l'expression qui permet de calculer la somme restant � Benjamin.
\end{enumerate}

\bigskip

\ul{\textbf{Exercice}} Benjamin a achet� 4 bouteilles de jus de fruits �
1,55 euro l'une. Avant cet achat, il disposait de 12 euros.

\begin{enumerate}
  \item Quelle somme lui reste-t-il apr�s son achat?
  \item Par quelle op�ration a-t-il fallu commencer le calcul?
  \item Ecrire l'expression qui permet de calculer la somme restant � Benjamin.
\end{enumerate}

\end{document}
