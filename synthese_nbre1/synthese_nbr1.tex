\documentclass[12pt, twoside]{article}
\usepackage[francais]{babel}
\usepackage[T1]{fontenc}
\usepackage[latin1]{inputenc}
\usepackage[left=1cm, right=1cm, top=1cm, bottom=1cm]{geometry}
\usepackage{float}
\usepackage{graphicx}
\usepackage{array}
\usepackage{multirow}
\usepackage{amsmath,amssymb,mathrsfs}
\pagestyle{empty}
\begin{document}

\section*{\center{Bilan}}

\subsection*{Ensemble de nombres}


\bigskip
\begin{center}
\begin{tabular}{|m{7cm}|m{7cm}|}
\hline
\textbf{Ce que je dois savoir} & \textbf{Ce que je dois savoir faire} \\
\hline
\begin{itemize}
  \item[$\bullet$] Je connais les diff�rents ensembles de nombres et leurs
  inclusions.
  \item [$\bullet$] J'identifie un nombre r�el sur la droite gradu�e.
\end{itemize}
&

\enskip
\begin{itemize}
  \item[$\bullet$] Classer les nombres dans le plus petit ensemble.
  \item[$\bullet$] Calculer avec des puissances, des racines carr�es et des
  fractions.
  \item[$\bullet$] Je sais construire certains nombres r�els sur la droite
  gradu�e.
\end{itemize} \\
\hline

\end{tabular}
\end{center}


\subsection*{Nombres premiers}


\bigskip
\begin{center}
\begin{tabular}{|m{7cm}|m{7cm}|}
\hline
\textbf{Ce que je dois savoir} & \textbf{Ce que je dois savoir faire} \\
\hline
\begin{itemize}
  \item[$\bullet$] Je connais la d�finition d'un nombre premier.
  \item [$\bullet$] Je connais les d�finitions de diviseurs et multiples.
  \item [$\bullet$] Je connais les premiers nombres premiers.
\end{itemize}
&

\enskip
\begin{itemize}
  \item[$\bullet$] Je sais d�terminer si un nombre est premier.
  \item[$\bullet$] Je sais d�composer un entier en produits de facteurs
  premiers.
  \item[$\bullet$] Je sais utiliser la d�composition en facteurs premiers pour
  calculer le $pgcd$ de $2$ nombres.
  \item[$\bullet$] Je sais utiliser la d�composition en facteurs premiers pour
  simplifier des radicaux.
  \end{itemize} \\
\hline

\end{tabular}
\end{center}


\subsection*{Nombre et ordre}


\bigskip
\begin{center}
\begin{tabular}{|m{7cm}|m{7cm}|}
\hline
\textbf{Ce que je dois savoir} & \textbf{Ce que je dois savoir faire} \\
\hline
\begin{itemize}
  \item[$\bullet$] Je connais les diff�rentes propri�t�s.
\end{itemize}
&

\enskip
\begin{itemize}
  \item[$\bullet$] Je sais classer deux nombres en utilisant les propri�t�s.
  \item[$\bullet$] Je sais utiliser les diff�rentes m�thodes de comparaison.
\end{itemize} \\
\hline

\end{tabular}
\end{center}

\end{document}
