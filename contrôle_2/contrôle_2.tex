\documentclass[12pt, twoside]{article}
\usepackage[francais]{babel}
\usepackage[T1]{fontenc}
\usepackage[latin1]{inputenc}
\usepackage[left=1cm, right=1cm, top=1cm, bottom=1cm]{geometry}
\usepackage{float}
\usepackage{graphicx}
\usepackage{array}
\usepackage{multirow}
\usepackage{amsmath,amssymb,mathrsfs}
\pagestyle{empty}

\begin{document} 
\begin{center}
{\fbox{$2^{de}5$ \qquad \qquad \textbf{\Large{Devoir surveill� 2}}
\qquad \qquad 24/10/2008}}
\end{center}

\bigskip
\bigskip



\textit{Devoir � rendre sur feuille double petits carreaux. La clart� et la
pr�cision de la r�daction seront prises en compte. Seul l'exercice 1 est �
r�diger sur la feuille photocopi�e(qu'il faut donc rendre).}


\bigskip




\textbf{\large{Exercice 2:}} \textit{3 points}

\medskip

Soient $I=]-\infty;-1[$, $J=[2;7[$ et $K=[-2;3]$. 
Repr�senter $I$, $J$ et $K$ sur la droite gradu�e et en d�duire $I \cap J$, $J
\cap K$ et $K \cap I$ (\textit{vous pouvez faire plusieurs sch�mas}).



\bigskip
\bigskip




\textbf{\large{Exercice 3:}} \textit{2 points} 

\begin{enumerate}
  \item Ranger les nombres suivants par ordre croissant: $\dfrac{10}{3}$;
  \enskip $-0,18$; \enskip $-0,4$; \enskip $\dfrac{1}{4}$; \enskip $-0,018$;
  \enskip $\dfrac{1}{3}$.
  \item Ranger les nombres suivants par ordre croissant: $0,53$; \enskip $\left
  (\dfrac{27}{6} \right)^{2}$; \enskip $1$; \enskip $\dfrac{25}{6}$; \enskip
  $(0,53)^{2}$; \enskip $\left( \dfrac{25}{6} \right )^{2}$.
\end{enumerate}



\bigskip
\bigskip




\textbf{\large{Exercice 4:}} \textit{4,5 points}
\medskip


Soient $x$ et $y$ deux r�els tesl que: $2 \leqslant x \leqslant 4$ et $-1
\leqslant y \leqslant 5$.
\begin{enumerate}
  \item Encadrer $x+y$.
  \item Encadrer $y-x$.
  \item Encadrer $\dfrac{x^{2}-y}{4}$.
  \item Encadrer $\dfrac{y}{x}+1$.
\end{enumerate}


\bigskip
\bigskip



\textbf{\large{Exercice 5:}} \textit{4 points}
\begin{enumerate}
  \item Soient $A=\dfrac{3+\pi}{3}$ et $B=\dfrac{4+\pi}{4}$. Comparer $A$ et
  $B$.
   \item Soient $C=\sqrt{4-2\sqrt{3}}$ et $D=1-\sqrt{3}$. Comparer $C$ et
  $D$.  
   \item Soient $a>0$, $b>0$, $E=\dfrac{a}{a+b}$ et $F=\dfrac{a-b}{a}$. Comparer
   $E$ et $F$.  
\end{enumerate}



\bigskip
\bigskip




\textbf{\large{Exercice 6:}} \textit{2,5 points dont un bonus}
\medskip

L'aire d'un rectangle est comprise entre $51\ m^{2}$ et $52 \ m^{2}$ inclus. Sa
largeur est comprise entre $3 \ m$ et $6 \ m$ inclus. Donner un encadrement de
sa largeur et de son p�rim�tre.


\end{document}
