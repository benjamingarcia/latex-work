\documentclass[12pt, twoside]{article}
\usepackage[francais]{babel}
\usepackage[T1]{fontenc}
\usepackage[latin1]{inputenc}
\usepackage[left=5mm, right=5mm, top=5mm, bottom=5mm]{geometry}
\usepackage{float}
\usepackage{graphicx}
\usepackage{array}
\usepackage{multirow}
\usepackage{amsmath,amssymb,mathrsfs}
\usepackage{soul}
\usepackage{textcomp}
\usepackage{eurosym}
 \usepackage{variations}
\usepackage{tabvar}

\pagestyle{empty}
\begin{document}

\begin{center}
\fbox{Correction du devoir maison 5}
\end{center}


\ul{Exercice 1:}

\begin{enumerate}
  \item \begin{enumerate}
          \item [a)] $2^2=4$ \qquad $4 \times 5=20$ \qquad $20+10=30$ \qquad Le
          calcul de Marc est exact.
          \item [b)] $0,1^2=0,01$ \qquad $0,01 \times 5=0,05$ \qquad
          $0,05+10=10,05$ \qquad Robin trouve 10,05.
\end{enumerate}

\item \begin{enumerate}
        \item[a)] $p(x)=x^2 \times 5 +10=5x^2+10$
        \item[b)] $p(-1)=(-1)^2 \times 5+10=1 \times 5+10=15$ \qquad $p(3)=3^2
        \times 5+10=9 \times 5+10=55$ \qquad $p(0)=0^2 \times 5+10=10$
        \item[c)]  $p(0,2)=0,2^2 \times 5+10=0,04 \times
        5+10=0,2+10=10,2$.
        
        $p(0,2)=10,2$ donc 0,2 est l'ant�c�dent de 10,2.
\end{enumerate}
\end{enumerate}

\bigskip

\ul{Exercice 2:} 

\begin{enumerate}
  \item Axe des abscisses: temps (en minutes) ; axe des ordonn�es: distance
  parcourue (en km)
  \item \begin{enumerate}
          \item La distance parcourue n'�volue pas entre la 20i�me et la
          30i�me minute: le coureur s'est donc arr�t� environ 10 minutes.
          \item Au bout de 5 minutes, il a parcouru un kilom�tre. 
          \item Il a mis environ 32,5 minutes pour parcourir 4 kilom�tres.
      
\end{enumerate}
\item \begin{enumerate}
        \item Le coureur a parcouru 2 kilom�tres en 10 minutes donc $d(10)=2$.
        L'image de 10 par $d$ est 2.
        \item $d(35)=6$; \quad  35 est l'ant�c�dent de 6 par $d$.
\end{enumerate}
\item 

\begin{tabular}{cc}

\begin{minipage}{6cm}

\begin{tabular}{|c|c|c|}
\hline
distance (en km) & 6 & y \\
\hline
temps (en min) & 35 & 60 \\
\hline
\end{tabular}

\end{minipage}
&
\begin{minipage}{12cm}
$y=6 \times 60 \div 35 \approx 10,3$

La vitesse moyenne du coureur est d'environ 10,3 km/h.
\end{minipage}

\end{tabular}
\end{enumerate}


\bigskip

\ul{Exercice 3:}

\begin{enumerate}
  \item Le premier graphique ne correspond pas au tableau de valeurs de la
  fonction $f$ car:
  
  $\bullet$ $f(3)=-3$. La repr�sentation
  graphique ne contient pas le point (-3;3) (� ne pas confondre avec (3;-3));
   
   $\bullet$ $f(1)=2$ mais le point (1;2)
   n'appartient pas � la courbe (ne pas confondre avec le point (2;1));
   
   $\bullet$ $f(2)=3$ mais le point (2;3)
   n'appartient pas � la courbe (ne pas confondre avec le point (3;2)).   
   
   \item Le deuxi�me graphique ne correspond pas au tableau de valeurs de la
  fonction $f$ car:  
  
   $\bullet$  $f(-2)=0$ mais le point (-2;0)
   n'appartient pas � la courbe (ne pas confondre avec le point (0;-2)). 
\end{enumerate}

\bigskip


\ul{Exercice 4:}

\begin{enumerate}
  \item effectif total: $2+4+5+3+7+3+2=26$ \qquad Il y a 26 personnes dans cette
  classe.
  \item Etendue: 17-8=9 \qquad Il y a 9 points d'�carts entre la note la plus
  basse et la note la plus haute.
  \item Calcul du premier quartile: 25\% de 26 \qquad $\dfrac{25}{100} \times
  26=6,5$ \qquad On prend la 7e valeur de la s�rie ordonn�e.
  
  Donc $Q_1=10$.
  
Calcul du troisi�me quartile: 75\% de 26 \qquad $\dfrac{75}{100} \times
  26=19,5$ \qquad On prend la 20e valeur de la s�rie ordonn�e.  
  
  Donc $Q_3=16$
  
  \item 6 personnes sur 26 ont eu moins de 10. \qquad $\dfrac{6}{26} \approx
  0,23<\dfrac{1}{4}$ 
  
   Donc l'affirmation ``moins d'un quart des �l�ves ont
  eu une note inf�rieure � 10/20'' est juste.
\end{enumerate}


\bigskip

\ul{Exercice 5:}


\begin{tabular}{l|l}
\begin{minipage}{12cm}
$A=\dfrac{3 \times 10^5 \times 4 \times (10^{-3})^2}{16 \times 10^{-4}}$

\enskip

$A=\dfrac{3 \times 4}{4 \times 4} \times \dfrac{10^5 \times 10^{-6}}{10^{-4}}$
\qquad car $(10^{-3})^2=10^{-3 \times 2}=10^{-6}$

\enskip

$A=\dfrac{3}{4} \times \dfrac{10^{-1}}{10^{-4}}$ \qquad  car $10^5 \times
10^{-6}=10^{5+(-6)}=10^{-1}$

\enskip


$A=0,75 \times 10^3$ \qquad car
$\dfrac{10^{-1}}{10^{-4}}=10^{-1-(-4)}=10^{-1+4}=10^3$


\enskip


$A= 750$ \qquad �criture d�cimale

\enskip

$A=7,5 \times 10^2$ \qquad notation scientifique
\end{minipage}

&
\begin{minipage}{7cm}
\qquad $B= \dfrac{7}{15}-\dfrac{2}{15} \times \dfrac{9}{4}$

\enskip

\qquad $B=\dfrac{7}{15}-\dfrac{2 \times 9}{15 \times 4}$

\enskip

\qquad $B=\dfrac{7}{15}-\dfrac{18}{60}$

\enskip

\qquad $B=\dfrac{28}{60}-\dfrac{18}{60}$ \qquad car
$\dfrac{7}{15}=\dfrac{28}{60}$

\enskip

\qquad $B=\dfrac{10}{60}=\dfrac{1}{6}$
\end{minipage}

\end{tabular}

\bigskip


\ul{Exercice 6:}

\enskip

On note $n$ le nombre de moutons.

Le nombre de poules est $\dfrac{n}{3}$.


Le nombre de pattes de l'ensemble des moutons est $4 \times n=4n$.

Le nombre de pattes de l'ensemble des poules est $2 \times
\dfrac{n}{3}=\dfrac{2n}{3} $.

Le nombre de pattes du chien est 4.


Le nombre total de pattes est $4n+\dfrac{2n}{3} +4$. D'apr�s l'�nonc�, ce total
vaut 172. 

On a donc l'�quation suivante: $4n+\dfrac{2n}{3} +4=172$


\enskip

\begin{tabular}{l|l|l}
\ul{R�duction de l'�quation:} & \qquad \ul{R�solution:} & \qquad 
\ul{V�rification:}  \\

\quad & \quad & \quad \\

$4n+\dfrac{2n}{3} +4=172$ & \qquad $\dfrac{14n}{3}+4-4=172-4$ &  \qquad Il y a
36 moutons et 12 poules. \\
\quad & \quad & \quad \\
$\dfrac{12n}{3}+\dfrac{2n}{3} +4=172$ & \qquad $\dfrac{14n}{3}=168$ & \qquad
$36 \times 4=144$  et $12 \times 2=24$ \\
\quad & \quad & \quad \\
$\dfrac{14n}{3}+4=172$ & \qquad $\dfrac{14}{3} \times n \div \dfrac{14}{3}=168
\div \dfrac{14}{3}$ &  \qquad Il y a 144 pattes de moutons et 24 de
poules. \\
\quad & \quad & \quad \\
\quad & \qquad $n=168 \times \dfrac{3}{14}$ & \qquad $144+24+4=172$ \\
\quad & \quad & \quad \\
\quad & \qquad $n=36$ & \qquad  \\
\end{tabular}


\bigskip


\ul{Exercice 7:}

$\bullet$ Pour calculer la longueur totale du parcours, il faut calculer
$AB+BC+CD+DE$.
Calculons les longueurs $BC$, $CD$ et $DE$.

\medskip


$\bullet$ Calcul de $BC$: Le triangle ABC est rectangle en A. D'apr�s le
th�or�me de Pythagore, on a: $AB^2+AC^2=BC^2$

$BC^2=300^2+400^2=9000+16000=25000$ \quad $BC$ est un nombre positif donc
$BC=\sqrt{25000}=500$ m.


\medskip

$\bullet$ Calcul de $CD$ et $DE$: Les droites (AB) et (DE) sont parall�les et
les droites (AE) et (BD) sont s�cantes en C. D'apr�s le th�or�me de Thal�s, on a:

\enskip

$\dfrac{CA}{CE}=\dfrac{CB}{CD}=\dfrac{AB}{ED}$ \qquad En rempla�ant par les
valeurs num�riques: $\dfrac{400}{1000}=\dfrac{500}{CD}=\dfrac{300}{ED}$

\enskip

Donc $CD=500 \times 1000 \div 400=1250$ m \quad et \quad $ED=300 \times 1000
\div 400=750$ m


\medskip 

$\bullet$ Longueur du parcours: $300+500+1250+750=2800$ m.
\end{document}