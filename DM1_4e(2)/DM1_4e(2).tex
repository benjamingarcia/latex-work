\documentclass[12pt, twoside]{article}
\usepackage[francais]{babel}
\usepackage[T1]{fontenc}
\usepackage[latin1]{inputenc}
\usepackage[left=1cm, right=1cm, top=1cm, bottom=1cm]{geometry}
\usepackage{float}
\usepackage{graphicx}
\usepackage{array}
\usepackage{multirow}
\usepackage{amsmath,amssymb,mathrsfs}
\usepackage{soul}
\usepackage{textcomp}
\usepackage{eurosym}
 \usepackage{variations}
\usepackage{tabvar}


\pagestyle{empty}

\begin{document}


\section*{\center{Devoir maison 1}}

\textit{Devoir � rendre sur feuille grand format petits
carreaux pour le \ul{lundi 5 octobre 2009}.}

\subsection*{Exercice 1}

Recopier et effectuer les calculs suivants en soulignant en vert � chaque �tape
le calcul en cours.

\enskip

\textit{Remarque: un r�sultat donn� directement sans les �tapes de calculs ne
donnera pas de point.}

\enskip

\begin{tabular}{cc}
\begin{minipage}{9cm}
$A=2 \times (-6)+ (-3)$

\enskip

$B=-7+10 \div (-3+1)$

\enskip

$C=8+4 \times \big( (-3)+(-4) \times 5 \big)$

\end{minipage}
&
\begin{minipage}{9cm}
$D=(-18) \div (-9)+ (-16) \div (+4)$

\bigskip

$E=\dfrac{(-3) \times (-4)+(-2)}{-3-2}$

\bigskip

$F=\dfrac{6+ \big( (-4) \times (-6)+(-10) \big)}{-2 \times 8 +  (-3) \times
(-5)}$
\end{minipage}
\end{tabular}


\subsection*{Exercice 2}


\begin{enumerate}
  \item Trouver deux nombres relatifs dont le produit est positif et la somme
  est n�gative. 
  \item Trouver deux nombres relatifs dont le produit est n�gatif et la somme
  est positive.
  \item Trouver deux nombres relatifs dont le produit et la somme sont positifs.
  \item Trouver deux nombres relatifs dont le produit et le somme sont
  n�gatifs.
\end{enumerate}

Pour chaque question, vous justifierez votre r�ponse en effectuant les
calculs (produit et somme des deux nombres trouv�s).



\subsection*{Exercice 3}

\begin{enumerate}
  \item Traduire les phrases suivantes par un calcul:
  
  \begin{enumerate}
    
\item Le quotient de -20 par le produit de 4 et de -5.

\item Le produit de la somme de 2 et de -3 par la diff�rence de -5 et de
    1.
  
\end{enumerate}

\item Effectuer ces calculs. 
\item Traduire les expressions math�matiques suivantes par des phrases:
\begin{enumerate}
  \item $3 \times (-2) - 8$
  \item $(3-7) \times  (-5)$
\end{enumerate}
\end{enumerate}
  

\subsection*{Exercice 4}

 Avec les nombres propos�s, retrouver le r�sultat annonc� en respectant
 les consignes suivantes:

$\bullet$ Chaque nombre est utilis� une fois.

$\bullet$ Tous les nombres sont utilis�s.

$\bullet$ Toutes les op�rations sont autoris�es.
 
\enskip

\begin{center}
-7 \qquad -25 \qquad 10 \qquad -8 \qquad -75
\end{center}

\enskip

R�sultat � trouver: 730

\end{document}
