\documentclass[12pt, twoside]{article}
\usepackage[francais]{babel}
\usepackage[T1]{fontenc}
\usepackage[latin1]{inputenc}
\usepackage[left=5mm, right=5mm, top=5mm, bottom=5mm]{geometry}
\usepackage{float}
\usepackage{graphicx}
\usepackage{array}
\usepackage{multirow}
\usepackage{amsmath,amssymb,mathrsfs}
\usepackage{soul}
\usepackage{textcomp}
\usepackage{eurosym}
 \usepackage{variations}
\usepackage{tabvar}

\pagestyle{empty}
\begin{document}

\section*{\center{Correction devoir maison 7}}

\enskip


\ul{Exercice 1}:


\begin{enumerate}
  \item L'expression 10x repr�sente la recette de la vente des places en virage.
  
  
  L'expression 15000-x repr�sente le nombre de places en tribune. 
  
  L'expression
  (15000-x)$\times$ 13 repr�sente la recette de la vente des places en tribune.
  
  
  \item Le montant total de la recette est: 10x+(15000-x)$\times$ 13.
  
  
  \item Si x=6500, on a alors: $10 \times 6500+(15000-6500) \times 13=175500$.
  S'il y a 6500 places en virage, le montant total de la recette est 175 500
  euros.
\end{enumerate}

\bigskip




\ul{Exercice 2}:


\enskip

\begin{tabular}{lll}

$A=2t+4-(8t-9)$ \qquad & \qquad $B=3a-7+(2a+5)-(-3a+10)$ \qquad & \qquad
$C=4y^2-2 \times 3y+2y \times 5y-6-4y$ \\

$A=2t+4-8t+9$ \qquad & \qquad $B=3a-7+2a+5+3a-10$ \qquad & \qquad 
$C=4y^2-6y+10y^2-6-4y$ \\

$A=-6t+4+9$ \qquad & \qquad $B=8a-7+5-10$ \qquad & \qquad  $C=14y^2-6y-6-4y$ \\

$A=-6t+13$ \qquad & \qquad $B=8a-12$ \qquad & \qquad  $C=14y^2-10y-6$ \\
\end{tabular}

\bigskip


\ul{Exercice 3}:

\begin{enumerate}
  \item  
   \ul{Donn�es}: MAN est un triangle rectangle en B. [AC] est la m�diane issue
   de l'angle droit. AC=2,3cm. 
  
  \ul{Propri�t�}: Si un triangle est rectangle alors la m�diane issue de
  l'angle droit a pour longueur la moiti� de celle de l'hypot�nuse. 
   
  
  \ul{Conclusion}: MN=2 $\times$ AC=2 $\times$ 2,3=4,6cm
 
  
   \medskip
   
     
  \item  
  
On sait que MBN est un triangle rectangle en B, [BD] est la m�diane issue
   de l'angle droit et MN=4,6cm.
  
 En utilisant la m�me propri�t� qu'� la question 1., on en d�duit que: 
BC=$\dfrac{MN}{2}$ =$\dfrac{4,6}{2}$ =2,3cm  
\end{enumerate}

\bigskip




\ul{Exercice 4}:

On construit le cercle de diam�tre [AB]. On place un point P sur ce cercle.

D'apr�s la propri�t�: ``si un triangle est inscrit dans un cercle de diam�tre
l'un de ses c�t�s alors ce triangle est rectangle et admet ce c�t� pour
hypot�nuse'', le triangle ABP est rectangle en P.


De m�me, on construit le cercle de diam�tre [CD]. On place P sur le cercle. Le
triangle CDP est rectangle en P.



Pour avoir ABP et CDP rectangles en P (un seul point), je dois donc placer P �
l'intersection de ces deux cercles.


\bigskip




\ul{Exercice 5}:

\begin{tabular}{cc}
\begin{minipage}{13cm}
2) \ul{donn�es}: IJK est inscrit dans le cercle $\mathcal{C}$. [IJ] est un
diam�tre du cercle.

\ul{propri�t�}: Si un triangle est inscrit dans un cercle de diam�tre l'un de
ses c�t�s alors ce triangle est rectangle et admet ce c�t� pour hypot�nuse.


\ul{conclusion}: IJK est rectangle en K.

  
\end{minipage}

&
\begin{minipage}{5cm}
\quad
\end{minipage}
\end{tabular}

\enskip


3) Le triangle IJK est rectangle en K. 
D'apr�s le th�or�me de Pythagore, on a:

\qquad \qquad 
\qquad \qquad $IJ^2=IK^2+JK^2$


\qquad \qquad 
\qquad \qquad $8^2=IK^2+3,5^2$


\qquad \qquad 
\qquad \qquad $64=IK^2+12,25$


\qquad \qquad 
\qquad \qquad $IK^2=64-12,25=51,75$

La longueur JK est positive donc $IK=\sqrt{51,75} \approx 7,2cm$.

\enskip

4) Le triangle IJK est rectangle en K donc $cos(\widehat{IJK})=\dfrac{KJ}{IJ}$.

 $cos(\widehat{IJK})=\dfrac{3,5}{8}$
donc $\widehat{IJK}=arccos (3,5 \div 8)\approx 64$�.  \qquad ( \fbox{2de}
\fbox{cos} � la calculatrice.)

\end{document}
