\documentclass[12pt, twoside]{article}
\usepackage[francais]{babel}
\usepackage[T1]{fontenc}
\usepackage[latin1]{inputenc}
\usepackage[left=7mm, right=7mm, top=4mm, bottom=4mm]{geometry}
\usepackage{float}
\usepackage{graphicx}
\usepackage{array}
\usepackage{multirow}
\usepackage{amsmath,amssymb,mathrsfs}
\usepackage{soul}
\usepackage{textcomp}
\usepackage{eurosym}
\usepackage{variations}
\usepackage{tabvar}

\pagestyle{empty}

\begin{document}

\begin{flushleft}
NOM PRENOM: \ldots \ldots \ldots \ldots \ldots \ldots \ldots \ldots \ldots
 
\bigskip

\end{flushleft}

\begin{center}
{\fbox{$5^{e}2$ \qquad \qquad \textbf{\Large{Contr�le de cours 7 (sujet 1)}}
\qquad \qquad 01/04/2010}}
\end{center}

\bigskip



\ul{Exercice 1}: Compl�ter par $=$ ou $\neq$. Justifier les r�ponses.\\
\begin{center}
$\dfrac{2}{3}$ \ldots \ldots$\dfrac{6}{9}$ \quad \quad \quad\quad\quad\quad
\qquad $\dfrac{8}{21}$ \ldots \ldots$\dfrac{16}{43}$ \quad \quad
\quad\quad\quad\quad \qquad$\dfrac{20}{70}$\ldots \ldots$\dfrac{10}{35}$ 
\end{center}

\bigskip
\bigskip
\ul{Exercice 2}: Compl�ter les in�galit�s. Justifier les r�ponses.
\begin{center}
$\dfrac{18}{24}=\dfrac{6}{\ldots\ldots}$ \quad \quad \quad\quad\quad\quad
\qquad $\dfrac{1}{6}=\dfrac{\ldots\ldots}{30}$
\quad \quad \quad\quad \quad\quad \qquad
$\dfrac{\ldots\ldots}{4}=\dfrac{28}{40}$
\end{center}

\bigskip
\bigskip
\ul{Exercice 3}: Simplifier les fractions le plus possible : \\
\begin{center}
$\dfrac{30}{25}=$\ldots\ldots\ldots\ldots\ldots\ldots\ldots\ldots\ldots\ldots\ldots\ldots\\
\bigskip
$\dfrac{16}{24}=$\ldots\ldots\ldots\ldots\ldots\ldots\ldots\ldots\ldots\ldots\ldots\ldots\\
\end{center}

\bigskip
\bigskip


\begin{flushleft}
NOM PRENOM: \ldots \ldots \ldots \ldots \ldots \ldots \ldots \ldots \ldots
 
\bigskip

\end{flushleft}

\begin{center}
{\fbox{$5^{e}2$ \qquad \qquad \textbf{\Large{Contr�le de cours 7 (sujet 2)}}
\qquad \qquad 01/04/2010}}
\end{center}

\bigskip



\ul{Exercice 1}: Completer par $=$ ou $\neq$. Justifier les r�ponses.\\
\begin{center}
$\dfrac{3}{5}$ \ldots $\dfrac{9}{15}$ \quad \quad \quad \quad\quad\quad\quad
\qquad $\dfrac{17}{25}$ \ldots $\dfrac{34}{50}$ \quad \quad \quad
\quad\quad\quad\quad \qquad $\dfrac{7}{26}$\ldots $\dfrac{14}{29}$
\end{center}

\bigskip
\bigskip
\ul{Exercice 2}: Completer les in�galit�s. Justifier les r�ponses.
\begin{center}
$\dfrac{12}{60}=\dfrac{2}{\ldots\ldots}$ \quad \quad \quad \quad\quad\quad\quad
\qquad $\dfrac{2}{7}=\dfrac{\ldots\ldots}{35}$ \quad \quad \quad \quad\quad
\quad\quad $\dfrac{\ldots\ldots}{5}=\dfrac{17}{50}$
\end{center}

\bigskip
\bigskip
\ul{Exercice 3}: Simplifier les fractions le plus possible : \\
\begin{center}
$\dfrac{20}{30}=$\ldots\ldots\ldots\ldots\ldots\ldots\ldots\ldots\ldots\ldots\ldots\ldots\\
\bigskip
$\dfrac{35}{25}=$\ldots\ldots\ldots\ldots\ldots\ldots\ldots\ldots\ldots\ldots\ldots\ldots\\
\end{center}
\end{document}
