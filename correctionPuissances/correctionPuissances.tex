%%This is a very basic article template.
%%There is just one section and two subsections.

\documentclass[12pt, twoside]{article}

\usepackage[francais]{babel}
\usepackage[T1]{fontenc}
\usepackage[latin1]{inputenc}
\usepackage[left=2cm, right=2cm, top=2cm, bottom=2cm]{geometry}
\usepackage{float}
\usepackage{graphicx}
\usepackage{array}
\usepackage{multirow}
\usepackage{amsmath, amssymb, mathrsfs}

\begin{document}

\section*{\center{Correction exercices puissances}}

\bigskip
\bigskip
\subsection*{Exercice 1}
$(\dfrac{-5}{11})^{3}\times(\dfrac{11}{5})^{4}=\dfrac{(-1)^3\times5^3\times11^4}{11^3\times5^4}=-\dfrac{11}{5}$

\subsection*{Exercice 2}
Le signe de l'expression
$\dfrac{(-1,2)^{4}\times(-2)^{13}\times(-1)^{2}}{(-5)^{6}\times(-7)^{21}}$ est "$+$".

\subsection*{Exercice 3}

\begin{equation}
A=\frac{(0,2)^{3}\times10^{4}}{2^{3}\times81}: \frac{8^{3}\times15}{12^{5}} =
\frac{(0,2)^{3}\times10^{4}}{2^{3}\times81}\times \frac{12^{5}}{8^{3}\times15}
$$\bigskip$$
=\frac{(10^{-1}\times2)^3\times(2\times5)^4\times(2\times6)^5}{2^3\times9^2\times(2^3)^3\times3\times5}
=\frac{10^{-3}\times2^3\times2^4\times5^4\times(2^2\times3)^5}{2^3\times(3^2)^2\times2^9\times3\times5}
$$\bigskip$$
=\frac{2^7\times(2\times5)^{-3}\times5^4\times2^{10}\times3^5}{2^{12}\times3^4\times3\times5}=\frac{2^{14}\times3^5\times5}{2^{12}\times3^3\times5}
$$\bigskip$$
=\frac{2^{14}}{2^{12}}=2^2=4$$\notag$$
\end{equation}
 $A=4$ donc $A$ est un nombre entier.
\subsection*{Exercice 4}
Soit $x=9\times10^7$ et $y=3.6\times10^8=36\times10^7$.

On a alors:
\bigskip
\begin{itemize}
  \item[$\bullet$] $xy=9\times10^7\times36\times10^7=324\times10^{14}$
  \medskip
  \item[$\bullet$] $\dfrac{x}{y}=\dfrac{9\times10^7}{36\times10^7}=\dfrac{1}{4}$
\end{itemize}
\end{document}
