\documentclass[12pt, twoside]{article}
\usepackage[francais]{babel}
\usepackage[T1]{fontenc}
\usepackage[latin1]{inputenc}
\usepackage[left=7mm, right=7mm, top=7mm, bottom=7mm]{geometry}
\usepackage{float}
\usepackage{graphicx}
\usepackage{array}
\usepackage{multirow}
\usepackage{amsmath,amssymb,mathrsfs}
\usepackage{soul}
\usepackage{textcomp}
\usepackage{eurosym}
 \usepackage{variations}
\usepackage{tabvar}


\pagestyle{empty}

\begin{document}


\section*{\center{Correction devoir surveill� 4}}


\subsection*{Exercice 2}


\textbf{Triangle ABC}:

\begin{itemize}
  \item [$\bullet$] Tracer un segment [AB] de longueur 4,5cm.
  \item [$\bullet$] Tracer la perpendiculaire � (AB) passant par B. Placer le
  point C sur cette droite tel que BC=6cm.
  \item [$\bullet$] Tracer le segment [AC].
  
\end{itemize}

\textbf{Triangle ABC}:

\begin{itemize}
  \item [$\bullet$] Construire l'angle $\widehat{ACD}$ de mesure 20�(�
  l'ext�rieur du triangle ABC). Prolonger la demi-droite obtenuue.
  
  \item [$\bullet$] Placer le point D sur cette demi-droite tel que DC=CA (en
  reportant la longueur � l'aide du compas).
  
  \item [$\bullet$] Tracer le segment [DC].
\end{itemize}


\subsection*{Exercice 3}

\ul{SUJET 1}:

\begin{enumerate}
  \item Le triangle AEI est tel que AI=5mm; AE=7mm et EI=1,5mm. 
  
  Le plus long
  c�t� est [AE]. 7<5+1,5 donc AEI est constructible.
  \item Le triangle IJK est tel que IJ=10,1m; IK=5,6m et KJ=4,8m.  
  
  Le plus long
  c�t� est [IJ]. 10,1>5,6+4,8 donc IJK n'est pas constructible. 
\end{enumerate}

\enskip

\ul{SUJET 2}:

\begin{enumerate}
  \item Le triangle AEI est tel que AE=5,5mm; AI=7mm et EI=1mm.

Le plus long
  c�t� est [AI]. 7<5,5+1 donc AEI est constructible.
  \item Le triangle IJK est tel que IJ=10,2m; IK=4,7m et KJ=5,6m.  
  
  Le plus long
  c�t� est [IJ]. 10,2>5,6+4,7 donc IJK n'est pas constructible. 
\end{enumerate}


\subsection*{Exercice 4}

\ul{SUJET 1}:

\begin{enumerate}
  \item On sait que AB=3cm; AC=7cm et BC=4cm. Donc $B \in [AC]$.
  \item R, S et T sont trois points tels que RS=11cm; RT=2cm et $T \in [RS]$.
  Donc TS=11-2=9cm.
\end{enumerate}

\enskip

\ul{SUJET 2}:

\begin{enumerate}
  \item On sait que AC=3cm; CB=7cm et AB=4cm. Donc $A \in [BC]$.
  \item O, U et I sont trois points tels que UO=11cm; UI=2cm et $I \in [UO]$.
  Donc IO=11-2=9cm.
\end{enumerate}


\subsection*{Exercice 6}

\begin{enumerate}
  \item Le traingle ABE est isoc�le en E. On en d�duit que les angles
  $\widehat{EAB}$ et $\widehat{ABE}$ ont m�me mesure. Donc
  $\widehat{ABE}=\widehat{EAB}=40$�.
  \item Le trinagle DBC est isoc�le en D. On en d�duit que les angles
  $\widehat{DBC}$ et $\widehat{DCB}$ ont m�me mesure. La somme des angles d'un
  triangle vaut 180�. Donc:
  $\widehat{DBC}=\widehat{DCB}=\dfrac{180-70}{2}=\dfrac{110}{2}=55$�.
  \item $\widehat{ABE}=40$�, $\widehat{EBD}=90$� et $\widehat{DBC}=55$�.
  
  40+90+55=185. L'angle $\widehat{ABC}$ n'est pas un angle plat donc les points
  A, B et C ne sont pas align�s.
\end{enumerate}

\end{document}
