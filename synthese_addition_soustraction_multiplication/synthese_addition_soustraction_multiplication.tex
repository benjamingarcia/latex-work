\documentclass[12pt, twoside]{article}
\usepackage[francais]{babel}
\usepackage[T1]{fontenc}
\usepackage[latin1]{inputenc}
\usepackage[left=8mm, right=8mm, top=8mm, bottom=8mm]{geometry}
\usepackage{float}
\usepackage{graphicx}
\usepackage{array}
\usepackage{multirow}
\usepackage{amsmath,amssymb,mathrsfs}
\pagestyle{empty}
\begin{document}

\center{\textbf{\Large{Bilan additions, soustractions et multiplications}}}




\bigskip

\begin{center}
\begin{tabular}{|m{9cm}|m{10cm}|}
\hline
\textbf{Ce que je dois savoir} & \textbf{Ce que je dois savoir faire} \\
\hline

\enskip


\begin{itemize}
  \item[$\bullet$] Je connais le vocabulaire: somme, diff�rence, termes,
  produit et facteurs.
  \item [$\bullet$] Je connais la m�thode pour additionner et soustraire des
  horaires.
  \item [$\bullet$] Je sais convertir des grandeurs (volume, masse et distance).
  
  \item [$\bullet$] Je connais mes tables de multiplication.
  \item [$\bullet$] Je connais la m�thode pour poser et effectuer une
  multiplication.
  \item [$\bullet$] Je connais la r�gle permettent de multiplier un nombre par
  10; par 100; par 1000\ldots
  \item [$\bullet$]  Je connais la r�gle permettant de multiplier un nombre par
  0,1; par 0,01; par 0,01 \ldots
  \item [$\bullet$] Je sais trouver un ordre de grandeur. 

\end{itemize}
&

\enskip
\begin{itemize}
  \item[$\bullet$] Je sais effectuer une addition ou une soustraction en ligne.
  \item[$\bullet$] Je sais effectuer une addition ou une soustraction en
  colonne.
  \item[$\bullet$] Je sais calculer mentalement une addition, une
  soustraction ou un produit (en utilisant des astuces).
 \item[$\bullet$] Je sais r�soudre des probl�mes d'horaires.
  \item[$\bullet$] Je sais effectuer des multiplications en ligne et en colonne.

  \item[$\bullet$] Je sais r�soudre des probl�mes avec plusieurs op�rations (+,
  -, x).
  \item[$\bullet$] Je sais multiplier par 10, par 100, par 0,1; par 0,01\ldots
  \item[$\bullet$] Je sais r�soudre des probl�mes en utilisant les ordres de
  grandeur.
  \item[$\bullet$] Je sais v�rifier un r�sultat � la calculatrice.
   \end{itemize} \\
\hline 

\end{tabular}
\end{center}

\bigskip


\bigskip

\bigskip


\bigskip

\center{\textbf{\Large{Bilan additions, soustractions et multiplications}}}




\bigskip

\begin{center}
\begin{tabular}{|m{9cm}|m{10cm}|}
\hline
\textbf{Ce que je dois savoir} & \textbf{Ce que je dois savoir faire} \\
\hline

\enskip


\begin{itemize}
  \item[$\bullet$] Je connais le vocabulaire: somme, diff�rence, termes,
  produit et facteurs.
  \item [$\bullet$] Je connais la m�thode pour additionner et soustraire des
  horaires.
  \item [$\bullet$] Je sais convertir des grandeurs (volume, masse et distance).
  
  \item [$\bullet$] Je connais mes tables de multiplication.
  \item [$\bullet$] Je connais la m�thode pour poser et effectuer une
  multiplication.
  \item [$\bullet$] Je connais la r�gle permettent de multiplier un nombre par
  10; par 100; par 1000\ldots
  \item [$\bullet$]  Je connais la r�gle permettant de multiplier un nombre par
  0,1; par 0,01; par 0,01 \ldots
  \item [$\bullet$] Je sais trouver un ordre de grandeur. 

\end{itemize}
&

\enskip
\begin{itemize}
  \item[$\bullet$] Je sais effectuer une addition ou une soustraction en ligne.
  \item[$\bullet$] Je sais effectuer une addition ou une soustraction en
  colonne.
  \item[$\bullet$] Je sais calculer mentalement une addition, une
  soustraction ou un produit (en utilisant des astuces).
 \item[$\bullet$] Je sais r�soudre des probl�mes d'horaires.
  \item[$\bullet$] Je sais effectuer des multiplications en ligne et en colonne.

  \item[$\bullet$] Je sais r�soudre des probl�mes avec plusieurs op�rations (+,
  -, x).
  \item[$\bullet$] Je sais multiplier par 10, par 100, par 0,1; par 0,01\ldots
  \item[$\bullet$] Je sais r�soudre des probl�mes en utilisant les ordres de
  grandeur.
  \item[$\bullet$] Je sais v�rifier un r�sultat � la calculatrice.
   \end{itemize} \\
\hline 

\end{tabular}
\end{center}
\end{document}
