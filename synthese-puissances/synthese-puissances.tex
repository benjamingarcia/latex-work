\documentclass[12pt, twoside]{article}
\usepackage[francais]{babel}
\usepackage[T1]{fontenc}
\usepackage[latin1]{inputenc}
\usepackage[left=5mm, right=5mm, top=5mm, bottom=5mm]{geometry}
\usepackage{float}
\usepackage{graphicx}
\usepackage{array}
\usepackage{multirow}
\usepackage{amsmath,amssymb,mathrsfs}
\pagestyle{empty}
\begin{document}

\section*{\center{Bilan puissances}}




\bigskip

\begin{center}
\begin{tabular}{|m{9cm}|m{9cm}|}
\hline


\textbf{Ce que je dois savoir} & \textbf{Ce que je dois savoir faire} \\

\hline

\enskip


\begin{itemize}
  \item [$\bullet$] Je connais la signification des notations $a^n$ et $a^{-n}$.
  
  
  \item [$\bullet$] Je connais le vocabulaire: carr�, cube, facteur, puissance,
  exposant.
  
  \item [$\bullet$] Je connais les r�gles de priorit� de calcul.
  
  \item [$\bullet$] Je connais la m�thode pour multiplier ou diviser par une
  puissance de 10.
  
  
  \item [$\bullet$] Je connais les trois r�gles de calcul (produit, quotient,
  puissance de puissance) pour le cas particulier des puissances de 10.
  
  \item [$\bullet$] Je sais reconna�tre si un nombre est en notation
  scientifique.

 \end{itemize}

&


\begin{itemize}
  \item[$\bullet$] Je sais uitliser la calculatrice pour calculer des
  puissances.


  \item[$\bullet$] Je sais calculer $a^0$ et $a^1$.

  
      
  \item[$\bullet$] Je sais calculer avec des puissances.
  
  \item[$\bullet$] Je sais �crire un produit avec la notation puissance.
  
  \item[$\bullet$] Je sais multiplier et diviser par une puissance de 10.
  
  \item[$\bullet$] Je sais appliquer les 3 r�gles sur les puissances de 10
  (produit, quotient, puissance de puissance).
  
  \item[$\bullet$] Je sais �crire un nombre en notation scientifique.
  
  \item[$\bullet$] Je sais utiliser la calculatrice pour les notations
  scientifiques.
 
\end{itemize} \\

\hline

\end{tabular}
\end{center}


\bigskip



\section*{\center{Bilan puissances}}




\bigskip

\begin{center}
\begin{tabular}{|m{9cm}|m{9cm}|}
\hline


\textbf{Ce que je dois savoir} & \textbf{Ce que je dois savoir faire} \\

\hline

\enskip


\begin{itemize}
  \item [$\bullet$] Je connais la signification des notations $a^n$ et $a^{-n}$.
  
  
  \item [$\bullet$] Je connais le vocabulaire: carr�, cube, facteur, puissance,
  exposant.
  
  \item [$\bullet$] Je connais les r�gles de priorit� de calcul.
  
  \item [$\bullet$] Je connais la m�thode pour multiplier ou diviser par une
  puissance de 10.
  
  
  \item [$\bullet$] Je connais les trois r�gles de calcul (produit, quotient,
  puissance de puissance) pour le cas particulier des puissances de 10.
  
  \item [$\bullet$] Je sais reconna�tre si un nombre est en notation
  scientifique.

 \end{itemize}

&


\begin{itemize}
  \item[$\bullet$] Je sais uitliser la calculatrice pour calculer des
  puissances.


  \item[$\bullet$] Je sais calculer $a^0$ et $a^1$.

  
      
  \item[$\bullet$] Je sais calculer avec des puissances.
  
  \item[$\bullet$] Je sais �crire un produit avec la notation puissance.
  
  \item[$\bullet$] Je sais multiplier et diviser par une puissance de 10.
  
  \item[$\bullet$] Je sais appliquer les 3 r�gles sur les puissances de 10
  (produit, quotient, puissance de puissance).
  
  \item[$\bullet$] Je sais �crire un nombre en notation scientifique.
  
  \item[$\bullet$] Je sais utiliser la calculatrice pour les notations
  scientifiques.
 
\end{itemize} \\

\hline

\end{tabular}
\end{center}



\end{document}
