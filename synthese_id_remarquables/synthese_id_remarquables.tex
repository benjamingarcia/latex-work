\documentclass[12pt, twoside]{article}
\usepackage[francais]{babel}
\usepackage[T1]{fontenc}
\usepackage[latin1]{inputenc}
\usepackage[left=7mm, right=7mm, top=7mm, bottom=7mm]{geometry}
\usepackage{float}
\usepackage{graphicx}
\usepackage{array}
\usepackage{multirow}
\usepackage{amsmath,amssymb,mathrsfs}
\pagestyle{empty}
\begin{document}

\section*{\center{Bilan identit�s remarquables}}




\bigskip

\begin{center}
\begin{tabular}{|m{9cm}|m{9cm}|}
\hline


\textbf{Ce que je dois savoir} & \textbf{Ce que je dois savoir faire} \\

\hline

\enskip


\begin{itemize}
  \item [$\bullet$] Je sais reconnaitre si une expression est une somme ou un
  produit.
  
  
  \item [$\bullet$] Je connais le vocabulaire: ``d�velopper'' et ``factoriser''.
  
  \item [$\bullet$] Je sais r�duire une expression.
  
  \item [$\bullet$] Je connais les trois identit�s remarquables dans les deux
  sens PAR COEUR.
  
  \item[$\bullet$] Je sais reconnaitre quelle identit� remarquable
  utiliser.
  
 
 \end{itemize}

&


\begin{itemize}
  \item[$\bullet$] Je sais d�velopper une expression en utilisant la
  distributivit� (simple ou double).


\item[$\bullet$] Je sais d�velopper une expression en utilisant les
identit�s remarquables.

\item[$\bullet$] Je sais factoriser une expression en trouvant un facteur
commun.

\item[$\bullet$] Je sais factoriser une expression en utilisant les
identit�s remarquables.



  
      
  \item[$\bullet$] Je sais r�soudre des probl�mes.
 
\end{itemize} \\

\hline

\end{tabular}
\end{center} 

\bigskip

\section*{\center{Bilan identit�s remarquables}}




\bigskip

\begin{center}
\begin{tabular}{|m{9cm}|m{9cm}|}
\hline


\textbf{Ce que je dois savoir} & \textbf{Ce que je dois savoir faire} \\

\hline

\enskip


\begin{itemize}
  \item [$\bullet$] Je sais reconnaitre si une expression est une somme ou un
  produit.
  
  
  \item [$\bullet$] Je connais le vocabulaire: ``d�velopper'' et ``factoriser''.
  
  \item [$\bullet$] Je sais r�duire une expression.
  
  \item [$\bullet$] Je connais les trois identit�s remarquables dans les deux
  sens PAR COEUR.
  
  \item[$\bullet$] Je sais reconnaitre quelle identit� remarquable
  utiliser.
  
 
 \end{itemize}

&


\begin{itemize}
  \item[$\bullet$] Je sais d�velopper une expression en utilisant la
  distributivit� (simple ou double).


\item[$\bullet$] Je sais d�velopper une expression en utilisant les
identit�s remarquables.

\item[$\bullet$] Je sais factoriser une expression en trouvant un facteur
commun.

\item[$\bullet$] Je sais factoriser une expression en utilisant les
identit�s remarquables.



  
      
  \item[$\bullet$] Je sais r�soudre des probl�mes.
 
\end{itemize} \\

\hline

\end{tabular}
\end{center}

\bigskip

\section*{\center{Bilan identit�s remarquables}}




\bigskip

\begin{center}
\begin{tabular}{|m{9cm}|m{9cm}|}
\hline


\textbf{Ce que je dois savoir} & \textbf{Ce que je dois savoir faire} \\

\hline

\enskip


\begin{itemize}
  \item [$\bullet$] Je sais reconnaitre si une expression est une somme ou un
  produit.
  
  
  \item [$\bullet$] Je connais le vocabulaire: ``d�velopper'' et ``factoriser''.
  
  \item [$\bullet$] Je sais r�duire une expression.
  
  \item [$\bullet$] Je connais les trois identit�s remarquables dans les deux
  sens PAR COEUR.
  
  \item[$\bullet$] Je sais reconnaitre quelle identit� remarquable
  utiliser.
  
 
 \end{itemize}

&


\begin{itemize}
  \item[$\bullet$] Je sais d�velopper une expression en utilisant la
  distributivit� (simple ou double).


\item[$\bullet$] Je sais d�velopper une expression en utilisant les
identit�s remarquables.

\item[$\bullet$] Je sais factoriser une expression en trouvant un facteur
commun.

\item[$\bullet$] Je sais factoriser une expression en utilisant les
identit�s remarquables.



  
      
  \item[$\bullet$] Je sais r�soudre des probl�mes.
 
\end{itemize} \\

\hline

\end{tabular}
\end{center}


\end{document}
