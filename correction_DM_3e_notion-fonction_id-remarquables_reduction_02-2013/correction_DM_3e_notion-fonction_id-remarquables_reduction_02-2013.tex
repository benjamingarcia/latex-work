\documentclass[12pt, twoside]{article}
\usepackage[francais]{babel}
\usepackage[T1]{fontenc}
\usepackage[latin1]{inputenc}
\usepackage[left=5mm, right=5mm, top=5mm, bottom=5mm]{geometry}
\usepackage{float}
\usepackage{graphicx}
\usepackage{array}
\usepackage{multirow}
\usepackage{amsmath,amssymb,mathrsfs}
\usepackage{soul}
\usepackage{textcomp}
\usepackage{eurosym}
 \usepackage{variations}
\usepackage{tabvar}

\pagestyle{empty}
\begin{document}

\begin{center}
\fbox{Correction du devoir maison 5}
\end{center}


\ul{Exercice 1:}

\begin{enumerate}
  \item \begin{enumerate}
          \item [a)] $2^2=4$ \qquad $4 \times 5=20$ \qquad $20+10=30$ \qquad Le
          calcul de Marc est exact.
          \item [b)] $0,1^2=0,01$ \qquad $0,01 \times 5=0,05$ \qquad
          $0,05+10=10,05$ \qquad Robin trouve 10,05.
\end{enumerate}

\item \begin{enumerate}
        \item[a)] $p(x)=x^2 \times 5 +10=5x^2+10$
        \item[b)] $p(-1)=(-1)^2 \times 5+10=1 \times 5+10=15$ \qquad $p(3)=3^2
        \times 5+10=9 \times 5+10=55$ \qquad $p(0)=0^2 \times 5+10=10$
        \item[c)]  $p(0,2)=0,2^2 \times 5+10=0,04 \times
        5+10=0,2+10=10,2$.
        
        $p(0,2)=10,2$ donc 0,2 est l'ant�c�dent de 10,2.
\end{enumerate}
\end{enumerate}

\bigskip

\ul{Exercice 2:} 

\begin{enumerate}
  \item Axe des abscisses: temps (en minutes) ; axe des ordonn�es: distance
  parcourue (en km)
  \item \begin{enumerate}
          \item La distance parcourue n'�volue pas entre la 20i�me et la
          30i�me minute: le coureur s'est donc arr�t� environ 10 minutes.
          \item Au bout de 5 minutes, il a parcouru un kilom�tre. 
          \item Il a mis environ 32,5 minutes pour parcourir 4 kilom�tres.
      
\end{enumerate}
\item \begin{enumerate}
        \item Le coureur a parcouru 2 kilom�tres en 10 minutes donc $d(10)=2$.
        L'image de 10 par $d$ est 2.
        \item $d(35)=6$; \quad  35 est l'ant�c�dent de 6 par $d$.
\end{enumerate}
\item 

\begin{tabular}{cc}

\begin{minipage}{6cm}

\begin{tabular}{|c|c|c|}
\hline
distance (en km) & 6 & y \\
\hline
temps (en min) & 35 & 60 \\
\hline
\end{tabular}

\end{minipage}
&
\begin{minipage}{12cm}
$y=6 \times 60 \div 35 \approx 10,3$

La vitesse moyenne du coureur est d'environ 10,3 km/h.
\end{minipage}

\end{tabular}
\end{enumerate}


\bigskip

\ul{Exercice 3:}

\begin{enumerate}
  \item Le premier graphique ne correspond pas au tableau de valeurs de la
  fonction $f$ car:
  
  $\bullet$ $f(3)=-3$. La repr�sentation
  graphique ne contient pas le point (-3;3) (� ne pas confondre avec (3;-3));
   
   $\bullet$ $f(1)=2$ mais le point (1;2)
   n'appartient pas � la courbe (ne pas confondre avec le point (2;1));
   
   $\bullet$ $f(2)=3$ mais le point (2;3)
   n'appartient pas � la courbe (ne pas confondre avec le point (3;2)).   
   
   \item Le deuxi�me graphique ne correspond pas au tableau de valeurs de la
  fonction $f$ car:  
  
   $\bullet$  $f(-2)=0$ mais le point (-2;0)
   n'appartient pas � la courbe (ne pas confondre avec le point (0;-2)). 
\end{enumerate}

\bigskip


\ul{Exercice 4:}

\begin{enumerate}
  \item $D=(2u+3)^2+(2u+3)(7u-2)=(2u)^2+2 \times 2u \times 3+3^2+2u \times
  7u+2u \times (-2)+3 \times 7u + 3 \times (-2)$
  
  \qquad \qquad \qquad \qquad \qquad \qquad \qquad  \quad
  $=4u^2+12u+9+14u^2-4u+21u-6=18u^2+29u+3$
  
  \enskip
  
  \item $D=(2u+3)\times (2u+3)+(2u+3)\times (7u-2)=(2u+3) \big[ (2u+3)+(7u-2)
  \big ]$
  
  \quad  $= (2u+3)(2u+3+7u-2)=(2u+3)(9u+1)$
 
  \enskip
  
    
  \item $D=(2u+3)(9u+1)=(2 \times (-4)+3)(9 \times (-4)+1)=(-8+3)(-36+1)=-5
  \times (-35)=175$
  
  \enskip
  
  \item $D=(2u+3)(9u+1)=2u \times 9u + 2u \times 1+ 3 \times 9u + 3 \times
  1=18u^2+2u+27u+3=18u^2+29u+3$
  
  On retrouve le m�me r�sultat qu'� la question 1.
\end{enumerate}


\bigskip

\ul{Exercice 5:}

\begin{enumerate}
  \item D�veloppons $A=(y+1)^2-(y-1)^2$ en utilisant les identit�s remarquables:
  
  $A=y^2+2\times y \times 1 - (y^2-2\times y \times
  1)=y^2+2y+1-(y^2-2y+1)=y^2+2y+1-y^2+2y-1=4y$
    
  \item Factorisons $A=(y+1)^2-(y-1)^2$ en utilisant les identit�s remarquables:
  
  $A=\big( (y+1)+(y-1) \big) \big( (y+1)-(y-1) \big)=(y+1+y-1)(y+1-y+1)=2y
  \times 2=4y$
  
  \item $1001^2-999^2=(1000+1)^2-(1000-1)^2$
  
  On retrouve l'expression A pour $y=1000$ donc $1001^2-999^2=4 \times
  1000=4000$.
\end{enumerate}


\bigskip


\ul{Exercice 6:}

\enskip

On note $n$ le nombre de moutons.

Le nombre de poules est $\dfrac{n}{3}$.


Le nombre de pattes de l'ensemble des moutons est $4 \times n=4n$.

Le nombre de pattes de l'ensemble des poules est $2 \times
\dfrac{n}{3}=\dfrac{2n}{3} $.

Le nombre de pattes du chien est 4.


Le nombre total de pattes est $4n+\dfrac{2n}{3} +4$. D'apr�s l'�nonc�, ce total
vaut 172. 

On a donc l'�quation suivante: $4n+\dfrac{2n}{3} +4=172$


\enskip

\begin{tabular}{l|l|l}
\ul{R�duction de l'�quation:} & \qquad \ul{R�solution:} & \qquad 
\ul{V�rification:}  \\

\quad & \quad & \quad \\

$4n+\dfrac{2n}{3} +4=172$ & \qquad $\dfrac{14n}{3}+4-4=172-4$ &  \qquad Il y a
36 moutons et 12 poules. \\
\quad & \quad & \quad \\
$\dfrac{12n}{3}+\dfrac{2n}{3} +4=172$ & \qquad $\dfrac{14n}{3}=168$ & \qquad
$36 \times 4=144$  et $12 \times 2=24$ \\
\quad & \quad & \quad \\
$\dfrac{14n}{3}+4=172$ & \qquad $\dfrac{14}{3} \times n \div \dfrac{14}{3}=168
\div \dfrac{14}{3}$ &  \qquad Il y a 144 pattes de moutons et 24 de
poules. \\
\quad & \quad & \quad \\
\quad & \qquad $n=168 \times \dfrac{3}{14}$ & \qquad $144+24+4=172$ \\
\quad & \quad & \quad \\
\quad & \qquad $n=36$ & \qquad  \\
\end{tabular}


\bigskip


\ul{Exercice 7:}


\begin{enumerate}
  \item A'B'C'D' est une r�duction de ABCD de coefficient 0,4. 
  
  $A'B'=0,4 AB=0,4 \times 12=4,8$ cm \qquad et \qquad $E'C'=0,4 EC=0,4 \times
  4=1,6$ cm
  
  $\widehat{E'B'C'}=\widehat{EBC}=30$� car les mesures d'angles sont conserv�es
  par r�duction.
  \item $Aire_{ABCD}=Base \times hauteur=12 \times 4=48cm^2$
  \item A'B'C'D' est une r�duction de ABCD de coefficient 0,4 donc
  $Aire_{A'B'C'D'}=0,4^2 \times Aire_{ABCD}$ .
  
  $Aire_{A'B'C'D'}=0,4^2 \times 48=7,68 cm^2$.
\end{enumerate}

\end{document}
