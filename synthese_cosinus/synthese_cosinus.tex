\documentclass[12pt, twoside]{article}
\usepackage[francais]{babel}
\usepackage[T1]{fontenc}
\usepackage[latin1]{inputenc}
\usepackage[left=8mm, right=8mm, top=8mm, bottom=8mm]{geometry}
\usepackage{float}
\usepackage{graphicx}
\usepackage{array}
\usepackage{multirow}
\usepackage{amsmath,amssymb,mathrsfs}
\pagestyle{empty}
\begin{document}

\section*{\center{Bilan cosinus d'un angle aigu}}




\bigskip

\begin{center}
\begin{tabular}{|m{9cm}|m{9cm}|}
\hline


\textbf{Ce que je dois savoir} & \textbf{Ce que je dois savoir faire} \\

\hline

\enskip


\begin{itemize}
  \item [$\bullet$] Je sais nommer un angle.
  
  
  \item [$\bullet$] Je sais reconna�tre le c�t� adjacent d'un angle.
  
  \item [$\bullet$] Je sais �crire la formule du cosinus d'un angle.

\item [$\bullet$] Je sais que le cosinus d'un angle est un nombre compris entre
0 et 1.

\item [$\bullet$] Je sais utiliser ma calculatrice pour calculer le cosinus
d'un angle.

\item [$\bullet$] Je sais utiliser ma calculatrice pour calculer une mesure
d'angle (connaissant son cosinus).
 \end{itemize}

&


\begin{itemize}
  \item[$\bullet$] Je sais calculer la mesure d'un angle.


\item[$\bullet$] Je sais calculer la longueur du c�t� adjacent � l'angle connu.

\item[$\bullet$] Je sais calculer la longueur de l'hypot�nuse.
\item[]
  \item[$\bullet$] Je sais r�diger ma r�ponse.

  
      
  \item[$\bullet$] Je sais r�soudre des probl�mes.
 
\end{itemize} \\

\hline

\end{tabular}
\end{center}


\bigskip



\section*{\center{Bilan cosinus d'un angle aigu}}




\bigskip

\begin{center}
\begin{tabular}{|m{9cm}|m{9cm}|}
\hline


\textbf{Ce que je dois savoir} & \textbf{Ce que je dois savoir faire} \\

\hline

\enskip


\begin{itemize}
  \item [$\bullet$] Je sais nommer un angle.
  
  
  \item [$\bullet$] Je sais reconna�tre le c�t� adjacent d'un angle.
  
  \item [$\bullet$] Je sais �crire la formule du cosinus d'un angle.

\item [$\bullet$] Je sais que le cosinus d'un angle est un nombre compris entre
0 et 1.

\item [$\bullet$] Je sais utiliser ma calculatrice pour calculer le cosinus
d'un angle.

\item [$\bullet$] Je sais utiliser ma calculatrice pour calculer une mesure
d'angle (connaissant son cosinus).
 \end{itemize}

&


\begin{itemize}
  \item[$\bullet$] Je sais calculer la mesure d'un angle.


\item[$\bullet$] Je sais calculer la longueur du c�t� adjacent � l'angle connu.

\item[$\bullet$] Je sais calculer la longueur de l'hypot�nuse.
\item[]
  \item[$\bullet$] Je sais r�diger ma r�ponse.

  
      
  \item[$\bullet$] Je sais r�soudre des probl�mes.
 
\end{itemize} \\

\hline

\end{tabular}
\end{center}

\bigskip


\section*{\center{Bilan cosinus d'un angle aigu}}




\bigskip

\begin{center}
\begin{tabular}{|m{9cm}|m{9cm}|}
\hline


\textbf{Ce que je dois savoir} & \textbf{Ce que je dois savoir faire} \\

\hline

\enskip


\begin{itemize}
  \item [$\bullet$] Je sais nommer un angle.
  
  
  \item [$\bullet$] Je sais reconna�tre le c�t� adjacent d'un angle.
  
  \item [$\bullet$] Je sais �crire la formule du cosinus d'un angle.

\item [$\bullet$] Je sais que le cosinus d'un angle est un nombre compris entre
0 et 1.

\item [$\bullet$] Je sais utiliser ma calculatrice pour calculer le cosinus
d'un angle.

\item [$\bullet$] Je sais utiliser ma calculatrice pour calculer une mesure
d'angle (connaissant son cosinus).
 \end{itemize}

&


\begin{itemize}
  \item[$\bullet$] Je sais calculer la mesure d'un angle.


\item[$\bullet$] Je sais calculer la longueur du c�t� adjacent � l'angle connu.

\item[$\bullet$] Je sais calculer la longueur de l'hypot�nuse.
\item[]
  \item[$\bullet$] Je sais r�diger ma r�ponse.

  
      
  \item[$\bullet$] Je sais r�soudre des probl�mes.
 
\end{itemize} \\

\hline

\end{tabular}
\end{center}
\end{document}
