\documentclass[12pt, twoside]{article}
\usepackage[francais]{babel}
\usepackage[T1]{fontenc}
\usepackage[latin1]{inputenc}
\usepackage[left=8mm, right=8mm, top=8mm, bottom=8mm]{geometry}
\usepackage{float}
\usepackage{graphicx}
\usepackage{array}
\usepackage{multirow}
\usepackage{amsmath,amssymb,mathrsfs}
\pagestyle{empty}
\begin{document}

\section*{\center{Bilan fractions}}




\bigskip

\begin{center}
\begin{tabular}{|m{9cm}|m{9cm}|}
\hline
\textbf{Ce que je dois savoir} & \textbf{Ce que je dois savoir faire} \\
\hline

\enskip


\begin{itemize}

  

  \item [$\bullet$] Je sais d�terminer le signe d'un quotient.

  \item [$\bullet$] Je sais reconna�tre une fraction irr�ductible.
  
  \item [$\bullet$] Je connais la propri�t� sur l'�galit� des produits en croix.
  
  \item [$\bullet$] Je connais la m�thode pour r�duire au m�me d�nominateur.
  
  \item [$\bullet$] Je connais la r�gle pour additionner et soustraire deux
  nombres en �criture fractionnaire.
  
  \item [$\bullet$] Je connais la r�gle pour multiplier deux nombres en
  �criture fractionnaire.
  
  
  \item [$\bullet$] Je connais la r�gle pour diviser deux nombres.
\end{itemize}
&

\enskip
\begin{itemize}
  \item[$\bullet$] Je sais compl�ter une �galit� de fractions.
  
      
  
 \item[$\bullet$] Je sais utiliser l'�galit� des produits en croix pour
 compl�ter une �galit� de fractions.

 
  \item[$\bullet$] Je sais utiliser l'�galit� des produits en croix pour tester
  si deux fractions sont �gales.
  
  \item[$\bullet$] Je sais simplifier une fraction.
  
  
  \item[$\bullet$] Je sais trouver l'inverse d'un nombre.
  
  
  \item[$\bullet$] Je sais appliquer les r�gles d'addition, soustraction
  , multiplication et division.
  
  
    \item [$\bullet$] Je sais simplifier un produit.

   
  
  
  \item[$\bullet$] Je sais effectuer une suite de calculs en respectant les
  priorit�s op�ratoires.
 
 
 \end{itemize} \\
\hline

\end{tabular}
\end{center}


\bigskip

\section*{\center{Bilan puissances}}

\begin{center}
\begin{tabular}{|m{9cm}|m{9cm}|}
\hline


\textbf{Ce que je dois savoir} & \textbf{Ce que je dois savoir faire} \\

\hline

\enskip


\begin{itemize}
  \item [$\bullet$] Je connais la signification des notations $a^n$ et $a^{-n}$.
  
  
  \item [$\bullet$] Je connais le vocabulaire: carr�, cube, facteur, puissance,
  exposant.
  
  \item [$\bullet$] Je connais les r�gles de priorit� de calcul.
  
  \item [$\bullet$] Je connais la m�thode pour multiplier ou diviser par une
  puissance de 10.
  
  \item [$\bullet$] Je connais les 5 r�gles de calculs pour effectuer des
  op�rations sur les puissances.
  
  
  \item [$\bullet$] Je sais reconna�tre si un nombre est en notation
  scientifique.

 \end{itemize}

&


\begin{itemize}
  \item[$\bullet$] Je sais uitliser la calculatrice pour calculer des
  puissances.


  \item[$\bullet$] Je sais calculer $a^0$ et $a^1$.

  
      
  \item[$\bullet$] Je sais calculer une puissance.
  
  
  \item[$\bullet$] Je sais multiplier et diviser par une puissance de 10.
  
  \item[$\bullet$] Je sais appliquer les 5 r�gles de calcul sur les puissances.
  
  \item[$\bullet$] Je sais simplifier une �criture en utilisant les r�gles de
  priorit�s et de calculs de puissance.
  
  \item[$\bullet$] Je sais �crire un nombre en notation scientifique.
  

 
\end{itemize} \\

\hline

\end{tabular}
\end{center}

\end{document}
