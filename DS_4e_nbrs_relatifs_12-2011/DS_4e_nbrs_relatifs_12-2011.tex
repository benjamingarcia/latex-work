\documentclass[12pt, twoside]{article}
\usepackage[francais]{babel}
\usepackage[T1]{fontenc}
\usepackage[latin1]{inputenc}
\usepackage[left=6mm, right=6mm, top=6mm, bottom=3mm]{geometry}
\usepackage{float}
\usepackage{graphicx}
\usepackage{array}
\usepackage{multirow}
\usepackage{amsmath,amssymb,mathrsfs}
\usepackage{soul}
\usepackage{textcomp}
\usepackage{eurosym}
 \usepackage{variations}
\usepackage{tabvar}


\pagestyle{empty}

\begin{document}

\begin{flushleft}
NOM PRENOM: \ldots \ldots \ldots \ldots \ldots \ldots \ldots \ldots \ldots
 \end{flushleft}


\medskip

\begin{center}
{\fbox{$4^{e}3$ \qquad \qquad \textbf{\Large{Devoir surveill� 3 (sujet 1)}}
\qquad \qquad 19/12/2013}}
\end{center}

\medskip

\textit{Remarque: Les exercices 2, 3 et 7 se font sur la photocopie.}
CALCULATRICE INTERDITE

 
\bigskip

\ul{\textbf{Exercice 1:}} (\textit{1,5 points})

\begin{tabular}{cc}
\begin{minipage}{12cm}
Dans le triangle RST, le point U appartient au segment [ST], le point V
appartient au segment [SR] et les droites (UV) et (RT) sont parall�les.

Ecrire l'�galit� des quotients.
\end{minipage}
&
\begin{minipage}{6cm}
\begin{center}
\includegraphics[width=4cm]{images/thales1.png}
\end{center}
\end{minipage}
\end{tabular}

\bigskip

\bigskip


\ul{\textbf{Exercice 2:}} (\textit{4,5 points}) Calculer mentalement:

\enskip

\begin{tabular}{ccc}
\begin{minipage}{6cm}
\begin{enumerate}
\item [a.] $-2 \times 6= \ldots$



\item [b.] $(-8)+(-2)=\ldots$


\item [c.] $14 \div(-7)=\ldots$
\end{enumerate}
\end{minipage}
&
\begin{minipage}{6cm}
\begin{enumerate}
\item[d.] $(-7)-(+2)=\ldots$



\item [e.] $-5+12=\ldots$



\item [f.] $-5 \times (-3)=\ldots$
\end{enumerate}
\end{minipage}
&
\begin{minipage}{6cm}
\begin{enumerate}
\item[g.] $-8-5=\ldots$


\item[h.] $(-25) \div (-5)=\ldots$
\item[i.] $(-4)-(-3)=\ldots$
\end{enumerate}
\end{minipage}
\end{tabular} 

\bigskip

\bigskip

\ul{\textbf{Exercice 3:}} (\textit{3 points}) D�terminer le signe des produits
suivants (on ne demande pas de les calculer):
 

\begin{enumerate}
  \item[a.]  $(-2) \times 3,78 \times (-1) \times (+5,2) \times (-2)$ 
  \qquad \quad \quad \quad signe: \ldots \ldots \ldots
  \item[b.] $1 \times (-5) \times (-10,2) \times (-3) \times (-6) \times (+2)$
   \quad \qquad  signe: \ldots \ldots \ldots
   \item[c.] $2 \times (+3) \times (+4,5) $ \qquad \qquad  \qquad \qquad
  \qquad \quad \quad \quad signe: \ldots \ldots \ldots
  
\end{enumerate} 

\bigskip

\bigskip

\ul{\textbf{Exercice 4:}} (\textit{4 points}) Effectuer le calcul suivant en
soulignant � chaque �tape le calcul en cours:

\begin{center}
$A=-6 -4 \times (+5)$ \qquad \qquad \qquad \qquad \qquad \qquad \qquad
\qquad $B=7 \times 2 - 3 \times 4 -8$

\end{center}


\enskip


\begin{center}
$C= 5 \times (-7+10)-12$ \qquad \qquad \qquad \qquad  \qquad \qquad \qquad
\qquad  $D=\dfrac{2-7 \times (-6) +1}{-5-1+(-3) \times (-5)}$

\end{center}

\bigskip

\bigskip



 

\ul{\textbf{Exercice 5:}} (\textit{4 points})

\begin{enumerate}
  \item Traduire les phrases suivantes par un calcul:

  \begin{enumerate}
  \item La diff�rence du produit de -4 par -5 et de 6.
  
 

  
  \item Le quotient de la somme de 3 et -2 par le produit de 7 par -1.
 
  \end{enumerate}
  
  \item Traduire les expressions math�matiques suivantes par
des phrases: 

\begin{enumerate}
  \item $(-6)+ 5 \times (-2)$
  
 
  \item $(-7-2) \times (-4)$
  
 
\end{enumerate}
\end{enumerate}

\bigskip




\bigskip

\ul{\textbf{Exercice 6:}} (\textit{2 points})

\begin{enumerate}
  \item Calculer B=3u-24 pour u=-2.
  \item b est un nombre n�gatif (avec b $\neq$ 0). Quel est le signe du nombre
  $\frac {b^2}{-2b}$ ? Justifier.
\end{enumerate}
 
\bigskip

\bigskip


\ul{\textbf{Exercice 7:}} (\textit{2 points})

Recopier et compl�ter les pointill�s par les signes d'op�rations qui conviennent
pour que les �galit�s soient vraies.

\begin{enumerate}
  \item[a.] \quad $5 \ldots 5 \ldots 5 \ldots 5 = 9$
  \item[b.] \quad $5 \ldots 5 \ldots 5 \ldots 5=-10$
\end{enumerate}








\pagebreak








\begin{flushleft}
NOM PRENOM: \ldots \ldots \ldots \ldots \ldots \ldots \ldots \ldots \ldots
 \end{flushleft}

\medskip

\begin{center}
{\fbox{$4^{e}3$ \qquad \qquad \textbf{\Large{Devoir surveill� 3 (sujet 2)}}
\qquad \qquad 19/12/2013}}
\end{center}



\textit{Remarque: Les exercices 2, 3 et 7 se font sur la photocopie.}
CALCULATRICE INTERDITE

 
\bigskip


\ul{\textbf{Exercice 1:}} (\textit{1,5 points})

\begin{tabular}{cc}
\begin{minipage}{12cm}
Dans le triangle GHI, le point E appartient au segment [IG], le point F
appartient au segment [HI] et les droites (EF) et (GH) sont parall�les.

Ecrire l'�galit� des quotients.
\end{minipage}
&
\begin{minipage}{6cm}
\begin{center}
\includegraphics[width=4cm]{images/thales2.png}
\end{center}
\end{minipage}
\end{tabular}

\bigskip

\bigskip


\ul{\textbf{Exercice 2:}} (\textit{4,5 points}) Calculer mentalement:

\enskip

\begin{tabular}{ccc}
\begin{minipage}{6cm}
\begin{enumerate}
\item [a.] $-3 \times 4= \ldots$



\item [b.] $(-4)+(-7)=\ldots$


\item [c.] $24 \div(-2)=\ldots$
\end{enumerate}
\end{minipage}
&
\begin{minipage}{6cm}
\begin{enumerate}
\item[d.] $(-6)-(+3)=\ldots$



\item [e.] $-8+11=\ldots$



\item [f.] $-8 \times (-2)=\ldots$
\end{enumerate}
\end{minipage}
&
\begin{minipage}{6cm}
\begin{enumerate}
\item[g.] $-3-8=\ldots$


\item[h.] $(-15) \div (-5)=\ldots$
\item[i.] $(-5)-(-6)=\ldots$
\end{enumerate}
\end{minipage}
\end{tabular} 

\bigskip

\bigskip

\ul{\textbf{Exercice 3:}} (\textit{3 points}) D�terminer le signe des produits
suivants (on ne demande pas de les calculer):
 

\begin{enumerate}
  \item[a.] $2 \times (+3) \times (+4,5) $ \qquad \qquad  \qquad \qquad
  \qquad \quad \quad \quad signe: \ldots \ldots \ldots  
  \item[b.]  $(-2) \times 3,78 \times (-1) \times (+5,2) \times (-2)$ 
  \qquad \quad \quad \quad signe: \ldots \ldots \ldots  
   \item[c.] $1 \times (-5) \times (-10,2) \times (-3) \times (-6) \times (+2)$
   \quad \qquad  signe: \ldots \ldots \ldots   
  
\end{enumerate} 

\bigskip

\bigskip


\ul{\textbf{Exercice 4:}} (\textit{4 points})Effectuer le calcul suivant en
soulignant � chaque �tape le calcul en cours:

\begin{center}
$A=-2 -3 \times (+4)$ \qquad \qquad \qquad \qquad \qquad \qquad \qquad
\qquad $B=3 \times 6 - 2 \times 7 -6$

\end{center}

\enskip

\begin{center}
$C= 4 \times (-6+8)-12$ \qquad \qquad \qquad \qquad  \qquad \qquad \qquad
\qquad  $D=\dfrac{2-7 \times (-5) +8}{-5-1+(-3) \times (-5)}$

\end{center}

\bigskip

\bigskip

 

\ul{\textbf{Exercice 5:}} (\textit{4 points})

\begin{enumerate}
  \item Traduire les phrases suivantes par un calcul:
\begin{enumerate}
  \item La diff�rence du produit de -3 par -2 et de 1.
 
  
  \item Le quotient de la somme de 6 et -7 par le produit de 1 par -2.
\end{enumerate}  
  \item Traduire les expressions math�matiques suivantes par
des phrases:  
\begin{enumerate}
  \item $(-5)+ 3 \times (-4)$
 
  \item $(-8-3) \times (-2)$
\end{enumerate}\item 
\end{enumerate}




\bigskip



\bigskip 

\ul{\textbf{Exercice 6:}} (\textit{2 points})
 

\begin{enumerate}
  \item Calculer B=3u-24 pour u=-2.
  \item b est un nombre n�gatif (avec b $\neq$ 0). Quel est le signe du nombre
  $\frac {b^2}{-2b}$ ? Justifier.
\end{enumerate}


\bigskip 

\bigskip

\ul{\textbf{Exercice 7:}} (\textit{2 points})

Recopier et compl�ter les pointill�s par les signes d'op�rations qui conviennent
pour que les �galit�s soient vraies.

\begin{enumerate}
  \item[a.] \quad $5 \ldots 5 \ldots 5 \ldots 5 = 24$
  \item[b.] \quad $5 \ldots 5 \ldots 5 \ldots 5=0$
\end{enumerate}

\end{document}

