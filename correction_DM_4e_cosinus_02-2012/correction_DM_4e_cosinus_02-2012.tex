\documentclass[12pt, twoside]{article}
\usepackage[francais]{babel}
\usepackage[T1]{fontenc}
\usepackage[latin1]{inputenc}
\usepackage[left=5mm, right=5mm, top=5mm, bottom=5mm]{geometry}
\usepackage{float}
\usepackage{graphicx}
\usepackage{array}
\usepackage{multirow}
\usepackage{amsmath,amssymb,mathrsfs}
\usepackage{soul}
\usepackage{textcomp}
\usepackage{eurosym}
 \usepackage{variations}
\usepackage{tabvar}

\pagestyle{empty}
\begin{document}


\begin{center}
\fbox{Correction du devoir maison 6}
\end{center}



\ul{Exercice 1}: Le triangle KJL est rectangle en K. On a donc:
 $cos\widehat{KJL}=\dfrac{KJ}{JL}$.
 On remplace par les valeurs num�riques: $cos(50)=\dfrac{KJ}{7}$ soit
 $KJ=7 \times cos(50) \approx 4,5 cm$.
 
 
 \bigskip
 
 
 \ul{Exercice 2}: Le triangle GHI est rectangle en G. On a donc:
 $cos\widehat{GHI}=\dfrac{GH}{HI}$.
 On remplace par les valeurs num�riques: $cos(30)=\dfrac{6}{HI}$ soit
 $HI=6 \div cos(30) \approx 7 cm$.


\bigskip

\ul{Exercice 3}:

\begin{enumerate}
  \item Dans le triangle TIR, le plus long c�t� est [TR] donc on calcule
  s�paremment $TR^2$ et $IT^2+IR^2$:
  
  
  
  
  $TR^2=30,6^2=936,36$ \qquad et \qquad
  $IT^2+IR^2=27^2+14,4^2=729+207,36=936,36$

  
  On constate que $TR^2=IT^2+IR^2$ donc d'apr�s la r�ciproque du th�or�me de
  Pythagore, le triangle TIR est rectangle en I.
  
 
  
  \item Le triangle TIR est rectangle en I. On a donc:
 $cos\widehat{IRT}=\dfrac{IR}{TR}$.
 On remplace par les valeurs num�riques: $cos(IRT)=\dfrac{14,4}{30,6}$. En
 utilisant la calculatrice ( \fbox{2nd}\quad \fbox{cos} ), on trouve:
 $\widehat{IRT}\approx61,9$�.
 \item $\widehat{RTI}= 180-90-61,9=28,1$�.
 

\end{enumerate}


\bigskip

\ul{Exercice 4}:

\begin{enumerate}
  \item [2.] Le triangle BMV est rectangle en B. On a donc:
 $cos\widehat{BVM}=\dfrac{BV}{MV}$.
 On remplace par les valeurs num�riques: $cos(25)=\dfrac{825}{MV}$ soit
 $MV=825 \div cos(25)\approx 910 m$.
 
 \item [3.] 
 
 \enskip
 
 \begin{tabular}{c|c}
 \begin{minipage}{11cm}
 \textbf{M�thode 1:}  Le triangle BMV est rectangle en B. D'apr�s le th�or�me
 de Pythagore, on a: $MV^2=BM^2+BV^2$.
 
 \quad
 
 On remplace par les valeurs num�riques:
 
 $910^2=BM^2+825^2$
 
 $828 100=BM^2+680 625$
 
 $BM^2=828 100 - 680 625$
 
 $BM^2=147 475$
 
 
 \quad
 
 
 BM est une valeur positive donc $BM= \sqrt{147 475} \approx 384 m$
 
 
 \end{minipage}
&
 \begin{minipage}{8cm}
 \textbf{M�thode 2:}  $\widehat{BMV}=180-90-25=65$�
 
 \quad
 
 
 Le triangle BMV est rectangle en B. On a donc:
 $cos\widehat{BMV}=\dfrac{BM}{MV}$.
 On remplace par les valeurs num�riques: $cos(65)=\dfrac{BM}{910}$ soit
 $BM=910 \times cos(65)\approx 385 m$.
 \end{minipage}
 \end{tabular}
 
 
\end{enumerate}


\bigskip

\ul{Exercice 5}:

\begin{enumerate}
  \item On a: (BC)$\perp$ (AD) et (DE) $\perp$ (AD). D'apr�s la propri�t� ``si
  deux droites sont perpendiculaires � une m�me droite alors elles sont
  parall�les entre elles'', on en d�duit que les droites (BC) et (DE) sont
  parall�les.
  
  \quad
  
  \item Dans le triangle ADE, on sait que C appartient au segment [AE], B
  appartient au segment [AD] et les droites (BC) et (DE) sont parall�les.
  D'apr�s le th�or�me de Thal�s, on a:
  
  $\dfrac{AC}{AE}=\dfrac{AB}{AD}=\dfrac{BC}{DE}$ \quad  soit
  $\dfrac{3}{8}=\dfrac{AB}{AD}=\dfrac{BC}{5}$ \quad donc $BC=3 \times 5 \div
  8=1,875cm$.
  
  \quad
  
  
  \item  Le triangle ABC est rectangle en B. On a donc:
 $cos\widehat{ACB}=\dfrac{BC}{CA}$.
 On remplace par les valeurs num�riques: $cos\widehat{ACB}=\dfrac{1,875}{3}$
 \quad En utilisant la calculatrice ( \fbox{2nd}\quad \fbox{cos} ), on trouve:
 $\widehat{ACB}\approx 51$�.
 
 \quad
 
 \item Les droites (BC) et (DE) sont parall�les donc les angles correspondants
 $\widehat{ACB}$ et $ \widehat{AED}$ ont m�me mesure. Donc $\widehat{AED}=51$�.
\end{enumerate}
\end{document}
