\documentclass[12pt, twoside]{article}
\usepackage[francais]{babel}
\usepackage[T1]{fontenc}
\usepackage[latin1]{inputenc}
\usepackage[left=7mm, right=7mm, top=7mm, bottom=7mm]{geometry}
\usepackage{float}
\usepackage{graphicx}
\usepackage{array}
\usepackage{multirow}
\usepackage{amsmath,amssymb,mathrsfs}
\usepackage{soul}
\usepackage{textcomp}
\usepackage{eurosym}
 \usepackage{variations}
\usepackage{tabvar}


\pagestyle{empty}
 
\begin{document}


\section*{\center{Correction devoir surveill� 5 (sujet 1)}}

\ul{Exercice 1}:
\quad   $\dfrac{4}{7}=\dfrac{8}{14}$ \qquad \qquad \qquad
$\dfrac{-2}{5}=\dfrac{6}{-15}$ \qquad \qquad
\qquad $\dfrac{-24}{16}=\dfrac{-3}{2}$ \qquad \qquad \qquad
$\dfrac{25}{4,5}=\dfrac{-250}{-45}$

\bigskip

\ul{Exercice 2}:\quad  
$\dfrac{42}{-140}=- \dfrac{14 \times 3}{14 \times
10}=-\dfrac{3}{10}$ \qquad  \qquad \qquad$\dfrac{-35}{-105}=\dfrac{5 \times 7}{5
\times 3 \times 7}=\dfrac{1}{3}$

\bigskip


\ul{Exercice 3}: \quad $\dfrac{2,4}{-3}=\dfrac{4}{-5}$ \quad car $2,4 \times
(-5) \div (-3)=4$ (�galit� des produits en croix)

\medskip


\quad $\dfrac{4}{-5,6}=\dfrac{-5}{7}$ \quad car $4 \times 7 \div
(-5)=-5,6$ (�galit� des produits en croix)

\bigskip


\ul{Exercice 4}: \quad
$B=\dfrac{8}{-5}+\dfrac{9}{5}=\dfrac{-8}{5}+\dfrac{9}{5}=\dfrac{1}{5}$ \qquad
\qquad 
$A=\dfrac{7}{6}+\dfrac{-1}{3}=\dfrac{7}{6}+\dfrac{-1
\times 2}{3 \times 2}=\dfrac{7}{6}+\dfrac{-2}{6}=\dfrac{5}{6}$


\bigskip

\medskip

$C=\dfrac{3}{5}-\dfrac{2}{7}=\dfrac{3
\times 7}{5 \times 7}-\dfrac{2 \times 5}{7
\times 5}=\dfrac{21}{35}-\dfrac{10}{35}=\dfrac{11}{35}$
\qquad $E=\dfrac{-2}{-11} \times \dfrac{-5}{-6} \times
\dfrac{-3}{20}=\dfrac{-2 \times 5 \times 3}{11 \times 2 \times 3 \times 5
\times4}=\dfrac{-1}{4 \times 11}=\dfrac{-1}{44}$

\bigskip

\medskip

$D=\dfrac{6}{-5} \times \dfrac{-7}{-6}=-\dfrac{7}{5}$ \qquad $F=\dfrac{7}{11}
\div \dfrac{-14}{3}=\dfrac{7}{11} \times \dfrac{3}{-14}=\dfrac{-3 \times 
7}{11 \times 2 \times 7}=\dfrac{-3}{22}$ 

\bigskip

\medskip

$G=\dfrac{-18}{13} \div \dfrac{27}{-26}=\dfrac{18}{13} \times
\dfrac{26}{27}=\dfrac{3 \times 2 \times 3 \times 2 \times 13}{13 \times 3
\times 3 \times 3}=\dfrac{4}{3}$


\bigskip

\ul{Exercice 5}: \quad $H=\dfrac{1}{3}+\dfrac{4}{3} \times
\dfrac{-6}{5}=\dfrac{1}{3}+\dfrac{-24}{15}=\dfrac{5}{15}+\dfrac{-24}{15}=\dfrac{-19}{15}$

\bigskip

\medskip

$I=\big( \dfrac{1}{5}-\dfrac{3}{10} \big) \times \big( \dfrac{1}{6}+\dfrac{5}{2}
\big )=\big( \dfrac{2}{10}-\dfrac{3}{10} \big) \times \big( \dfrac{1}{6}+
\dfrac{15}{6} \big)=\dfrac{-1}{10} \times
\dfrac{16}{6}=-\dfrac{16}{60}=-\dfrac{4}{15}$


\bigskip

\ul{Exercice 6}: 

\begin{enumerate}
  \item Un tiers de 15000: $\dfrac{1}{3} \times 15000=5000$. Un cinqui�me de
  15000: $\dfrac{1}{5} \times 15000=3000$. A la livraison, il reste donc �
  payer 15000-5000-3000=7000. Cela repr�sente
  $\dfrac{7000}{15000}=\dfrac{7}{15}$ du prix total.
  
  \item Les 7000 \euro vont �tre payer en 10 mensualit�s
  c'est-�-dire 700 \euro par mois pendant 10 mois. Cela repr�sente
  $\dfrac{700}{15000}=\dfrac{7}{150}$ du prix total.
  
  
\end{enumerate}
 \pagebreak
 
 
 \section*{\center{Correction devoir surveill� 5 (sujet 2)}}

\ul{Exercice 1}: \quad
 $\dfrac{3}{8}=\dfrac{6}{16}$ \qquad \qquad \qquad
$\dfrac{-3}{5}=\dfrac{24}{-40}$ \qquad \qquad
\qquad $\dfrac{-32}{40}=\dfrac{-8}{10}$ \qquad \qquad \qquad
$\dfrac{22}{3,5}=\dfrac{-220}{-35}$

\bigskip


\ul{Exercice 2}: \quad 
 $\dfrac{-105}{-165}=\dfrac{5 \times
7 \times 3}{5 \times 3 \times 11}=\dfrac{7}{11}$ \qquad \qquad \qquad
$\dfrac{-52}{68}=- \dfrac{4 \times 13}{4 \times 17}=- \dfrac{13}{17}$


\bigskip


\ul{Exercice 3}: \quad $\dfrac{-0,6}{-2,4}=\dfrac{0,8}{3,2}$ \quad car $0,8
\times (-2,4) \div 3,2=-0,6$ (�galit� des produits en croix)

\medskip


\quad $\dfrac{-819}{195}=\dfrac{63}{-15}$ \quad car $195 \times 63 \div
(-819)=-15$ (�galit� des produits en croix)

\bigskip


\ul{Exercice 4}: \quad 
$B=\dfrac{9}{-4}+\dfrac{10}{4}=\dfrac{-9}{4}+\dfrac{10}{4}=\dfrac{1}{4}$ \qquad
\qquad  $A=\dfrac{-8}{5}+\dfrac{23}{50}=\dfrac{-8 \times 10}{5 \times
10}+\dfrac{23}{50}=\dfrac{-80}{50}+\dfrac{23}{50}=\dfrac{-57}{50}$

\bigskip

\medskip

$C=\dfrac{-3}{4}+\dfrac{-1}{5}=\dfrac{-3
\times 5}{4 \times 5}+\dfrac{-1 \times 4}{5
\times 4}=\dfrac{-15}{20}+\dfrac{-4}{20}=\dfrac{-19}{20}$ \qquad $F=\dfrac{6}{7}
\div \dfrac{-18}{5}=\dfrac{6}{7} \times \dfrac{5}{-18}=\dfrac{-2 \times 3
\times 5}{7 \times 2 \times 3 \times 3}=\dfrac{-5}{21}$


\bigskip

\medskip

$D=\dfrac{8}{-7} \times \dfrac{-6}{-8}=-\dfrac{6}{7}$ \qquad \qquad 
$E=\dfrac{-3}{-14} \times \dfrac{7}{-10} \times \dfrac{-2}{-9}=\dfrac{-3
\times 7 \times 2}{2 \times 7 \times 2 \times 5 \times 3 \times
3}=\dfrac{-1}{3 \times 2 \times 5}=\dfrac{-1}{30}$ 

\bigskip

\medskip

$G=\dfrac{-17}{15}
 \div \dfrac{-12}{20}=\dfrac{17}{15} \times \dfrac{20}{12}=\dfrac{17 \times
 5 \times 4}{5 \times 3 \times 3 \times 4}=\dfrac{17}{9}$
 
 
 \bigskip
 
 
 \ul{Exercice 5}: \quad $H=\dfrac{1}{3}-\dfrac{1}{3} \times
\dfrac{5}{7}=\dfrac{1}{3}-\dfrac{5}{21}=\dfrac{7}{21}-\dfrac{5}{21}=\dfrac{2}{21}$

\bigskip

\bigskip

\medskip

$I=\dfrac{2 \times \dfrac{3}{7}}{\dfrac{5}{3}-1}=\dfrac{6}{7} \div  \big(
\dfrac{5}{3}-\dfrac{3}{3} \big)=\dfrac{6}{7} \div \dfrac{2}{3}=\dfrac{6}{7}
\times \dfrac{3}{2}=\dfrac{9}{7}$


\bigskip

\bigskip

\ul{Exercice 6}: 

\begin{enumerate}
  \item Un tiers de 15000: $\dfrac{1}{3} \times 15000=5000$. Un cinqui�me de
  15000: $\dfrac{1}{5} \times 15000=3000$. A la livraison, il reste donc �
  payer 15000-5000-3000=7000. Cela repr�sente
  $\dfrac{7000}{15000}=\dfrac{7}{15}$ du prix total.
  
  \item Les 7000 \euro vont �tre payer en 10 mensualit�s
  c'est-�-dire 700 \euro par mois pendant 10 mois. Cela repr�sente
  $\dfrac{700}{15000}=\dfrac{7}{150}$ du prix total.
  
  
\end{enumerate}

\pagebreak

\end{document}
