\documentclass[12pt, twoside]{article}
\usepackage[francais]{babel}
\usepackage[T1]{fontenc}
\usepackage[latin1]{inputenc}
\usepackage[left=5mm, right=5mm, top=5mm, bottom=5mm]{geometry}
\usepackage{float}
\usepackage{graphicx}
\usepackage{array}
\usepackage{multirow}
\usepackage{amsmath,amssymb,mathrsfs}
\usepackage{soul}
\usepackage{textcomp}
\usepackage{eurosym}
 \usepackage{variations}
\usepackage{tabvar}

\pagestyle{empty}
\begin{document}

\begin{center}
\fbox{Correction du devoir maison 6}
\end{center}


\ul{Exercice 1:}

\begin{enumerate}
  \item Algorithme d'Euclide: 
  
  
  \bigskip
  
  \bigskip
  
  \bigskip
  
 donc PGCD(1394;255)=17
 
 \item Pour faire des colliers identiques avec 1394 graines d'a�a� et 255
 graines de palmier p�che, il faut trouver un diviseur commmun � 1394 et 255.
 Pour r�aliser le plus grand nombre de colliers, il faut trouver le PGCD.
 
 \enskip
 
 On peut donc faire 17 colliers identiques compos�s de 82 graines d'a�a� ($1394
 \div 17=82$) et de 15 graines de palmier p�che ($255 \div 17=15$).
\end{enumerate}


\bigskip

\ul{Exercice 2:}

\begin{enumerate}
  \item  $\bullet$ 1035 est divisible par 5 mais pas par 2(car son dernier
  chiffre est 5). 1035 est divisible par 3 et par 9 (car 1+0+3+5=9 est divisible par 3 et par
  9).
  
  \enskip

  
 $\bullet$ 774 est divisible par 2 mais pas par 5 (car son dernier chiffre est
 4). 774 est divisible par 3 et par 9 (car 7+7+4=18 est divisible par 3 et par
  9).  
  

  \enskip
  
  $\bullet$ 774 est divisible par 2 mais pas par 5 (car son dernier chiffre est
  2).774 n' est pas divisible par 3 et  ni par 9 (car 3+2+2=7 n' est pas divisible par 3
 ni par 9).    

\enskip

\item D'apr�s le tableau, les fractions $\dfrac{774}{1035}$ et
$\dfrac{322}{774}$ ne sont pas irr�ductibles car on peut simplifier la premi�re
par 3 (ou 9) et la deuxi�me par 2.

\enskip

\item Le tableau ne nous permet pas de conclure sur l'irr�ductibilit� de la
fraction $\dfrac{322}{1035}$ (on sait seulement que l'on ne peut pas simplifier
par 2, par 3, par 5 ou par 9).

\enskip

\item Algorithme d'Euclide:

\bigskip

\bigskip

\bigskip

donc PGCD(322;1035)=23

\item La fraction $\dfrac{322}{1035}$ peut �tre simplifi�e par 23, elle n'est
donc pas irr�ductible.
\end{enumerate}

\bigskip

\ul{Exercice 3:}

\begin{enumerate}
  
  
  \bigskip
  
  \bigskip
  
  
  \bigskip
  
  
  \item [2.] Dans le triangle AMD, le plus long c�t� est [AD] donc on calcule
  s�parement $AD^2$ et $MD^2+MA^2$.
  
  $AD^2=4^2=16$ \quad et \quad $MD^2+MA^2=3,2^+2,4^2=10,24+5,76=16$
  
  On constate que $AD^2=MD^2+MA^2$, donc d'apr�s la r�ciproque du th�or�me de
  Pythagore, le traingle AMD est rectangle en D.
  
  \item [3.] Le triangle AMD est rectangle en M. On a:
  $cos(\widehat{DAM})=\dfrac{AM}{AD}$.
  
  $cos(\widehat{DAM})=\dfrac{2,4}{4}$ \quad donc \quad $arccos(\widehat{DAM})
  \approx 53$�.
  
  \enskip
  
  
  On pouvait �galement uiliser pour cette question: 
  $sin(\widehat{DAM})=\dfrac{DM}{AD}=\dfrac{3,2}{4}$ ou 
  $tan(\widehat{DAM})=\dfrac{DM}{AM}=\dfrac{3,2}{2,4}$.
  
  \item [4.] Le triangle DAI est rectangle en D. On a
  $tan(\widehat{DAI})=\dfrac{DI}{DA}$.
  
  $tan(53)=\dfrac{DI}{4}$ \quad donc \quad $DI=tan(53) \times 4 \approx 5,3$ cm.
  
\end{enumerate}

\bigskip

\ul{Exercice 4:} 

\enskip

 $\bullet$ Le triangle BDC est rectangle en B: \quad 
 $sin(\widehat{BDC})=\dfrac{BC}{CD}$ \quad et \quad $tan(\widehat{BDC})=\dfrac{BC}{BD}$.


On remplace par les valeurs num�riques:
\quad $sin(8)=\dfrac{20}{CD}$ \quad donc \quad $CD=20 \div sin(8) \approx 144$
dm


 $tan(8)=\dfrac{20}{BD}$ \quad donc \quad $BD=20 \div tan(8) \approx 142$ m. 


\bigskip


$\bullet$ Le triangle ABD est rectangle en B: \quad
$cos(\widehat{BDA})=\dfrac{BD}{AD}$ \quad et \quad $tan(\widehat{BDA})=\dfrac{AB}{BD}$.


On remplace par les valeurs num�riques: 
$cos(52)=\dfrac{142}{AD}$ \quad donc \quad $AD=142 \div cos(52) \approx 231$
dm


$tan(52)=\dfrac{AB}{142}$ \quad donc \quad $AB=142 \times tan(52) \approx 182$
m.


\bigskip

$\bullet$ $\mathcal{P}_{ADC}=AB+BC+CD+DA \approx 182+20+144+231 \approx 577$ dm.
 
\bigskip

\ul{Exercice 5:} 

\enskip


Le triangle ABC est rectangle en C. On a : \quad 
$(cos(\widehat{ABC}))^2+(sin(\widehat{ABC}))^2=1$ \quad et \quad
$tan(\widehat{ABC})=\dfrac{sin(\widehat{ABC})}{cos(\widehat{ABC})}$.

\bigskip

On remplace par les valeurs num�riques:

\enskip


$0,8^2+sin(\widehat{ABC})^2=1$ 

\enskip


$0,64+sin(\widehat{ABC})^2=1$

\enskip

$sin(\widehat{ABC})^2=1-0,64$

\enskip

$sin(\widehat{ABC})^2=0,36$  \qquad \qquad $sin(\widehat{ABC})$ est une valeur
positive donc $sin(\widehat{ABC})=\sqrt{0,36}=0,6$.

\bigskip

On en d�duit:
$tan(\widehat{ABC})=\dfrac{0,6}{0,8}=0,75$


\end{document}
