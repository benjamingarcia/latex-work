\documentclass[12pt, twoside]{article}
\usepackage[francais]{babel}
\usepackage[T1]{fontenc}
\usepackage[latin1]{inputenc}
\usepackage[left=7mm, right=7mm, top=7mm, bottom=7mm]{geometry}
\usepackage{float}
\usepackage{graphicx}
\usepackage{array}
\usepackage{multirow}
\usepackage{amsmath,amssymb,mathrsfs}
\usepackage{soul}
\usepackage{textcomp}
\usepackage{eurosym}
 \usepackage{variations}
\usepackage{tabvar}

\pagestyle{empty}
\begin{document}

\begin{center}
\fbox{Correction du devoir maison 3}
\end{center}

\enskip


\ul{Exercice 1}: 

\begin{enumerate}
  \item L'expression 10x repr�sente la recette de la vente des places en virage.
  
  
  L'expression 15000-x repr�sente le nombre de places en tribune. 
  
  L'expression
  (15000-x)$\times$ 13 repr�sente la recette de la vente des places en tribune.
  
  \enskip
  
  \item Le montant total de la recette est: 10x+(15000-x)$\times$ 13.
  
  \enskip
  
  \item Si x=6500, on a alors: $10 \times 6500+(15000-6500) \times 13=175500$.
  S'il y a 6500 places en virage, le montant total de la recette est 175 500
  euros.
\end{enumerate}


\bigskip

\medskip


\ul{Exercice 2}: 

\begin{enumerate}
  \item Le p�rim�tre de ABCD est: $2\times (x+3) + 2 \times (x+2)$.
  
 \enskip
  
  R�duisons cette expression: 
 $2\times (x+3) + 2 \times (x+2)=2x+ 2 \times3 +2x+ 2 \times 2=2x+6+2x+4=4x+10$
 

 \enskip
 
 
  Calculons la valeur de cette expression pour x=6: 
  
  $4 \times 6+10=34$. Donc le
  p�rim�tre de ABCD est 34cm.
  
  
  \enskip
  
  \item L'aire de ABCD est: (x+3)(x+2). Cette expression est factoris�e,
  d�veloppons et r�duisons cette expression:
  
  $(x+3)(x+2)=x \times x+ x \times 2+ 3 \times x+ 3 \times
  2=x^2+2x+3x+6=x^2+5x+6$.
  
 \enskip
 
  
  Calculons cette aire pour x=6cm: on remplace x par 6 dans la formule de
  l'aire (celle factoris�e ou celle r�duite au choix). 
  
  $6^2+5 \times 6+6=36+30+6=72$.
 \end{enumerate} 

\bigskip


\medskip


\ul{Exercice 3}:

\enskip

\begin{tabular}{c|c|c}
A=(2x+7)+(3x-6)-(-5-x) & \qquad  B=-(5+4y)-(4-4y)+(-3+y)
\qquad & C=[(x+2)-(3x+5)]\\

A= 2x+7+3x-6+5+x \quad \qquad & B=-5-4y-4+4y-3+y \quad \quad & \quad \quad
\thinspace  $C=3 \times [x+2-3x-5] $\\

A= 2x+3x+x+7-6+5 \quad \qquad & B=-4y+4y+y-5-4-3 \quad \quad & $C= 3 \times
(-2x-3)$\\

A=6x+6 \qquad \qquad \qquad \quad \quad & B=y-12 \qquad \qquad \qquad \qquad &
\qquad \quad $C=3 \times (-2x)+3 \times (-3)$ \\


\quad & \qquad & C=-6x-9 \qquad \qquad  \qquad \\
\end{tabular}


\bigskip

\medskip


\ul{Exercice 4}:
 
\enskip

\begin{tabular}{c|c}
D=2x(5x-7)-12(x-1) \qquad \qquad \qquad \qquad \qquad \qquad  \qquad&
E=(2x-4)(-3+5x) \qquad \qquad \qquad \qquad \qquad \qquad 
\\

$D=2x \times 5x-2x \times 7-12 \times x-12 \times (-1)$ \qquad \qquad & \qquad
$E=2x \times (-3)+ 2x \times 5x-4 \times (-3)-4 \times 5x$ \\

$D=10x^2-14x-12x+12$ \quad \qquad \qquad \qquad \qquad \quad &
$E=-6x+10x^2+12-20x$ \qquad \qquad \qquad \qquad \\

$D=10x^2-26x+12$ \qquad \qquad \qquad \qquad \qquad \qquad \quad &
$E=10x^2-26x+12$ \qquad \qquad \qquad \qquad \quad  \quad \\
\end{tabular}


\enskip

On en d�duit que D=E.


\bigskip

\medskip



\ul{Exercice 5}:

\begin{enumerate}
  \item Un croissant co�te x \euro. Un pain au chocolat co�te 0,20 \euro de
  plus qu'un croissant donc un pain au chocolat co�te x+0,20 \euro.
  \item La d�pense totale est: 7(x+0,20)+4x.
  \item $7(x+0,20)+4x=7 \times x + 7 \times 0,20 +4x=7x+1,40+4x=11x+1,40$.
  \item On remplace x par 0,80 dans l'expression pr�c�dente: $11 \times 0,80
  +1,40=10,2$. Pascal doit payer 10,2 \euro.
\end{enumerate}
\end{document}
