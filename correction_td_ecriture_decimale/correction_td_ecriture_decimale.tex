\documentclass[12pt, twoside]{article}
\usepackage[francais]{babel}
\usepackage[T1]{fontenc}
\usepackage[latin1]{inputenc}
\usepackage[left=9mm, right=9mm, top=8mm, bottom=8mm]{geometry}
\usepackage{float}
\usepackage{graphicx}
\usepackage{array}
\usepackage{multirow}
\usepackage{amsmath,amssymb,mathrsfs}
\usepackage{soul}
\usepackage{textcomp}
\pagestyle{empty}
\begin{document}


\section*{\center{Correction du td �criture d�cimale}}

\subsection*{Exercice 2}
\begin{tabular}{cc}
\begin{minipage}{10cm}
$251,3= 2,513 \times 10^{2}$
\medskip


$0,0019= 1,9 \times 10^{-3}$
\medskip


$150=1,5 \times 10^{2}$
\medskip


$150 \times 10^{3}=1,5 \times 10^{2} \times 10^{3}=1,5 \times 10^{5}$
\medskip


$150 \times 10^{-2}= 1,5 \times 10^{2} \times 10^{-2}=1,5$
\end{minipage}
&
\begin{minipage}{5cm}
$1024=1,024 \times 10^{3}$
\medskip


$0,125=1,25 \times 10^{-1}$
\medskip


$1,31 \times 1000=1,31 \times 10^{3}$
\medskip


$18000,35=1,800035 \times 10^{4}$
\end{minipage}
\end{tabular}


\subsection*{Exercice 3}

\begin{enumerate}
  \item $1$ an=$365$ jours \\
   \quad $1$ jour= $24$h\\
   \quad $1$ h=$60$min \\
   \quad $1$min=$60$s\\
   donc $1$ an$=365 \times 24 \times 60 \times 60$s$=31 \thinspace 536
   \thinspace 000$s. La lumi�re parcourt $300 \thinspace 000$km par seconde.
   Elle parcourt donc en un an: $300 \thinspace 000 \times 31 \thinspace 536
   \thinspace 000=9 \thinspace 460 \thinspace 800 \thinspace 000 \thinspace
   000$km $\simeq 9 \times 10^{12}$ km.
  \item La distance terre/soleil est $150 \times 10^{9}$ km. La lumi�re parcourt
  $300 \thinspace 000$km en une seconde donc pour faire $150 \times 10^{9}$ km,
  il lui faut $\dfrac{150 \times 10^{9}}{300 \thinspace 000}=50 \times 10^{4}$
  s. Dans un jour, il y a $24 \times 60 \times 60=86 \thinspace 400$ s donc $50
  \times 10^{4}$ s$=\dfrac{50 \times 10^{4}}{86 \thinspace 400}$ jours $\simeq
  5,78$ jours $\simeq 6$ jours. Finalement, la lumi�re met $6$ jours pour venir
  du soleil.
  \item La lumi�re provenant de l'�toile met $4,5$ ann�es pour nous parvenir.\\
  $1$ an$= 31 \thinspace 536 \thinspace 000$s (d'apr�s la premi�re question)
  d'o� $4,5$ ann�es $= 4,5 \times 31 \thinspace 536 \thinspace 000= 141
  \thinspace 912 \thinspace 000$s. Or la lumi�re parcourt $300 \thinspace
  000$km en une seconde donc en $141 \thinspace 912 \thinspace 000$s elle
  parcourt:\\
   $300 \thinspace 000 \times 141 \thinspace 912 \thinspace 000=42 \thinspace
   573 \thinspace 600 \thinspace 000$ km.


\medskip

\textbf{Autre r�ponse possible:} L'ann�e lumi�re �tant la distance parcourue par
la lumi�re pendant une ann�e, en $4,5$ ans la lumi�re parcourt donc: $4,5 \
a.l$.
  
  \item $1\ a.l$ est la distance parcourue par la lumi�re en un an. D'apr�s la
  premi�re question: \\
 $1 \ a.l=9,4608 \times 10^{12}$ km donc $350 \ a.l=350 \times 9,4608 \times
 10^{12}$ km=$3,311 \thinspace 28 \times 10^{15}$ km.
\end{enumerate}


\end{document}
