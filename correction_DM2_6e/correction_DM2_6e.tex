\documentclass[12pt, twoside]{article}
\usepackage[francais]{babel}
\usepackage[T1]{fontenc}
\usepackage[latin1]{inputenc}
\usepackage[left=7mm, right=7mm, top=7mm, bottom=7mm]{geometry}
\usepackage{float}
\usepackage{graphicx}
\usepackage{array}
\usepackage{multirow}
\usepackage{amsmath,amssymb,mathrsfs}
\usepackage{soul}
\usepackage{textcomp}
\usepackage{eurosym}
 \usepackage{variations}
\usepackage{tabvar}

\pagestyle{empty}
\begin{document}


\section*{\center{Correction devoir maison 2}}



\subsection*{Exercice 4}

\begin{enumerate}
  \item Tracer un triangle IJK.
  \item Tracer la droite perpendiculaire � (JK) passant par I. Elle coupe (JK)
  en C.
  \item Tracer la droite parall�le � (IK) passant par J. Elle coupe (CI) en D.
  \item Coder la figure.
\end{enumerate}


\subsection*{Exercice 5}

\begin{enumerate}
  \item On sait que: $(d_1) \perp (d_3)$; \qquad $(d_1) \perp (d_4)$; \qquad
  $(d_2) \perp (d_4)$.
  \item On sait que: $(d_3) \perp (d_1)$ et $(d_4) \perp (d_1)$.
  
  Or si deux droites sont perpendiculaires � la m�me droite alors elles sont
  parall�les entre elles.
  
  Donc: $(d_3) // (d_4)$


  \item  On sait que: $(d_3) // (d_4)$ et $(d_4) \perp (d_5)$.
  
  Or si deux droites sont parall�les alors toutes perpendiculaire � l'une est
  perpendiculaire � l'autre.
  
  Donc: $(d_3) \perp (d_5)$  
  
  \item ABCD est un quadrilat�re ayant quatre angles droits donc c'est un
  rectangle.
\end{enumerate}


\bigskip

\bigskip

\section*{\center{Correction devoir maison 2}}



\subsection*{Exercice 4}

\begin{enumerate}
  \item Tracer un triangle IJK.
  \item Tracer la droite perpendiculaire � (JK) passant par I. Elle coupe (JK)
  en C.
  \item Tracer la droite parall�le � (IK) passant par J. Elle coupe (CI) en D.
  \item Coder la figure.
\end{enumerate}


\subsection*{Exercice 5}

\begin{enumerate}
  \item On sait que: $(d_1) \perp (d_3)$; \qquad $(d_1) \perp (d_4)$; \qquad
  $(d_2) \perp (d_4)$.
  \item On sait que: $(d_3) \perp (d_1)$ et $(d_4) \perp (d_1)$.
  
  Or si deux droites sont perpendiculaires � la m�me droite alors elles sont
  parall�les entre elles.
  
  Donc: $(d_3) // (d_4)$


  \item  On sait que: $(d_3) // (d_4)$ et $(d_4) \perp (d_5)$.
  
  Or si deux droites sont parall�les alors toutes perpendiculaire � l'une est
  perpendiculaire � l'autre.
  
  Donc: $(d_3) \perp (d_5)$  
  
  \item ABCD est un quadrilat�re ayant quatre angles droits donc c'est un
  rectangle.
\end{enumerate}
\end{document}
