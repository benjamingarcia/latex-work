\documentclass[12pt, twoside]{article}
\usepackage[francais]{babel}
\usepackage[T1]{fontenc}
\usepackage[latin1]{inputenc}
\usepackage[left=5mm, right=5mm, top=5mm, bottom=5mm]{geometry}
\usepackage{float}
\usepackage{graphicx}
\usepackage{array}
\usepackage{multirow}
\usepackage{amsmath,amssymb,mathrsfs}
\usepackage{soul}
\usepackage{textcomp}
\usepackage{eurosym}
 \usepackage{variations}
\usepackage{tabvar}


\pagestyle{empty}

\begin{document}


\begin{center}
\textbf{\large{Correction devoir surveill� 7}}
\end{center}

\medskip



\ul{\textbf{Exercice 2:}} Dans le triangle ABC, on sait que E appartient � [AB], F
appartient � [AC] et les droites (BC) et (EF) sont parall�les. D'apr�s la propri�t� de
proportionnalit� des longueurs, on a:

\begin{center}
$\dfrac{AE}{AB}=\dfrac{AF}{AC}=\dfrac{EF}{BC}$
\end{center}


AB=AE+EB=2+1,5=3,5 cm. En rempla�ant par les valeurs num�riques:
$\dfrac{2}{3,5}=\dfrac{AF}{7}=\dfrac{EF}{5}$

Calcul de EF: $EF=\dfrac{2 \times 5}{3,5} \approx 2,86$cm \qquad  \qquad Calcul
de AF: $EF=\dfrac{2 \times 7}{3,5}=4$cm
donc FC=AC-AF=7-4=3


\bigskip

\bigskip

\ul{\textbf{Exercice 3:}} Dans le triangle ABE, on sait que C appartient � [EA], D
appartient � [EB] et les droites (CD) et (AB) sont parall�les. D'apr�s la propri�t� de
proportionnalit� des longueurs, on a:

\begin{center}
$\dfrac{EC}{EA}=\dfrac{ED}{EB}=\dfrac{CD}{AB}$
\end{center}


EC=EA-AC=2,4-2=0,4m. En rempla�ant par les valeurs num�riques:
$\dfrac{0,4}{2,4}=\dfrac{ED}{EB}=\dfrac{0,25}{AB}$

Calcul de AB: $AB=\dfrac{2,4 \times 0,25}{0,4}=1,5$ m.

\bigskip


\bigskip

\ul{\textbf{Exercice 5:}}  $\dfrac{60}{18}=\dfrac{30}{9}=\dfrac{3 \times 2 \times 5}{3 \times
3}=2^1 \times 5^1 \times 3^{-1}$

\medskip


$\dfrac{2 \times 2 \times 3 \times 5}{2 \times 2 \times 2 \times 5 \times 5}=
\dfrac{3}{2 \times 5}=3^1 \times 2^{-1} \times 5^{-1}$ \qquad \qquad $\dfrac{3
\times 3 \times 5 \times 2}{3 \times 3 \times 5 \times 3 \times 5}= \dfrac{2}{3 \times 5}=2^1 \times 3^{-1} \times 5^{-1}$


\bigskip

\bigskip


\ul{\textbf{Exercice 6:}}

 $L_1=3 \times 2^3-(3 \times 2)^3=3 \times 8 -
6^3=24-216=-192$ \qquad $M_1=\big( \dfrac{5}{10} \big) ^4=\dfrac{5}{10}
\times \dfrac{5}{10} \times \dfrac{5}{10} \times \dfrac{5}{10}=\dfrac{5^4}{10^4}=\dfrac{625}{10000}$

\enskip

$N_1= (3-5)^4+1,2 \times 10^2=(-2)^4+1,2 \times 100= 16+120=136$ \qquad  
$L_2=2 \times 3^3-(2 \times 3)^2= 2 \times 27 -6^2= 54-36=18$


\enskip

$M_2=\big( \dfrac{10}{7} \big) ^4=\dfrac{10}{7} \times \dfrac{10}{7} \times
\dfrac{10}{7} \times \dfrac{10}{7}=\dfrac{10^4}{7^4}=\dfrac{10000}{2401}$
\qquad $N_2=(4-6)^4+2,3 \times 10^3=(-2)^4+1300=16+1300=1316$

\bigskip


\bigskip

\ul{\textbf{Exercice 7:}} $\bullet$ (IE) // (GF) et (IG) // (EF) donc le
quadrilat�re IEFG est un parall�logramme.

$\bullet$ IEFG est un parall�logramme donc ses c�t�s oppos�s ont m�me mesure:
IG=EF=0,5 et IE=FG=3.

$\bullet$ DH=DI+IH=4+1=5

\enskip

$\bullet$ Dans le triangle HDF, on sait que G appartient � [HF], I appartient �
[HD] et les droites (IG) et (DF) sont parall�les. D'apr�s la propri�t� de
proportionnalit� des longueurs, on a:
$\dfrac{HI}{HD}=\dfrac{HG}{HF}=\dfrac{IG}{DF}$



Calcul de DF: $DF=\dfrac{5 \times 0,5}{1}=2,5$.

\enskip

$\bullet$ Dans le triangle HDF, on sait que I appartient � [DH], E appartient �
[DF] et les droites (IE) et (HF) sont parall�les. D'apr�s la propri�t� de
proportionnalit� des longueurs, on a:
$\dfrac{DI}{DH}=\dfrac{DE}{DF}=\dfrac{EI}{FH}$



Calcul de DF: $FH=\dfrac{5 \times 3}{4}=3,75$.
 
\end{document}
