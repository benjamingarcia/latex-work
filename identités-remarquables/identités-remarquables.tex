\documentclass[12pt, twoside]{article}
\usepackage[francais]{babel}
\usepackage[T1]{fontenc}
\usepackage[latin1]{inputenc}
\usepackage[left=5mm, right=5mm, top=5mm, bottom=5mm]{geometry}
\usepackage{float}
\usepackage{graphicx}
\usepackage{array}
\usepackage{multirow}
\usepackage{amsmath,amssymb,mathrsfs} 
\usepackage{soul}
\usepackage{textcomp}
\usepackage{eurosym}
 \usepackage{variations}
\usepackage{tabvar}
 
\pagestyle{empty}

\begin{document}


\begin{center}
\textbf{\Large{\ul{Identit�s remarquables}}}
\end{center}


\bigskip


\textit{\ul{Activit�:}} D�velopper les expressions suivantes: $(a+b)(a-b)$;
$(a+b)^2$ et $(a-b)^2$.


\bigskip



\begin{center}
\fbox{
\begin{minipage}{18cm}


\bigskip

\bigskip


\begin{center}
\begin{tabular}{cc}
FORME FACTORISEE \qquad \qquad & \qquad \qquad FORME DEVELOPPEE \\
\end{tabular}
\end{center}


\medskip


\begin{tabular}{lllllll}
\textit{carr� d'une somme}& $\longrightarrow$ & $(a+b)^2$ & = & $a^2+2ab+b^2$ &
\quad & \quad \\

\quad & & &  &  & & \\

\textit{carr� d'une diff�rence} & $\longrightarrow$ &  $(a-b)^2$ & = &
$a^2-2ab+b^2$ & \quad & \quad \\

\quad & & &  & & &  \\

\textit{produit d'une somme} & $\longrightarrow$ &
$(a+b)(a-b)$ & = & $a^2-b^2$ & $\longleftarrow$ & \textit{diff�rence de deux
carr�s} \\

\textit{et d'une diff�rence} & \quad &  \quad &  \quad & \quad  & \quad   & \\
\end{tabular}

\bigskip 

\bigskip


\end{minipage}
}

\end{center}


\bigskip




\ul{Exemples de d�veloppements}:

\begin{enumerate}
  \item \textit{En utilisant la distributivit�:}
  
  
  $\bullet$  $7(4y-9)=7 \times 4y - 7 \times 9=28y-63$
  
  \enskip
  
  \item \textit{En utilisant la double distributivit�:}
  
   
   $\bullet$  $(4a-5)(-5a-7)=4a \times (-5a) +4a \times (-7)+(-5)
   \times (-5a)+(-5) \times (-7)=-20a^2-3a+35$
  
  \enskip  
  
  
  \item \textit{En utilisant les identit�s remarquables:}
  
  $\bullet$ $(6-2u)^2=6^2-2 \times 6 \times 2u+ (2u)^2=36-24u+4u^2$
  
  
  $\bullet$ $(3t+1)^2=(3t)^2+2 \times 3t \times 1 + 1^2=9t^2+6t+1$
  
  
  $\bullet$ $(8-2y)(8+2y)=8^2-(2y)^2=64-4y^2$
\end{enumerate}

\bigskip


\bigskip




\ul{Exemples de factorisations}:

\begin{enumerate}
  
  \item \textit{En trouvant un facteur commun:}
  
   
   $\bullet$  $18-24d=6 \times 3 - 6 \times 4d=6(3-4d)$
  
  \enskip  
  
  
  \item \textit{En utilisant les identit�s remarquables:}
  
  $\bullet$ $81-36u^2=9^2-(6u)^2=(9-6u)(9+6u)$
  
  
  $\bullet$ $x^2+6x+9=x^2 + 2 \times x \times 3 + 3^2=(x+3)^2$
  
  
  $\bullet$ $4y^2-20y+25=(2y)^2-2 \times 2y \times 5 + 5^2=(2y-5)^2$
\end{enumerate}

\bigskip
 

\ul{Remarque}: En g�n�ral, le carr� d'une somme n'est pas �gal � la somme des
carr�s: $(a+b)^2 \neq a^2 + b^2$.


$(3+5)^2=8^2=64$ \quad et \quad $3^2+5^2=9+25=34$

\bigskip


\textit{ex 56 p 122 G-H-K-J; ex 62 p 122; ex 66 p 122; }

\textit{ex 35 p 120; ex 37 p 120; ex 43 p 121; ex 54 p 122;}


\textit{Calculer mentalement: $19 \times 99 + 19$; $101^2-101$;
$85^2-15^2$; $103^2$ ; $98^2$; $101 \times 99$} 

\textit{ex 98 p 125.}


\pagebreak


\begin{center}
\textbf{\Large{\ul{Identit�s remarquables}}}
\end{center}


\bigskip


\textit{\ul{Activit�:}} D�velopper les expressions suivantes: $(a+b)(a-b)$;
$(a+b)^2$ et $(a-b)^2$.


\bigskip



\begin{center}
\fbox{
\begin{minipage}{18cm}


\bigskip

\bigskip


\begin{center}
\begin{tabular}{cc}
FORME FACTORISEE \qquad \qquad & \qquad \qquad FORME DEVELOPPEE \\
\end{tabular}
\end{center}


\medskip


\begin{tabular}{lllllll}
\textit{carr� d'une somme}& $\longrightarrow$ & $(a+b)^2$ & = & $a^2+2ab+b^2$ &
\quad & \quad \\

\quad & & &  &  & & \\

\textit{carr� d'une diff�rence} & $\longrightarrow$ &  $(a-b)^2$ & = &
$a^2-2ab+b^2$ & \quad & \quad \\

\quad & & &  & & &  \\

\textit{produit d'une somme} & $\longrightarrow$ &
$(a+b)(a-b)$ & = & $a^2-b^2$ & $\longleftarrow$ & \textit{diff�rence de deux
carr�s} \\

\textit{et d'une diff�rence} & \quad &  \quad &  \quad & \quad  & \quad   & \\
\end{tabular}

\bigskip 

\bigskip


\end{minipage}
}

\end{center}


\bigskip




\ul{Exemples de d�veloppements}:

\begin{enumerate}
  \item \textit{En utilisant la distributivit�:}
  
  
  $\bullet$  $7(4y-9)=$
  
  \enskip
  
  \item \textit{En utilisant la double distributivit�:}
  
   
   $\bullet$  $(4a-5)(-5a-7)=$
  
  \enskip  
  
  
  \item \textit{En utilisant les identit�s remarquables:}
  
  $\bullet$ $(6-2u)^2=$
  
  \enskip
  
  $\bullet$ $(3t+1)^2=$
  
  \enskip
  
  $\bullet$ $(8-2y)(8+2y)=$
\end{enumerate}

\bigskip


\bigskip




\ul{Exemples de factorisations}:

\begin{enumerate}
  
  \item \textit{En trouvant un facteur commun:}
  
   
   $\bullet$  $18-24d=$
  
  \enskip  
  
  
  \item \textit{En utilisant les identit�s remarquables:}
  
  $\bullet$ $81-36u^2=$
  
\enskip

  
  $\bullet$ $x^2+6x+9=$
  
 \enskip
  
  
  $\bullet$ $4y^2-20y+25=$
\end{enumerate}

\bigskip
 

\ul{Remarque}: En g�n�ral, le carr� d'une somme n'est pas �gal � la somme des
carr�s: $(a+b)^2 \neq a^2 + b^2$.


$(3+5)^2=8^2=64$ \quad et \quad $3^2+5^2=9+25=34$

\bigskip


\textit{ex 56 p 122 G-H; ex 62 p 122 A-B; ex 66 p 122; }

\textit{ex 35 p 120 E-F-G; ex 37 p 120 A-B-C; ex 43 p 121 A-B-D; ex 54 p 122
A-B-C-D-F-G-J;}


\textit{Calculer mentalement: $19 \times 99 + 19$;  \quad $101^2-101$;
\quad $85^2-15^2$; \quad  $103^2$ ; \quad $98^2$; \quad $101 \times 99$} 



\end{document}
