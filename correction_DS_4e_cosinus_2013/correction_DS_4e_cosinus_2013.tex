\documentclass[12pt, twoside]{article}
\usepackage[francais]{babel}
\usepackage[T1]{fontenc}
\usepackage[latin1]{inputenc}
\usepackage[left=5mm, right=5mm, top=5mm, bottom=5mm]{geometry}
\usepackage{float}
\usepackage{graphicx}
\usepackage{array}
\usepackage{multirow}
\usepackage{amsmath,amssymb,mathrsfs}
\usepackage{soul}
\usepackage{textcomp}
\usepackage{eurosym}
 \usepackage{variations}
\usepackage{tabvar}

\pagestyle{empty}
\begin{document}


\begin{center}
\fbox{Correction du contr�le 6 (sujet 1)}
\end{center}


 
\bigskip

\ul{Exercice 3}: Le triangle DEF est rectangle en E. On a donc:
 $cos\widehat{EDF}=\dfrac{ED}{DF}$.
 On remplace par les valeurs num�riques: $cos(26)=\dfrac{5}{DF}$ \qquad soit
 $DF=5 \div cos(26) \approx 5,6 cm$.
 
 
 \bigskip

\ul{Exercice 4}: 

\begin{enumerate}
  \item  Le triangle GHI est rectangle en H. On a donc:
 $cos\widehat{IGH}=\dfrac{GH}{GI}$.
 On remplace par les valeurs num�riques: $cos(61)=\dfrac{GH}{6,4}$ soit
 $GH=6,4 \times cos(61) \approx 3,10m$. 
 \item 
 
 \enskip
 
 
 \begin{tabular}{ccc}
       \begin{minipage}{8cm}
  M�thode 1: Avec Pythagore
  
  \enskip
  
  
  Le triangle GHI est rectangle en H. D'apr�s le th�or�me de Pythagore, on a:
  
  
  $GI^2=GH^2+HI^2$    
  
  $6,4^2=3,1^2+HI^2$ 
  
  $40,96=9,61+ HI^2$
  
  
  $HI^2=40,96-9,61=31,35$
  
  \enskip
  
  Donc $HI=\sqrt{31,35} \approx 5,6m$.
  
       \end{minipage}
&
\qquad \qquad 
&
 \begin{minipage}{8cm}
 M�thode 2: Avec cosinus
 
 
 \enskip
 
 
 $\widehat {GIH}=180-90-61=29$�
 
 
 \enskip
 
  Le triangle GHI est rectangle en H. On a donc:
 $cos\widehat{GIH}=\dfrac{IH}{GI}$.
 
 
 On remplace: $cos(29)=\dfrac{HI}{6,4}$
 
  soit
 $IH=6,4 \times cos(29) \approx 5,6m$. 
      
       \end{minipage}
       \end{tabular}
\end{enumerate}

\bigskip

\ul{Exercice 5}:

\begin{enumerate}
  \item Le triangle KLM est rectangle en K. On a donc:
 $cos\widehat{KLM}=\dfrac{KL}{LM}$. 
 
 
 On remplace: $cos\widehat{KLM}=\dfrac{6}{11}$
   \qquad soit
 $\widehat{KLM}= arccos(\dfrac{6}{11})  \approx 57$�
 \item $\widehat{KML}=180-90-57=33$�
\end{enumerate}


\bigskip

\ul{Exercice 6}: Voir correction contr�le 4

\begin{enumerate}
  \item Th�or�me de Pythagore dans le triangle ADC 
  \item Th�or�me de Thal�s dans le triangle ACB
  \item R�ciproque de Pythagore dans le triangle ACB
  \item Le triangle ABC est rectangle en C. On a donc:
 $cos\widehat{ABC}=\dfrac{BC}{AB}$. 
 
 
 On remplace: $cos\widehat{ABC}=\dfrac{6}{10}$
 \qquad  soit
 $\widehat{ABC}= arccos(\dfrac{6}{10})  \approx 53$�  
 \item (MN) et (BC) sont parall�les et les angles $\widehat{ABC}$
 et $\widehat{AMN}$ sont correspondants 
 donc $\widehat{ABC}=\widehat{AMN}=53$�
\end{enumerate}



\pagebreak



\begin{center}
\fbox{Correction du contr�le 6 (sujet 2)}
\end{center}



\ul{Exercice 3}: Le triangle NOP est rectangle en O. On a donc:
 $cos\widehat{NPO}=\dfrac{OP}{PN}$.
 On remplace par les valeurs num�riques: $cos(42)=\dfrac{6}{PN}$  \qquad soit
 $PN=6 \div cos(42) \approx 8,1 cm$.
 
 
 \bigskip

\ul{Exercice 4}: 

\begin{enumerate}
  \item  Le triangle RST est rectangle en S. On a donc:
 $cos\widehat{RTS}=\dfrac{TS}{TR}$.
 On remplace par les valeurs num�riques: $cos(61)=\dfrac{ST}{8,3}$ soit
 $ST=8,3 \times cos(61) \approx 4,02m$. 
 \item 
 
 \enskip
 
 
 \begin{tabular}{ccc}
       \begin{minipage}{8cm}
  M�thode 1: Avec Pythagore
  
  \enskip
  
  
  Le triangle RST est rectangle en S. D'apr�s le th�or�me de Pythagore, on a:
  
  
  $RT^2=SR^2+ST^2$    
  
  $8,3^2=SR^2+4,02^2$ 
  
  $68,89=SR^2+ 16,1604$
  
  
  $SR^2=68,89-16,1604=52,7296$
  
  \enskip
  
  Donc $SR=\sqrt{52,7296} \approx 7,26m$.
  
       \end{minipage}
&
\qquad \qquad 
&
 \begin{minipage}{8cm}
 M�thode 2: Avec cosinus
 
 
 \enskip
 
 
 $\widehat {TRS}=180-90-61=29$�
 
 
 \enskip
 
  Le triangle TRS est rectangle en S. On a donc:
 $cos\widehat{TRS}=\dfrac{RS}{RT}$.
 
 
 On remplace: $cos(29)=\dfrac{RS}{8,3}$
 
  soit
 $RS=8,3 \times cos(29) \approx 7,26m$. 
      
       \end{minipage}
       \end{tabular}
\end{enumerate}

\bigskip

\ul{Exercice 5}:

\begin{enumerate}
  \item Le triangle KLM est rectangle en K. On a donc:
 $cos\widehat{KLM}=\dfrac{KL}{LM}$. 
 
 
 On remplace: $cos\widehat{KLM}=\dfrac{6}{11}$
   \qquad soit
 $\widehat{KLM}= arccos(\dfrac{6}{11})  \approx 57$�
 \item $\widehat{KML}=180-90-57=33$�
\end{enumerate}


\bigskip

\ul{Exercice 6}: Voir correction contr�le 4

\begin{enumerate}
  \item Th�or�me de Pythagore dans le triangle ADC 
  \item Th�or�me de Thal�s dans le triangle ACB
  \item R�ciproque de Pythagore dans le triangle ACB
  \item Le triangle ABC est rectangle en C. On a donc:
 $cos\widehat{ABC}=\dfrac{BC}{AB}$. 
 
 
 On remplace: $cos\widehat{ABC}=\dfrac{6}{10}$
 \qquad  soit
 $\widehat{ABC}= arccos(\dfrac{6}{10})  \approx 53$ � 
 \item (MN) et (BC) sont parall�les et les angles $\widehat{ABC}$
 et $\widehat{AMN}$ sont correspondants donc $\widehat{ABC}=\widehat{AMN}=53$�
\end{enumerate}
\end{document}


