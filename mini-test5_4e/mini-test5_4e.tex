\documentclass[12pt, twoside]{article}
\usepackage[francais]{babel}
\usepackage[T1]{fontenc}
\usepackage[latin1]{inputenc}
\usepackage[left=7mm, right=7mm, top=7mm, bottom=7mm]{geometry}
\usepackage{float}
\usepackage{graphicx}
\usepackage{array}
\usepackage{multirow}
\usepackage{amsmath,amssymb,mathrsfs}
\usepackage{soul}
\usepackage{textcomp}
\usepackage{eurosym}
 \usepackage{variations}
\usepackage{tabvar}

\pagestyle{empty}

\begin{document}

\begin{flushleft}
NOM PRENOM: \ldots \ldots \ldots \ldots \ldots \ldots \ldots \ldots \ldots
 
\bigskip

\end{flushleft}

\begin{center}
{\fbox{$4^{e}4$ \qquad \qquad \textbf{\Large{Mini-test 3 (sujet 1)}}
\qquad \qquad \ldots/01/2015}}
\end{center}



\bigskip 


\ul{Exercice 1}: Compl�ter les �galit�s. Justifier les r�ponses.
\begin{center}
 $\dfrac{3}{7}=\dfrac{\ldots\ldots}{-21}$ \quad\quad\quad\quad\quad\quad\qquad
 $\dfrac{8}{\ldots\ldots}=\dfrac{-16}{40}$ \quad\quad\quad\quad\quad\quad \qquad
 $\dfrac{-2}{-5}=\dfrac{12}{\ldots\ldots}$  
\end{center}


\bigskip
\bigskip
\ul{Exercice 2}: Compl�ter par $=$ ou $\neq$ en justifiant les r�ponses.
\begin{center}
$\dfrac{-56}{-57}\ldots\ldots\dfrac{57}{58}$
\quad\quad\quad\quad\quad\quad\qquad
$\dfrac{-9,1}{5,2}\ldots\ldots\dfrac{79,8}{-45,6}$
\quad\quad\quad\quad\quad\quad\qquad $\dfrac{-4}{27}\ldots\ldots\dfrac{2}{13,5}$
\end{center}

 
\bigskip
\bigskip
\ul{Exercice 3}: Simplifier les �critures fractionnaires suivantes.
\begin{center}
$\dfrac{-81}{-36}=$\ldots\ldots\ldots\ldots\ldots\ldots\ldots\ldots\ldots\ldots\ldots\ldots\\
\bigskip
$\dfrac{-15}{40}=$\ldots\ldots\ldots\ldots\ldots\ldots\ldots\ldots\ldots\ldots\ldots\ldots\\
\bigskip
$\dfrac{48}{72}=$\ldots\ldots\ldots\ldots\ldots\ldots\ldots\ldots\ldots\ldots\ldots\ldots\\
\bigskip
$\dfrac{21}{-56}=$\ldots\ldots\ldots\ldots\ldots\ldots\ldots\ldots\ldots\ldots\ldots\ldots\\
\end{center}


\bigskip
\bigskip

\begin{flushleft}
NOM PRENOM: \ldots \ldots \ldots \ldots \ldots \ldots \ldots \ldots \ldots
 
\bigskip

\end{flushleft}

\begin{center}
{\fbox{$4^{e}4$ \qquad \qquad \textbf{\Large{Mini-test 3 (sujet 2)}}
\qquad \qquad \ldots/01/2015}}
\end{center}



\bigskip 


\ul{Exercice 1}: Compl�ter les �galit�s. Justifier les r�ponses.
\begin{center}
 $\dfrac{4}{9}=\dfrac{\ldots\ldots}{-18}$ \quad\quad\quad\quad\quad\quad\qquad
 $\dfrac{8}{\ldots\ldots}=\dfrac{-24}{30}$  \quad\quad\quad\quad\quad\quad\qquad
 $\dfrac{-1}{-6}=\dfrac{7}{\ldots\ldots}$  
\end{center}


\bigskip
\bigskip
\ul{Exercice 2}: Compl�ter par $=$ ou $\neq$ en justifiant les r�ponses.
\begin{center}
$\dfrac{-27}{-28}\ldots\ldots\dfrac{28}{29}$
\quad\quad\quad\quad\quad\quad \qquad
$\dfrac{-12,8}{15}\ldots\ldots\dfrac{20,6}{17,3}$ 
\quad\quad\quad\quad\quad\quad \qquad
$\dfrac{17,36}{-22,32}\ldots\ldots\dfrac{-28,7}{36,9}$ \qquad
\end{center}


\bigskip
\bigskip
\ul{Exercice 3}: Simplifier les �critures fractionnaires suivantes.
\begin{center}
$\dfrac{-32}{-88}=$\ldots\ldots\ldots\ldots\ldots\ldots\ldots\ldots\ldots\ldots\ldots\ldots\\
\bigskip
$\dfrac{-21}{63}=$\ldots\ldots\ldots\ldots\ldots\ldots\ldots\ldots\ldots\ldots\ldots\ldots\\
\bigskip
$\dfrac{15}{25}=$\ldots\ldots\ldots\ldots\ldots\ldots\ldots\ldots\ldots\ldots\ldots\ldots\\
\bigskip
$\dfrac{-56}{21}=$\ldots\ldots\ldots\ldots\ldots\ldots\ldots\ldots\ldots\ldots\ldots\ldots\\
\end{center}


\end{document}
