\documentclass[12pt, twoside]{article}
\usepackage[francais]{babel}
\usepackage[T1]{fontenc}
\usepackage[latin1]{inputenc}
\usepackage[left=10mm, right=10mm, top=7mm, bottom=7mm]{geometry}
\usepackage{float}
\usepackage{graphicx}
\usepackage{array}
\usepackage{multirow}
\usepackage{amsmath,amssymb,mathrsfs} 
\usepackage{soul}
\usepackage{textcomp}
\usepackage{eurosym}
\usepackage{lscape}
 \usepackage{variations}
\usepackage{tabvar}
 
\pagestyle{empty}

\title{\ul{\textbf{Puissances}}}
\date{}

\begin{document}
\maketitle


\section{Puissances d'un nombre}

\subsection{Vocabulaire et notation}


\ul{D�finition}: Pour tout nombre relatif $a$ non nul et tout nombre entier $n$
positif non nul:

\enskip


\begin{center}
\fbox{


$a^n= \underbrace{a \times a \times  \ldots \times a}_{\text{n facteurs}} $
\quad \quad et \quad \quad $a^{-n}=\underbrace{\dfrac{1}{a \times a \times 
\ldots \times a}}_{\text{n facteurs}}=\dfrac{1}{a^n}$

}
 \end{center}

\bigskip

\ul{Remarques}:
$a^n$ se lit ``a puissance n'' ou ``a exposant n''.

Si a est diff�rebt de 0, $a^{-n}=\dfrac{1}{a^n}$ d�signe l'inverse de $a^n$.

\bigskip

\ul{Cas particuliers}:

\enskip

\begin{itemize}
  \item [$\bullet$] $a^2=a \times a$ et se lit ``a au carr�'''.
  
  
  \item [$\bullet$] $a^3=a \times a \times a$ et se lit ``a au cube''.
  
  
  \item [$\bullet$] $a^1=a$ et $a^{-1}$ est l'inverse de $a$.
  
  \item [$\bullet$] \ul{Convention}: $a^0=1$.
\end{itemize}


\bigskip

\ul{Exemples}:


\begin{center}
\begin{tabular}{lll}
$5^2=\ldots \ldots$ \qquad \qquad  & \qquad  \qquad $4^3=\ldots \ldots$ \qquad
\qquad & \qquad  \qquad $3^1=\ldots \ldots$ \\

$2010^0=\ldots \ldots$ \qquad \qquad  & \qquad  \qquad $8^{-1}=\dfrac{1}{8}=0,125$
\qquad \quad & \qquad  \qquad $10^{-3}=\dfrac{1}{10^3}=\dfrac{1}{1000}=0,001$ \\

\end{tabular}
 \end{center} 

 \enskip
 
 
 \ul{Avec la calculatrice}: 
 
 
 
\subsection{Calculer avec des puissances}


\ul{R�gles de priorit�}: 

$\bullet$ En l'absences de parenth�ses, on calcule les puissances avant
d'effectuer les autres op�rations.

$\bullet$ En pr�sence de parenth�ses, on effectue d'abord les calculs entre
parenth�ses.


\medskip

\ul{Exemples}:

$5 \times 3^2=5 \times 9 =45$ \qquad \qquad $(5+3)^2=8^2=64$ \qquad \qquad
$\left( \dfrac{5}{3} \right) ^3=\dfrac{5^3}{3^3}$ \quad car
$\left( \dfrac{5}{3} \right) ^3=\dfrac{5}{3} \times \dfrac{5}{3} \times
\dfrac{5}{3}$.



\section{Cas particulier: puissances de 10}

\subsection{Multiplier par une puissance de 10}

\ul{Propri�t�s}: Pour tout nombre entier positif $n$ (non nul):




\begin{center}
\fbox{
\begin{minipage}{18cm}
 
$10^n=\underbrace{10 \times 10 \times \ldots \times 10}_{\text{n
facteurs}}=1 \underbrace{0 \ldots 0}_{\text{n z�ros}}$
\qquad  \qquad et \qquad \qquad 
$10^{-n}=\underbrace{\dfrac{1}{10 \times 10 \times \ldots \times 10}}_{\text{n
facteurs}}=\underbrace{0,0 \ldots 0}_{\text{n z�ros}} 1$
\end{minipage}
}
\end{center}

\enskip

\ul{Exemples}: $10^5=10000$ \qquad \qquad $10^{-5}=0,00001$

\bigskip

\ul{Propri�t�s}:


\textbf{Multiplier} un nombre par $10^n$ revient � d�caler la virgule de
\textbf{n rangs vers la droite} (on compl�te par des z�ros si n�c�ssaire).


\textbf{Multiplier} un nombre par $10^{-n}$ revient � d�caler la virgule de
\textbf{n rangs vers la gauche} (on compl�te par des z�ros si n�c�ssaire).

\enskip

\ul{Remarque}: Multiplier par $10^{-n}$ revient � diviser par $10^n$.


\enskip


\ul{Exemples}: $10^0=\ldots$ \qquad \qquad \qquad \qquad $208,564 \times
10^2=\ldots \ldots$ \qquad \qquad \qquad \qquad $38,5 \times 10^{-3}=\ldots
\ldots$


\subsection{R�gles de calcul}




\ul{Propri�t�}: pour tous nombres entiers relatifs $m$ et $p$:



\begin{center}

\fbox{

\begin{tabular}{ll}
$\bullet$ $10^m \times 10^p=10^{m+p}$ \qquad & \qquad r�gle du \textbf{produit}
de deux puissances de 10 \\

\quad & \quad \\

$\bullet$ $\dfrac{10^m}{10^p}=10^{m-p}$ \qquad & \qquad r�gle du
\textbf{quotient} de deux puissances de 10 \\

\quad & \quad \\

$\bullet$ $\left( 10^m \right) ^p =10^{m \times p}$ \qquad & \qquad r�gle des
\textbf{puissances} de puissances de 10 \\
\end{tabular}
 }
\end{center}

\enskip


\ul{Exemples}: \quad $10^4 \times 10^3= 10^{4+3}=10^7=10000000$ \qquad \qquad
$\dfrac{10}{10^{-3}}=\dfrac{10^1}{10^{-3}}=10^{1-(-3)}=10^{1+3}=10^4$


$\left( 10^3 \right) ^{-2}=10^{3 \times (-2)}=10^{-6}$


\subsection{Ecriture scientifique}



\ul{D�finitions}: Tout nombre d�cimal non nul peut �tre �crit en
\textbf{notation scientifique}, c'est-�-dire sous la forme $a \times 10^n$, o�
$a$ est un nombre d�cimal ayant un seul chiffre avant la virgule diff�rent de
z�ro et $n$ un entier relatif.

Le nombre $a$ est appel� \textbf{mantisse}.

\enskip

\ul{Exemple}: Ecrire le nombre $A=6430,2$ en notation scientifique.

$A=643,2= 6,4302 \times 10^3$.


\enskip

\ul{A la calculatrice}:

\end{document}
