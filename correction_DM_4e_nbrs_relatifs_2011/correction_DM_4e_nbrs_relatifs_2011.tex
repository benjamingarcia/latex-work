\documentclass[12pt, twoside]{article}
\usepackage[francais]{babel}
\usepackage[T1]{fontenc}
\usepackage[latin1]{inputenc}
\usepackage[left=5mm, right=5mm, top=3mm, bottom=3mm]{geometry}
\usepackage{float}
\usepackage{graphicx}
\usepackage{array}
\usepackage{multirow}
\usepackage{amsmath,amssymb,mathrsfs}
\usepackage{soul}
\usepackage{textcomp}
\usepackage{eurosym}
 \usepackage{variations}
\usepackage{tabvar}
\usepackage{lscape}


\pagestyle{empty}

\begin{document}




\section*{\center{Correction devoir maison 3}}

\subsection*{Exercice 1}

\begin{align*}
A & =(-2) \times 5 + 3 + (-4) \times (-2)  & B & =-3 \times 7 +(-44) \div (-6+2) \\ 
& = -10 + 3 + (-4) \times (-2) & &= -3 \times 7 +(-44) \div (-4) \\
& = -10 + 3 + 8 & & = -21 + (-44) \div (-4)\\
& = -7 + 8 & & = -21+ 11\\
& = 1 & & =-10\\
\end{align*}


\begin{align*}
C & =-11+2 \times (-3+ 7 \times 5)  & D & = (-16) \div (-8) + (-24) \div 4\\
& = -11+2 \times (-3 +35)           &   & = 2 + (-24) \div 4 \\
& = -11+2 \times 32                 &   & =2+ (-6) \\
&= -11+ 64 =53                         &   & =-4 \\
\end{align*}


\subsection*{Exercice 2}

Pour chaque question, il y a une infinit� de solutions, nous allons en donner un
exemple de chaque.

\begin{enumerate}
  \item -2 et -5: le produit est positif (car $-2 \times
  (-5)= +10$) et la somme est n�gative (car $-2+(-5)=-7$).
  \item -2 et 5: le produit est n�gatif (car $-2 \times
  5= -10$) et la somme est positive (car $-2+5=3$). 
   \item 2 et 5: le produit est positif (car $2 \times
  5= +10$) et la somme est positive (car $2+5=7$).  
   \item 2 et -5: le produit est n�gatif (car $2 \times
  (-5)= -10$) et la somme est n�gative (car $2+(-5)=-3$). 
\end{enumerate}

 
\subsection*{Exercice 3}

\begin{enumerate}
  \item $\big ( -3 + (-5) \big ) \times \big ( 6-(-8) \big)$
  \qquad \qquad 2. $\dfrac{-75}{8-14}$ \quad ou \quad $-75 \div (8-14)$
  
  \item[3.] La somme de 25 et du produit de 7 par (-2).
  
  \item [4.] Le quotient de la diff�rence de 4 et de (-6) par le produit de (-3)
  et de (-2).
\end{enumerate}

\subsection*{Exercice 4}

$-25 -75=-100$ \qquad \qquad 
$-9 \times (-100)=900$ \qquad \qquad 
$7 \times 10=70$ \qquad \qquad 
$900-70=830$

 \subsection*{Exercice 5}

Je cherche deux nombres entiers relatifs dont le produit est -28.
Les couples possibles sont:

(1;-28) \quad ou \quad (-1; 28) \quad ou \quad (2;-14) \quad ou \quad (-2; 14)
\quad ou \quad (4; -7) \quad ou \quad (-4; 7).


La somme des nombres trouv�s est -3. En calculant les diff�rentes sommes, la
seule solution est: 4 et (-7).

On a bien: $4 \times (-7)=-28$ et $-7+4=-3$

\subsection*{Exercice 6}

a) $(5-4) \times  3 + 2 =1 \times 3 + 2 = 3 +2=5$ \qquad \qquad autre
solution: $(5-4)\times(3+2))1 \times 5 =5$


b) $5-4 \times (3+2)=5-4 \times 5=5-20=-15$


c) $5-(4 \times 3+2)=5-(12+2)=5-14=-9$
\end{document}


