\documentclass[12pt, twoside]{article}
\usepackage[francais]{babel}
\usepackage[T1]{fontenc}
\usepackage[latin1]{inputenc}
\usepackage[left=8mm, right=8mm, top=8mm, bottom=8mm]{geometry}
\usepackage{float}
\usepackage{graphicx}
\usepackage{array}
\usepackage{multirow}
\usepackage{amsmath,amssymb,mathrsfs}
\pagestyle{empty}
\begin{document}

\section*{\center{Bilan proportionnalit� des longueurs dans le triangle}}




\bigskip

\begin{center}
\begin{tabular}{|m{9cm}|m{9cm}|}
\hline


\textbf{Ce que je dois savoir} & \textbf{Ce que je dois savoir faire} \\

\hline

\enskip


\begin{itemize}
  \item [$\bullet$] Je connais la prori�t� de proportionnalit� des longueurs
  dans un triangle.
  
  
  \item [$\bullet$] Je sais reconna�tre dans quel cas on peut utiliser cette
  propri�t�.
  
  \item [$\bullet$] Je connais la m�thode pour calculer une longueur (produits
  en croix).

 \end{itemize}

&


\begin{itemize}
  \item[$\bullet$] Je sais �crire l'�galit� des rapports de longueurs dans un
  triangle.


  \item[$\bullet$] Je sais r�diger ma r�ponse.

  
      
  \item[$\bullet$] Je sais r�soudre des probl�mes.
 
\end{itemize} \\

\hline

\end{tabular}
\end{center}


\bigskip


\section*{\center{Bilan proportionnalit� des longueurs dans le triangle}}




\bigskip

\begin{center}
\begin{tabular}{|m{9cm}|m{9cm}|}
\hline


\textbf{Ce que je dois savoir} & \textbf{Ce que je dois savoir faire} \\

\hline

\enskip


\begin{itemize}
  \item [$\bullet$] Je connais la prori�t� de proportionnalit� des longueurs
  dans un triangle.
  
  
  \item [$\bullet$] Je sais reconna�tre dans quel cas on peut utiliser cette
  propri�t�.
  
  \item [$\bullet$] Je connais la m�thode pour calculer une longueur (produits
  en croix).

 \end{itemize}

&


\begin{itemize}
  \item[$\bullet$] Je sais �crire l'�galit� des rapports de longueurs dans un
  triangle.


  \item[$\bullet$] Je sais r�diger ma r�ponse.

  
      
  \item[$\bullet$] Je sais r�soudre des probl�mes.
 
\end{itemize} \\

\hline

\end{tabular}
\end{center}


\bigskip


\section*{\center{Bilan proportionnalit� des longueurs dans le triangle}}




\bigskip

\begin{center}
\begin{tabular}{|m{9cm}|m{9cm}|}
\hline


\textbf{Ce que je dois savoir} & \textbf{Ce que je dois savoir faire} \\

\hline

\enskip


\begin{itemize}
  \item [$\bullet$] Je connais la prori�t� de proportionnalit� des longueurs
  dans un triangle.
  
  
  \item [$\bullet$] Je sais reconna�tre dans quel cas on peut utiliser cette
  propri�t�.
  
  \item [$\bullet$] Je connais la m�thode pour calculer une longueur (produits
  en croix).

 \end{itemize}

&


\begin{itemize}
  \item[$\bullet$] Je sais �crire l'�galit� des rapports de longueurs dans un
  triangle.


  \item[$\bullet$] Je sais r�diger ma r�ponse.

  
      
  \item[$\bullet$] Je sais r�soudre des probl�mes.
 
\end{itemize} \\

\hline

\end{tabular}
\end{center}


\end{document}
