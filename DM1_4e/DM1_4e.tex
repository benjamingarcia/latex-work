\documentclass[12pt, twoside]{article}
\usepackage[francais]{babel}
\usepackage[T1]{fontenc}
\usepackage[latin1]{inputenc}
\usepackage[left=1cm, right=1cm, top=1cm, bottom=1cm]{geometry}
\usepackage{float}
\usepackage{graphicx}
\usepackage{array}
\usepackage{multirow}
\usepackage{amsmath,amssymb,mathrsfs}
\usepackage{soul}
\usepackage{textcomp}
\usepackage{eurosym}
 \usepackage{variations}
\usepackage{tabvar}


\pagestyle{empty}

\begin{document}


\section*{\center{Devoir maison 1}}

\textit{Devoir � rendre sur feuille grand format petits
carreaux pour le \ul{vendredi 25 septembre 2009}.}

\subsection*{Exercice 1}

Recopier et effectuer les calculs suivants en soulignant en vert � chaque �tape
le calcul en cours.

\enskip

\textit{Remarque: un r�sultat donn� directement sans les �tapes de calculs ne
donnera pas de point.}

\enskip

\begin{tabular}{cc}
\begin{minipage}{9cm}
$A=(-3) \times 5+1+(-2) \times (-2)$

\enskip

$B=-3 \times 8 + (-33) \div (-5+2)$

\enskip

$C=11+2 \times \big( (-3)+(-7) \times 3 \big)$

\end{minipage}
&
\begin{minipage}{9cm}
$D=(-14) \div (-7)+ (-15) \div (+5)$

\bigskip

$E=\dfrac{(-1) \times (-2)-(-3) \times 4}{2-2 \times (-6)}$

\bigskip

$F=\dfrac{15+ \big( (-3) \times (-2)+(-5) \big)}{-2 \times 7 - \big( (-6)+8
\times (-3) \big)}$
\end{minipage}
\end{tabular}


\subsection*{Exercice 2}


\begin{enumerate}
  \item Trouver deux nombres relatifs dont le produit est positif et la somme
  est n�gative. 
  \item Trouver deux nombres relatifs dont le produit est n�gatif et la somme
  est positive.
  \item Trouver deux nombres relatifs dont le produit et la somme sont positifs.
  \item Trouver deux nombres relatifs dont le produit et le somme sont
  n�gatifs.
\end{enumerate}

Pour chaque question, vous justifierez votre r�ponse en effectuant les
calculs (produit et somme des deux nombres trouv�s).



\subsection*{Exercice 3}

\begin{enumerate}
  \item Traduire les phrases suivantes par un calcul:
  
  \begin{enumerate}
    \item Le produit de la somme de -6 et de -4 par la diff�rence de -10 et de
    -8.
    \item Le quotient de -100 par la diff�rence de -24 et de -14.
    \item La diff�rence du quotient de 12 par -2 et du produit de 3 par -5.
\end{enumerate}

\item Effectuer ces calculs. 
\end{enumerate}
  

\subsection*{Exercice 4}

 Avec les nombres propos�s, retrouver le r�sultat annonc� en respectant
 les consignes suivantes:

$\bullet$ Chaque nombre est utilis� une fois.

$\bullet$ Tous les nombres sont utilis�s.

$\bullet$ Toutes les op�rations sont autoris�es.

\enskip

\begin{center}
-3 \qquad -5 \qquad 25 \qquad -100 \qquad 7
\end{center}

\enskip

R�sultat � trouver: -650

\end{document}
