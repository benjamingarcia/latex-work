\documentclass[12pt, twoside]{article}
\usepackage[francais]{babel}
\usepackage[T1]{fontenc}
\usepackage[latin1]{inputenc}
\usepackage[left=5mm, right=5mm, top=3mm, bottom=3mm]{geometry}
\usepackage{float}
\usepackage{graphicx}
\usepackage{array}
\usepackage{multirow}
\usepackage{amsmath,amssymb,mathrsfs}
\usepackage{soul}
\usepackage{textcomp}
\usepackage{eurosym}
 \usepackage{variations}
\usepackage{tabvar}


\pagestyle{empty}

\begin{document}



\section*{\center{Devoir maison 8}}

\begin{center}
\fbox{
\begin{minipage}{19cm}
\textit{Devoir � rendre sur feuille grand format pour le \ul{mardi 7 mai
2013}. Chaque r�ponse sera justifi�e.}
 \end{minipage}
} 
\end{center}
 
 
\bigskip

\ul{Exercice 1:} \textit{(4,5 points)} 

\enskip

Un train relie deux villes A et C en passant par B. 

La vitesse moyenne d'un train entre A et B est de 230 km/h. Le temps n�cessaire
� ce parcours est de 3h30min. 

La distance entre les villes B et C est de 180 km. Le temps mis sur cette
portion du trajet est de 1h30min.

\begin{enumerate}
  \item Quelle distance s�pare les villes A et B?
  \item Quelle est la vitesse moyenne du train entre les villes B et C?
  \item Quelle est la distance totale parcourue par le train?
  \item Quelle est la vitesse moyenne du train sur l'ensemble du parcours?
  \item Convertir cette vitesse en m/s.
\end{enumerate}


\bigskip

\ul{Exercice 2:} \textit{(3,5 points)} 


\enskip

La lumi�re parcourt environ $3 \times 10^5$ km par seconde. La distance du
soleil � la Terre est d'environ $1,5 \times 10^8$ km.

\begin{enumerate}
  \item 
  Combien de temps la lumi�re met-elle pour parcourir la distance du soleil �
  la Terre?
  \item Calculer la distance que la lumi�re parcourt en une ann�e.
  
  \enskip
  
  (\textit{Remarque: cette distance s'appelle une ann�e lumi�re.})
\end{enumerate}


\bigskip

\ul{Exercice 3:} \textit{(3 points)} 

\enskip

Donner la notation scientifique des nombres ci-dessous:

$A=27000$ \qquad \qquad $B=0,00034$ \qquad \qquad $C=0,067 \times 10^3$


\bigskip

\ul{Exercice 4:} \textit{(2,5 points)} 

\enskip

Ecrire sous la forme d'une seule puissance:


$D=3^6 \times 3^5$	\qquad \qquad $E=\dfrac{12^7}{12^4}$ \qquad \qquad
$F=(5,4^{-3})^2$ \qquad \qquad $G=2^3 \times 125$


\bigskip

\ul{Exercice 5:} \textit{(4,5 points)} 


\enskip

Calculer chaque expresion en \textbf{d�taillant chaque �tape de calcul}.


\enskip



$H=13-2^4 \times (5-2 \times 4)^2$ \qquad \qquad $I=\dfrac{10^2-6 \times
3^2}{2^6-3 \times 4^2}$ \qquad \qquad $J=\dfrac{4 \times 8 - 3^2}{7 \times
2^3-(5 \times 2)^2}$



\bigskip

\ul{Exercice 6:} \textit{(3 points)} 

\enskip

$K=\dfrac{42 \times 10^{-3} \times 5 \times (10^2)^3}{3 \times 10^6}$


\enskip


\begin{enumerate}
  \item Calculer le nombre K en d�taillant les �tapes et donner son r�sultat
  sous la forme d'un nombre d�cimal.
  \item Donner son �criture scientifique.
\end{enumerate}

\end{document}
