%%This is a very basic article template.
%%There is just one section and two subsections.

\documentclass[12pt, twoside]{article}

\usepackage[francais]{babel}
\usepackage[T1]{fontenc}
\usepackage[latin1]{inputenc}
\usepackage[left=2cm, right=2cm, top=2cm, bottom=2cm]{geometry}
\usepackage{float}
\usepackage{graphicx}
\usepackage{array}
\usepackage{multirow}

\begin{document}


\section*{\center{Les puissances}}

\subsection*{Ce qu'il faut savoir $\ldots$}

\textbf{D�finition:}
Soit $a$ un nombre r�el et $n$ un nombre entier naturel non nul. On appelle
puissance $n^{ieme}$ de $a$ et on note $a^{n}$ le nombre $a^{n}=a*a*\ldots*a$
compos� de $n$ facteurs. On pose lorsque $a$ est non nul: $a^{0}=1$ et
$a^{-n}=\frac{1}{a^{n}}$.


\bigskip
\textbf{Exemple} (� compl�ter):

\bigskip
\textbf{R�gles de calcul} (� compl�ter): Soient $a$ et $b$ deux r�els. On a:
\begin{center}
$a^{m}*a^{n}=\ldots$\\
$(a^{n})^{m}=\ldots$\\
$(ab)^{n}=\ldots$\\
$(\frac{a}{b})^{n}=\ldots$ avec $b$ non nul
\end{center}

\subsection*{Applications}
Les exercices de niveau $\diamond$ et $\diamond \diamond$ sont � savoir faire et
� finir pour le prochain module (jeudi 25/09). Les exercices $\diamond \diamond
\diamond$ et $\diamond \diamond\diamond \diamond$ sont plus difficiles et
facultatifs.

\bigskip
 \textbf{Exercice 1:}($\diamond$)
 Simplifier les �critures suivantes:
 $\frac{(-12)^{3}*5^{4}*8^{3}}{6^{4}*10^{3}*16}$ et $(\frac{-5}{11})^{3}*(\frac{11}{5})^{4}$
 
 \bigskip
 \textbf{Exercice 2:}($\diamond$)
 Sans effectuer de calcul d�terminer le signe de
 $\frac{(-1,2)^{4}*(-2)^{13}*(-1)^{2}}{(-5)^{6}*(-7)^{21}}$.
 
 \bigskip
 \textbf{Exercice 3:}($\diamond \diamond$)
 Le nombre $\frac{(0,2)^{3}*10^{4}}{2^{3}*81}: \frac{8^{3}*15}{12^{5}}$
 est-il entier?
 
 
 \bigskip
 \textbf{Exercice 4:}($\diamond \diamond$)
 Calculer $xy$, $x+y$, $x-y$, $\frac{x}{y}$ avec $x=9*10^{7}$ et $y=3,6*10^{8}$.
 
 
\bigskip 
\textbf{Exercice 5:}($\diamond \diamond$)
On appelle carr� multiplicativement magique, un carr� dont le produit est le
m�me sur chaque ligne, chaque colonne et chaque diagonale. Compl�ter le tableau
pour qu'il soit multiplicativement magique.\\
\bigskip
\begin{center}
$\begin{array}{|c|c|c|}
 \hline 2^{2}*3^{4}*5 & & \\
 \hline 3^{4}*5^{2} & 2^{2}*3^{3}*5^{2} & 2^{4}*(3*5)^{2} \\
 \hline  & & \\
 \hline
 \end{array}$
\end{center}
 
\bigskip 
\textbf{Exercice 6:}($\diamond \diamond \diamond$)
$1m^{3}$ d'eau de mer contient $0,004 mg$ d'or (et oui!!). Le volume total
d'eau de mer sur la Terre est $1,3*10^{6} km^{3}$. Quelle est la masse totale
d'or en tonnes que contient toute l'eau de mer?

\bigskip
\textbf{Exercice 7:}($\diamond\diamond\diamond\diamond$)
A l'aide de la calculatrice, v�rifier l'�galit� suivante:\\
 $(99 \thinspace 999 \thinspace 999)^{2}+ (20  \thinspace 000)^{2}= (100
 \thinspace 000 \thinspace 001)^{2}$. D�montrer cette �galit�.
\end{document}
