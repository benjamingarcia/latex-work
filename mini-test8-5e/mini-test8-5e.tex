\documentclass[12pt, twoside]{article}
\usepackage[francais]{babel}
\usepackage[T1]{fontenc}
\usepackage[latin1]{inputenc}
\usepackage[left=7mm, right=7mm, top=4mm, bottom=4mm]{geometry}
\usepackage{float}
\usepackage{graphicx}
\usepackage{array}
\usepackage{multirow}
\usepackage{amsmath,amssymb,mathrsfs}
\usepackage{soul}
\usepackage{textcomp}
\usepackage{eurosym}
\usepackage{variations}
\usepackage{tabvar}

\pagestyle{empty}

\begin{document}

\begin{flushleft}
NOM PRENOM: \ldots \ldots \ldots \ldots \ldots \ldots \ldots \ldots \ldots
 
\bigskip

\end{flushleft}

\begin{center}
{\fbox{$5^{e}1$ \qquad \qquad \textbf{\Large{Contr�le de cours 8 (sujet 1)}}
\qquad \qquad 10/05/2010}}
\end{center}

\bigskip



\ul{Exercice 1}: (\textit{4 points}) Compl�ter les phrases suivantes:

\begin{enumerate}
  \item (DH) est la \ldots \ldots \ldots \ldots \ldots \ldots \ldots \ldots
  \ldots issue de \ldots \ldots \ldots \ldots .
 
  \enskip
  
  \item ($d_1$) est la \ldots \ldots \ldots \ldots \ldots \ldots \ldots \ldots
  \ldots de \ldots \ldots \ldots \ldots \ldots \ldots.

\enskip


  \item (AI) est la \ldots \ldots \ldots \ldots \ldots \ldots \ldots \ldots
  \ldots relative \ldots \ldots \ldots \ldots .
 
  \enskip
  
  
  \item ($d_2$) est la \ldots \ldots \ldots \ldots \ldots \ldots \ldots \ldots
  \ldots de \ldots \ldots \ldots \ldots \ldots \ldots.
\end{enumerate}

\bigskip


\ul{Exercice 2}: (\textit{3 points}) Sur le triangle TOC ci-dessous, tracer �
main lev�e et \textbf{coder}:

\enskip

$\bullet$ en bleu la m�diatrice de [TO]

\enskip

$\bullet$ en noir la m�diane relative � [OC]

\enskip


$\bullet$ en vert la hauteur issue de O.



\bigskip

\ul{Exercice 3}: (\textit{3 points})

\begin{enumerate}
  
  
  \item A, B, C et D sont 4 points tels que AB=AC=AD. Compl�ter la phrase:
  
   ``Le point \ldots \ldots est le centre du cercle circonscrit au triangle
   \ldots \ldots \ldots.''
      
  \enskip   
                   
   \item   Parmi les trois triangles OUI, TAC et MON, indiquer celui pour lequel
   le point K est le centre du cercle circonscrit. Justifier la r�ponse.


\bigskip

\bigskip

\bigskip

\bigskip


\bigskip

\bigskip

\bigskip

\ldots \ldots \ldots \ldots \ldots \ldots \ldots \ldots \ldots \ldots \ldots
\ldots \ldots \ldots \ldots \ldots \ldots \ldots \ldots \ldots \ldots \ldots
\ldots \ldots \ldots \ldots \ldots \ldots \ldots \ldots \ldots \ldots \ldots
\ldots 

\ldots \ldots \ldots \ldots \ldots \ldots \ldots \ldots \ldots \ldots \ldots
\ldots \ldots \ldots \ldots \ldots \ldots \ldots \ldots \ldots \ldots \ldots
\ldots \ldots \ldots \ldots \ldots \ldots \ldots \ldots \ldots \ldots \ldots
\ldots  

\enskip

   \item D'apr�s la figure ci-dessous, $\mathcal{C}$ est le cercle circonscrit
   au triangle \ldots \ldots \ldots.   
\end{enumerate}

\pagebreak


\begin{flushleft}
NOM PRENOM: \ldots \ldots \ldots \ldots \ldots \ldots \ldots \ldots \ldots
 
\bigskip

\end{flushleft}

\begin{center}
{\fbox{$5^{e}1$ \qquad \qquad \textbf{\Large{Contr�le de cours 8 (sujet 2)}}
\qquad \qquad 10/05/2010}}
\end{center}

\bigskip



\ul{Exercice 1}: (\textit{4 points}) Compl�ter les phrases suivantes:

\begin{enumerate}
  \item (KI) est la \ldots \ldots \ldots \ldots \ldots \ldots \ldots \ldots
  \ldots issue de \ldots \ldots \ldots \ldots .
 
  \enskip
  
  \item ($d_1$) est la \ldots \ldots \ldots \ldots \ldots \ldots \ldots \ldots
  \ldots de \ldots \ldots \ldots \ldots \ldots \ldots.

\enskip


  \item (OT) est la \ldots \ldots \ldots \ldots \ldots \ldots \ldots \ldots
  \ldots relative \ldots \ldots \ldots \ldots .
 
  \enskip
  
  
  \item ($d_2$) est la \ldots \ldots \ldots \ldots \ldots \ldots \ldots \ldots
  \ldots de \ldots \ldots \ldots \ldots \ldots \ldots.
\end{enumerate}

\bigskip


\ul{Exercice 2}: (\textit{3 points}) Sur le triangle TOC ci-dessous, tracer �
main lev�e et \textbf{coder}:

\enskip

$\bullet$ en noir la m�diatrice de [OC]

\enskip

$\bullet$ en bleu la m�diane relative � [OT]

\enskip


$\bullet$ en vert la hauteur issue de T.



\bigskip

\ul{Exercice 3}: (\textit{3 points})

\begin{enumerate}
  
  
  \item A, B, C et D sont 4 points tels que BA=BC=BD. Compl�ter la phrase:
  
   ``Le point \ldots \ldots est le centre du cercle circonscrit au triangle
   \ldots \ldots \ldots.''
      
  \enskip   
                   
   \item   Parmi les trois triangles OUI, TAC et MON, indiquer celui pour lequel
   le point K est le centre du cercle circonscrit. Justifier la r�ponse.


\bigskip

\bigskip

\bigskip

\bigskip


\bigskip

\bigskip

\bigskip

\ldots \ldots \ldots \ldots \ldots \ldots \ldots \ldots \ldots \ldots \ldots
\ldots \ldots \ldots \ldots \ldots \ldots \ldots \ldots \ldots \ldots \ldots
\ldots \ldots \ldots \ldots \ldots \ldots \ldots \ldots \ldots \ldots \ldots
\ldots 

\ldots \ldots \ldots \ldots \ldots \ldots \ldots \ldots \ldots \ldots \ldots
\ldots \ldots \ldots \ldots \ldots \ldots \ldots \ldots \ldots \ldots \ldots
\ldots \ldots \ldots \ldots \ldots \ldots \ldots \ldots \ldots \ldots \ldots
\ldots  

\enskip

   \item D'apr�s la figure ci-dessous, $\mathcal{C}$ est le cercle circonscrit
   au triangle \ldots \ldots \ldots.   
\end{enumerate}

 \end{document}
