\documentclass[12pt, twoside]{article}
\usepackage[francais]{babel}
\usepackage[T1]{fontenc}
\usepackage[latin1]{inputenc}
\usepackage[left=7mm, right=7mm, top=7mm, bottom=7mm]{geometry}
\usepackage{float}
\usepackage{graphicx}
\usepackage{array}
\usepackage{multirow}
\usepackage{amsmath,amssymb,mathrsfs}
\usepackage{soul}
\usepackage{textcomp}
\usepackage{eurosym}
 \usepackage{variations}
\usepackage{tabvar}

\pagestyle{empty}
\begin{document}




\begin{center}
\fbox{Correction contr�le 3 (sujet 1)}
\end{center}

\ul{Exercice 4}:

\begin{tabular}{llll}
\begin {minipage}{4cm}
$A=-6-4 \times (+5)$

$A=-6-20$

$A=-26$
\end{minipage}
&
\begin {minipage}{4,5cm}
$B=7 \times 2 - 3 \times 4 -8$

$B=14-12-8$

$B=2-8$

$B=-6$
\end{minipage}
&
\begin {minipage}{4,5cm}
$C=5 \times (-7+10) -12$

$C=5 \times 3 -12$

$C=15 -12$

$C=3$
\end{minipage}

\begin {minipage}{5cm}
$D=\dfrac{2-7 \times (-6) + 1}{-5-1+ (-3) \times (-5)}$

\bigskip


$D=\dfrac{2-(-42)+1}{-5-1+15}$

\bigskip


$D=\dfrac{2+42+1}{-6+15}$


\bigskip


$D=\dfrac{45}{9}=5$
\end{minipage}
\end{tabular}

\bigskip


\ul{Exercice 5}:

\enskip

\begin{tabular}{cc}
\begin{minipage}{8cm}
\begin{enumerate}
  \item $(-4) \times (-5)-6$
  \item $\big( 3+(-2) \big) \div \big(7 \times (-1) \big)$ ou
  $\dfrac{3+(-2)}{7 \times (-1)}$
  \end{enumerate}
\end{minipage}
&
\begin{minipage}{10cm}
\begin{enumerate}
  \item [3.] La somme de (-6) et du produit de 5 par (-2).
  \item [4.] Le produit de la diff�rence de (-7) et 2 par (-4).
\end{enumerate}
\end{minipage}
\end{tabular}




\bigskip

\ul{Exercice 6}:

\begin{enumerate}
  \item $B=3u-24=3 \times u-24=3 \times (-2)-24=-6-24=-30$
  \item b est un nombre n�gatif. 
   $(-2) \times b$ est le produit de deux nombres
  n�gatifs donc $(-2) \times b$ est positif.
  
  $b^2=b \times b$ est le produit de deux nombres
  n�gatifs donc $b^2=b \times b$ est positif.  
  
  $\dfrac{b^2}{-2b}$ est le quotient de deux nombres positifs, c'est donc un
  nombre positif.
\end{enumerate}



\bigskip

\begin{center}
\fbox{Correction contr�le 3 (sujet 2)}
\end{center}

\ul{Exercice 4}:

\begin{tabular}{llll}
\begin {minipage}{4cm}
$A=-2-3 \times (+4)$

$A=-2-12$

$A=-14$
\end{minipage}
&
\begin {minipage}{4,5cm}
$B=3 \times 6 - 2 \times 7 -6$

$B=18-14-6$

$B=4 - 6$

$B=-2$
\end{minipage}
&
\begin {minipage}{4,5cm}
$C=4 \times (-6+8) -12$

$C=4 \times 2 -12$

$C=8 -12$

$C=-4$
\end{minipage}

\begin {minipage}{5cm}
$D=\dfrac{2-7 \times (-5) + 8}{-5-1+ (-3) \times (-5)}$

\bigskip


$D=\dfrac{2-(-35)+8}{-5-1+15}$

\bigskip


$D=\dfrac{2+35+8}{-6+15}$


\bigskip


$D=\dfrac{45}{9}=5$
\end{minipage}
\end{tabular}

\bigskip


\ul{Exercice 5}:

\enskip

\begin{tabular}{cc}
\begin{minipage}{8cm}
\begin{enumerate}
  \item $(-3) \times (-2)-1$
  \item $\big( 6+(-7) \big) \div \big(1 \times (-2) \big)$ ou
  $\dfrac{6+(-7)}{1 \times (-2)}$
  \end{enumerate}
\end{minipage}
&
\begin{minipage}{10cm}
\begin{enumerate}
  \item [3.] La somme de (-5) et du produit de 3 par (-4).
  \item [4.] Le produit de la diff�rence de (-8) et 3 par (-2).
\end{enumerate}
\end{minipage}
\end{tabular}




\bigskip

\ul{Exercice 6}:

\begin{enumerate}
  \item $B=3u-24=3 \times u-24=3 \times (-2)-24=-6-24=-30$
  \item b est un nombre n�gatif. 
   $(-2) \times b$ est le produit de deux nombres
  n�gatifs donc $(-2) \times b$ est positif.
  
  $b^2=b \times b$ est le produit de deux nombres
  n�gatifs donc $b^2=b \times b$ est positif.  
  
  $\dfrac{b^2}{-2b}$ est le quotient de deux nombres positifs, c'est donc un
  nombre positif.
\end{enumerate}
\end{document}
