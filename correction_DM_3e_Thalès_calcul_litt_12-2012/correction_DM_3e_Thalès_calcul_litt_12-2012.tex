\documentclass[12pt, twoside]{article}
\usepackage[francais]{babel}
\usepackage[T1]{fontenc}
\usepackage[latin1]{inputenc}
\usepackage[left=5mm, right=5mm, top=5mm, bottom=5mm]{geometry}
\usepackage{float}
\usepackage{graphicx}
\usepackage{array}
\usepackage{multirow}
\usepackage{amsmath,amssymb,mathrsfs}
\usepackage{soul}
\usepackage{textcomp}
\usepackage{eurosym}
 \usepackage{variations}
\usepackage{tabvar}

\pagestyle{empty}
\begin{document}

\begin{center}
\fbox{Correction du devoir maison 3}
\end{center}

\enskip


\ul{Exercice 1}:
Les droites (BA) et (DE) sont s�cantes en C. Les droites (BE) et (DA) sont
parall�les. D'apr�s le th�or�me de Thal�s, on a:
$\dfrac{CB}{CA}=\dfrac{CE}{CD}=\dfrac{BE}{DA}$


Donc 
$\dfrac{2,5}{CA}=\dfrac{2,4}{4,8}=\dfrac{BE}{3,6}$. On trouve: \quad 
$CA=2,5 \times 4,8 \div 2,4 = 5$ \quad et \quad $BE=2,4 \times 3,6 \div 4,8=1,8$

\bigskip


\ul{Exercice 2}:


\begin{enumerate}
  \item Dans le triangle ODC, le plus long c�t� est [DO]. On calcule
  s�paremment $DO^2$ et $CO^2+CD^2$.
  
  $DO^2=35^2=1225$ \quad et \quad $CO^2+CD^2=28^2+21^2=784+441=1225$
  
  On constate que $DO^2=CO^2+CD^2$, donc d'apr�s la r�ciproque du th�or�me de
  Pythagore, le triangle ODC est rectangle en C.
  
  \item DCO est un triangle rectangle en O et OAC est un triangle rectangle
  en A. Donc les droites (DC) et (AB) sont toutes les deux perpendiculaires � la
  droite (AC). D'apr�s la propri�t� ``si deux droite sont perpendiculaires �
  une m�me troisi�me droite alors ces droites sont parall�les entre elles'', on
  peut en d�duire que (DC) et (AB) sont parall�les.
  
  \item Les droites (DB) et (AC) sont s�cantes en O. Les droites (DC) et (BA)
  sont parall�les. D'apr�s le th�or�me de Thal�s, on a:
$\dfrac{OD}{OB}=\dfrac{OC}{OA}=\dfrac{DC}{AB}$. 


Donc 
$\dfrac{35}{OB}=\dfrac{28}{42}=\dfrac{21}{AB}$. 
On trouve: \quad 
$OB=35 \times 428 \div 284 = 52,5$ \quad et \quad $BA=21 \times 42 \div
28=31,5$
\end{enumerate}


\bigskip

\ul{Exercice 3}: BREV est un rectangle donc ses c�t�s oppos�s ont m�me longueur
et sont parall�les.


\begin{enumerate}
  \item $TE=VE-VT=13-9,6=3,4$cm
  
  \item BRVE est un rectangle donc le triangle BVT est rectangle en V. D'apr�s
  le th�or�me de Pythagore, on a: $BT^2=BV^2+VT^2=7,2^2+9,6^2=92,16+51,84=144$.
  
  
  La longuer BT est positive donc $BT=\sqrt{144}=12$cm.
  
  \item Les droites (BN) et (VE) sont s�cantes en T. Les droites (BV) et (EN)
  sont parall�les. D'apr�s le th�or�me de Thal�s, on a:
$\dfrac{TB}{TN}=\dfrac{TV}{TE}=\dfrac{BV}{EN}$. 


Donc 
$\dfrac{12}{TN}=\dfrac{9,6}{3,4}=\dfrac{7,2}{EN}$. 
On trouve: \quad 
$TN=12 \times 3,4 \div 9,6 = 4,25$ \quad et \quad $EN=7,2 \times 3,4 \div
9,6=2,55$  
\end{enumerate}


\bigskip

\ul{Exercice 4}:


$A=(9a-3)(8-7a)-(3a-5)=9a \times 8+9a \times (-7a)+(-3) \times 8+(-3) \times
(-7a)-(3a-5)$

 
\qquad \qquad \qquad \qquad \qquad \qquad
\qquad $=72a-63a^2-24+21a-3a+5=-63a^2+90a-19$


\enskip


$B=2+3u(-7+4u^2)+(-8u^3+4)=2+3u \times (-7)+3u \times 4u^2 +
(-8u^3+4)=2-21u+12u^3-8u^3+4=4u^3-21u+6$

\enskip

$C=5y(10-4y)+(-6+8y)(y-7)=5y(10-4y) + (-6) \times y+(-6) \times (-7)+8y \times
y+ 8y \times (-7)$

\quad $=5y \times 10+5y \times
(-4y)-6y+42+8y^2-56y=50y-20y^2-6y+42+8y^2-56y=-12y^2-12y+42$

\enskip

$D=7-(2t+3)(-4t-5)=7-[2t \times (-4t)+2t \times (-5)+3 \times (-4t)+3 \times
(-5)]$

\quad $=7-(-8t^2-10-12t-15)=7+8t^2+10t+12t+15=8t^2+22t+22$

\bigskip


\ul{Exercice 5}: 
\quad $E=20y^2-4y=5\times y \times 4 \times y- 4 \times y \times 1=4y(5y-1)$

\enskip

$F=18-72w=18 \times 1-18 \times 4 \times w=18(1-4w)$

\enskip

$G=-3d^3+21d^2=3 \times d^2 \times (-d)+ 3 \times d^2 \times 7=3d^2(-d+7)$

\enskip

$H=(9z+3)(3z-1)-2(9z+3)=(9z+3)\times (3z-1)-2 \times
(9z+3)=(9z+3)((3z-1)-2)=(9z-3)(3z-3)$

\quad $=(3\times 3z-3\times 1)(3 \times z - 3
\times 1)=3\times (3z-1) \times 3 \times (z-1)=9(3z-1)(z-1)$
\end{document}
