\documentclass{article}

\usepackage[francais]{babel}
\usepackage[T1]{fontenc}
\usepackage[latin1]{inputenc}
\usepackage[left=1cm, right=1cm, top=6mm, bottom=6mm]{geometry}
\usepackage{float}
\usepackage{graphicx}
\usepackage{array}
\usepackage{multirow}
\usepackage{amsmath, amssymb, mathrsfs}

\begin{document}

\begin{flushleft}
\textbf{Groupe 1}

\end{flushleft}
\bigskip

\begin{center}
\textbf{\Large{Correction contr�le de cours 1}}
\end{center}


\bigskip
\textbf{Exercice 1:} $\dfrac{9}{4}=\dfrac{18}{8}$ et $2=\dfrac{16}{8}$ donc
$\dfrac{15}{8}<2<\dfrac{17}{8}<\dfrac{9}{4}$.


\bigskip
\textbf{Exercice 2:} $0,6<1$ donc $(0,6)^{2}<0,6<1$. $\dfrac{10}{6}>1$ donc
$1<\dfrac{10}{6}<\dfrac{10}{6}^{2}$. Finalement, on a:
$(0,6)^{2}<0,6<1<\dfrac{10}{6}<\dfrac{10}{6}^{2}$


\bigskip
\textbf{Exercice 3:} 
  \begin{center}
  \begin{tabular}{|m{10cm}|c|}
 \hline Quand on �l�ve un nombre positif au carr�, on obtient un r�sultat plus 
grand que le nombre de d�part & $\circ$ toujours \quad $\bullet$ parfois \quad
$\circ$ jamais \\[4mm]
 \hline Soit $x$ un nombre r�el. On a: $1+x^{2} \geqslant 1$ & $\bullet$
toujours \quad $\circ$ parfois \quad $\circ$ jamais  \\[2mm]
 \hline  Soit $x$ un r�el v�rifiant l'in�quation $-3x+5 \leqslant 2$ alors $x
 \leqslant 1$ & $\circ$ Vrai \quad $\bullet$ Faux  \\[2mm]
 \hline $\sqrt{8}+3<\sqrt{10}+ \pi$ & $\bullet$ Vrai \quad $\circ$ Faux  \\[2mm]
 \hline
 \end{tabular}
  \end{center}

\bigskip
\bigskip
\bigskip
\bigskip
\bigskip
\bigskip

\begin{flushleft}
\textbf{Groupe 2}

\end{flushleft}
\bigskip

\begin{center}
\textbf{\Large{Correction contr�le de cours 1}}
\end{center}


\bigskip
\textbf{Exercice 1:} $\dfrac{7}{3}=\dfrac{21}{9}$ et $2=\dfrac{18}{9}$ donc
$2<\dfrac{19}{9}<\dfrac{20}{9}<\dfrac{7}{3}$.


\bigskip
\textbf{Exercice 2:} $0,3<1$ donc $(0,3)^{2}<0,3<1$. $\dfrac{10}{3}>1$ donc
$1<\dfrac{10}{3}<\dfrac{10}{3}^{2}$. Finalement, on a:
$(0,3)^{2}<0,3<1<\dfrac{10}{3}<\dfrac{10}{3}^{2}$


\bigskip
\textbf{Exercice 3:} 
 \begin{center}
  \begin{tabular}{|m{10cm}|c|}
  \hline Soit $x$ un nombre r�el. On a: $3+x^{2} \geqslant 3$ & $\bullet$
toujours \quad $\circ$ parfois \quad $\circ$ jamais  \\[2mm]
 \hline Quand on �l�ve un nombre positif au carr�, on obtient un r�sultat plus 
grand que le nombre de d�part & $\circ$ toujours \quad $\bullet$ parfois \quad
$\circ$ jamais \\[4mm]
 
 \hline  Soit $x$ un r�el v�rifiant l'in�quation $-2x+8 \leqslant 4$ alors $x
 \leqslant 2$ & $\circ$ Vrai \quad $\bullet$ Faux  \\[2mm]
 \hline $\sqrt{6}+3,1<\sqrt{7}+ \pi$ & $\bullet$ Vrai \quad $\circ$ Faux 
 \\[2mm]
 \hline
 \end{tabular}
  \end{center}



\end{document}