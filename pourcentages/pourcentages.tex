\documentclass[12pt, twoside]{article}
\usepackage[francais]{babel}
\usepackage[T1]{fontenc}
\usepackage[latin1]{inputenc}
\usepackage[left=1cm, right=1cm, top=1cm, bottom=1cm]{geometry}
\usepackage{float}
\usepackage{graphicx}
\usepackage{array}
\usepackage{multirow}
\usepackage{amsmath,amssymb,mathrsfs}
\usepackage{soul}
\usepackage{textcomp}
\usepackage{eurosym}
 
\pagestyle{empty}
\begin{document}
\section*{\center{Calculs de pourcentages}}

\subsection*{1) Ouvrir et enregister un fichier}

Ouvrir votre fichier excel (dans le menu d�marrer).

\medskip

Pour enregistrer votre fichier:
\begin{itemize}
  \item [$\bullet$] cliquer sur fichier (en haut � gauche)
  \item [$\bullet$]  cliquer sur  enregistrer sous 
  \item [$\bullet$] nommer votre fichier: nom-prenom-2de5-pourcentages (laisser
  la partie .xls derri�re le fichier)
  \item [$\bullet$] cliquer sur poste de travail puis sur votre prenom.nomPeda1
  et enfin sur votre dossier math (qui est d�j� cr�e normalement)
  \item [$\bullet$] enregister
\end{itemize}




\subsection*{2) Augmentation et r�duction}

\subsubsection*{2.1) Augmentation}

Le but est de trouver une formule permettant de calculer la valeur d'un nombre
 r�el apr�s qu'on lui ait appliqu� une augmentation d'un certain
 pourcentage.
 
 \enskip
 

\begin{itemize}
  \item [$\bullet$] Dans la cellule A1, entrer: Pourcentage (en \%)
  \item [$\bullet$] Dans la cellule B1, entrer un nombre entier compris entre 1 et 99
  \item [$\bullet$] Dans la cellule A3, entrer: Valeur initiale
  \item [$\bullet$] Dans la cellule B3, entrer: Augmentation
  \item [$\bullet$] Dans la cellule C3, entrer: Valeur apr�s augmentation
\end{itemize} 


 
\subsubsection*{2.2)R�duction}



\end{document}
