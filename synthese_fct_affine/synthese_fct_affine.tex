\documentclass[12pt, twoside]{article}
\usepackage[francais]{babel}
\usepackage[T1]{fontenc}
\usepackage[latin1]{inputenc}
\usepackage[left=8mm, right=8mm, top=8mm, bottom=8mm]{geometry}
\usepackage{float}
\usepackage{graphicx}
\usepackage{array}
\usepackage{multirow}
\usepackage{amsmath,amssymb,mathrsfs}
\pagestyle{empty}
\begin{document}

\section*{\center{Bilan fonctions affines et �quations de droites}}




\bigskip
\begin{center}
\begin{tabular}{|m{9cm}|m{10cm}|}
\hline
\textbf{Ce que je dois savoir} & \textbf{Ce que je dois savoir faire} \\
\hline

\enskip


\begin{itemize}
  \item[$\bullet$] Je sais reconna�tre l'expression d'une fonction affine
  \item [$\bullet$] Je connais le sens de variation et le signe d'une fonction
  affine
  \item [$\bullet$] Je connais la propri�t� sur la proportionnalit� entre
  l'accroissement de l'image et l'accroissement de la variable
  \item [$\bullet$] Je connais l'�quation d'une droite parall�le � l'axe des
  ordonn�es
  \item [$\bullet$] Je connais la caract�risation des droites parall�les
\end{itemize}
&

\enskip
\begin{itemize}
  \item[$\bullet$] Je sais reconna�tre le graphe d'une fonction affine
  \item[$\bullet$] Je sais d�terminer l'expression d'une fonction affine
  graphiquement
  \item[$\bullet$] Je sais calculer l'expression d'une fonction affine 
 \item[$\bullet$] Je sais tracer la repr�sentation graphique d'une fonction
 affine � partir du coefficient directeur de la droite et un point
 appartenant � cette droite
  \item[$\bullet$] Je sais d�terminer une �quation de droite
  \item[$\bullet$] Je sais r�soudre un syst�me � deux �quations et deux
  inconnues graphiquement et par le calcul
 
 \end{itemize} \\
\hline

\end{tabular}
\end{center}

\bigskip

\section*{\center{Bilan fonctions affines et �quations de droites}}




\bigskip
\begin{center}
\begin{tabular}{|m{9cm}|m{10cm}|}
\hline
\textbf{Ce que je dois savoir} & \textbf{Ce que je dois savoir faire} \\
\hline

\enskip


\begin{itemize}
  \item[$\bullet$] Je sais reconna�tre l'expression d'une fonction affine
  \item [$\bullet$] Je connais le sens de variation et le signe d'une fonction
  affine
  \item [$\bullet$] Je connais la propri�t� sur la proportionnalit� entre
  l'accroissement de l'image et l'accroissement de la variable
  \item [$\bullet$] Je connais l'�quation d'une droite parall�le � l'axe des
  ordonn�es
  \item [$\bullet$] Je connais la caract�risation des droites parall�les
\end{itemize}
&

\enskip
\begin{itemize}
  \item[$\bullet$] Je sais reconna�tre le graphe d'une fonction affine
  \item[$\bullet$] Je sais d�terminer l'expression d'une fonction affine
  graphiquement
  \item[$\bullet$] Je sais calculer l'expression d'une fonction affine 
 \item[$\bullet$] Je sais tracer la repr�sentation graphique d'une fonction
 affine � partir du coefficient directeur de la droite et un point
 appartenant � cette droite
  \item[$\bullet$] Je sais d�terminer une �quation de droite
  \item[$\bullet$] Je sais r�soudre un syst�me � deux �quations et deux
  inconnues graphiquement et par le calcul
 
 \end{itemize} \\
\hline

\end{tabular}
\end{center}

\end{document}
