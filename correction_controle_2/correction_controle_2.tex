%%This is a very basic article template.
%%There is just one section and two subsections.
\documentclass{article}
\usepackage[francais]{babel}
\usepackage[T1]{fontenc}
\usepackage[latin1]{inputenc}
\usepackage[left=1cm, right=1cm, top=1cm, bottom=1cm]{geometry}
\usepackage{float}
\usepackage{graphicx}
\usepackage{array}
\usepackage{multirow}
\usepackage{amsmath,amssymb,mathrsfs}
\usepackage{soul}
\usepackage[dvipsnames]{xcolor}
\pagestyle{empty}
\begin{document} 

\section*{\center{Correction du contr�le $2$}}

\subsection*{Exercice 4}

$x$ et $y$ $2$ r�els tels que : $2\leqslant x \leqslant4$ et
$-1\leqslant y \leqslant5$


$\bullet$ encadrons $x+y$: 
\[ + 
\left.
\begin{aligned} 
	2\leqslant \ x\  \leqslant 4\\
	-1\leqslant \ y\  \leqslant 5\\
	\hline
\end{aligned}
\right. \]
\[
\begin{aligned}
	1\leqslant x+y \leqslant 9
\end{aligned}
\]


\bigskip
$\bullet$ encadrons $y-x$:\qquad $-1\leqslant y \leqslant5$
\smallskip



\fbox{Je ne peux pas faire la soustraction directement.} J'�cris donc
$y-x=y+(-x)$ et j'encadre $y$ puis $(-x)$ et je peux ensuite \fbox{additionner}
les encadrements de $y$ et de $(-x)$.


\begin{minipage}{7cm}
\begin{align*}
2 & \leqslant x \leqslant 4 \\
-2 & \geqslant -x \geqslant -4 \\
-4 & \leqslant -x \leqslant -2 \\
\end{align*}
\end{minipage}
\begin{minipage}{7cm}
\[ + 
\left.
\begin{aligned} 
	-1\leqslant y\  \leqslant 5\\
	-4\leqslant \ -x\  \leqslant -2\\
	\hline
\end{aligned}
\right. \]
\[
\begin{aligned}
	-5\leqslant y+(-x) \leqslant 3 \\
	-5\leqslant y-x \leqslant 3 
\end{aligned}
\]
\end{minipage}


\bigskip
$\bullet$ encadrons $\dfrac{x^2-y}{4}$ :\quad J'encadre $x^2$ (je peux car
$2\leqslant x \leqslant4$ donc on a QUE des nombres positifs).


$2^2 \leqslant x^2 \leqslant 4^2$ \textbf{car $0<2\leqslant x \leqslant 4$}


$x^2-y = x^2+(-y)$ donc j'encadre $-y$


\begin{minipage}{7cm}
\begin{align*}
-1 & \leqslant y \leqslant 5 \\
1 & \geqslant -y \geqslant -5 \\
-5 & \leqslant -y \leqslant 1 \\
\end{align*}
\end{minipage}
\begin{minipage}{7cm}
\[ + 
\left.
\begin{aligned} 
	4\leqslant x^2\  \leqslant 16\\
	-5\leqslant \ -y\  \leqslant 1\\
	\hline
\end{aligned}
\right. \]
\[
\begin{aligned}
	-1\leqslant x^2+(-y) \leqslant 17 \\
	-1\leqslant x^2-y \leqslant 17 
\end{aligned}
\]
\end{minipage}


$\dfrac{1}{4}>0$ donc $\dfrac{-1}{4} \leqslant \dfrac{x^2-y}{4} \leqslant
\dfrac{17}{4}$


\bigskip

$\bullet$ Autre exemple : Soit $x$ et $y$ deux r�els tesl que: $3\leqslant x
\leqslant 5$ et $6\leqslant y \leqslant 7$. Encadrer $\dfrac{x}{y}+8$

\fbox{Je ne peut pas faire la division directement.}

J'�cris donc $\dfrac{x}{y}=x\times\dfrac{1}{y}$ et j'encadre $x$ puis
$\dfrac{1}{y}$. Je peux ensuite \fbox{multiplier} les encadrements de $x$ et
$\dfrac{1}{y}$ \ul{CAR TOUS LES NOMBRES SONT POSITIFS}.


\begin{minipage}{7cm}
\begin{align*}
3 & \leqslant x \leqslant 5 \\
6 & \leqslant y \leqslant 7 \\
\dfrac{1}{6} & \geqslant \dfrac{1}{y} \geqslant  \dfrac{1}{7} \\
\dfrac{1}{7} & \leqslant \dfrac{1}{y} \leqslant  \dfrac{1}{6} \\
\end{align*}
\end{minipage}
\begin{minipage}{7cm}
\[ \times 
\left.
\begin{aligned} 
	3\leqslant \ x\  \leqslant 5\\
	\dfrac{1}{7}\leqslant \ \dfrac{1}{y}\  \leqslant \dfrac{1}{6}\\
	\hline
\end{aligned}
\right. \]
\[
\begin{aligned}
	\dfrac{3}{7}\leqslant \dfrac{x}{y} \leqslant \dfrac{5}{6}
\end{aligned}
\]
\end{minipage}

$\dfrac{3}{7}+8 \leqslant \dfrac{x}{y}+8 \leqslant \dfrac{5}{6}+8$.

\subsection*{Exercice 5}
1) $A=\dfrac{3+\pi}{3}$ \qquad  $B=\dfrac{4+\pi}{4}$ 

\medskip


J'utilise la m�thode de la diff�rence (je calcule $A-B$ et je regarde le
signe, si $A-B>0$ alors $A>B$, si $A-B<0$ alors $A<B$).


$A-B=\dfrac{3+\pi}{3}-\dfrac{4+\pi}{4}=\dfrac{(3+\pi)\times
\color{red}{4}}{12}$$-\dfrac{(4+\pi)\times\color{red}{3}}{12}=\dfrac{\color{red}{12+4\pi
-(12+3\pi)}}{12}=\begin{color}{red}\dfrac{\pi}{12}>\end{color}0$

\smallskip 
Donc $A \begin{color}{red}>\end{color} B$


\medskip


\ul{Autre m�thode possible}: J'utilise la m�thode du quotient et je compare
� 1 : si $\dfrac{A}{B}<1$ alors $A \begin{color}{red}<\end{color} B$ et si 
$\dfrac{A}{B}>1$ alors $A \begin{color}{red}>\end{color} B$.

\medskip

$\dfrac{A}{B} = \dfrac{\dfrac{3+\pi}{3}}{\dfrac{4+\pi}{4}} =
\dfrac{3+\pi}{3} \times
\begin{color}{red}\dfrac{4}{4+\pi} =
\dfrac{12+4\pi}{12+3\pi} >\end{color}1$ \qquad donc $A \begin{color}{red}>\end{color} B$.
\begin{color}{red}
car $12+3\pi<12+4\pi$
\end{color}


\bigskip
2) $C=\sqrt{4-2\sqrt{3}}$ et $D=\sqrt{3}-1$


$C>0$ et $D>0$ donc je peux comparer leurs carr�s.

\smallskip
$C^2=\begin{color}{red}4-2\sqrt{3}\end{color}$ \ \ $D^2=(\sqrt{3}-1)^2 =
\begin{color}{red}3-2\sqrt{3}+1 = 4-2\sqrt{3}\end{color}$ \qquad donc $C^2
\begin{color}{red}=\end{color} D^2$ et $C \begin{color}{red}=\end{color} D$


\bigskip
3)Soit $a>0$ et $b>0$; $E=\dfrac{a}{a+b}$ et $F=\dfrac{a-b}{a}$
\smallskip

J'utilise la m�thode de la diff�rence.
\smallskip

$E-F = \dfrac{a}{a+b}-\dfrac{a-b}{a}=\dfrac{a\times
\begin{color}{red}a\end{color} -(a-b)
\begin{color}{red}(a+b)\end{color}}{a(a+b)} =
\dfrac{\color{red}{a^2-(a^2-b^2)}}{a(a+b)}\begin{color}{red}=\dfrac{b^2}{a(a+b)}\end{color}$
\smallskip


$a>0$ et $b>0$ donc $a+b \begin{color}{red}>\end{color} 0$ et
$a(a+b)\begin{color}{red}>\end{color} 0$. Un carr� est toujours positif ou nul
donc $b^{2} \begin{color}{red}>\end{color} 0$


J'en d�duis $E-F = \dfrac{b^2}{a(a+b)}\ \begin{color}{red}>\end{color} 0$ d'o� $E \begin{color}{red}>\end{color} F$


\subsection*{Exercice 6}
On note $\mathcal{A}$ l'aire du rectangle, $P$ son p�rim�tre,
$l$ sa largeur et $L$ sa longueur.


On a: $51\leqslant \mathcal{A} \leqslant 52$ et $3\leqslant l
\leqslant 6$.


Or $\mathcal{A}=L\times l$ donc $L=\dfrac{\mathcal{A}}{\color{red}{l}} =
\mathcal{A}\times\color{red}{\dfrac{1}{l}}$


J'encadre donc $\dfrac{1}{l}$


\begin{align*}
3 & \leqslant l \leqslant 6 \\
\color{red}{\dfrac{1}{3}} & \geqslant \dfrac{1}{l} \geqslant
\color{red}{\dfrac{1}{6}} \\ \color{red}{\dfrac{1}{6}} & \leqslant \dfrac{1}{l} \leqslant \color{red}{\dfrac{1}{3}} \\
\end{align*}


\textbf{$0<51\leqslant\mathcal{A}\leqslant52$} donc
$\begin{color}{red}51\times\dfrac{1}{6}\end{color}\leqslant
\mathcal{A} \times \dfrac{1}{l} \leqslant \color{red}{52\times\dfrac{1}{3}}$

d'o� $\begin{color}{red}\dfrac{51}{6}\end{color}\leqslant L \leqslant
\color{red}{\dfrac{52}{3}}$.

\medskip


La formule du p�rim�tre est \fbox{$P=\color{red}{2(L+l)}$} d'o� : \qquad
$2\times(\begin{color}{red}\dfrac{51}{6}+3\end{color})\leqslant P\leqslant
2\times(\begin{color}{red}\dfrac{52}{3}+6\end{color})$

\medskip

$\begin{color}{red}\dfrac{69}{6}\end{color}\leqslant
P\leqslant \begin{color}{red}\dfrac{70}{3}\end{color}$
\end{document}
