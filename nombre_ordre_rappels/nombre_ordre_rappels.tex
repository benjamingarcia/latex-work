\documentclass[12pt, twoside]{article}

\usepackage[francais]{babel}
\usepackage[T1]{fontenc}
\usepackage[latin1]{inputenc}
\usepackage[left=1cm, right=1cm, top=1cm, bottom=1cm]{geometry}
\usepackage{float}
\usepackage{graphicx}
\usepackage{array}
\usepackage{multirow}
\usepackage{amsmath,amssymb,mathrsfs}

\begin{document}
 

\section*{\center{Nombres et ordres}}

\textbf{D�finition}: Soit $a$ et $b$ deux r�els. $a>b$ ou ($b<a$) signifie que
$a-b>0$. De m�me, $a \geqslant b $ ou ($b \leqslant a$) signifie qua $a-b
\geqslant 0$.

\bigskip
\textbf{Propri�t�}:
Soit $a$ et $b$ deux r�els avec $a>b$, $a>0$ et $b>0$ alors $\dfrac{1}{a} <
\dfrac{1}{b}$.

\bigskip
\textit{Exemples:}$\dfrac{1}{4}< \dfrac{1}{2}$ \qquad
$\dfrac{1}{\pi}<\dfrac{1}{3}$.
\bigskip

\textbf{Propri�t�s}: 
\begin{enumerate}
  \item Soit $a,b,c$ trois r�els avec $a>b$ alors $a+c > b+c$.
  \item Soit $a,b,c$ trois r�els avec $a>b$ alors $a-c > b-c$.
\end{enumerate}

\medskip

Quand on ajoute ou retranche un m�me nombre � chaque membre d'une in�galit�, on
ne change pas le sens d'une in�galit�.

\bigskip

\textit{Exemples}: $2,7+10^{25}<2,76+10^{25}$\qquad
$3-\sqrt{7}<3-\sqrt{7}$.

\bigskip

\textbf{Propri�t�s}: 
\begin{enumerate}
  \item Soit $a,b,c$ trois r�els avec $a>b$  et $c>0$ alors $ac> bc$ et
$\dfrac{a}{c} > \dfrac{b}{c}$.\\
Quand on multiplie ou divise chaque membre d'une in�galit� par un m�me nombre
strictement POSITIF, on ne change pas le sens de l'in�galit�.\\
\textit{Exemples}: $-2 \times 3>-1 \times 3$ \qquad $\dfrac{7}{2}>
\dfrac{3}{2}$.

\bigskip

  \item Soit $a,b,c$ trois r�els avec $a>b$  et $c<0$ alors $ac< bc$ et
$\dfrac{a}{c}< \dfrac{b}{c}$.\\
Quand on multiplie ou divise chaque membre d'une in�galit� par un m�me nombre
strictement NEGATIF, on change le sens de l'in�galit�.\\
\textit{Exemple}: $\dfrac{1}{-4}> \dfrac{3}{-4}$ 
\end{enumerate}

\end{document}
