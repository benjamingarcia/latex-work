\documentclass[12pt, twoside]{article}
\usepackage[francais]{babel}
\usepackage[T1]{fontenc}
\usepackage[latin1]{inputenc}
\usepackage[left=5mm, right=5mm, top=5mm, bottom=5mm]{geometry}
\usepackage{float}
\usepackage{graphicx}
\usepackage{array}
\usepackage{multirow}
\usepackage{amsmath,amssymb,mathrsfs}
\usepackage{soul}
\usepackage{textcomp}
\usepackage{eurosym}
 \usepackage{variations}
\usepackage{tabvar}


\pagestyle{empty}

\begin{document}


\section*{\center{Devoir maison 4}}


\bigskip


 


\fbox{

\begin{minipage}{18cm}
\textit{Devoir � rendre sur feuille grand format petits
carreaux pour le \textbf{mercredi 25 f�vrier 2015}.}
\end{minipage}
}

\enskip


 \textit{Remarque: Chaque r�ponse devra �tre justifi�e. \textbf{Un r�sultat non
 justifi� donnera 0 point.}}


\bigskip




\ul{Exercice 1}: \textit{(3 points)}

\enskip

Barbara a 30 \euro. Elle d�pense le tiers pour l'achat d'un CD puis
$\dfrac{3}{4}$ de ce qu'il reste pour l'achat d'un livre. Combien lui reste-t-il apr�s ses
deux achats? Ecrivez tous vos calculs.


\bigskip

\bigskip

\ul{Exercice 2}: (\textit{4,5 points})

\enskip

Calculer et donner le r�sultat sous forme de \textbf{fraction irr�ductible}.
Chaque �tape sera marqu�e.

\enskip

$A=\dfrac{1}{10}+\dfrac{5}{4}-\dfrac{5}{12}$ \qquad \qquad \qquad
\qquad $B=\dfrac{34}{15} \times \dfrac{-60}{17} \times \dfrac{3}{4}$ \qquad
\qquad \qquad \qquad $C=-\dfrac{12}{15} \div \dfrac{16}{5}$


\bigskip


\bigskip


\ul{Exercice 3}: \textit{(3,5 points)}

\enskip

Lors d'un triathlon (�preuve sportive de natation, cyclisme et course � pied)
d'une distance totale de 50km:

$\dfrac{1}{25}$ du parcours est effectu� � la nage et 26 \% � pied.

\begin{enumerate}
  \item Quelle est la distance effectu�e � v�lo? Justifier votre r�ponse.
  \item Quelle fraction de la distance totale cela repr�sente-t-il? Justifier
  votre r�ponse.
\end{enumerate}

\bigskip


\bigskip

\ul{Exercice 4}: \textit{(5 points)}

ABC est un triangle dont les dimensions sont: AB=1dm, BC=$\dfrac{4}{3}$dm et
AC=$\dfrac{5}{3}$dm.

\begin{enumerate}
  \item Quel est le c�t� le plus long? Justifier votre r�ponse.
  \item D�montrer que le triangle est rectangle en B.
  \item Calculer l'aire et le p�rim�tre de ce triangle. (Cette question peut se
  faire en admettant la question pr�c�dente.)
\end{enumerate}

\bigskip



\bigskip

\ul{Exercice 5}: \textit{(4 points)}

\enskip

\textit{Extrait du brevet}

\begin{enumerate}
  \item Soit $D=\dfrac{8}{3}-\dfrac{5}{3} \div \dfrac{20}{21}$. 
  
  \enskip
  
  Calculer D en
  d�taillant les �tapes du calcul et �crire le r�sultat sous la forme d'une
  fraction irr�ductible.
  \item Effectuer le calcul suivant en d�taillant les �tapes du calcul:
  $E=(2+\dfrac{2}{3}) \div (\dfrac{4}{5}-\dfrac{2}{3})$. 
  
  Le r�sultat sera donn� sous la forme d'un
  entier.
\end{enumerate}




\bigskip


\bigskip


\ul{Exercice BONUS}: \textit{(2 points)}

\enskip

$F=\dfrac{\quad \dfrac{1}{3}+\dfrac{3}{4}\quad }{\quad
\dfrac{5}{4}-\dfrac{7}{3}\quad }$. 
D�montrer que le nombre F
 est un nombre entier.
\end{document}
