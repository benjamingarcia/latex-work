\documentclass[12pt, twoside]{article}
\usepackage[francais]{babel}
\usepackage[T1]{fontenc}
\usepackage[latin1]{inputenc}
\usepackage[left=7mm, right=7mm, top=7mm, bottom=7mm]{geometry}
\usepackage{float}
\usepackage{graphicx}
\usepackage{array}
\usepackage{multirow}
\usepackage{amsmath,amssymb,mathrsfs}
\usepackage{soul}
\usepackage{textcomp}
\usepackage{eurosym}
 \usepackage{variations}
\usepackage{tabvar}

\pagestyle{empty}

\begin{document}

\begin{flushleft}
NOM PRENOM: \ldots \ldots \ldots \ldots \ldots \ldots \ldots \ldots \ldots
 
\bigskip

\end{flushleft}

\begin{center}
{\fbox{$4^{e}3$ \qquad \qquad \textbf{\Large{Contr�le de cours 4 (sujet 1)}}
\qquad \qquad 22/01/2010}}
\end{center}



\bigskip 


\ul{Exercice 1}: a, b, c, d et k sont des nombres quelconques. Compl�ter les
deux formules suivantes (on pourra faire appara�tre les fl�ches pour s'aider.

\enskip


k(a+b)= \ldots \ldots \ldots \ldots \ldots \ldots \ldots \ldots \ldots \ldots
\qquad \qquad \qquad (a+b)(c+d)= \ldots \ldots \ldots \ldots \ldots \ldots
\ldots \ldots \ldots \ldots 


\bigskip


\bigskip


\ul{Exercice 2}: Supprimer les parenth�ses et r�duire.

\begin{center}
\begin{tabular}{c|c}
$A=(x+3)-(x+5)-(-2x-7)$ \qquad  \qquad  \qquad \qquad & \qquad  $B=
-4x+x^2-(6+5x^2)+3x+(10-8x^2)+2x$\\

\quad & \quad \\

\quad & \quad \\

\quad & \quad \\
\quad & \quad \\
\quad & \quad \\
\quad & \quad \\
\quad & \quad \\

\quad & \quad \\

\quad & \quad \\

\end{tabular}
\end{center}

\bigskip

\bigskip




\ul{Exercice 3}: D�velopper et r�duire.

\begin{center}
\begin{tabular}{c|c}
$C=3(a-2)+5(3-2a)$ \qquad  \qquad  \qquad \qquad & \qquad  $D=2x(-x+5)-x(1-x)$\\

\quad & \quad \\

\quad & \quad \\

\quad & \quad \\
\quad & \quad \\
\quad & \quad \\
\quad & \quad \\
\quad & \quad \\
\quad & \quad \\
\quad & \quad \\
\quad & \quad \\
\quad & \quad \\

$E=(y+3)(y-2)$ \qquad  \qquad  \qquad \qquad \qquad \qquad & \qquad 
$F=(3x-2)(5x+1)$\\

\quad & \quad \\

\quad & \quad \\

\quad & \quad \\

\quad & \quad \\

\quad & \quad \\
\quad & \quad \\
\quad & \quad \\
\quad & \quad \\
\quad & \quad \\
\quad & \quad \\
\quad & \quad \\
\end{tabular}
\end{center}


\pagebreak


\begin{center}
{\fbox{$4^{e}3$ \qquad \qquad \textbf{\Large{Contr�le de cours 4 (sujet 2)}}
\qquad \qquad 22/01/2010}}
\end{center}



\bigskip


\ul{Exercice 1}: Supprimer les parenth�ses et r�duire.

\begin{center}
\begin{tabular}{c|c}
$A=(y+7)-(2y-3)-(-y-4)$ \qquad  \qquad  \qquad \qquad & \qquad  $B=
-2x+3x^2-(4+5x^2)+x+(12-7x^2)+4x$\\

\quad & \quad \\

\quad & \quad \\

\quad & \quad \\
\quad & \quad \\
\quad & \quad \\
\quad & \quad \\
\quad & \quad \\

\quad & \quad \\

\quad & \quad \\

\end{tabular}
\end{center}

\bigskip

\bigskip

\ul{Exercice 2}: a, b, c, d et k sont des nombres quelconques. Compl�ter les
deux formules suivantes (on pourra faire appara�tre les fl�ches pour s'aider.

\enskip


k(a+b)= \ldots \ldots \ldots \ldots \ldots \ldots \ldots \ldots \ldots \ldots
\qquad \qquad \qquad (a+b)(c+d)= \ldots \ldots \ldots \ldots \ldots \ldots
\ldots \ldots \ldots \ldots 


\bigskip


\bigskip


\ul{Exercice 3}: D�velopper et r�duire.

\begin{center}
\begin{tabular}{c|c}
$C=4(b-3)+6(2-3b)$ \qquad  \qquad  \qquad \qquad & \qquad 
$D=3x(-2x+6)-x(2-x)$\\

\quad & \quad \\

\quad & \quad \\

\quad & \quad \\
\quad & \quad \\
\quad & \quad \\
\quad & \quad \\
\quad & \quad \\
\quad & \quad \\
\quad & \quad \\
\quad & \quad \\
\quad & \quad \\

$E=(y+5)(2y-3)$ \qquad  \qquad  \qquad \qquad \qquad \qquad & \qquad 
$F=(4x-1)(3x+2)$\\

\quad & \quad \\

\quad & \quad \\

\quad & \quad \\

\quad & \quad \\

\quad & \quad \\
\quad & \quad \\
\quad & \quad \\
\quad & \quad \\
\quad & \quad \\
\quad & \quad \\
\quad & \quad \\
\end{tabular}
\end{center}


\pagebreak



\begin{flushleft}
NOM PRENOM: \ldots \ldots \ldots \ldots \ldots \ldots \ldots \ldots \ldots
 
\bigskip

\end{flushleft}

\begin{center}
{\fbox{$4^{e}3$ \qquad \qquad \textbf{\Large{Contr�le de cours 6 (sujet 1)}}
\qquad \qquad 03/02/2011}}
\end{center}



\bigskip 


\ul{Exercice 1}: a, b, c, d et k sont des nombres quelconques. Compl�ter les
deux formules suivantes.

\enskip


k(a+b)= \ldots \ldots \ldots \ldots \ldots \ldots \ldots \ldots \ldots \ldots
\qquad \qquad \qquad (a+b)(c+d)= \ldots \ldots \ldots \ldots \ldots \ldots
\ldots \ldots \ldots \ldots 


\bigskip


\bigskip


\ul{Exercice 2}: R�duire si possible.

\enskip

3a - 5a = 

\enskip

8z+7 = 

\enskip

4y $\times$ (-3)y = 

\enskip


9$t^2$ + 6$t$ -3 + 2$t$ = 

\bigskip

\bigskip




\ul{Exercice 3}: D�velopper.

\begin{center}
\begin{tabular}{c|c}

$E=(y+3)(y-2)$ \qquad  \qquad  \qquad \qquad \qquad \qquad & \qquad 
$F=(3u-2)(5u+1)$\\

\quad & \quad \\

\quad & \quad \\

\quad & \quad \\

\quad & \quad \\

\quad & \quad \\
\quad & \quad \\
\quad & \quad \\
\quad & \quad \\
\quad & \quad \\
\quad & \quad \\
\quad & \quad \\
\end{tabular}
\end{center}


\bigskip

\bigskip


\ul{Exercice 4}: R�duire l'expression suivante:

\enskip

A=3(a - 2) + (3 - 4a)(a + 4)



\bigskip


\bigskip


\bigskip


\bigskip


\bigskip


\pagebreak


\begin{flushleft}
NOM PRENOM: \ldots \ldots \ldots \ldots \ldots \ldots \ldots \ldots \ldots
 
\bigskip

\end{flushleft}

\begin{center}
{\fbox{$4^{e}3$ \qquad \qquad \textbf{\Large{Contr�le de cours 6 (sujet 2)}}
\qquad \qquad 03/02/2011}}
\end{center}



\bigskip 


\ul{Exercice 1}: a, b, c, d et k sont des nombres quelconques. Compl�ter les
deux formules suivantes.

\enskip


k(a+b)= \ldots \ldots \ldots \ldots \ldots \ldots \ldots \ldots \ldots \ldots
\qquad \qquad \qquad (a+b)(c+d)= \ldots \ldots \ldots \ldots \ldots \ldots
\ldots \ldots \ldots \ldots 


\bigskip


\bigskip


\ul{Exercice 2}: R�duire si possible.

\enskip

4a + 8 = 

\enskip


5u - 9u =

\enskip

5b $\times$ (-2)b = 

\enskip


5$z^2$ + 3$z$ -2 + 7$z$ = 

\bigskip

\bigskip




\ul{Exercice 3}: D�velopper.

\begin{center}
\begin{tabular}{c|c}

$E=(t+6)(t-3)$ \qquad  \qquad  \qquad \qquad \qquad \qquad & \qquad 
$F=(3a-2)(4a+5)$\\

\quad & \quad \\

\quad & \quad \\

\quad & \quad \\

\quad & \quad \\

\quad & \quad \\
\quad & \quad \\
\quad & \quad \\
\quad & \quad \\
\quad & \quad \\
\quad & \quad \\
\quad & \quad \\
\end{tabular}
\end{center}


\bigskip

\bigskip


\ul{Exercice 4}: R�duire l'expression suivante:

\enskip

A=4(b - 1) + (5 - 3b)(1 + 2b)



\bigskip


\bigskip


\bigskip


\bigskip


\bigskip

\end{document}
