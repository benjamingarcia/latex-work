%%This is a very basic article template.
%%There is just one section and two subsections.
\documentclass{article}

\usepackage[francais]{babel}
\usepackage[T1]{fontenc}
\usepackage[latin1]{inputenc}
\usepackage[left=2cm, right=2cm, top=2cm, bottom=2cm]{geometry}
\usepackage{float}
\usepackage{graphicx}
\usepackage{array}
\usepackage{multirow}
\usepackage{textcomp}
\usepackage{amsmath, amssymb, mathrsfs}

\begin{document}


\section*{Correction td racine}

\subsection*{Exercice 3}
1)$$2\sqrt{3}-4\sqrt{3}+\sqrt{27} = -2\sqrt{3}+\sqrt{9\times3} =
-2\sqrt{3}+3\sqrt{3} = \sqrt{3}$$
 

2)$$\sqrt{45}+3\sqrt{5}-3\sqrt{20} = \sqrt{9\times5}+3\sqrt{5}-3\sqrt{4\times5} =
3\sqrt{5}+3\sqrt{5}-3\times2\sqrt{5} = 0$$


3)$$
  \dfrac{3\sqrt{5}\times\sqrt{12}}{2\sqrt{15}} =
  \dfrac{3\sqrt{5}\times\sqrt{3\times4}}{2\sqrt{3\times5}} =
  \dfrac{3\sqrt{5}\times2\sqrt{3}}{2\sqrt{3}\times\sqrt{5}} = 3$$


4)$$(\sqrt{3})^2-(\dfrac{2\sqrt{2}}{3})^2 = 3-\dfrac{(2\sqrt{2})^2}{3^2} =
3-\dfrac{2\sqrt{2}\times2\sqrt{2}}{9} = 3-\dfrac{4\times2}{9} =
\dfrac{27}{9}-\dfrac{8}{9} = \dfrac{19}{9}$$

\subsection*{Exercice 4}
1)$$(4-\sqrt{2})^2 = 4^2 - 2\times4\times\sqrt{2}+(\sqrt{2})^2 = 16-8\sqrt{2}+2 = 18-8\sqrt{2}$$

2)$$(3\sqrt{2}-\sqrt{3})^2 = 
   (3\sqrt{2})^2-2\times3\sqrt{2}\times\sqrt{3}+(\sqrt{3})^2 =
   9\times2-6\sqrt{6}+3 = 21-6\sqrt{6}$$

3)$$(4+2\sqrt{7})(6+\sqrt{7}) =
   24+4\sqrt{7}+12\sqrt{7}+2\sqrt{7}\times\sqrt{7} = 38+16\sqrt{7}$$


4)$$(3+\sqrt{5})(3-\sqrt{5})+(1-\sqrt{2})^2 = 9-(\sqrt{5})^2+1-2\sqrt{2}+2 =
7-2\sqrt{2}$$


 \subsection*{Exercice 5}
1)$$\sqrt{3^4\times5^2} = \sqrt{3^2\times3^2\times5^2} =3 \times 3 \times 5=45$$


2)$$\sqrt{9\times10^6} = \sqrt{3^2\times(10^3)^2} = 3\times10^3=3000$$


3)$$\sqrt{16\times10^{-8}} = \sqrt{\dfrac{16}{10^8}}=
\dfrac{\sqrt{16}}{\sqrt{10^8}} = \dfrac{4}{\sqrt{(10^4)^2}} = \dfrac{4}{10^4} 
= 4\times10^{-4}$$


4)$$\sqrt{8,1\times10^9} = \sqrt{81\times10^8} = \sqrt{9^2\times(10^4)^2} =
9\times10^4$$


\subsection*{Exercice 6}
$$\dfrac{1}{\sqrt{2}-1}-\dfrac{1}{\sqrt{2}+1} =
\dfrac{1\times(\sqrt{2}+1)}{(\sqrt{2}-1)(\sqrt{2}+1)}-\dfrac{1\times(\sqrt{2}-1)}{(\sqrt{2}+1)(\sqrt{2}-1)}
= \dfrac{\sqrt{2}+1-(\sqrt{2}-1)}{(\sqrt{2}-1)(\sqrt{2}+1)}=
\frac{\sqrt{2}+1-\sqrt{2}+1}{(\sqrt{2}-1)(\sqrt{2}+1)}= \dfrac{2}{2-1} = 2$$

Donc $\dfrac{1}{\sqrt{2}-1}-\dfrac{1}{\sqrt{2}+1}$ est un entier naturel.
\end{document}
