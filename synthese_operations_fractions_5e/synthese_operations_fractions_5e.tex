\documentclass[12pt, twoside]{article}
\usepackage[francais]{babel}
\usepackage[T1]{fontenc}
\usepackage[latin1]{inputenc}
\usepackage[left=7mm, right=7mm, top=7mm, bottom=7mm]{geometry}
\usepackage{float}
\usepackage{graphicx}
\usepackage{array}
\usepackage{multirow}
\usepackage{amsmath,amssymb,mathrsfs}
\pagestyle{empty}
\begin{document}

\section*{\center{Bilan op�rations sur les fractions}}




\bigskip

\begin{center}
\begin{tabular}{|m{9cm}|m{9cm}|}
\hline


\textbf{Ce que je dois savoir} & \textbf{Ce que je dois savoir faire} \\

\hline

\enskip


\begin{itemize}
  \item [$\bullet$] Je connais la r�gle pour additionner et soustraire deux
  nombres en �criture fractionnaire.
  
  \item [$\bullet$] Je sais r�duire au m�me d�nominateur.
  
  \item [$\bullet$] Je sais simplifier une fraction.
  
  \item [$\bullet$] Je connais la m�thode pour multiplier deux nombres en
  �criture fractionnaire.
 
 \item [$\bullet$] Je sais �crire un nombre entier sous forme de fraction.

 \end{itemize}

&

  \enskip

\begin{itemize}
  
  

   
  
  \item[$\bullet$] Je sais additionner et soustraire des
  nombres en �criture fractionnaire de m�me d�nominateur.


  \item[$\bullet$] Je sais additionner et soustraire des
  nombres en �criture fractionnaire ayant des  d�nominateurs 
  diff�rents.

  
      
  \item[$\bullet$] Je sais multiplier des
  nombres en �criture fractionnaire.
  
  \item[$\bullet$] Je sais simplifier un produit ``au fur et � mesure''.
  
  \item[$\bullet$] Je sais effectuer un calcul en respectant les r�gles
  de priorit�s.
 
\end{itemize} \\

\hline

\end{tabular}
\end{center}


\bigskip

\bigskip

\bigskip

\bigskip

\section*{\center{Bilan op�rations sur les fractions}}




\bigskip

\begin{center}
\begin{tabular}{|m{9cm}|m{9cm}|}
\hline


\textbf{Ce que je dois savoir} & \textbf{Ce que je dois savoir faire} \\

\hline

\enskip


\begin{itemize}
  \item [$\bullet$] Je connais la r�gle pour additionner et soustraire deux
  nombres en �criture fractionnaire.
  
  \item [$\bullet$] Je sais r�duire au m�me d�nominateur.
  
  \item [$\bullet$] Je sais simplifier une fraction.
  
  \item [$\bullet$] Je connais la m�thode pour multiplier deux nombres en
  �criture fractionnaire.
 
 \item [$\bullet$] Je sais �crire un nombre entier sous forme de fraction.

 \end{itemize}

&

  \enskip

\begin{itemize}
  
  

   
  
  \item[$\bullet$] Je sais additionner et soustraire des
  nombres en �criture fractionnaire de m�me d�nominateur.


  \item[$\bullet$] Je sais additionner et soustraire des
  nombres en �criture fractionnaire ayant des  d�nominateurs 
  diff�rents.

  
      
  \item[$\bullet$] Je sais multiplier des
  nombres en �criture fractionnaire.
  
  \item[$\bullet$] Je sais simplifier un produit ``au fur et � mesure''.
  
  \item[$\bullet$] Je sais effectuer un calcul en respectant les r�gles
  de priorit�s.
 
\end{itemize} \\

\hline

\end{tabular}
\end{center}



\end{document}
