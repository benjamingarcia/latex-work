\documentclass[12pt, twoside]{article}
\usepackage[francais]{babel}
\usepackage[T1]{fontenc}
\usepackage[latin1]{inputenc}
\usepackage[left=5mm, right=5mm, top=5mm, bottom=5mm]{geometry}
\usepackage{float}
\usepackage{graphicx}
\usepackage{array}
\usepackage{multirow}
\usepackage{amsmath,amssymb,mathrsfs} 
\usepackage{soul}
\usepackage{textcomp}
\usepackage{eurosym}
\usepackage{lscape}
 \usepackage{variations}
\usepackage{tabvar}
 
\pagestyle{empty}



\begin{document}

\begin{center}
\LARGE{\ul{\textbf{Racines carr�es}}}
\end{center}

\bigskip

\section{D�finitions}

\ul{D�finition:} Si $a$ est un nombre positif ou nul alors il existe un nombre
positif ou nul not� $\sqrt{a}$ (se lit ``racine carr�e de a'') dont le carr�
est �gal � $a$.

Si $a \geqslant 0$ alors $\sqrt{a} \geqslant 0$ et $(\sqrt{a})^2=a$.


\enskip

\ul{Remarques:}

$\bullet$ La racine carr� d'un nombre n�gatif n'existe pas (et n'a pas de sens)
car un carr� est toujours positif.

$\bullet$ $\sqrt{a}$ peut �tre un nombre positif entier, d�cimal, rationnel ou
irrationnel.


\enskip

\ul{Exemples:} 

$\bullet$ Nombre entier:
\qquad
$\sqrt{9}=\ldots$ car \ldots \ldots \ldots \ldots \ldots \qquad \qquad \qquad
 $\sqrt{64}=\ldots$ car \ldots \ldots \ldots \ldots \ldots

$\bullet$ Nombre d�cimal:
\qquad
$\sqrt{2,25}=\ldots$ car \ldots \ldots \ldots \ldots \ldots \qquad \qquad \qquad
$\sqrt{11,56}=\ldots$ car \ldots \ldots \ldots \ldots \ldots

$\bullet$ Nombre irrationnel:
\qquad
$\sqrt{2} \approx \ldots$ (arrondi au dixi�me) \qquad \qquad \qquad \qquad
$\sqrt{3}\approx \ldots$ (arrondi au centi�me)



\enskip

\ul{Application:} D�veloppe et r�duis l'expression $(5-\sqrt{7})(5+\sqrt{7})$.

\bigskip


\bigskip

\bigskip


\textit{ex 23 p 50; ex 21 p 50 }


\section{Equations du type $x^2=a$}


\ul{Propri�t�:} L'�quation $x^2=a$ poss�de:

$\bullet$ aucune solution si $a < 0$

$\bullet$ deux solutions $-\sqrt{a}$ et $\sqrt{a}$ si $a>0$

$\bullet$ une unique solution $0$ si $a=0$.

\enskip

\ul{Exercice:} R�soudre les �quations suivantes:

$5x^2=30$ \qquad \qquad \qquad \qquad \qquad \qquad  $y^2+7=2$ \qquad \qquad
\qquad  \qquad \qquad \qquad $3t^2-7=8$

\bigskip



\bigskip

\bigskip

\bigskip

\bigskip

\bigskip

\textit{ex 10 p 49; ex 11 c) et d) p 49; ex 13 a) p 49; ex 14 a) et b) p 49}


\section{Racines carr�es et op�rations}

\subsection{Racine et produit}


\ul{Propri�t�:} Pour tous nombres positifs ou nuls $a$ et $b$, on a :
\quad  \fbox{$\sqrt{a \times b}=\sqrt{a} \times \sqrt{b}$}


\enskip

\ul{Exemple 1:} Calcule sans calculatrice:


$\sqrt{81 \times 4}$ \qquad \qquad \qquad \qquad \qquad  $\sqrt{25 \times 36}$
\qquad \qquad \qquad \qquad \qquad $\sqrt{2} \times \sqrt{50}$ \qquad \qquad
\qquad \qquad \qquad $\sqrt{27} \times \sqrt{3}$

\bigskip

\bigskip

\bigskip



\bigskip



\ul{Propri�t� (cons�quences):} Pour tout nombre $a$ positif ou nul, on a:
\qquad \fbox{$\sqrt{a^2}=a$}.

\enskip

\ul{D�monstration:}

\bigskip

\ul{Exemple 2:} Ecris les expressions suivantes sous la forme $a \sqrt{b}$ avec
$a$ et $b$ des entiers positifs et $b$ le plus petit possible.

$\sqrt{12}$ \qquad \qquad \qquad \qquad \qquad $\sqrt{45}$ \qquad \qquad \qquad
\qquad \qquad $\sqrt{32}$ \qquad \qquad \qquad \qquad \qquad $\sqrt{700}$


\bigskip

\bigskip

\bigskip

\bigskip


\bigskip

\bigskip

\bigskip


\ul{Exemple 3:} Ecris les expressions suivantes sous la forme $\sqrt{a}$ avec
$a$ entier positif.

$2 \sqrt{5}$ \qquad \qquad \qquad \qquad \qquad $4 \sqrt{3}$ \qquad \qquad
\qquad \qquad \qquad $10 \sqrt{11}$ \qquad \qquad \qquad \qquad \qquad $7
\sqrt{2}$

\bigskip

\bigskip

\bigskip

\bigskip

\bigskip

\bigskip

\bigskip


\textit{ex 32 p 50; ex 36 p 50; ex 27 p 50; ex 29 p 50; ex 38 p 51; ex 39 p 51;
ex 43 p 51}

\subsection{Racine et quotient}

\ul{Propri�t�:} Pour tous nombres positifs ou nuls $a$ et $b$ (avec $b \neq 0$),
on a : \quad  \fbox{$\sqrt{\dfrac{a}{b}}= \dfrac{\sqrt{a}}{\sqrt{b}} $}


\enskip

\ul{Exemple:} Calcule sans calculatrice:

$ \sqrt{\dfrac{1}{9}}$ \qquad \qquad \qquad \qquad \qquad \qquad$
\sqrt{\dfrac{36}{25}}$ \qquad \qquad \qquad \qquad \qquad \qquad
$\dfrac{\sqrt{80}}{\sqrt{5}}$ \qquad \qquad \qquad \qquad \qquad \qquad
$\dfrac{\sqrt{242}}{\sqrt{2}}$

\bigskip

\bigskip

\bigskip

\bigskip

\bigskip

\bigskip

\bigskip

\textit{ex 42 p 51; ex 26 p 50}

\subsection{Addition et soustraction}


ATTENTION!!! \quad $\sqrt{a+b} \neq \sqrt{a} + \sqrt{b}$ \qquad et \qquad
$\sqrt{a-b} \neq \sqrt{a} - \sqrt{b}$

\ul{Exemples:}

\enskip


$\sqrt{9+16} \neq \sqrt{9}+\sqrt{16}$ \qquad et \qquad $\sqrt{25-9} \neq
\sqrt{25}-\sqrt{9}$

\bigskip

\bigskip

\bigskip

$\sqrt{45}+\sqrt{5} \neq \sqrt{45+5}$ \qquad mais \qquad
$\sqrt{45}+\sqrt{5}=\sqrt{5 \times 9}+\sqrt{5}=\sqrt{9} \times \sqrt{5}
+\sqrt{5}=3\sqrt{5}+1\sqrt{5}=4\sqrt{5}$

\bigskip


$\sqrt{32}-5\sqrt{200}=$

\bigskip

\bigskip

\bigskip

\textit{ex 48 A-C p 51; ex 49 F-H p 51; ex 60 p 53; ex 62 p 53}
\end{document}
