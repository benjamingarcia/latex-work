\documentclass[12pt, twoside]{article}
\usepackage[francais]{babel}
\usepackage[T1]{fontenc}
\usepackage[latin1]{inputenc}
\usepackage[left=5mm, right=5mm, top=5mm, bottom=5mm]{geometry}
\usepackage{float}
\usepackage{graphicx}
\usepackage{array}
\usepackage{multirow}
\usepackage{amsmath,amssymb,mathrsfs}
\usepackage{soul}
\usepackage{textcomp}
\usepackage{eurosym}
 \usepackage{variations}
\usepackage{tabvar}

\pagestyle{empty}
\begin{document}

\section*{\center{Correction devoir maison 3}}

\enskip


\ul{Exercice 1}:

\begin{enumerate}
  \item  
   \ul{Donn�es}: MAN est un triangle rectangle en B. [AC] est la m�diane issue
   de l'angle droit. AC=2,3cm. 
  
  \ul{Propri�t�}: Si un triangle est rectangle alors la m�diane issue de
  l'angle droit a pour longueur la moiti� de celle de l'hypot�nuse. 
   
  
  \ul{Conclusion}: MN=2 $\times$ AC=2 $\times$ 2,3=4,6cm
 
  
   \medskip
   
     
  \item  
  
On sait que MBN est un triangle rectangle en B, [BD] est la m�diane issue
   de l'angle droit et MN=4,6cm.
  
 En utilisant la m�me propri�t� qu'� la question 1., on en d�duit que: 
BC=$\dfrac{MN}{2}$ =$\dfrac{4,6}{2}$ =2,3cm  
\end{enumerate}

\bigskip

\ul{Exercice 2}:

On construit le cercle de diam�tre [AB]. On place un point P sur ce cercle.

D'apr�s la propri�t�: ``si un triangle est inscrit dans un cercle de diam�tre
l'un de ses c�t�s alors ce triangle est rectangle et admet ce c�t� pour
hypot�nuse'', le triangle ABP est rectangle en P.


De m�me, on construit le cercle de diam�tre [CD]. On place P sur le cercle. Le
triangle CDP est rectangle en P.


Pour avoir ABP et CDP rectangles en P (un seul point), je dois donc placer P �
l'intersection de ces deux cercles.


\bigskip


\ul{Exercice 3}:

\begin{tabular}{cc}
\begin{minipage}{13cm}
2) \ul{donn�es}: IJK est inscrit dans le cercle $\mathcal{C}$. [IJ] est un
diam�tre du cercle.

\ul{propri�t�}: Si un triangle est inscrit dans un cercle de diam�tre l'un de
ses c�t�s alors ce triangle est rectangle et admet ce c�t� pour hypot�nuse.


\ul{conclusion}: IJK est rectangle en K.

  
\end{minipage}

&
\begin{minipage}{5cm}
\quad
\end{minipage}
\end{tabular}

\enskip


3) Le triangle IJK est rectangle en K. 
D'apr�s le th�or�me de Pythagore, on a:

\qquad \qquad 
\qquad \qquad $IJ^2=IK^2+JK^2$


\qquad \qquad 
\qquad \qquad $8^2=IK^2+3,5^2$


\qquad \qquad 
\qquad \qquad $64=IK^2+12,25$


\qquad \qquad 
\qquad \qquad $IK^2=64-12,25=51,75$

La longueur JK est positive donc $IK=\sqrt{51,75} \approx 7,2cm$.

\bigskip

\ul{Exercice 4}:

\begin{enumerate}
  \item $\big ( -3 + (-5) \big ) \times \big ( 6-(-8) \big)$
  \qquad \qquad 2. $\dfrac{-75}{8-14}$ \quad ou \quad $-75 \div (8-14)$
  
  \item[3.] La somme de 25 et du produit de 7 par (-2).
  
  \item [4.] Le quotient de la diff�rence de 4 et de (-6) par le produit de (-3)
  et de (-2).
\end{enumerate}

\bigskip

\ul{Exercice 5}:

a) $(5-4) \times  3 + 2 =1 \times 3 + 2 = 3 +2=5$ \qquad \qquad autre
solution: $(5-4)\times(3+2))1 \times 5 =5$


b) $5-4 \times (3+2)=5-4 \times 5=5-20=-15$


c) $5-(4 \times 3+2)=5-(12+2)=5-14=-9$


\bigskip

\ul{Exercice 6}:


$a$ est un nombre positif. 


$a^2=a \times a$ est un nombre positif car le produit de deux nombres positifs
est positif.


$-a$ est l'opppos� de $a$, c'est donc un nombre n�gatif.

$(-a)^3=(-a) \times  (-a) \times (-a)$ est un nombre n�gatif car c'est le
produit de trois nombres n�gatifs.


$a^2 \times (-a)^3$ est le produit d'un nombre positif par un nombre n�gatif, le
r�sultat sera donc n�gatif.


\end{document}
