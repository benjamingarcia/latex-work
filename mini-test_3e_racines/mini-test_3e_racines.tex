\documentclass[12pt, twoside]{article}
\usepackage[francais]{babel}
\usepackage[T1]{fontenc}
\usepackage[latin1]{inputenc}
\usepackage[left=5mm, right=5mm, top=3mm, bottom=3mm]{geometry}
\usepackage{float}
\usepackage{graphicx}
\usepackage{array}
\usepackage{multirow}
\usepackage{amsmath,amssymb,mathrsfs}
\usepackage{soul}
\usepackage{textcomp}
\usepackage{eurosym}
 \usepackage{variations}
\usepackage{tabvar}

\pagestyle{empty}

\begin{document}

\begin{flushleft}
NOM PRENOM: \ldots \ldots \ldots \ldots \ldots \ldots \ldots \ldots \ldots
 
\enskip

\end{flushleft}

\begin{center}
{\fbox{$3^{e}2$ \qquad \qquad \textbf{\Large{Mini-test 14 (sujet 1)}}
\qquad \qquad 05/06/2013}}
\end{center}



\enskip 


\ul{Exercice 1}: \textit{(4 points)}


\enskip


Calculer \textbf{si possible} et donner la valeur exacte.


\begin{center}
$\sqrt{2013^2}=\ldots \ldots$ \qquad \qquad \quad $\sqrt{-36}=\ldots \ldots $
\qquad \qquad \quad $2 \times \sqrt{7} \times \sqrt{7}= \ldots \ldots$ \qquad
\qquad \quad $(3 \sqrt{5})^2=\ldots \ldots$
\end{center}


\bigskip

\ul{Exercice 2}: \textit{(5 points)}

\enskip

Calculer en d�taillant les calculs.

\begin{tabular}{lll}
 $\sqrt3 \times \sqrt{12}=$
  \ldots \ldots \ldots \ldots 
  \qquad \qquad \qquad \qquad & $\sqrt{\dfrac{25}{4}}=$ \ldots \ldots \ldots
  \ldots & \qquad \qquad \qquad \qquad $\dfrac{\sqrt{28}}{\sqrt{7}}=$ \ldots
  \ldots \ldots \ldots
\\
  
  \quad & \quad & \quad \\
  
 $\sqrt{9 \times 81}=$ \ldots \ldots \ldots \ldots 
 
   \qquad \qquad \qquad \qquad &  $\sqrt{36+64}=$ \ldots \ldots \ldots \ldots 
  & \quad  \\
   

  
   
   
\end{tabular}
  
  \bigskip
  
  
\ul{Exercice 3}: \textit{(2 points)}

\enskip

D�velopper et r�duire.

\enskip

$2 \sqrt{5} (\sqrt{5} + 1)=$ \ldots \ldots \ldots \ldots \ldots \ldots
  \ldots \ldots \ldots \ldots \ldots \ldots \ldots \ldots
  
  
  \enskip
  
  
  $(2+4 \sqrt{3})(2-4 \sqrt{3})=$ \ldots \ldots \ldots \ldots \ldots \ldots
  \ldots \ldots \ldots \ldots \ldots \ldots \ldots \ldots
  
  
    \bigskip
  
  
\ul{Exercice 4}: \textit{(2 points)}

\enskip
  
 Ecrire sous la forme $\sqrt{a}$ avec $a$ nombre entier positif.

 \begin{center}
 $3\sqrt{7}$=  \ldots \ldots \ldots \qquad \qquad \qquad \qquad \qquad $5
 \sqrt{3} \times 2 \times \sqrt{10}=$ \ldots \ldots \ldots
\end{center}

   \bigskip
  
  
\ul{Exercice 5}: \textit{(3 points)}

\enskip

Ecrire sous la forme $a\sqrt{b}$ avec $a$ et $b$ deux nombres entiers positifs
et $b$ le plus petit possible.

\enskip

$\sqrt{72}=$ \ldots \ldots \ldots


\enskip


$\sqrt{13} +5 \sqrt{13} - 9\sqrt{13}=$ \ldots \ldots \ldots


\enskip


$\sqrt{20}-4 \sqrt{5}+\sqrt{45}=$ \ldots \ldots \ldots





\bigskip
  
  
\ul{Exercice 6}: \textit{(4 points)}


\enskip

R�soudre les �quations suivantes.

\enskip


\begin{tabular}{l|l|l|l}

 \quad \quad $y^2=-4$ \quad \quad  \quad \quad & \quad \quad  \quad
 \quad $x^2=11$ \quad \quad  \quad \quad & \quad \quad  \quad \quad $10u^2-5=-5$
 \quad \quad  \quad \quad &  \quad \quad \quad \quad $7a^2=42$ \\

\quad & \quad & \quad  & \quad \\

\quad & \quad & \quad & \quad \\

\quad & \quad & \quad  & \quad \\

\quad & \quad & \quad  & \quad \\


\end{tabular}

\pagebreak

\begin{flushleft}
NOM PRENOM: \ldots \ldots \ldots \ldots \ldots \ldots \ldots \ldots \ldots
 
\enskip

\end{flushleft}

\begin{center}
{\fbox{$3^{e}2$ \qquad \qquad \textbf{\Large{Mini-test 14 (sujet 2)}}
\qquad \qquad 05/06/2013}}
\end{center}



\enskip 


\ul{Exercice 1}: \textit{(4 points)}


\enskip


Calculer \textbf{si possible} et donner la valeur exacte.


\begin{center}
$\sqrt{1998^2}=\ldots \ldots$ \qquad \qquad \quad $\sqrt{-25}=\ldots \ldots $
\qquad \qquad \quad $3 \times \sqrt{5} \times \sqrt{5}= \ldots \ldots$ \qquad
\qquad \quad $(2 \sqrt{7})^2=\ldots \ldots$
\end{center}


\bigskip

\ul{Exercice 2}: \textit{(5 points)}

\enskip

Calculer en d�taillant les calculs.

\begin{tabular}{lll}
 $\sqrt{5} \times \sqrt{20}=$
  \ldots \ldots \ldots \ldots 
  \qquad \qquad \qquad \qquad & $\sqrt{\dfrac{36}{4}}=$ \ldots \ldots \ldots
  \ldots & \qquad \qquad \qquad \qquad $\dfrac{\sqrt{24}}{\sqrt{6}}=$ \ldots
  \ldots \ldots \ldots
\\
  
  \quad & \quad & \quad \\
  
 $\sqrt{9 \times 64}=$ \ldots \ldots \ldots \ldots 
 
   \qquad \qquad \qquad \qquad &  $\sqrt{36-25}=$ \ldots \ldots \ldots \ldots 
  & \quad  \\
   

  
   
   
\end{tabular}
  
  \bigskip
  
  
\ul{Exercice 3}: \textit{(2 points)}

\enskip

D�velopper et r�duire.

\enskip

$3 \sqrt{7} (\sqrt{7} + 4)=$ \ldots \ldots \ldots \ldots \ldots \ldots
  \ldots \ldots \ldots \ldots \ldots \ldots \ldots \ldots
  
  
  \enskip
  
  
  $(3+2 \sqrt{6})(3-2 \sqrt{6})=$ \ldots \ldots \ldots \ldots \ldots \ldots
  \ldots \ldots \ldots \ldots \ldots \ldots \ldots \ldots
  
  
    \bigskip
  
  
\ul{Exercice 4}: \textit{(2 points)}

\enskip
  
 Ecrire sous la forme $\sqrt{a}$ avec $a$ nombre entier positif.

 \begin{center}
 $2\sqrt{11}$=  \ldots \ldots \ldots \qquad \qquad \qquad \qquad \qquad $2
 \sqrt{6} \times 5 \times \sqrt{10}=$ \ldots \ldots \ldots
\end{center}

   \bigskip
  
  
\ul{Exercice 5}: \textit{(3 points)}

\enskip

Ecrire sous la forme $a\sqrt{b}$ avec $a$ et $b$ deux nombres entiers positifs
et $b$ le plus petit possible.

\enskip

$\sqrt{800}=$ \ldots \ldots \ldots


\enskip


$\sqrt{17} +4 \sqrt{17} - 7\sqrt{17}=$ \ldots \ldots \ldots


\enskip


$\sqrt{45}-3 \sqrt{5}+\sqrt{20}=$ \ldots \ldots \ldots





\bigskip
  
  
\ul{Exercice 6}: \textit{(4 points)}


\enskip

R�soudre les �quations suivantes.

\enskip


\begin{tabular}{l|l|l|l}

 \quad \quad $u^2=-9$ \quad \quad  \quad \quad & \quad \quad  \quad
 \quad $t^2=13$ \quad \quad  \quad \quad & \quad \quad  \quad \quad $8x^2-7=-7$
 \quad \quad  \quad \quad &  \quad \quad \quad \quad $6y^2=42$ \\

\quad & \quad & \quad  & \quad \\

\quad & \quad & \quad & \quad \\

\quad & \quad & \quad  & \quad \\

\quad & \quad & \quad  & \quad \\


\end{tabular}
\end{document}
