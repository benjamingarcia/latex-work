\documentclass[12pt, twoside]{article}
\usepackage[francais]{babel}
\usepackage[T1]{fontenc}
\usepackage[latin1]{inputenc}
\usepackage[left=8mm, right=8mm, top=8mm, bottom=8mm]{geometry}
\usepackage{float}
\usepackage{graphicx}
\usepackage{array}
\usepackage{multirow}
\usepackage{amsmath,amssymb,mathrsfs}
\pagestyle{empty}
\begin{document}

\section*{\center{Bilan enchainements d'op�rations sur les nombres}}




\bigskip

\begin{center}
\begin{tabular}{|m{9cm}|m{9cm}|}
\hline
\textbf{Ce que je dois savoir} & \textbf{Ce que je dois savoir faire} \\
\hline

\enskip


\begin{itemize}
  \item[$\bullet$] Je connais les r�gles de priorit� de calculs.
  
\enskip

  \item [$\bullet$] Je connais la signification de $a^2$ et $a^3$ et je sais les
  calculer sur des exemples.
  
  \enskip
  
  \item [$\bullet$] Je sais rajouter les signes ``$\times$'' qui sont
  sous-entendus.

\enskip

  \item [$\bullet$] Je peux supprimer un signe ``$\times$'' devant une lettre
  ou une parenth�se.
 
 \enskip
  
  \item [$\bullet$] Je connais le vocabulaire associ� aux diff�rentes
  op�rations.
 
 \enskip
  
  \item [$\bullet$] Je connais la m�thode pour calculer une expression
  litt�rale pour une certaine valeur de lettres.
  
 
\end{itemize}
&

\enskip
\begin{itemize}
  \item[$\bullet$] Je sais effectuer des calculs en respectant les priorit�s.

\enskip

  \item[$\bullet$] Je sais effectuer des calculs avec des expressions
  fractionnaires.

 
\enskip  
      
  \item[$\bullet$] Je sais reconna�tre si une expression est une somme, un
  produit, un quotient ou une diff�rence.
  
\enskip
  
 \item[$\bullet$] A partir d'une phrase utilisant le vocabulaire des
 op�rations, je sais �crire le calcul correspondant.
 
\enskip
 
  \item[$\bullet$] Je sais r�soudre un probl�me et �crire l'expression
  permettant de r�soudre le probl�me. 
  

 \end{itemize} \\
\hline

\end{tabular}
\end{center}


\bigskip


\bigskip


\section*{\center{Bilan enchainements d'op�rations sur les nombres}}




\bigskip

\begin{center}
\begin{tabular}{|m{9cm}|m{9cm}|}
\hline
\textbf{Ce que je dois savoir} & \textbf{Ce que je dois savoir faire} \\
\hline

\enskip


\begin{itemize}
  \item[$\bullet$] Je connais les r�gles de priorit� de calculs.
  
\enskip

  \item [$\bullet$] Je connais la signification de $a^2$ et $a^3$ et je sais les
  calculer sur des exemples.
  
  \enskip
  
  \item [$\bullet$] Je sais rajouter les signes ``$\times$'' qui sont
  sous-entendus.

\enskip

  \item [$\bullet$] Je peux supprimer un signe ``$\times$'' devant une lettre
  ou une parenth�se.
 
 \enskip
  
  \item [$\bullet$] Je connais le vocabulaire associ� aux diff�rentes
  op�rations.
 
 \enskip
  
  \item [$\bullet$] Je connais la m�thode pour calculer une expression
  litt�rale pour une certaine valeur de lettres.
  
 
\end{itemize}
&

\enskip
\begin{itemize}
  \item[$\bullet$] Je sais effectuer des calculs en respectant les priorit�s.

\enskip

  \item[$\bullet$] Je sais effectuer des calculs avec des expressions
  fractionnaires.

 
\enskip  
      
  \item[$\bullet$] Je sais reconna�tre si une expression est une somme, un
  produit, un quotient ou une diff�rence.
  
\enskip
  
 \item[$\bullet$] A partir d'une phrase utilisant le vocabulaire des
 op�rations, je sais �crire le calcul correspondant.
 
\enskip
 
  \item[$\bullet$] Je sais r�soudre un probl�me et �crire l'expression
  permettant de r�soudre le probl�me. 
  

 \end{itemize} \\
\hline

\end{tabular}
\end{center}
\end{document}
