\documentclass[12pt, twoside]{article}
\usepackage[francais]{babel}
\usepackage[T1]{fontenc}
\usepackage[latin1]{inputenc}
\usepackage[left=1cm, right=1cm, top=1cm, bottom=1cm]{geometry}
\usepackage{float}
\usepackage{graphicx}
\usepackage{array}
\usepackage{multirow}
\usepackage{amsmath,amssymb,mathrsfs}
\usepackage{soul}
\pagestyle{empty}
\begin{document}


\section*{Ecriture d�cimale et calculatrice}

\subsection*{1) Ecriture scientifique d'un d�cimal}

Tout nombre d�cimal peut s'�crire sous la forme $a \times 10^{p}$ o� $a \in
\mathbb{D}$ tel que $1 \leqslant a < 10$ et $p \in \mathbb{Z}$. C'est
l'�criture scientifique du d�cimal.

\bigskip
\textit{Exemples:}\\
$487,3692=4,873692 \times 10^{2}$ s'�crit � la calculatrice : $4,873692$ EE
$2$\\
$0,000258=2,58 \times 10^{-4}$ s'�crit � la calculatrice: $2,58$ EE $-4$.

\bigskip
\textit{Remarques:} $487,3692$ a pour ordre de grandeur $5 \times 10^{2}$.\\
$0,000258$ a pour ordre de gandeur $3 \times 10^{-4}$.

\subsection*{2) Arrondis et calculatrice}

\textit{Exemple:} Donner une valeur arrondi � $10^{-4}$ pr�s de $\pi$. On a
$\pi \simeq 3,1415$ \fbox{$9$}.\\ $3,1416$ est une valeur approch�e de $\pi$.
C'est sa valeur arrondie au milli�me pr�s.


\bigskip
Un nombre d�cimal a une �criture d�cimale \ldots \ldots \ldots \ldots. Il y a deux
types de nombres non d�cimaux: les rationnels (et dans ce  cas on a une �criture
d�cimale \ldots \ldots \ldots \ldots \ldots \ldots \ldots \ldots \ldots) et les
irrationnels (et dans ce cas on a une �criture d�cimale \ldots \ldots \ldots \ldots).\\
 La calculatrice donne une valeur arrondie des r�els non d�cimaux.

\bigskip
\textbf{PROBLEME 1:} Comment �tre s�r du r�sultat de ma calculatrice?
\medskip

\textbf{R�ponse:} Si le nombre affich� � l'�cran a au maximum \ldots chiffres
apr�s la virgule, le d�cimal affich� est la VALEUR EXACTE de ce nombre.
\medskip

\textit{Exemples:}
\begin{enumerate}
  \item pour $\dfrac{9}{2 \thinspace 000 \thinspace 000}$, la calculatrice
  affiche $0,000 \thinspace 004 \thinspace 5$. C'est la valeur \ldots \ldots
  \ldots.
  \item pour $\dfrac{2}{3}$, la calculatrice affiche $0,666 \thinspace 666
  \thinspace 667$. On ne peut pas conclure pour le moment.
\end{enumerate}

\bigskip

\textbf{PROBLEME 2:} En arrondissant, la calculatrice peut ``masquer'' la
p�riode des rationnels ou ne pas l'afficher (elle est trop grande).



\textit{Exemples:}Pour $\dfrac{2}{3}$, l'arrondi de la calculatrice donne
``l'impression'' qu'il n'y a pas de p�riode et pour $\dfrac{7}{523}$ on ne voit
pas la p�riode sur l'�cran.



\textbf{Comment savoir alors si un rationnel est d�cimal?}
\medskip


\textbf{Th�or�me:}
\bigskip
\bigskip
\bigskip
\bigskip
\bigskip



\textit{Exemples:}
 $\dfrac{43}{5}$ est un d�cimal car $5=5^{1} \times 2^{0}$.
 \smallskip
 
 
 $\dfrac{51}{21}$ \ldots \ldots \ldots \ldots un d�cimal.
 \smallskip 
 
 
 $\dfrac{207}{200}$ \ldots \ldots \ldots \ldots un d�cimal car \ldots \ldots
 \ldots \ldots \ldots.
 \smallskip
 
 
 $\dfrac{52}{150}$ \ldots \ldots \ldots \ldots un d�cimal.
 
\end{document}
