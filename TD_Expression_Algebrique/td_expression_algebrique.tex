%%This is a very basic article template.
%%There is just one section and two subsections.
\documentclass[12pt, twoside]{article}
\usepackage[francais]{babel}
\usepackage[T1]{fontenc}
\usepackage[latin1]{inputenc}
\usepackage[left=8mm, right=8mm, top=8mm, bottom=8mm]{geometry}
\usepackage{float}
\usepackage{graphicx}
\usepackage{array}
\usepackage{multirow}
\usepackage{amsmath,amssymb,mathrsfs}
\usepackage{textcomp}
\usepackage{soul}

\begin{document}

\section*{\center{TD expressions alg�briques et �quations}}

\textbf{Exercice 1}
Indiquer pour chaque expression s'il s'agit d'une somme ou d'un produit, puis retrouver les expressions �gales.
\bigskip


$
\begin{array}{lcl}
A= (x-3)(x+3) & \ \ \  &F= 2x+3x^2\\
B=x^2-9 & \ \ \ &G=6ab\\
C=x�-6x+9& \ \ \ &H=16ab\\
D= 4a\times b + a\times b\times 2& \ \ \ & I=(x-3)�\\
E=4a\times (b+b) \times 2& \ \ \ & J=x(2+3x)\\
\end{array}
$
\bigskip


\textbf{Exercice 2}
Indiquer pour chaque expression s'il s'agit d'une somme, d'un produit ou d'un quotient; puis d�velopper les produits,
factoriser les sommes et simplifier les quotients.
\bigskip

$
\begin{array}{lcl}
A=(x-3)(2x-7) & \ \ \  &E=\dfrac{x+1}{x�-1}\\
\ \\
B=(x-3)(2x-7)-(x-3)(2x+1) & \ \ \  &F=27x�+18+3\\
\ \\
C=\dfrac{2x+6}{4x�+12x} & \ \ \  &G=(5-x)�\\
\ \\
D=(x-2)�-25 & \ \ \  &H=2x�+20x+50\\
\end{array}
$
\bigskip


\textbf{Exercice 3}
D�velopper, r�duire et ordonner : 
\bigskip


$
\begin{array}{l}
A=-x(x+2)-3(x�-2)+2x�-6\\
B=x-2-5(x-3)+3(-x-4)\\
C=2x(2x-3)-5(x-1)(x+2)\\
\end{array}
$
\bigskip


\ul{Rappel}: Pour d�velopper un produit de trois facteurs, on multiplie les deux premiers puis le r�sultat obtenu par le
troisi�me.
\bigskip


\textbf{Exercice 4}
R�soudre les �quations suivantes ; 
\bigskip


$
\begin{array}{l}
1)  \ x�=36\\
\ \\
2)  \ x�=5\\
\ \\
3)\  x�=-3\\
\ \\
4)\  \dfrac{5x-7}{3x+2}=0\\
\ \\
5) \ \dfrac{2(7x-1)+(2x+3)(7x-1)}{x�-1}\\
\end{array}
$
\bigskip


\textbf{Exercice 5}
On pose : $A=2x�+12x+18$


1) Factoriser $A$


2) Utiliser le r�sultat pr�c�dent pour r�soudre l'�quatino $2x�+12x+18=(x+3)(x+5)$


\end{document}
