\documentclass[12pt, twoside]{article}
\usepackage[francais]{babel}
\usepackage[T1]{fontenc}
\usepackage[latin1]{inputenc}
\usepackage[left=5mm, right=5mm, top=5mm, bottom=5mm]{geometry}
\usepackage{float}
\usepackage{graphicx}
\usepackage{array}
\usepackage{multirow}
\usepackage{amsmath,amssymb,mathrsfs}
\usepackage{soul}
\usepackage{textcomp}
\usepackage{eurosym}
 \usepackage{variations}
\usepackage{tabvar}

\pagestyle{empty}
\begin{document}


\section*{\center{Correction contr�le 2}}



\subsection*{Exercice 3}


 


$A \notin [PI]$ \qquad  \qquad $A \in (PI)$ \qquad \qquad $R \notin (AP)$
\qquad \qquad
$A \in [RH)$ \qquad \qquad  $R \in (AR]$



\subsection*{Exercice 5}

\begin{enumerate}
  \item Les droites (AD) et (BC) sont parall�les.
  \item Lesdroites (AD) et (BC) sont parall�les parce qu'elles sont toutes les
  deux perpendiculaires � la droite (AB).
  \item Les droites (DC) et (BC) sont perpendiculaires.
  \item Si deux droites sont parall�les alors toute perpendiculaire � l'une est
  perpendiculaire � l'autre.
  
  (AD) // (BC) et (AB) $\perp$ (AD) donc (BC) $\perp$ (AD).
  \item ABCD est un qudrilat�re ayant quatre angles droits: c'est un rectangle.
  
    
\end{enumerate}



\subsection*{Exercice 6}

\begin{enumerate}
  \item Tracer deux droites $(d_1)$ et $(d_2)$ s�cantes en E.
 \item  Placer un point B qui n'appartient ni � la droite $(d_1)$ ni � la
  droite $(d_2)$.  
  \item Tracer la droite parall�le � la droite $(d_1)$ et passant par B; elle
  coupe $(d_2)$ au point A.
  
  \item Tracer la droite perpendiculaire � la droite $(d_1)$ et passant par A;
  elle coupe $(d_1)$ en C. 
\end{enumerate}



\section*{\center{Correction contr�le 2}}



\subsection*{Exercice 3}


$A \notin [PI]$ \qquad  \qquad $A \in (PI)$ \qquad \qquad $R \notin (AP)$
\qquad \qquad
$A \in [RH)$ \qquad \qquad  $R \in (AR]$


\subsection*{Exercice 5}

\begin{enumerate}
  \item Les droites (AD) et (BC) sont parall�les.
  \item Lesdroites (AD) et (BC) sont parall�les parce qu'elles sont toutes les
  deux perpendiculaires � la droite (AB).
  \item Les droites (DC) et (BC) sont perpendiculaires.
  \item Si deux droites sont parall�les alors toute perpendiculaire � l'une est
  perpendiculaire � l'autre.
  
  (AD) // (BC) et (AB) $\perp$ (AD) donc (BC) $\perp$ (AD).
  \item ABCD est un qudrilat�re ayant quatre angles droits: c'est un rectangle.
  
    
\end{enumerate}



\subsection*{Exercice 6}

\begin{enumerate}
  \item Tracer deux droites $(d_1)$ et $(d_2)$ s�cantes en E.
 \item  Placer un point B qui n'appartient ni � la droite $(d_1)$ ni � la
  droite $(d_2)$.  
  \item Tracer la droite parall�le � la droite $(d_1)$ et passant par B; elle
  coupe $(d_2)$ au point A.
  
  \item Tracer la droite perpendiculaire � la droite $(d_1)$ et passant par A;
  elle coupe $(d_1)$ en C. 
\end{enumerate}
\end{document}
