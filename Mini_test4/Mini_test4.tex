\documentclass{article}

\usepackage[francais]{babel}
\usepackage[T1]{fontenc}
\usepackage[latin1]{inputenc}
\usepackage[left=1cm, right=1cm, top=6mm, bottom=6mm]{geometry}
\usepackage{float}
\usepackage{graphicx}
\usepackage{array}
\usepackage{multirow}
\usepackage{amsmath, amssymb, mathrsfs}

\begin{document}

\begin{flushleft}
NOM PRENOM: \ldots \ldots \ldots \ldots \ldots \ldots \ldots \ldots \ldots

\bigskip
\end{flushleft}
\begin{center}
{\fbox{$2^{de}5$ \qquad \qquad \textbf{\Large{Contr�le de cours 4 (sujet 1)}}
\qquad \qquad 06/03/2009}}
\end{center}


\bigskip
\textbf{Exercice 1:} 
On consid�re 2 triangles $ABC$ et $EFD$ tel que : $BC=DE$,  $\widehat{ABC}=\widehat{EDF}$ et 
$\widehat{ACB}=\widehat{FED}$

\medskip
1)Faire un sch�ma et le coder (� main lev�e).

\bigskip
\bigskip

\bigskip
\bigskip
\medskip
2)Est-on s�r que les triangles $ABC$ et $EFD$ sont isom�triques?
\begin{tabular}{cc}
oui & non
\end{tabular}

\medskip
3)Pourquoi?

\bigskip

\bigskip
\bigskip

4)Peut-on en d�duire que $AC=EF$?
\begin{tabular}{cc}
oui & non
\end{tabular}

\medskip
5)Peut-on en d�duire que $AC=FD$?
\begin{tabular}{cc}
oui & non
\end{tabular}


\bigskip
\bigskip
\textbf{Exercice 2:} 
On consid�re 2 triangles $GHI$ et $UVW$ tel que : $UV=GI$,  $\widehat{GHI}=\widehat{UVW}$ et $GH=UW$.

\medskip
1)Faire un sch�ma et le coder (� main lev�e).

\bigskip
\bigskip

\bigskip
\bigskip
\bigskip
\medskip
2)Est-on s�r que les triangles $GHI$ et $UVW$ sont isom�triques?
\begin{tabular}{cc}
oui & non
\end{tabular}

\medskip
3)Pourquoi?

\bigskip
\bigskip

\medskip
4)Peut-on en d�duire que $HI=VW$?
\begin{tabular}{cc}
oui & non
\end{tabular}

\bigskip
\textbf{Exercice 3:} 
Soit $PQR$ un triangle retangle en $P$ et $STU$ un triangle retangle en $T$. On suppose que $ST=PQ$ et $SU=QR$.

\medskip
1)Faire un sch�ma et le coder (� main lev�e).

\bigskip
\bigskip

\bigskip
\bigskip

\medskip
2)Est-on s�r que les triangles $PQR$ et $STU$ sont isom�triques?
\begin{tabular}{cc}
oui & non
\end{tabular}

\medskip
3)Pourquoi?


\bigskip
\bigskip
\bigskip
\bigskip
\bigskip
\textbf{Exercice 4:} 
Soit $KLM$ et $MNP$ 2 triangles tels que : $\widehat{LKM}=\widehat{MPN}$,  $\widehat{KML}=\widehat{PNM}$
et $\widehat{MLK}=\widehat{NMP}$.

\medskip
1)Est-on s�r que les triangles $KLM$ et $MNP$ sont isom�triques?
\begin{tabular}{cc}
oui & non
\end{tabular}

\medskip
2)Pourquoi?

\pagebreak
\begin{flushleft}
NOM PRENOM: \ldots \ldots \ldots \ldots \ldots \ldots \ldots \ldots \ldots

\bigskip
\end{flushleft}
\begin{center}
{\fbox{$2^{de}5$ \qquad \qquad \textbf{\Large{Contr�le de cours 4 (sujet 2)}}
\qquad \qquad 06/03/2009}}
\end{center}


\bigskip
\textbf{Exercice 1:} 
On consid�re 2 triangles $MNO$ et $PQR$ tel que : $PQ=MO$,  $\widehat{MNO}=\widehat{PRQ}$ et $MN=PR$.

\medskip
1)Faire un sch�ma et le coder (� main lev�e).

\bigskip
\bigskip

\bigskip
\bigskip
\bigskip
\medskip
2)Est-on s�r que les triangles $MNO$ et $PQR$ sont isom�triques?
\begin{tabular}{cc}
oui & non
\end{tabular}

\medskip
3)Pourquoi?

\bigskip
\bigskip

\medskip
4)Peut-on en d�duire que $NO=QR$?
\begin{tabular}{cc}
oui & non
\end{tabular}

\bigskip
\bigskip
\textbf{Exercice 2:} 
On consid�re 2 triangles $GHI$ et $KLJ$ tel que : $HI=JK$,  $\widehat{GHI}=\widehat{KJL}$
et $\widehat{GIH}=\widehat{LKJ}$

\medskip
1)Faire un sch�ma et le coder (� main lev�e).

\bigskip
\bigskip

\bigskip
\bigskip
\medskip
2)Est-on s�r que les triangles $GHI$ et $KLJ$ sont isom�triques?
\begin{tabular}{cc}
oui & non
\end{tabular}

\medskip
3)Pourquoi?

\bigskip

\bigskip
\bigskip

4)Peut-on en d�duire que $GI=KL$?
\begin{tabular}{cc}
oui & non
\end{tabular}

\medskip
5)Peut-on en d�duire que $GI=LJ$?
\begin{tabular}{cc}
oui & non
\end{tabular}


\bigskip
\textbf{Exercice 3:} 
Soit $TUS$ un triangle retangle en $T$ et $XYZ$ un triangle retangle en $Y$. On suppose que $XY=TU$ et $XZ=US$.

\medskip
1)Faire un sch�ma et le coder (� main lev�e).

\bigskip
\bigskip

\bigskip
\bigskip

\medskip
2)Est-on s�r que les triangles $XYZ$ et $TUS$ sont isom�triques?
\begin{tabular}{cc}
oui & non
\end{tabular}

\medskip
3)Pourquoi?


\bigskip
\bigskip
\bigskip
\bigskip
\bigskip
\textbf{Exercice 4:} 
Soit $BAC$ et $DFE$ 2 triangles tels que : $\widehat{ABC}=\widehat{DEF}$,  $\widehat{BCA}=\widehat{EFD}$
et $\widehat{CAB}=\widehat{FDE}$.

\medskip
1)Est-on s�r que les triangles $BAC$ et $MNP$ sont isom�triques?
\begin{tabular}{cc}
oui & non
\end{tabular}

\medskip
2)Pourquoi?

\end{document}
