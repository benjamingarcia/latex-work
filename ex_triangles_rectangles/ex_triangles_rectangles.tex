\documentclass[12pt, twoside]{article}
\usepackage[francais]{babel}
\usepackage[T1]{fontenc}
\usepackage[latin1]{inputenc}
\usepackage[left=7mm, right=7mm, top=7mm, bottom=7mm]{geometry}
\usepackage{float}
\usepackage{graphicx}
\usepackage{array}
\usepackage{multirow}
\usepackage{amsmath,amssymb,mathrsfs} 
\usepackage{soul}
\usepackage{textcomp}
\usepackage{eurosym}
 \usepackage{variations}
\usepackage{tabvar} 

\begin{document}


\textbf{Activit� 1:} 

\begin{enumerate}
  \item Tracer 3 triangles rectangles et construire leur cercle circonscrit.
  \item Que peut-on dire de la position du centre du cercle circonscrit?
  \item Trace un triangle GHK rectangle en K. Soit I le milieu de l'hypot�nuse
  [GH]. \textit{On veut montrer que I est le centre du cercle circonscrit �
  GHK.}
  \item Soit L le sym�trique de K par rapport � I. Quelle est la nature  du
  quadrilat�re GKHL? Explique pourquoi.
  \item Que peut-on en d�duire sur les longueurs IG,IH et IK? Quel est le
  centre du cercle circonscrit au triangle GHK?
  \item Ecris la propri�t� que tu viens de d�montrer.
\end{enumerate}

\bigskip

\textbf{Activit� 2:}

\begin{enumerate}
  \item Tracer 3 cercles de centre O. 
  \item Tracer dans chacun d'eux un diam�tre [AB].
\item Placer un point M sur chaque cercle (autre que A et B).
\item Quelle conjecture peut-on faire sur le triangle ABM?
\item Sur l'un des cercles, placer le point P tel que [MP] soit un diam�tre du
cercle.
\item D�montrer que AMPB est un rectangle.
\item Conclure sur la nature du triangle AMB.
\item Ecrire la propri�t� que l'on vient de d�montrer.
\end{enumerate}

\bigskip


\textbf{Activit� 1:} 

\begin{enumerate}
  \item Tracer 3 triangles rectangles et construire leur cercle circonscrit.
  \item Que peut-on dire de la position du centre du cercle circonscrit?
  \item Trace un triangle GHK rectangle en K. Soit I le milieu de l'hypot�nuse
  [GH]. \textit{On veut montrer que I est le centre du cercle circonscrit �
  GHK.}
  \item Soit L le sym�trique de K par rapport � I. Quelle est la nature  du
  quadrilat�re GKHL? Explique pourquoi.
  \item Que peut-on en d�duire sur les longueurs IG,IH et IK? Quel est le
  centre du cercle circonscrit au triangle GHK?
  \item Ecris la propri�t� que tu viens de d�montrer.
\end{enumerate}

\bigskip

\textbf{Activit� 2:}

\begin{enumerate}
  \item Tracer 3 cercles de centre O. 
  \item Tracer dans chacun d'eux un diam�tre [AB].
\item Placer un point M sur chaque cercle (autre que A et B).
\item Quelle conjecture peut-on faire sur le triangle ABM?
\item Sur l'un des cercles, placer le point P tel que [MP] soit un diam�tre du
cercle.
\item D�montrer que AMPB est un rectangle.
\item Conclure sur la nature du triangle AMB.
\item Ecrire la propri�t� que l'on vient de d�montrer.
\end{enumerate}
\end{document}
