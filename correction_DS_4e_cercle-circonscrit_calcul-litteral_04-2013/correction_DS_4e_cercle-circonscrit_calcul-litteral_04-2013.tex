\documentclass[12pt, twoside]{article}
\usepackage[francais]{babel}
\usepackage[T1]{fontenc}
\usepackage[latin1]{inputenc}
\usepackage[left=5mm, right=5mm, top=5mm, bottom=5mm]{geometry}
\usepackage{float}
\usepackage{graphicx}
\usepackage{array}
\usepackage{multirow}
\usepackage{amsmath,amssymb,mathrsfs}
\usepackage{soul}
\usepackage{textcomp}
\usepackage{eurosym}
 \usepackage{variations}
\usepackage{tabvar}

\pagestyle{empty}
\begin{document}


\begin{center}
\fbox{Correction du contr�le 7}
\end{center}


 
\bigskip

\ul{Exercice 1}: 


$A=7+(3-5y)=7+3-5y=10-5y$ 

\enskip


 $B=-(5t-4)+t=-5t+4+t=-4t+4$

\enskip

 $C=(-9u+5)-(-14u+10)=-9u+5+14u-10=5u-5$
 
 
 \bigskip
 
 
 \ul{Exercice 2}: 
 
 $b+4$: longueur DC  
 
 \enskip
 
 $4 \times x$: aire du
 rectangle DEFG 
 
 \enskip
 
 $(a+b+4) \times 2$: p�rim�tre du rectangle ABCD

\enskip

 $a-x$: longueur AE 

\enskip

$(4+x)\times 2$: p�rim�tre du
 rectangle DEFG 

\enskip

 $a \times (b+4)$: aire du rectangle ABCD 


\enskip

$a \times (b+4)-4 \times x$: aire de ABCGFE (partie color�e).
 
 
 \bigskip
 
 \ul{Exercice 4:}
 
 \begin{enumerate}
   \item $\widehat{EDF}=180-29-61=90$� donc le triangle DEF est rectangle en D.
   \item DEF est un triangle rectangle en D. [DJ] est la m�diane issue
   de l'angle droit (car J est le milieu de [EF]) et EF=7cm.
   
   D'apr�s la propri�t� ``si un triangle est rectangle alors la m�diane issue
   de l'angle droit a pour longueur la moiti� de celle de son hypot�nuse'', on
   peut en d�duire que DJ=3,5 cm.
   
 \end{enumerate}
 
 
  \bigskip
 
 \ul{Exercice 5:}
 
 
 \begin{enumerate}
   \item LOU est un triangle rectangle en L et M est le milieu de son
   hypot�nuse. D'apr�s la propri�t� ``si un triangle est rectangle alors le
   centre de son cercle circonscrit est le milieu de son hypot�nuse'', on  en
   d�duit que M est le centre du cercle circonscrit de diam�tre [OU].
   \item Par le m�me raisonnnement, on en d�duit que M est le centre du cercle
   circonscrit du triangle KOU de diam�tre [OU]. Donc les points O, K, L et M
   appartiennent au m�me cercle.
 \end{enumerate}
 
 
 \bigskip
 
 \ul{Exercice 6:}
 
 \begin{enumerate}
   \item ABC est un triangle inscrit dans un cercle de diam�tre [BC]. D'apr�s
   la propri�t� ``si un triangle est inscrit dans un cercle de diam�tre l'un de
   ses c�t�s alors ce triangle est rectangle et admet ce c�t� pour
   hypot�nuse'', on en d�duit que ABC est rectangle en A.
   \item Le triangle ABC est rectangle en A. D'apr�s le th�or�me de Pythagore,
   on a: $BC^2=AB^2+AC^2$
   
   $8^2=AB^2+5^2$
   
   $64=AB^2+25$
   
   $AB^2=64-25=39$  \quad AB est une mesure positive donc $AB=\sqrt{39}\approx
   6,2$ cm.
   
   \item Le triangle ABC est rectangle en A. On a:
   $cos(\widehat{ACB})=\dfrac{AC}{BC}$. En rempla�ant par les valeurs
   num�riques, on trouve: $cos(\widehat{ACB})=\dfrac{5}{8}$.
   
   On en d�duit  $\widehat{ACB}=arccos(5 \div 8) \approx 51$�.
 \end{enumerate}
\end{document}
