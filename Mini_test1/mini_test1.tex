\documentclass{article}

\usepackage[francais]{babel}
\usepackage[T1]{fontenc}
\usepackage[latin1]{inputenc}
\usepackage[left=1cm, right=1cm, top=6mm, bottom=6mm]{geometry}
\usepackage{float}
\usepackage{graphicx}
\usepackage{array}
\usepackage{multirow}
\usepackage{amsmath, amssymb, mathrsfs}

\begin{document}

\begin{flushleft}
NOM PRENOM: \ldots \ldots \ldots \ldots \ldots \ldots \ldots \ldots \ldots

\bigskip
\end{flushleft}
\begin{center}
{\fbox{$2^{de}5$ \qquad \qquad \textbf{\Large{Contr�le de cours 1 (groupe 1)}}
\qquad \qquad 10/10/2008}}
\end{center}



\bigskip
\textbf{Exercice 1:} Ranger les nombres suivants dans l'ordre croissant (on ne
demande pas la justification):
\begin{center}
$\dfrac{17}{8}$  \quad \quad $\dfrac{9}{4}$ \quad  \quad $2$ \quad
\quad $\dfrac{15}{8}$
\end{center}
\medskip
\quad \ldots \ldots \ldots \ldots \ldots \ldots \ldots \ldots \ldots \ldots
\ldots \ldots \ldots \ldots \ldots \ldots \ldots \ldots \ldots \ldots \ldots
\ldots \ldots \ldots \ldots \ldots \ldots \ldots \ldots \ldots \ldots \ldots
\ldots \ldots 

\bigskip
\textbf{Exercice 2:} Ranger les nombres suivants dans l'ordre croissant (on ne
demande pas la justification):
\begin{center}
$1$ \quad \quad $0,6$ \quad \quad $\dfrac{10}{6}^{2}$ \quad \quad $(0,6)^{2}$
\quad \quad  $\dfrac{10}{6}$
\end{center}
\medskip

\quad \ldots \ldots \ldots \ldots \ldots \ldots \ldots \ldots \ldots \ldots
\ldots \ldots \ldots \ldots \ldots \ldots \ldots \ldots \ldots \ldots \ldots
\ldots \ldots \ldots \ldots \ldots \ldots \ldots \ldots \ldots \ldots \ldots
\ldots \ldots 

\bigskip 
\textbf{Exercice 3:} Compl�ter le tableau ci-dessous en cochant la bonne
r�ponse. \textit{Attention: pour chacune des questions suivantes une r�ponse correcte donne $1$ point; une r�ponse incorrecte donne $-0,5$; une
  absence de r�ponse donne $0$ point}.
  \medskip
  \begin{center}
  \begin{tabular}{|m{10cm}|c|}
 \hline Quand on �l�ve un nombre positif au carr�, on obtient un r�sultat plus 
grand que le nombre de d�part & $\circ$ toujours \quad $\circ$ parfois \quad
$\circ$ jamais \\[4mm]
 \hline Soit $x$ un nombre r�el. On a: $1+x^{2} \geqslant 1$ & $\circ$
toujours \quad $\circ$ parfois \quad $\circ$ jamais  \\[2mm]
 \hline  Soit $x$ un r�el v�rifiant l'in�quation $-3x+5 \leqslant 2$ alors $x
 \leqslant 1$ & $\circ$ Vrai \quad $\circ$ Faux  \\[2mm]
 \hline $\sqrt{8}+3<\sqrt{10}+ \pi$ & $\circ$ Vrai \quad $\circ$ Faux  \\[2mm]
 \hline
 \end{tabular}
  \end{center}


\bigskip
\bigskip
\begin{flushleft}
NOM PRENOM: \ldots \ldots \ldots \ldots \ldots \ldots \ldots \ldots \ldots

\bigskip
\end{flushleft}
\begin{center}
{\fbox{$2^{de}5$ \qquad \qquad \textbf{\Large{Contr�le de cours 1 (groupe 2)}}
\qquad \qquad 10/10/2008}}
\end{center}



\bigskip
\textbf{Exercice 1:} Ranger les nombres suivants dans l'ordre croissant (on ne
demande pas la justification):
\begin{center}
$\dfrac{20}{9}$  \quad \quad $\dfrac{7}{3}$ \quad  \quad $2$ \quad
\quad $\dfrac{19}{9}$
\end{center}
\medskip
\quad \ldots \ldots \ldots \ldots \ldots \ldots \ldots \ldots \ldots \ldots
\ldots \ldots \ldots \ldots \ldots \ldots \ldots \ldots \ldots \ldots \ldots
\ldots \ldots \ldots \ldots \ldots \ldots \ldots \ldots \ldots \ldots \ldots
\ldots \ldots 

\bigskip
\textbf{Exercice 2:} Ranger les nombres suivants dans l'ordre croissant (on ne
demande pas la justification):
\begin{center}
$0,3$ \quad \quad $1$ \quad \quad $\dfrac{10}{3}^{2}$ \quad \quad $(0,3)^{2}$
\quad \quad  $\dfrac{10}{3}$
\end{center}
\medskip

\quad \ldots \ldots \ldots \ldots \ldots \ldots \ldots \ldots \ldots \ldots
\ldots \ldots \ldots \ldots \ldots \ldots \ldots \ldots \ldots \ldots \ldots
\ldots \ldots \ldots \ldots \ldots \ldots \ldots \ldots \ldots \ldots \ldots
\ldots \ldots 

\bigskip
\textbf{Exercice 3:} Compl�ter le tableau ci-dessous en cochant la bonne
r�ponse. \textit{Attention: pour chacune des questions suivantes une r�ponse correcte donne $1$ point; une r�ponse incorrecte donne $-0,5$; une
  absence de r�ponse donne $0$ point}.
  \medskip
  \begin{center}
  \begin{tabular}{|m{10cm}|c|}
  \hline Soit $x$ un nombre r�el. On a: $3+x^{2} \geqslant 3$ & $\circ$
toujours \quad $\circ$ parfois \quad $\circ$ jamais  \\[2mm]
 \hline Quand on �l�ve un nombre positif au carr�, on obtient un r�sultat plus 
grand que le nombre de d�part & $\circ$ toujours \quad $\circ$ parfois \quad
$\circ$ jamais \\[4mm]
 
 \hline  Soit $x$ un r�el v�rifiant l'in�quation $-2x+8 \leqslant 4$ alors $x
 \leqslant 2$ & $\circ$ Vrai \quad $\circ$ Faux  \\[2mm]
 \hline $\sqrt{6}+3,1<\sqrt{7}+ \pi$ & $\circ$ Vrai \quad $\circ$ Faux  \\[2mm]
 \hline
 \end{tabular}
  \end{center}
\end{document}
