\documentclass[12pt, twoside]{article}
\usepackage[francais]{babel}
\usepackage[T1]{fontenc}
\usepackage[latin1]{inputenc}
\usepackage[left=5mm, right=5mm, top=5mm, bottom=5mm]{geometry}
\usepackage{float}
\usepackage{graphicx}
\usepackage{array}
\usepackage{multirow}
\usepackage{amsmath,amssymb,mathrsfs}
\usepackage{soul}
\usepackage{textcomp}
\usepackage{eurosym}
 \usepackage{variations}
\usepackage{tabvar}

\pagestyle{empty}
\begin{document}


\begin{center}
\fbox{Correction du devoir maison 2}
\end{center}



\ul{Exercice 3}:

\begin{enumerate}
  
  \item  sch�ma:
  
  \bigskip
  
  \bigskip
  
  \item  Le triangle ABC est rectangle en B. D'apr�s le th�or�me de Pythagore, on a:
  
  $AC^2=AB^2+BC^2$
  
  $AC^2=4^2+2,5^2$
  
  $AC^2=16+6,25$
  
  $AC^2=22,25$
  
  
  AC est une longueur positive
  donc $AC=\sqrt{22,25} \approx 4,7$.
  
  
  \medskip
  
 Le triangle DBC est rectangle en B. D'apr�s le th�or�me de Pythagore, on a:
  
  $DC^2=DB^2+BC^2$
  
  $DC^2=6^2+2,5^2$
  
  $DC^2=36+6,25$
  
  $DC^2=42,25$
  
  
  DC est une longueur positive
  donc $DC=\sqrt{42,25}=6,5$.
  
  
  \medskip
  
  La longueur totale de l'arbre est: AC+CD $\approx $ 4,7+6,5=11,2
  
\end{enumerate}

\bigskip


\ul{Exercice 4}:

\begin{enumerate}
  \item L'�chelle $\dfrac{1}{200}$ signifie que 1cm sur mon dessin repr�sente
  200cm sur le terrain. Je peux faire un tableau de proportionnalit�:
  
  \enskip
  
  \begin{tabular}{cc}
\begin{minipage}{13cm}
\begin{tabular}{|c|c|c|c|c|}
\hline
dimension sur le sch�ma (en cm) & 1 &  \ldots & \dots &  \dots \\
\hline
dimension r�elle du terrain (en cm) & 200 & 1200 & 900 & 1500 \\
\hline
\end{tabular}
\end{minipage}
&
\begin{minipage}{5cm}

12m=1200cm ; 9m=900cm  et 15m=1500cm




\end{minipage}


\end{tabular} 

\medskip

$\dfrac{1 \times 1200}{200}=6$ \quad $\dfrac{1 \times 900}{200}=4,5$ \quad
$\dfrac{1 \times 1500}{200}=7,5$ 

\enskip

Je dois donc construire un triangle DEF avec
DE= 6cm; EF=4,5 cm et DF=7,5 cm. 


\item Dans le triangle DEF, le plus long c�t� est [DF]. Donc on calcule s�par�ment
$DF^2$ et $EF^2+ED^2$:

\begin{center}
\begin{tabular}{ccl}
$DF^2=7,5^2$  &  \qquad \qquad \qquad \qquad &$EF^2+ED^2=4,5^2+6^2$ \\
$DF^2=56,25$  &  \qquad \qquad& $EF^2+ED^2=20,25+36$\\
 \quad  &   \qquad \qquad &$EF^2+ED^2=56,25$\\   
\end{tabular}
  \end{center}


On constate que  $DF^2= EF^2+ED^2$. Donc d'apr�s la r�ciproque du th�or�me de
Pythagore, le triangle DEF est rectangle en E.

\item  Calculons l'aire du triangle rectangle DEF (on utilise les mesures
r�elles).

  $\mathcal{A}_{DEF}=\dfrac{DE \times EF}{2}=\dfrac{ 12 \times 9}{2}=54$
 \qquad Le parc a une aire de 54 $m^2$.
\end{enumerate}






\bigskip


\ul{Exercice 5}:

\begin{enumerate}
  \item   25 \% de 40: $\dfrac{25}{100} \times 40=10$ \quad La lampe a une r�duction de
10 euros donc elle co�te 30 \euro (40-10=30)


20 \% de 30: $\dfrac{20}{100}\times 30 =6$ \quad La lampe a une augmentation de
6 euros donc elle co�te 36 \euro (30+6=36)  

\item 
\begin{tabular}{cc}
\begin{minipage}{8cm}
\begin{tabular}{|c|c|c|}
\hline
prix initial en euros & 40 & 100 \\
\hline
r�duction en euros & 4 & \ldots \\
\hline
\end{tabular}
\end{minipage}
&
\begin{minipage}{10cm}
Montant total de la r�duction: 40-36=4

\enskip


$\dfrac{4 \times 100}{40}=10$ 

\enskip

La r�duction de 6 \euro repr�sente 10 \%
du montant initial. 
\end{minipage}


\end{tabular}
\end{enumerate}

\end{document}
