\documentclass[12pt, twoside]{article}
\usepackage[francais]{babel}
\usepackage[T1]{fontenc}
\usepackage[latin1]{inputenc}
\usepackage[left=8mm, right=8mm, top=8mm, bottom=8mm]{geometry}
\usepackage{float}
\usepackage{graphicx}
\usepackage{array}
\usepackage{multirow}
\usepackage{amsmath,amssymb,mathrsfs}
\pagestyle{empty}
\begin{document}

\section*{\center{Bilan intervalles}}




\bigskip
\begin{center}
\begin{tabular}{|m{9cm}|m{10cm}|}
\hline
\textbf{Ce que je dois savoir} & \textbf{Ce que je dois savoir faire} \\
\hline
\begin{itemize}
  \item[$\bullet$] Je connais la d�finition d'un intervalle et je comprends le
  sens des crochets.
  \item [$\bullet$] Je connais la d�finition d'une union et d'une intersection
  d'intervalles.
\end{itemize}
&

\enskip
\begin{itemize}
  \item[$\bullet$] Je sais reconna�tre un intervalle � partir d'un encadrement
  ou d'une repr�sentation sur la droite gradu�e.
  \item[$\bullet$] Je peux repr�senter la droite gradu�e et �crire des
  encadrements � partir d'un intervalle.
  \item[$\bullet$] Je suis capable de d�terminer l'intersection de plu
 sieurs intervales.
 \item[$\bullet$] je suis capable de d�terminer l'union de plusieurs
 intervalles. \end{itemize} \\
\hline

\end{tabular}
\end{center}

\section*{\center{Bilan intervalles}}




\bigskip
\begin{center}
\begin{tabular}{|m{9cm}|m{10cm}|}
\hline
\textbf{Ce que je dois savoir} & \textbf{Ce que je dois savoir faire} \\
\hline
\begin{itemize}
  \item[$\bullet$] Je connais la d�finition d'un intervalle et je comprends le
  sens des crochets.
  \item [$\bullet$] Je connais la d�finition d'une union et d'une intersection
  d'intervalles.
\end{itemize}
&

\enskip
\begin{itemize}
  \item[$\bullet$] Je sais reconna�tre un intervalle � partir d'un encadrement
  ou d'une repr�sentation sur la droite gradu�e.
  \item[$\bullet$] Je peux repr�senter la droite gradu�e et �crire des
  encadrements � partir d'un intervalle.
  \item[$\bullet$] Je suis capable de d�terminer l'intersection de plu
 sieurs intervales.
 \item[$\bullet$] je suis capable de d�terminer l'union de plusieurs
 intervalles. \end{itemize} \\
\hline

\end{tabular}
\end{center}


\section*{\center{Bilan intervalles}}




\bigskip
\begin{center}
\begin{tabular}{|m{9cm}|m{10cm}|}
\hline
\textbf{Ce que je dois savoir} & \textbf{Ce que je dois savoir faire} \\
\hline
\begin{itemize}
  \item[$\bullet$] Je connais la d�finition d'un intervalle et je comprends le
  sens des crochets.
  \item [$\bullet$] Je connais la d�finition d'une union et d'une intersection
  d'intervalles.
\end{itemize}
&

\enskip
\begin{itemize}
  \item[$\bullet$] Je sais reconna�tre un intervalle � partir d'un encadrement
  ou d'une repr�sentation sur la droite gradu�e.
  \item[$\bullet$] Je peux repr�senter la droite gradu�e et �crire des
  encadrements � partir d'un intervalle.
  \item[$\bullet$] Je suis capable de d�terminer l'intersection de plu
 sieurs intervales.
 \item[$\bullet$] je suis capable de d�terminer l'union de plusieurs
 intervalles. \end{itemize} \\
\hline

\end{tabular}
\end{center}




\end{document}
