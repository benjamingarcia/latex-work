\documentclass[12pt, twoside]{article}
\usepackage[francais]{babel}
\usepackage[T1]{fontenc}
\usepackage[latin1]{inputenc}
\usepackage[left=8mm, right=8mm, top=8mm, bottom=8mm]{geometry}
\usepackage{float}
\usepackage{graphicx}
\usepackage{array}
\usepackage{multirow}
\usepackage{amsmath,amssymb,mathrsfs}
\pagestyle{empty}
\begin{document}

\section*{\center{Bilan nombres relatifs}}




\bigskip

\begin{center}
\begin{tabular}{|m{9cm}|m{9cm}|}
\hline
\textbf{Ce que je dois savoir} & \textbf{Ce que je dois savoir faire} \\
\hline

\enskip


\begin{itemize}
  \item[$\bullet$] Je connais la r�gle pour multiplier deux nombres relatifs.
  
  \item [$\bullet$] Je connais la r�gle pour multiplier plusieurs nombres
  relatifs.
  
  \item [$\bullet$] Je connais la r�gle pour diviser deux nombres relatifs.

  \item [$\bullet$] Je connais les r�gles pour additionner deux nombres relatifs
  de m�me signe ou de signes contraires.
  
  \item [$\bullet$] Je connais la r�gle pour soustraire deux nombres relatifs.
  
  \item [$\bullet$] Je connais l'ordre de priorit� des diff�rentes op�rations.
  
  \item [$\bullet$] Je connais le vocabulaire associ� aux diff�rentes
  op�rations.
\end{itemize}
&

\enskip
\begin{itemize}
  \item[$\bullet$] Je sais multiplier des nombres relatifs.

\enskip

  \item[$\bullet$] Je sais diviser des nombres relatifs.

 \enskip
  
      
  \item[$\bullet$] Je sais appliquer la r�gle pour additionner et soustraire
  des nombres relatifs.
  
  \enskip
  
 \item[$\bullet$] Je sais effectuer des calculs sur les nombres relatifs en
 respectant les priorit�s de calculs et en appliquant les diff�rentes r�gles. 
 
 \enskip
 
  \item[$\bullet$] Je sais d�crire un calcul � l'aide du vocabulaire.
  
  \enskip
  
  \item[$\bullet$] Je sais �crire une expression litt�rale � partir d'une
  phrase utilisant le vocabulaire associ� aux diff�rentes op�rations.
 
 
 \end{itemize} \\
\hline

\end{tabular}
\end{center}

\bigskip

\bigskip

\section*{\center{Bilan nombres relatifs}}




\bigskip

\begin{center}
\begin{tabular}{|m{9cm}|m{9cm}|}
\hline
\textbf{Ce que je dois savoir} & \textbf{Ce que je dois savoir faire} \\
\hline

\enskip


\begin{itemize}
  \item[$\bullet$] Je connais la r�gle pour multiplier deux nombres relatifs.
  
  \item [$\bullet$] Je connais la r�gle pour multiplier plusieurs nombres
  relatifs.
  
  \item [$\bullet$] Je connais la r�gle pour diviser deux nombres relatifs.

  \item [$\bullet$] Je connais les r�gles pour additionner deux nombres relatifs
  de m�me signe ou de signes contraires.
  
  \item [$\bullet$] Je connais la r�gle pour soustraire deux nombres relatifs.
  
  \item [$\bullet$] Je connais l'ordre de priorit� des diff�rentes op�rations.
  
  \item [$\bullet$] Je connais le vocabulaire associ� aux diff�rentes
  op�rations.
\end{itemize}
&

\enskip
\begin{itemize}
  \item[$\bullet$] Je sais multiplier des nombres relatifs.

\enskip

  \item[$\bullet$] Je sais diviser des nombres relatifs.

 \enskip
  
      
  \item[$\bullet$] Je sais appliquer la r�gle pour additionner et soustraire
  des nombres relatifs.
  
  \enskip
  
 \item[$\bullet$] Je sais effectuer des calculs sur les nombres relatifs en
 respectant les priorit�s de calculs et en appliquant les diff�rentes r�gles. 
 
 \enskip
 
  \item[$\bullet$] Je sais d�crire un calcul � l'aide du vocabulaire.
  
  \enskip
  
  \item[$\bullet$] Je sais �crire une expression litt�rale � partir d'une
  phrase utilisant le vocabulaire associ� aux diff�rentes op�rations.
 
 
 \end{itemize} \\
\hline

\end{tabular}
\end{center}
\end{document}
