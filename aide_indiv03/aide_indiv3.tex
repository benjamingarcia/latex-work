\documentclass[12pt, twoside]{article}
\usepackage[francais]{babel}
\usepackage[T1]{fontenc}
\usepackage[latin1]{inputenc}
\usepackage[left=8mm, right=8mm, top=8mm, bottom=8mm]{geometry}
\usepackage{float}
\usepackage{graphicx}
\usepackage{array}
\usepackage{multirow}
\usepackage{amsmath,amssymb,mathrsfs}
\usepackage{soul}


\pagestyle{empty}
\begin{document}

\begin{flushright}
$2^{de}5$
\end{flushright}
\section*{\center{Aide individualis�e: r�vision du contr�le}}

\textit{Travail � faire sans calculatrice}
\subsection*{Exerceice 1}
Ranger les nombres suivants dans l'ordre croissant: $-2$; \enskip
$\dfrac{1}{4}$; \enskip $1$; \enskip $-\dfrac{5}{3}$; \enskip
$\left(\dfrac{1}{4} \right)^{2}$; \enskip $\left(\dfrac{1}{4} \right)^{3}$;
\enskip $\dfrac{-7}{3}$.


\subsection*{Exercice 2}
Soient $I$, $J$, $K$ et $L$ des intervalles de $\mathbb{R}$ tels que:
$I=]-\infty;2]$, $J=[4;7]$, $K=]2;17[$ et $L=]-1;+\infty[$. Donner $I \cap J $,
$I \cap K$,\enskip $I \cap L$,\enskip $K \cap J$,\enskip $L \cap J$ et $K \cap
L$.

\subsection*{Exercice 3} 
Soit $x$ et $y$ deux nombres r�els tel que: $-5<x<-1$ et $3<y<4$.\\
 Encadrer $x+y$,\enskip  $x-y$,\enskip  $y-x+2$,\enskip  $\dfrac{x}{y}$ et
 $\dfrac{y^{2}-x}{7}$.

\subsection*{Exercice 4}
\begin{enumerate}
  \item Comparer $\dfrac{3}{7}$ et $\dfrac{11}{9}$.
  \item Comparer $\sqrt{1+2\sqrt{6}}$ et $\sqrt{2}+\sqrt{3}$.
\end{enumerate}

\end{document}
