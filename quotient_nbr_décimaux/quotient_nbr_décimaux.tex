\documentclass[12pt, twoside]{article}
\usepackage[francais]{babel}
\usepackage[T1]{fontenc}
\usepackage[latin1]{inputenc}
\usepackage[left=5mm, right=5mm, top=5mm, bottom=5mm]{geometry}
\usepackage{float}
\usepackage{graphicx}
\usepackage{array}
\usepackage{multirow}
\usepackage{amsmath,amssymb,mathrsfs} 
\usepackage{soul}
\usepackage{textcomp}
\usepackage{eurosym}
\usepackage{lscape}
 \usepackage{variations}
\usepackage{tabvar}
 

\begin{document}


\begin{center}

\textbf{\ul{Quotient de deux nombres d�cimaux}}

\end{center}

\ul{Exercice 1}: Effectuer � la main les divisions suivantes:


$8,64 \div 3,6$ \quad et \quad $81,3 \div 5,42$.

\enskip

\ul{Exercice 2}: Un r�ti de 1,260kg co�te 18,90 \euro. Quel est le prix d'un
kilo?

\enskip

\ul{Exercice 3}: L'aire d'une feuille de classeur est �gale � $623,7cm^2$ et sa
largeur mesure $29,7cm$. 

Calculer sa longueur.

\medskip

\begin{center}

\textbf{\ul{Quotient de deux nombres d�cimaux}}

\end{center}

\ul{Exercice 1}: Effectuer � la main les divisions suivantes:


$8,64 \div 3,6$ \quad et \quad $81,3 \div 5,42$.

\enskip

\ul{Exercice 2}: Un r�ti de 1,260kg co�te 18,90 \euro. Quel est le prix d'un
kilo?

\enskip

\ul{Exercice 3}: L'aire d'une feuille de classeur est �gale � $623,7cm^2$ et sa
largeur mesure $29,7cm$. 

Calculer sa longueur.

\medskip

\begin{center}

\textbf{\ul{Quotient de deux nombres d�cimaux}}

\end{center}

\ul{Exercice 1}: Effectuer � la main les divisions suivantes:


$8,64 \div 3,6$ \quad et \quad $81,3 \div 5,42$.

\enskip

\ul{Exercice 2}: Un r�ti de 1,260kg co�te 18,90 \euro. Quel est le prix d'un
kilo?

\enskip

\ul{Exercice 3}: L'aire d'une feuille de classeur est �gale � $623,7cm^2$ et sa
largeur mesure $29,7cm$. 

Calculer sa longueur.

\medskip


\begin{center}

\textbf{\ul{Quotient de deux nombres d�cimaux}}

\end{center}

\ul{Exercice 1}: Effectuer � la main les divisions suivantes:


$8,64 \div 3,6$ \quad et \quad $81,3 \div 5,42$.

\enskip

\ul{Exercice 2}: Un r�ti de 1,260kg co�te 18,90 \euro. Quel est le prix d'un
kilo?

\enskip

\ul{Exercice 3}: L'aire d'une feuille de classeur est �gale � $623,7cm^2$ et sa
largeur mesure $29,7cm$. 

Calculer sa longueur.

\medskip

\begin{center}

\textbf{\ul{Quotient de deux nombres d�cimaux}}

\end{center}

\ul{Exercice 1}: Effectuer � la main les divisions suivantes:


$8,64 \div 3,6$ \quad et \quad $81,3 \div 5,42$.

\enskip

\ul{Exercice 2}: Un r�ti de 1,260kg co�te 18,90 \euro. Quel est le prix d'un
kilo?

\enskip

\ul{Exercice 3}: L'aire d'une feuille de classeur est �gale � $623,7cm^2$ et sa
largeur mesure $29,7cm$. 

Calculer sa longueur.





\end{document}
