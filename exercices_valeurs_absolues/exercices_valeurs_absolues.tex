\documentclass[12pt, twoside]{article}
\usepackage[francais]{babel}
\usepackage[T1]{fontenc}
\usepackage[latin1]{inputenc}
\usepackage[left=9mm, right=9mm, top=8mm, bottom=8mm]{geometry}
\usepackage{float}
\usepackage{graphicx}
\usepackage{array}
\usepackage{multirow}
\usepackage{amsmath,amssymb,mathrsfs}
\usepackage{soul}
\pagestyle{empty}
\begin{document}


\section*{\center{Valeurs absolues}}

\subsection*{Exercice 1}

Calculer: $|(-5)^{3}|$; $|2^{2}-2^{3}|$; $|\pi-3|$; 
$|\sqrt{2}-\dfrac{577}{408}|$; $|-3|+|5|$ et $|-3+5|$.\\
Que remarque t'on sur les deux derniers calculs?

\subsection*{Exercice 2}

Traduire les in�galit�s ci-dessous � l'aide d'intervalles:\\
$|x| \leqslant5$;\quad $|x-3| <5$;\quad $|x-3| \geqslant 5$ et $|2x-3| \leqslant
5$.

\subsection*{Exercice 3}
Traduire � l'aide de valeurs absolues: \\
$x \in [2;4]$; \quad $x \in [-2;3]$; \quad $x \in [\sqrt{2};\dfrac{3}{2}]$. \\
(\textit{indication}: utiliser la m�thode vue dans le cours.)

\subsection*{Exercice 4}
Faire l'exercice $130$ p $30$ du livre.

\section*{\center{Valeurs absolues}}

\subsection*{Exercice 1}

Calculer: $|(-5)^{3}|$; $|2^{2}-2^{3}|$; $|\pi-3|$; 
$|\sqrt{2}-\dfrac{577}{408}|$; $|-3|+|5|$ et $|-3+5|$.\\
Que remarque t'on sur les deux derniers calculs?

\subsection*{Exercice 2}

Traduire les in�galit�s ci-dessous � l'aide d'intervalles:\\
$|x| \leqslant5$;\quad $|x-3| <5$;\quad $|x-3| \geqslant 5$ et $|2x-3| \leqslant
5$.

\subsection*{Exercice 3}
Traduire � l'aide de valeurs absolues: \\
$x \in [2;4]$; \quad $x \in [-2;3]$; \quad $x \in [\sqrt{2};\dfrac{3}{2}]$. \\
(\textit{indication}: utiliser la m�thode vue dans le cours.)

\subsection*{Exercice 4}
Faire l'exercice $130$ p $30$ du livre.
\end{document}
