\documentclass[12pt, twoside]{article}
\usepackage[francais]{babel}
\usepackage[T1]{fontenc}
\usepackage[latin1]{inputenc}
\usepackage[left=7mm, right=7mm, top=7mm, bottom=7mm]{geometry}
\usepackage{float}
\usepackage{graphicx}
\usepackage{array}
\usepackage{multirow}
\usepackage{amsmath,amssymb,mathrsfs}
\usepackage{soul}
\usepackage{textcomp}
\usepackage{eurosym}
 \usepackage{variations}
\usepackage{tabvar}

\begin{document}
 
\begin{center}
\fbox{\textbf{Exercices: Priorit�s et conventions}}
\end{center}

\bigskip

\ul{\textbf{Activit� 1:}} On consid�re les 2 nombres A et B ci-dessous. Les
calculer astucieusement, c'est-�-dire en regroupant convenablement certains
termes pour faciliter les calculs.

\begin{center}
$A=4,5+2,6+3+5,5+7,4$  \qquad $B=0,25+1,2+0,75+0,8$
\end{center}

A-t-on le droit de changer l'ordre des termes d'une somme? 


\begin{center}
$C=19-6-13+8$ 
\end{center}

A-t-on le droit de changer l'ordre des termes d'une diff�rence? 

\bigskip

\medskip

\ul{\textbf{Activit� 2:}} Benjamin a achet� 4 bouteilles de jus de fruits �
1,55 euro l'une. Avant cet achat, il disposait de 12 euros.

\begin{enumerate}
  \item Quelle somme lui reste-t-il apr�s son achat?
  \item Par quelle op�ration a-t-il fallu commencer le calcul?
  \item Ecrire l'expression qui permet de calculer la somme restant � Benjamin.
\end{enumerate}

\bigskip

\medskip


\ul{\textbf{Exercices 1,2 et 3:}} \textit{(n�1, 2 et 3 p14 s�samath)}

\begin{enumerate}
  \item Reproduis les deux tableaux ci-dessous et associe chaque suite
  d'op�rations � son r�sultat:
  
  \enskip
  
  
  \begin{tabular}{cc}
  \begin{minipage}{9cm}


 $3+2 \times 5$ 
 
 
   $15 \times 4 \div 3$ 
    
   $19-4 \times 4$ 
   
   
  $50-7 \times 4 +9$ 
   
   
   $17,7-11,7+0.3 \times 2$ 

  \end{minipage}
  &
  \begin{minipage}{9cm}
  
   $3$ 
   
   
   $6,6$ 
   
   $13$  
   
   $31$ 
  
  $20$ 

  \end{minipage}
  \end{tabular}

\bigskip




\begin{tabular}{cc}
\begin{minipage}{9cm}
\item Effectue les calculs suivants en soulignant �

chaque �tape le calcul en cours:
 

\enskip


$A=41-12-5$

$B=24,1-0,7+9,4$

$C=35 \div 7-3$

$D=24 \div2 \div 3$

$E=58-14+21 \div 3-1$

$F= 6 \times 8-3+9 \times 5$
\end{minipage}
&
\begin{minipage}{9cm}
\item Effectue les calculs suivants en soulignant � chaque �tape le calcul en
cours:

\enskip

$G=53-(12+21)$

$H=2+(4,7-0,3) \times 10$

$I=15+25 \times 4-13$

$J=31-[8-(0,8+2,1)]$

$K=27-[9+2 \times 0,5]$

$L=(39+10) \times (18-11)$
\end{minipage}
\end{tabular}

\end{enumerate}

\bigskip

\medskip

\ul{\textbf{Exercice 4:}} \textit{(n�6 p14 s�samath)} Calcule, � la main,
chaque expression puis v�rifie � la calculatrice:
\begin{center}
$A= 12 - \dfrac{0,9 \times 30}{3}$ \qquad \qquad $B=\dfrac{12-5 \times 2}{15+2,5
\times 2}$ \qquad \qquad $C=8 \times 7-3 \times \dfrac{24 \div 3+8}{200 \times
0,02}$
\end{center}

\bigskip

\medskip

\ul{\textbf{Exercice 5:}} Dans les calculs ci-dessous, placer correctement des
parenth�ses pour que l'�galit� soit vraie:

\center{$7+14+3 \times 2=41$ \qquad \qquad $7+14+3 \times 2=48$}
 

\end{document}
