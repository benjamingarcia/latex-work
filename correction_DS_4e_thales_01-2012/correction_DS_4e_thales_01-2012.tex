\documentclass[12pt, twoside]{article}
\usepackage[francais]{babel}
\usepackage[T1]{fontenc}
\usepackage[latin1]{inputenc}
\usepackage[left=5mm, right=5mm, top=5mm, bottom=5mm]{geometry}
\usepackage{float}
\usepackage{graphicx}
\usepackage{array}
\usepackage{multirow}
\usepackage{amsmath,amssymb,mathrsfs}
\usepackage{soul}
\usepackage{textcomp}
\usepackage{eurosym}
 \usepackage{variations}
\usepackage{tabvar}

\pagestyle{empty}
\begin{document}

\begin{center}
\fbox{Correction du devoir surveill� 4}
\end{center}


\enskip


\ul{Exercice 1}: 
Dans le triangle ABC, le point E appartient au segment [AB], le point F
appartient au segment [AC] et les droites (FE) et (BC) sont parall�les. D'apr�s
le th�or�me de Thal�s, on a:

\enskip


$\dfrac{AF}{AC}=\dfrac{AE}{AB}=\dfrac{EF}{BC}$
\qquad \qquad 
On remplace:
 $\dfrac{AF}{9,8}=\dfrac{4,5}{7}=\dfrac{2,7}{BC}$ \quad car $AB= AE+EB=4,5+2,5=7
 $ cm

\enskip


Donc $BC= 7 \times 2,7 \div 4,5=4,2 $ cm  \qquad et \qquad $AF=4,5 \times 9,8
\div 7 = 6,3 $  cm

\bigskip





\ul{Exercice 2}: 

\begin{enumerate}
  \item [2.] Les droites (OP) et (SB) sont toutes les deux perpendiculaires �
  la droite (BT). D'apr�s la propri�t� ``si deux sont perpendicualries � la
  m�me droite alors elles sont parall�les entre elles'', on peut en d�duire que
  les droites (OP) et (SB) sont parall�les.
  \item [3.] 

Dans le triangle BST, le point O appartient au segment [ST], le point P
appartient au segment [TB] et les droites (OP) et (SB) sont parall�les. D'apr�s
le th�or�me de Thal�s, on a:

\enskip

$\dfrac{TO}{TS}=\dfrac{TP}{TB}=\dfrac{OP}{SB}$
\qquad \qquad 
On remplace:
 $\dfrac{1,6}{113,6}=\dfrac{1,4}{SB}$ \quad car $TB= 1,6+112=113,6
 $ m

\enskip

Donc $SB= 1,4 \times 113,6 \div 1,6=99,4$ m  \qquad La falaise a une hauteur de
99,4 m�tres.

\end{enumerate}


\bigskip



\ul{Exercice 3}: 
Dans le triangle ABE, le point D appartient au segment [EB], le point C
appartient au segment [AE] et les droites (CD) et (AB) sont parall�les. D'apr�s
le th�or�me de Thal�s, on a:

\enskip


$\dfrac{EC}{EA}=\dfrac{ED}{EB}=\dfrac{CD}{AB}$
\qquad \qquad 
On remplace:
 $\dfrac{0,4}{2,4}=\dfrac{0,25}{AB}=\dfrac{ED}{EB}$ \quad car
 $EC=EA-AC=2,40-2=0,4 $ m

\enskip


Donc $AB= 2,4 \times 0,25 \div 0,4=1,5 $ m  \qquad Les deux pieds sont s�par�s
de 1,5 m�tre.

\bigskip

\ul{Exercice 4}: 

\begin{enumerate}
  \item Le triangle ADC est rectangle en D. D'apr�s le th�or�me de Pythagore,
  on a: $AC^2=AD^2+DC^2$.
  
  $AC^2=4,8^2+6,4$
  
  $AC^=23,04+40,96=64$
  
  AC est une longueur positive donc $AC=\sqrt{64}=8$cm.
  
  \item Dans le triangle ABC, le point M appartient au segment [AB], le point N
appartient au segment [AC] et les droites (NM) et (BC) sont parall�les. D'apr�s
le th�or�me de Thal�s, on a:

\enskip

$\dfrac{AN}{AC}=\dfrac{AM}{AB}=\dfrac{MN}{BC}$
\qquad \qquad 
On remplace:
 $\dfrac{4}{8}=\dfrac{AM}{10}=\dfrac{3}{BC}$ 
\qquad \qquad
Donc $BC= 3 \times 8 \div 4=6 $ cm

\enskip

  \item Dans le triangle ABC, le plus long c�t� est [AB]. On calcule
  s�paremment $AB^2$ et $AC^2+CB^2$:
  
  $AB^2=10^2=100$ \quad et \quad $AC^2+CB^2=8^2+6^2=64+36=100$
  
  On constate que $AB^2=CA^2+CB^2$ donc d'apr�s la r�ciproque du th�or�me de
  Pythagore, on en d�duit que le triangle ABC est rectangle en C.
  
  \item $\mathcal{A}_{ABCD}=\mathcal{A}_{ABC}+\mathcal{A}_{ACD}$
  
  ABC est un triangle rectangle en C donc $\mathcal{A}_{ABC}=\dfrac{AC
  \times AB}{2}=\dfrac{8 \times 6}{2}=24cm^2$.
  
  
  ADC est un triangle rectangle en D donc $\mathcal{A}_{ADC}=\dfrac{AD
  \times AC}{2}=\dfrac{4,8 \times 6,4}{2}=15,36cm^2$
  
  L'aire totale est: $15,36+ 24 = 39,36 cm^2$
  
\end{enumerate}


\end{document}
