\documentclass[12pt, twoside]{article}
\usepackage[francais]{babel}
\usepackage[T1]{fontenc}
\usepackage[latin1]{inputenc}
\usepackage[left=8mm, right=8mm, top=6mm, bottom=6mm]{geometry}
\usepackage{float}
\usepackage{graphicx}
\usepackage{array}
\usepackage{multirow}
\usepackage{amsmath,amssymb,mathrsfs}
\usepackage{soul}


\pagestyle{empty}
\begin{document}

\section*{\center{Devoir maison 2 (seconde chance)}}


\textit{Devoir � rendre sur feuille double grand format petits carreaux pour le
\ul{samedi 25 octobre 2008}. Toutes les r�ponses seront fournies sur votre
copie double, pour l'exercice 1, rendre le tableau.}

\bigskip
\textbf{Exercice 1:}


\medskip
\begin{tabular}{|m{35mm}|c|m{35mm}|m{35mm}|m{35mm}|}
\hline
Texte & Intervalles & Repr�sentation sur une droite gradu�e & Ensemble
des r�els $x$ v�rifiant: & L'ensemble est-il born�? (oui ou non)\\
\hline
& &  & & \\[7mm]
\hline
& & & $\dfrac{-21}{8}\leqslant x < \dfrac{7}{2}$ & \\[7mm]
\hline
& &  & & \\[7mm]
\hline
l'ensemble des r�els sup�rieurs ou �gaux � $\sqrt{2}$ & & & & \\[7mm]
\hline
& $]4,2;-\infty[$ & & & \\[7mm]
\hline
l'ensemble des r�els compris strictement entre $-1$ et $3$ & & & & \\[7mm]
\hline
& & & $x<-4$ & \\[7mm]
\hline
l'ensemble des r�els sup�rieurs ou �gaux � $-27$ & & & & \\[7mm]
\hline
& $[\pi;+\infty[$ & & & \\[6mm]
\hline
\end{tabular}


\bigskip

\textbf{Exercice 2:}
Un insecticide �lectrique est efficace pour une pi�ce de $15\ m^{3}$ maximum.
On veut chasser les moustiques d'une pi�ce cubique (la pi�ce a la forme d'un
cube). On ne connait pas la taille pr�cise des murs; le c�t� de cette pi�ce est
compris entre $2,97\ m$ et $3,08\ m$.
\begin{enumerate}
  \item Trouver un encadrement du volume de cette pi�ce.
  \item Faut-il acheter un deuxi�me insecticide?
\end{enumerate}


\bigskip


\textbf{Exercice 3:} Le triangle $ABC$ est tel que: \\
$AC=2 \times \sqrt{2}+\sqrt{3}$,\enskip $BC=\sqrt{2}+\sqrt{3}$ \enskip et
$\sqrt{2}+\sqrt{3} \leqslant AB \leqslant 1+ \sqrt{6}$.
\begin{enumerate}
  \item Calculer $AC^{2}$ et $BC^{2}$.
  \item D�terminer un encadrement de $AB^{2}$.
  \item En d�duire que le triangle $ABC$ peut �tre rectangle en $B$ (il es
  conseill� de faire un sch�ma pour rep�rer l'hypot�nuse.)
\end{enumerate}

\bigskip

\textbf{Exercice 4:} Le p�rim�tre d'un rectangle est compris entre $45 \ m$ et
$55 \ m$, sa longueur entre $16 \ m$ et $19 \ m$. Donner un encadrement de
sa largeur not� $l$ et de son aire.


\end{document}
