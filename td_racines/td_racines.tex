\documentclass[12pt, twoside]{article}
\usepackage[francais]{babel}
\usepackage[T1]{fontenc}
\usepackage[latin1]{inputenc}
\usepackage[left=1cm, right=1cm, top=8mm, bottom=8mm]{geometry}
\usepackage{float}
\usepackage{graphicx}
\usepackage{array}
\usepackage{multirow}
\usepackage{amsmath,amssymb,mathrsfs}
\pagestyle{empty}
\begin{document}

\begin{flushright}
$2^{de}5$
\end{flushright}

\section*{\center{TD Racines}}


\subsection*{Rappels}

\textbf{D�finition:} $a$ �tant un nombre \textbf{positif} (ou nul), $\sqrt{a}$
est le nombre positif (ou nul) qui �lev� au carr� donne $a$:  \quad $\sqrt{a}
\times \sqrt{a}=(\sqrt{a})^2=a$ avec $a\geqslant 0$.
 
 \bigskip
 \textbf{R�gles:}
 \begin{enumerate}
   \item \fbox{$\sqrt{a}$ n'est d�finie que si $a$ est un nombre
   \textbf{positif} (ou nul).}
  \medskip 
\item $a$ �tant un nombre positif, il existe $2$ nombres, $\sqrt{a}$ et
$-\sqrt{a}$ qui �lev�s au carr� donnent $a$.
\item $a$ et $b$ �tant des nombres positifs et $b \not =0$, on a:\\
$\sqrt{a} \times \sqrt{b}=\sqrt{ab}$\\ 
$\sqrt{a^{2}}=a$ car $a \geqslant 0$\\ 
$\dfrac{\sqrt{a}}{\sqrt{b}}=\sqrt{\dfrac{a}{b}}$.
 \end{enumerate}
 
 \textit{ATTENTION:} il faut calculer le nombre sous le radical $\sqrt{\   }$
 \textbf{avant} de calculer la racine carr�e!\\
 
 \subsection*{Applications}
 
 
 \textbf{Exercice 1:} ($\diamond$) Calculer $\sqrt{a+b}$ et $\sqrt{a}+\sqrt{b}$
 pour:\\
 $a=1 $ et $b=3$ \\
 $a=9$ et $b=16$\\
 Que remarque t'on?
 
 
 \bigskip
 \textbf{Exercice 2:} ($\diamond$) Ecrire les nombres suivants sous forme
 $a\sqrt{b}$ o� $a$ et $b$ sont des entiers naturels et $b$ le plus petit
 possible: \\
 $\sqrt{27}$, \thinspace $\sqrt{200}$,  \thinspace $\sqrt{8}$, \thinspace
 $\sqrt{75}$, \thinspace $\sqrt{18}$.
 
 
 \bigskip
 \textbf{Exercice 3:} ($\diamond$) R�duire au maximum les nombres
 suivants:
 \begin{enumerate}
   \item [1)]$2 \sqrt{3} - 4 \sqrt{3} + \sqrt{27}$
   \item [2)]$\sqrt{45} + 3 \sqrt{5} - 3 \sqrt{20}$ 
   \item [3)]$\dfrac{3  \sqrt{5} \times \sqrt{12}}{2 \sqrt{15}}$
   \item [4)]$(\sqrt{3})^{2}- (\dfrac{2 \sqrt{2}}{3})^{2}$
 \end{enumerate} 
 
 
 \bigskip
 \begin{tabular}{c|c}
\begin{minipage}{9cm}
	\textbf {Exercice 4:} ($\diamond \diamond$) D�velopper chaque nombre:
 \begin{enumerate}
   \item[1)] $(4- \sqrt{2})^{2}$
   \item [2)]$(3 \sqrt{2} - \sqrt{3})^{2}$
   \item [3)]$(4+ 2 \sqrt{7})(6 + \sqrt{7})$
   \item [4)]$(3 + \sqrt{5})(3- \sqrt{5}) + (1- \sqrt{2})^{2}$ 
 \end{enumerate}
\end{minipage}
&
\begin{minipage}{9cm}
\textbf{Exercice 5:} ($\diamond \diamond \diamond$) D�terminer la racine
 carr�e de:
 \begin{enumerate}
   \item [1)] $3^{4} \times  5^{2}$
   \item [2)] $9 \times 10^{6}$
   \item [3)] $16 \times 10^{-8}$
   \item [4)] $8,1 \times 10^{9}$
 \end{enumerate}	
\end{minipage} 
\end{tabular}
 
 
 
 \bigskip
 \textbf{Exercice 6:} ($\diamond \diamond \diamond  \diamond$) Le nombre
 $\dfrac{1}{\sqrt{2} -1}- \dfrac{1}{\sqrt{2} + 1}$ est-il entier?

 
\end{document}
