\documentclass[12pt, twoside]{article}
\usepackage[francais]{babel}
\usepackage[T1]{fontenc}
\usepackage[latin1]{inputenc}
\usepackage[left=5mm, right=5mm, top=5mm, bottom=5mm]{geometry}
\usepackage{float}
\usepackage{graphicx}
\usepackage{array}
\usepackage{multirow}
\usepackage{amsmath,amssymb,mathrsfs} 
\usepackage{soul}
\usepackage{textcomp}
\usepackage{eurosym}
\usepackage{lscape}
 \usepackage{variations}
\usepackage{tabvar}
 
\pagestyle{empty}


\begin{document}

\begin{center}
\Large{\ul{\textbf{In�quations}}}
\end{center}


\section{Vocabulaire}

\ul{D�finitions:}

$\bullet$ Une \textbf{in�quation � une inconnue} $x$ est une in�galit� qui est
soit vrai soit fausse selon les valeurs de $x$.

$\bullet$ Les valeurs de $x$ pour lesquelles l'in�galit� est vraie sont les
\textbf{solutions de l'in�quation}.


$\bullet$ \textbf{R�soudre une in�quation} c'est trouver \ul{toutes} ses
solutions.

\bigskip

\ul{Exemple}: $3x-3 < x+2$ est une in�quation d'inconnue $x$.

Pour $x=0$, $3 \times 0-3=0-3=-3$ et $0+2=2$; $-3<2$ donc 0 est \ldots \ldots
\ldots \ldots \ldots \ldots


Pour $x=4$, $3 \times 4 -3=12-3=9$ et $4+2=6$ ; $9>6$ donc 4 \ldots \ldots
\ldots \ldots \ldots \ldots


\bigskip

\textit{ex 43 p 138}

\section{M�thode pour r�soudre une in�quation}

\ul{But:} se ramener � une in�quation du type $ax<b$ ou $ax \leqslant b$ ou
$ax>b$ ou $ax \geqslant b$

\bigskip

Pour cela, on utilisera deux propri�t�s:

\enskip


\fbox{
\begin{minipage}{18cm}
\ul{Propri�t�:} Quand on additionne (ou soustrait) un m�me nombre aux deux
membres d'une in�galit�, on ne change pas son sens.


\enskip

\ul{Propri�t�:} Quand on multiplie (ou divise)  les deux membres d'une in�galit�
par un m�me nombre non nul

$\bullet$ positif, on ne change pas son sens;

$\bullet$ n�gatif, on change son sens.
\end{minipage}}


\bigskip


\ul{Exemple 1:} R�soudre l'in�quation $3x-3< x+2$


\bigskip

\bigskip

\bigskip

\bigskip

\bigskip

\bigskip

\bigskip

\bigskip

\bigskip

\bigskip

\bigskip

\bigskip

\ul{Exemple 2:} R�soudre l'in�quation $3x-3 \geqslant 5x+2$

\bigskip

\bigskip

\bigskip

\bigskip

\bigskip

\bigskip

\bigskip

\bigskip

\bigskip



\bigskip

\bigskip

\bigskip

\bigskip

\bigskip

\bigskip

\bigskip

\bigskip

\bigskip



\textit{ex 45 p 138; ex 51 p 139; ex 52 a) et c) p 139; ex 54 a) et b) p 139}

\textit{ex 42 p 138; ex 72 p 141}
\end{document}
