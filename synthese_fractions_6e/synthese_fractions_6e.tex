\documentclass[12pt, twoside]{article}
\usepackage[francais]{babel}
\usepackage[T1]{fontenc}
\usepackage[latin1]{inputenc}
\usepackage[left=8mm, right=8mm, top=8mm, bottom=8mm]{geometry}
\usepackage{float}
\usepackage{graphicx}
\usepackage{array}
\usepackage{multirow}
\usepackage{amsmath,amssymb,mathrsfs}
\pagestyle{empty}
\begin{document}

\center{\textbf{\Large{Bilan fractions}}}




\bigskip

\begin{center}
\begin{tabular}{|m{9cm}|m{10cm}|}
\hline
\textbf{Ce que je dois savoir} & \textbf{Ce que je dois savoir faire} \\
\hline

\enskip 

\begin{itemize}
  \item[$\bullet$] Je sais que la fraction $\dfrac{a}{b}$ est le quotient de a
  par b, c'est-�-dire $\dfrac{a}{b}=a \div b$.  
  \item [$\bullet$] Je connais le vocabulaire: num�rateur, d�nominateur,
  �criture fractionnaire, fractions, �criture d�cimale. 
  
  
  \item [$\bullet$] Je sais que $\dfrac{a}{b} \times b = a$.
  \item [$\bullet$] Je sais placer ou lire une fraction sur une demi-droite
  gradu�e.
  \item [$\bullet$] Je connais la propri�t� sur les fractions �gales.
  \item [$\bullet$] Je connais les diff�rentes m�thodes pour prendre une
  fraction d'un nombre (par exemple: prendre $\dfrac{3}{10}$ de 400).

\end{itemize}
&

\enskip
\begin{itemize}
  \item[$\bullet$] Sur une figure, je sais colorier l'aire repr�sent�e par une
  fraction.
  \item[$\bullet$] Je sais calculer la valeur exacte d'une fraction en �criture
  d�cimale (lorsque c'est possible).
  \item[$\bullet$] Je sais donner une valeur approch�e d'une fraction.
 \item[$\bullet$] Je sais reconna�tre deux fractions �gales.
  \item[$\bullet$] Je sais compl�ter une �galit� de fractions.
  \item[$\bullet$] Je sais prendre une fraction d'un nombre (m�thode au choix).
  \item[$\bullet$] Je sais r�soudre des probl�mes faisant inervenir des
  fractions.
   \end{itemize} \\
\hline 

\end{tabular}
\end{center}


\bigskip

\bigskip

\bigskip

\center{\textbf{\Large{Bilan fractions}}}




\bigskip

\begin{center}
\begin{tabular}{|m{9cm}|m{10cm}|}
\hline
\textbf{Ce que je dois savoir} & \textbf{Ce que je dois savoir faire} \\
\hline

\enskip 

\begin{itemize}
  \item[$\bullet$] Je sais que la fraction $\dfrac{a}{b}$ est le quotient de a
  par b, c'est-�-dire $\dfrac{a}{b}=a \div b$.  
  \item [$\bullet$] Je connais le vocabulaire: num�rateur, d�nominateur,
  �criture fractionnaire, fractions, �criture d�cimale. 
  
  
  \item [$\bullet$] Je sais que $\dfrac{a}{b} \times b = a$.
  \item [$\bullet$] Je sais placer ou lire une fraction sur une demi-droite
  gradu�e.
  \item [$\bullet$] Je connais la propri�t� sur les fractions �gales.
  \item [$\bullet$] Je connais les diff�rentes m�thodes pour prendre une
  fraction d'un nombre (par exemple: prendre $\dfrac{3}{10}$ de 400).

\end{itemize}
&

\enskip
\begin{itemize}
  \item[$\bullet$] Sur une figure, je sais colorier l'aire repr�sent�e par une
  fraction.
  \item[$\bullet$] Je sais calculer la valeur exacte d'une fraction en �criture
  d�cimale (lorsque c'est possible).
  \item[$\bullet$] Je sais donner une valeur approch�e d'une fraction.
 \item[$\bullet$] Je sais reconna�tre deux fractions �gales.
  \item[$\bullet$] Je sais compl�ter une �galit� de fractions.
  \item[$\bullet$] Je sais prendre une fraction d'un nombre (m�thode au choix).
  \item[$\bullet$] Je sais r�soudre des probl�mes faisant inervenir des
  fractions.
   \end{itemize} \\
\hline 

\end{tabular}
\end{center}
\end{document}
