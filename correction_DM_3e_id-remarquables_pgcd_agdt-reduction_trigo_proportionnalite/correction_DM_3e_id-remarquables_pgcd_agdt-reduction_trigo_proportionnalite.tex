\documentclass[12pt, twoside]{article}
\usepackage[francais]{babel}
\usepackage[T1]{fontenc}
\usepackage[latin1]{inputenc}
\usepackage[left=5mm, right=5mm, top=5mm, bottom=5mm]{geometry}
\usepackage{float}
\usepackage{graphicx}
\usepackage{array}
\usepackage{multirow}
\usepackage{amsmath,amssymb,mathrsfs}
\usepackage{soul}
\usepackage{textcomp}
\usepackage{eurosym}
 \usepackage{variations}
\usepackage{tabvar}

\pagestyle{empty}
\begin{document}

\begin{center}
\fbox{Correction du devoir maison 6}
\end{center}

\ul{Exercice 1:}

\enskip


\begin{enumerate}
  \item La travers�e avec Catamaran Express a dur� 30 min. 
  
 \enskip
  
  
  \begin{tabular}{cc}
  \begin{minipage}{9cm}
    \begin{tabular}{|l|c|c|}
  \hline
  Temps (en min) & 30 & 60 \\
  \hline
  Distance (en km) & 17 & \quad \\
  \hline
  \end{tabular}
  \end{minipage}
  &
 \begin{minipage}{9cm}
  $17 \times 60 \div 30=34$ 
  
  
   La vitesse moyenne est donc de 34 km/h.
  \end{minipage}  
  \end{tabular}
  

  
  
  \item vitesse moyenne: 20 km/h ; distance du parcours: 17 km.
 
 \enskip
 
  
   \begin{tabular}{cc}
  \begin{minipage}{9cm}
      \begin{tabular}{|l|c|c|}
  \hline
  Temps (en min) & 60 & \quad  \\
  \hline
  Distance (en km) & 20 & 17 \\
  \hline
  \end{tabular}
  \end{minipage}
  &
 \begin{minipage}{9cm}
 

 
  $60 \times 17 \div 20 = 51$ 
  
  
   La dur�e du parcours est donc de 51
  minutes. Le d�part �tant � 6h, le bateau arrrivera � 6h 51min.
  \end{minipage}  
  \end{tabular}
\end{enumerate}

\bigskip

\ul{Exercice 2:}

\enskip

Montant de la r�duction: $28-18,20=9,80$

Elle a eu une r�duction de 9,80 \euro.

\enskip

Calcul du pourcentage: $\dfrac{9,80}{28}\times 100 =35$

B�a a b�n�fici� d'une remise de 35 \%.
\bigskip


\ul{Exercice 3:}

\enskip

\begin{enumerate}
  \item Algorithme d'Euclide: 
  
  \bigskip
  
  \bigskip
   
  \bigskip
  
  \bigskip
  
  \bigskip
  
 donc PGCD(1394;255)=17
 
 \item Pour faire des colliers identiques avec 1394 graines d'a�a� et 255
 graines de palmier p�che, il faut trouver un diviseur commmun � 1394 et 255.
 Pour r�aliser le plus grand nombre de colliers, il faut trouver le PGCD.
 
 \enskip
 
 On peut donc faire 17 colliers identiques compos�s de 82 graines d'a�a� ($1394
 \div 17=82$) et de 15 graines de palmier p�che ($255 \div 17=15$).
\end{enumerate}


\bigskip

\ul{Exercice 4:} 

\enskip

 $\bullet$ Le triangle BDC est rectangle en B: \quad 
 $sin(\widehat{BDC})=\dfrac{BC}{CD}$ \quad et \quad $tan(\widehat{BDC})=\dfrac{BC}{BD}$.


On remplace par les valeurs num�riques:
\quad $sin(8)=\dfrac{20}{CD}$ \quad donc \quad $CD=20 \div sin(8) \approx 144$
dm


 $tan(8)=\dfrac{20}{BD}$ \quad donc \quad $BD=20 \div tan(8) \approx 142$ m. 


\bigskip


$\bullet$ Le triangle ABD est rectangle en B: \quad
$cos(\widehat{BDA})=\dfrac{BD}{AD}$ \quad et \quad $tan(\widehat{BDA})=\dfrac{AB}{BD}$.


On remplace par les valeurs num�riques: 
$cos(52)=\dfrac{142}{AD}$ \quad donc \quad $AD=142 \div cos(52) \approx 231$
dm


$tan(52)=\dfrac{AB}{142}$ \quad donc \quad $AB=142 \times tan(52) \approx 182$
m.


\bigskip

$\bullet$ $\mathcal{P}_{ADC}=AB+BC+CD+DA \approx 182+20+144+231 \approx 577$ dm.
 
\pagebreak


\ul{Exercice 5:}

\enskip

\begin{enumerate}
  \item $D=(2u+3)^2+(2u+3)(7u-2)=(2u)^2+2 \times 2u \times 3+3^2+2u \times
  7u+2u \times (-2)+3 \times 7u + 3 \times (-2)$
  
  \qquad \qquad \qquad \qquad \qquad \qquad \qquad  \quad
  $=4u^2+12u+9+14u^2-4u+21u-6=18u^2+29u+3$
  
  \enskip
  
  \item $D=(2u+3)\times (2u+3)+(2u+3)\times (7u-2)=(2u+3) \big[ (2u+3)+(7u-2)
  \big ]$
  
  \quad  $= (2u+3)(2u+3+7u-2)=(2u+3)(9u+1)$
 
  \enskip
  
    
  \item $D=(2u+3)(9u+1)=(2 \times (-4)+3)(9 \times (-4)+1)=(-8+3)(-36+1)=-5
  \times (-35)=175$
  
  \enskip
  
  \item $D=(2u+3)(9u+1)=2u \times 9u + 2u \times 1+ 3 \times 9u + 3 \times
  1=18u^2+2u+27u+3=18u^2+29u+3$
  
  On retrouve le m�me r�sultat qu'� la question 1.
\end{enumerate}


\bigskip

\ul{Exercice 6:}

\begin{enumerate}
  \item D�veloppons $A=(y+1)^2-(y-1)^2$ en utilisant les identit�s remarquables:
  
  $A=y^2+2\times y \times 1 - (y^2-2\times y \times
  1)=y^2+2y+1-(y^2-2y+1)=y^2+2y+1-y^2+2y-1=4y$
  
  \enskip
  
    
  \item Factorisons $A=(y+1)^2-(y-1)^2$ en utilisant les identit�s remarquables:
  
  $A=\big( (y+1)+(y-1) \big) \big( (y+1)-(y-1) \big)=(y+1+y-1)(y+1-y+1)=2y
  \times 2=4y$
  
  \enskip
  
  \item $1001^2-999^2=(1000+1)^2-(1000-1)^2$
  
  On retrouve l'expression A pour $y=1000$ donc $1001^2-999^2=4 \times
  1000=4000$.
\end{enumerate}


\bigskip

\ul{Exercice 7:}


\begin{enumerate}
  \item A'B'C'D' est une r�duction de ABCD de coefficient 0,4. 
  
  $A'B'=0,4 AB=0,4 \times 12=4,8$ cm \qquad et \qquad $E'C'=0,4 EC=0,4 \times
  4=1,6$ cm
  
  $\widehat{E'B'C'}=\widehat{EBC}=30$� car les mesures d'angles sont conserv�es
  par r�duction.
  
  \enskip
  
  \item $Aire_{ABCD}=Base \times hauteur=12 \times 4=48cm^2$
  
  \enskip
  
  \item A'B'C'D' est une r�duction de ABCD de coefficient 0,4 donc
  $Aire_{A'B'C'D'}=0,4^2 \times Aire_{ABCD}$ .
  
  $Aire_{A'B'C'D'}=0,4^2 \times 48=7,68 cm^2$.
\end{enumerate}


\end{document}
