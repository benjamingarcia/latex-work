\documentclass[12pt, twoside]{article}
\usepackage[francais]{babel}
\usepackage[T1]{fontenc}
\usepackage[latin1]{inputenc}
\usepackage[left=5mm, right=5mm, top=3mm, bottom=3mm]{geometry}
\usepackage{float}
\usepackage{graphicx}
\usepackage{array}
\usepackage{multirow}
\usepackage{amsmath,amssymb,mathrsfs}
\usepackage{soul}
\usepackage{textcomp}
\usepackage{eurosym}
 \usepackage{variations}
\usepackage{tabvar}


\pagestyle{empty}

\begin{document}

\begin{center}
\ul{\textbf{Correction devoir maison 5}}
\end{center}

\medskip

\ul{\textbf{Exercice 2}}

\enskip

\begin{tabular}{l|l|l}

$A=\dfrac{1}{10}+\dfrac{5}{4}-\dfrac{5}{12}$  & \quad \quad 
$B=\dfrac{34}{15}\times \dfrac{-60}{17} \times \dfrac{3}{4}$  \quad \quad &
 \quad \quad $C=\dfrac{-12}{15} \div \dfrac{16}{5}$\\

\quad &\quad & \quad \\

$A=\dfrac{1 \times 6}{10 \times 6}+\dfrac{5 \times 15}{4 \times 15}-\dfrac{5
\times 5}{12 \times 5}$  \quad \quad &  \quad \quad $B=\dfrac{34 \times (-60)
\times 3}{15 \times 17 \times 4}$  \quad \quad &  \quad \quad 
$C=-\dfrac{12}{15} \times \dfrac{5}{16}$\\

\quad &\quad & \quad \\

$A=\dfrac{6}{60}+\dfrac{75}{60}-\dfrac{25}{60}$ & \quad \quad $B=-\dfrac{2
\times 17 \times 3 \times 5 \times 4 \times 3}{3 \times 5 \times 17 \times 4}$
\quad \quad &  \quad \quad $C=-\dfrac{12 \times 5}{15 \times 16}$ \\

\quad &\quad & \quad \\

 $A=\dfrac{56}{60}=\dfrac{14 \times 4}{15 \times 4}= \dfrac{14}{15}$ &  \quad
 \quad $B=-\dfrac{2 \times 3}{1}=-\dfrac{6}{1}=-6$  \quad \quad &  \quad \quad $C=-\dfrac{4 \times 3
 \times 5}{3 \times 5 \times 4 \times 4}=-\dfrac{1}{4}$\\

\end{tabular}


\medskip


\ul{\textbf{Exercice 3}} 

\begin{enumerate}
  \item $\bullet$ $\dfrac{1}{25}$ du parcours est effectu� � la nage. La
  distance totale est de 50km. Donc la distance parcourue � la
  nage est: $\dfrac{1}{25}\times 50=
  \dfrac{50}{25}=2$km. 
   
  
  $\bullet$ 26 \% du parcours est effectu� � pied. 26\%=$\dfrac{26}{100}$ et
  $\dfrac{26}{100} \times 50=13$km.
 

 $\bullet$ 50-(2+13)=50-15=35. La distance faite � v�lo est de 35km.
 
 (autre m�thode: $\dfrac{1}{25}=\dfrac{2}{50}$ et 26
 \%=$\dfrac{26}{100}=\dfrac{13}{50}$ donc sur 50 km, 2 sont faits � la nage et
 13 � pied).
  \item 35km sur la distance totale repr�sente:
  $\dfrac{35}{50}=\dfrac{70}{100}=70 \%=\dfrac{7}{10}$.
\end{enumerate}


\medskip


\ul{\textbf{Exercice 4}}

\begin{enumerate}
  \item AB=1dm=$\dfrac{3}{3}$dm, BC=$\dfrac{4}{3}$dm et AC=$\dfrac{5}{3}$dm.
  
  $\dfrac{5}{3}>\dfrac{4}{3}>\dfrac{3}{3}$ donc AC est le plus grand c�t�.
    (autre m�thode: on trouve BC=$\dfrac{4}{3}
  \approx 1,33$ et AC=$\dfrac{5}{3} \approx 1,66$.)
  \item  Dans le triangle ABC, le plus long c�t� est [AC]. Donc on calcule
  s�par�ment $AC^2$ et $BC^2+BA^2$:

\begin{tabular}{cc}
\begin{minipage}{10cm}
\begin{tabular}{c|c}
$AC^2=(\dfrac{5}{3})^2$  &  $BC^2+BA^2=(\dfrac{4}{3})^2+1^2 $\\
\quad & \quad \\
$AC^2=\dfrac{5}{3} \times \dfrac{5}{3}$  &  $BC^2+BA^2=\dfrac{4}{3}\times
\dfrac{4}{3}+1$\\

\quad & \quad \\
 
$AC^2=\dfrac{25}{9}$  &   $BC^2+BA^2=\dfrac{16}{9}+\dfrac{9}{9}=\dfrac{25}{9}$\\   
\end{tabular}
\end{minipage}

&
\begin{minipage}{8cm}
On constate que  $AC^2=BC^2+BA^2$. D'apr�s la r�ciproque du th�or�me de
Pythagore,   le triangle ABC est rectangle en B.
\end{minipage}
\end{tabular}


  



\item Le triangle ABC est rectangle en B donc l'aire est donn�e par la formule:
$\dfrac{AB \times BC}{2}$.

$\mathcal{A}_{ABC}=\dfrac{1 \times \dfrac{4}{3}}{2}=\dfrac{4}{3} \times
\dfrac{1}{2}=\dfrac{4}{6}=\dfrac{2}{3}dm^2$.
Le p�rim�tre de ABC est:
AB+AC+BC$=\dfrac{4}{3}+\dfrac{3}{3}+\dfrac{5}{3}=\dfrac{12}{3}=4$dm.
\end{enumerate}



\medskip


\ul{\textbf{Exercice 5}}

\enskip


$D=\dfrac{8}{3}-\dfrac{5}{3} \div \dfrac{20}{21}=\dfrac{8}{3}-\dfrac{5}{3}
\times \dfrac{21}{20}=\dfrac{8}{3}-\dfrac{5 \times 21}{3 \times
20}=\dfrac{8}{3}-\dfrac{5 \times 3 \times 7}{3 \times 5 \times
4}=\dfrac{8}{3}-\dfrac{ 7}{ 4}=\dfrac{8 \times 4}{3 \times 4}-\dfrac{ 7 \times 3}{ 4 \times
3}=\dfrac{32}{12}-\dfrac{21}{12}=\dfrac{11}{12}$


\bigskip



$E=(2+\dfrac{2}{3}) \div
(\dfrac{4}{5}-\dfrac{2}{3})=(\dfrac{2 \times 3}{1 \times 3}+\dfrac{2}{3}) \div
(\dfrac{4 \times 3}{5 \times 3}-\dfrac{2 \times 5}{3 \times 5})=(\dfrac{6}{ 3}+\dfrac{2}{3}) \div
(\dfrac{12}{15}-\dfrac{10}{15})=\dfrac{8}{3} \div \dfrac{2}{15}=\dfrac{8}{3}
\times \dfrac{15}{2}=\dfrac{120}{6}=20$


\bigskip



$F=\dfrac{\dfrac{1}{3}+\dfrac{3}{4}}{\dfrac{5}{4}-\dfrac{7}{3}}=\dfrac{\dfrac{4}{12}+\dfrac{9}{12}}{\dfrac{15}{12}-\dfrac{28}{12}}=\dfrac{13}{12}
\div \dfrac{-13}{12}=\dfrac{13}{12} \times \dfrac{-12}{13}=-1$


\end{document}
