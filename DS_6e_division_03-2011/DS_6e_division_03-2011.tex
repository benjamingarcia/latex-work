\documentclass[12pt, twoside]{article}
\usepackage[francais]{babel}
\usepackage[T1]{fontenc}
\usepackage[latin1]{inputenc}
\usepackage[left=5mm, right=5mm, top=5mm, bottom=5mm]{geometry}
\usepackage{float}
\usepackage{graphicx}
\usepackage{array}
\usepackage{multirow}
\usepackage{amsmath,amssymb,mathrsfs}
\usepackage{textcomp}
\pagestyle{empty}
\usepackage{soul}
\usepackage{eurosym}


\begin{document} 



\begin{flushleft}
NOM PRENOM: \ldots \ldots \ldots \ldots \ldots \ldots \ldots \ldots \ldots
 \end{flushleft}


\begin{center}
{\fbox{$6^{e}\ldots$ \qquad \qquad \textbf{\Large{Devoir surveill� 6 }}
\qquad \qquad \ldots/05/2013}}
\end{center}


\bigskip
 

\textit{L'exercice 1 est � faire sur
la photocopie. Les autres sont � faire sur votre feuille pr�par�e.}

\bigskip


\ul{\textbf{Exercice 1:}} \textit{(3 points)}

\enskip

Compl�ter: 

\enskip

$83 \div 10 = \ldots \ldots \ldots$	\qquad \qquad $35,1 \times  \ldots \ldots
\ldots=3510$ \qquad \qquad $74 \div 74= \ldots \ldots \ldots$


\enskip

$75,9 \times 0,1 = \ldots \ldots \ldots $ \qquad \qquad $56 \times 1000=\ldots
\ldots \ldots $ \qquad \qquad $95 \div 1=\ldots \ldots \ldots $


\bigskip

\ul{\textbf{Exercice 2:}} \textit{(3,5 points)}

\begin{enumerate}
  \item Poser et effectuer la division euclidienne de 1045 par 6. Ecrire
  l'�galit� euclidienne obtenue.
  \item Damien dispose de 260 timbres. Il veut les ranger dans un album. Sur
  chaque page, on peut coller 12 timbres.
  
  \begin{enumerate}
    \item Combien de pages compl�tes peut-il remplir?
    \item Combien de timbres lui reste-t-il?
    \end{enumerate}
\end{enumerate}

\bigskip

\ul{\textbf{Exercice 3:}} \textit{(2 points)}

\enskip

Julien a fait du v�lo pendant 6732 secondes.

\begin{enumerate}
\item Exprimer cette dur�e en minutes et secondes.
\item Exprimer cette dur�e en heures, minutes et secondes.

\end{enumerate}

\bigskip

\ul{\textbf{Exercice 4:}} \textit{(2 points)}

\enskip

Poser et effectuer les divisions d�cimales: \qquad a) $73 \div 8$
\qquad \qquad b) $795,6 \div 6$.
\bigskip

\ul{\textbf{Exercice 5:}} \textit{(2 points)}

\begin{enumerate}
  \item Donner deux multiples de 12.
  \item Donner deux diviseurs de 12.
\end{enumerate}
\bigskip


\ul{\textbf{Exercice 6:}} \textit{(4,5 points)}


\begin{enumerate}
  \item Ecrire le crit�re de divisibilit� par 9.
  \item 2761 est-il divisible par 2? Justifier.
  \item 6780 est-il divisible par 5? Justifier.
  \item 432 est-il divisible par 3? Justifier.
  \item Invente un nombre de quatre chiffres qui est divisible par 5 et par 9.
\end{enumerate}
\bigskip
 
\ul{\textbf{Exercice 7:}} \textit{(2 points)}

\enskip

Mamie Georgette veut acheter 8 places de cin�ma avec un billet de 50 \euro.
Devant la caisse, elle s'aper�oit qu'il lui manque 2 \euro. Combien co�te une
place de cin�ma?

\bigskip

\ul{\textbf{Exercice 8:}} \textit{(2 points)}

\enskip

A la boulangerie, Ingrid a achet� six g�teaux et deux baguettes � 0,85 \euro
l'une. Elle a pay� 16,70 \euro. Combien co�te un g�teau?
\end{document}
