\documentclass[12pt, twoside]{article}
\usepackage[francais]{babel}
\usepackage[T1]{fontenc}
\usepackage[latin1]{inputenc}
\usepackage[left=5mm, right=5mm, top=5mm, bottom=5mm]{geometry}
\usepackage{float}
\usepackage{graphicx}
\usepackage{array}
\usepackage{multirow}
\usepackage{amsmath,amssymb,mathrsfs}
\pagestyle{empty}
\begin{document}

\center{\textbf{\Large{Bilan distances}}}




\bigskip

\begin{center}
\begin{tabular}{|m{95mm}|m{95mm}|}
\hline
\textbf{Ce que je dois savoir} & \textbf{Ce que je dois savoir faire} \\
\hline

\enskip


\begin{itemize}
  \item[$\bullet$] Je connais les notations pour un segment: 
  
  segment et
  longueur.
  \item [$\bullet$] Je connais la d�finition de milieu d'un segment.
  \item [$\bullet$] Je connais le vocabulaire du cercle: centre, rayon,
  diam�tre, arc de cercle et corde.
  \item [$\bullet$] Je sais calculer un rayon connaissant le diam�tre.
  \item [$\bullet$] Je sais calculer un diam�tre connaissant le rayon.
  \item [$\bullet$] Je connais les propri�t�s du cercle.
  \item [$\bullet$] Je connais le vocabulaire: �quidistant, m�diatrice.
  \item [$\bullet$] Je connais les 2 propri�t�s de la m�diatrice.
  \item [$\bullet$] Je sais coder une figure.

\end{itemize}
&

\enskip
\begin{itemize}
  \item[$\bullet$] Je sais reconna�tre deux longueurs identiques � partir du
  codage.
  \item[$\bullet$] Je sais calculer des longueurs de segments.
  \item[$\bullet$] Je sais placer des points � une distance donn�e en utilisant
  le compas.
 \item[$\bullet$] Je sais construire un triangle connaissant les trois
 longueurs.
  \item[$\bullet$] Je sais construire la m�diatrice d'un segment (avec
  l'�querre et avec le compas).
   \item[$\bullet$] Je sais utiliser les propri�t�s des m�diatrices pour
   r�soudre des probl�mes.
  \item[$\bullet$] Je sais r�aliser une figure � partir d'un programme de
  construction.
  \item[$\bullet$] Je sais �crire un programme de construction.
    \end{itemize} \\
\hline 

\end{tabular}
\end{center}


\bigskip



\center{\textbf{\Large{Bilan distances}}}




\bigskip

\begin{center}
\begin{tabular}{|m{95mm}|m{95mm}|}
\hline
\textbf{Ce que je dois savoir} & \textbf{Ce que je dois savoir faire} \\
\hline

\enskip


\begin{itemize}
  \item[$\bullet$] Je connais les notations pour un segment: 
  
  segment et
  longueur.
  \item [$\bullet$] Je connais la d�finition de milieu d'un segment.
  \item [$\bullet$] Je connais le vocabulaire du cercle: centre, rayon,
  diam�tre, arc de cercle et corde.
  \item [$\bullet$] Je sais calculer un rayon connaissant le diam�tre.
  \item [$\bullet$] Je sais calculer un diam�tre connaissant le rayon.
  \item [$\bullet$] Je connais les propri�t�s du cercle.
  \item [$\bullet$] Je connais le vocabulaire: �quidistant, m�diatrice.
  \item [$\bullet$] Je connais les 2 propri�t�s de la m�diatrice.
  \item [$\bullet$] Je sais coder une figure.

\end{itemize}
&

\enskip
\begin{itemize}
  \item[$\bullet$] Je sais reconna�tre deux longueurs identiques � partir du
  codage.
  \item[$\bullet$] Je sais calculer des longueurs de segments.
  \item[$\bullet$] Je sais placer des points � une distance donn�e en utilisant
  le compas.
 \item[$\bullet$] Je sais construire un triangle connaissant les trois
 longueurs.
  \item[$\bullet$] Je sais construire la m�diatrice d'un segment (avec
  l'�querre et avec le compas).
   \item[$\bullet$] Je sais utiliser les propri�t�s des m�diatrices pour
   r�soudre des probl�mes.
  \item[$\bullet$] Je sais r�aliser une figure � partir d'un programme de
  construction.
  \item[$\bullet$] Je sais �crire un programme de construction.
    \end{itemize} \\
\hline 

\end{tabular}
\end{center}
\end{document}
