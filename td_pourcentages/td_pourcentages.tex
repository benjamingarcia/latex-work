\documentclass[12pt, twoside]{article}
\usepackage[francais]{babel}
\usepackage[T1]{fontenc}
\usepackage[latin1]{inputenc}
\usepackage[left=7mm, right=7mm, top=7mm, bottom=7mm]{geometry}
\usepackage{float}
\usepackage{graphicx}
\usepackage{array}
\usepackage{multirow}
\usepackage{amsmath,amssymb,mathrsfs}
\usepackage{soul}
\usepackage{textcomp}
\usepackage{eurosym}
 \usepackage{variations}
\usepackage{tabvar}
 
\pagestyle{empty}
\begin{document}

\begin{center}
{\fbox{ \textbf{\Huge{TD: Pourcentages}}}}
\end{center}

\bigskip

\section{Calculs et pourcentages}

\subsection{Calculer un pourcentage}

Calculer le pourcentage repr�sent� par $p$ individus d'une population $P$ dans
une population de r�f�rence de $N$ de $n$ individus, c'est faire:


\begin{center}
\fbox{
$\dfrac{p}{n}=\dfrac{\text{effectif de} P}{\text{effectif de}
N}=\dfrac{t}{100}=t \%$ }
\end{center}


On dit que $P$ repr�sente $t\%$ de $N$.

\subsection{Calculer t\% de n}

Calculer t\% de n ou prendre t\% de n, c'est faire: n$\times
\dfrac{\text{t}}{100}$.

\begin{center}
\fbox{effectif de N 
\begin{LARGE}
{$\xrightarrow[]{\times t \% }$}
\end{LARGE} 
effectif de P}
\end{center}

\subsection{Augmenter de t\%}

Augmenter $a$ de t\%, c'est faire: $a+a\times \dfrac{t}{100}$ ou encore $a
\times (1+\dfrac{t}{100})$.

\ul{Exemple}: augmenter $a$ de $5\% $ c'est multiplier $a$ par $1,05$.

\begin{center}
\fbox{valeur intiale 
\begin{LARGE}
{$\xrightarrow[]{\times (1+t \%) }$}
\end{LARGE} 
valeur finale}
\end{center}

Le taux t du poucentage d'�volution se retrouve � partir des valeurs initiale
et finale par:

\begin{center}
\fbox{
$t=\dfrac{\text{valeur finale}-\text{valeur initiale}}{\text{valeur
initiale}} \times 100$}
\end{center}

\subsection{Baisser de t\%}

Baisser $a$ de $t\% $, c'est faire: $a-a\times \dfrac{t}{100}$ ou encore $a
\times (1-\dfrac{t}{100})$.


\ul{Exemple}: baisser $a$ de $5\% $ c'est multiplier $a$ par $0,95$.

\begin{center}
\fbox{valeur intiale 
\begin{LARGE}
{$\xrightarrow[]{\times (1-t \%) }$}
\end{LARGE} 
valeur finale}
\end{center}

\ul{Remarque}: On retrouve le taux t du pourcentage d'�volution par la m�me
formule que pour l'augmentation.

\subsection{Les augmentations ou baisses successives ne s'ajoutent pas}

Appliquer deux augmentations successives de $10\% $ puis de $20 \% $, c'est
augmenter de $32 \% $ et non de $30 \% $; les coefficients 1,10 et 1,20 se
multiplient.

\pagebreak

\section{Applications}

\begin{enumerate}
  \item Dans une classe de 29 �l�ves, il y a 6 filles. Quel est le pourcentage
  de filles dans la classe?
  
  \enskip
  
  \item Trois personnes A,B et C ont mis� respectivement 5 euros, 15 euros et
  20 euros sur la m�me grille du loto. Ils ont gagn� 100000 euros. Comment les
  leur r�partir?
  
  \enskip
  
  \item Un prix de 230\euro \enskip augmente de 10\%. Quel est le montant de
  l'augmentation? Que devient le prix apr�s augmentation?
  
  \enskip
  
  \item Quel est le prix TTC d'un article qui co�te 245\euro \enskip hors taxe
  avec une TVA de 5,5\%? De 19,6\%?
  
  \enskip
  
  \item Losqu'un capital est plac� ``avec int�r�ts'' comme � la Caisse
  d'�pargne, au bout de chaque ann�e, l'int�r�t est cumul� avec le capital. Si
  le taux annuel est de 2\%, que devient un capital C plac� un an? Plac� 3 ans?
  Et s'il est plac� n ann�es?
  
  \enskip
  
  
  \item Si les prix augmentent de 3\% par an, quel est le pourcentage
  d'augmentation en 3 ans? En 10 ans?
  
  \enskip
  
  \item Est-il vrai qu'augmenter un article de 10\% puis le baisser de 10\%
  �quivaut � le baisser de 10\% puis de l'augmenter de 10\%?
  
  \enskip
  
  \item Quel est le pourcentage de diminution si un prix de 240\euro \enskip devient 190\euro?
  
  \enskip
  
  
  \item Quelle fonction lin�aire doit-on appliquer � un prix qui augmente de
  4\% puis de 1,5\%?
  
  \enskip
  
  \item On augmente de 3\% le c�t� d'un carr�. Quel est le pourcentage
  d'augmentation de son p�rim�tre? Celui de son aire?
  
  \enskip
  
  
  \item On accro�t de 10\% l'ar�te d'un cube. Quel est le pourcentage
  d'augmentation de son volume?
\end{enumerate}
\end{document}
