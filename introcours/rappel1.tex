%%This is a very basic article template.
%%There is just one section and two subsections.
\documentclass[12pt, twoside]{article}
\usepackage[francais]{babel}
\usepackage[T1]{fontenc}
\usepackage[latin1]{inputenc}
\usepackage[left=2cm, right=3cm, top=1cm, bottom=2cm]{geometry}

\begin{document}


\section*{Quelques rappels du coll�ge}

\textbf{D�finition:} Soit $a$ et $b$ deux entiers naturels ; on dit que
$b$ est un diviseur de $a$ s'il existe un entier naturel $c$ tel que : $a=b*c$  ($c$ est
�galement un diviseur de $b$). On dit aussi que $a$ est un multiple de $b$ ( $a$
est �galement un multiple de $c$). On peut dire aussi que $a$ est divisible par
$b$ ( $a$ est �galement divisible par $c$). 

\bigskip
\textbf{Exemple:} $52=4*13$\\ 
$4$ et $13$ sont des diviseurs de $52$.


\bigskip

\textbf{Comment trouver tous les diviseurs d'un entier?}


\textbf{M�thode:} On teste chaque diviseur entier gr�ce aux crit�res de
divisibilit� rappel�s ci-dessous. Les diviseurs vont par deux, sauf si le
nombre est un carr�.


\textbf{Exemple:} $80=1*80=2*40=4*20=5*16=8*10$.\\
On s'arr�te puisque le prochain
serait $10*8$  d�j� �crit. L'ensemble des diviseurs de $80$ est donc :
$\{1,2,4,5,8,10,16,20,40,80\}$.

\bigskip
\textbf{Propri�t�s:} Un nombre entier est divisible:
\begin{itemize}
  \item[$\star$] par $2$ si et seulement si son chiffre des unit�s est
  $0,2,4,6,$ou $8$
  \item[$\star$] par $3$ si et seulement si la somme des chiffres est
  divisible par $3$
  \item[$\star$] par $5$ si et seulement si le chiffre des unit�s ets $0$ ou
  $5$
  \item[$\star$] par $9$ si et seulement si la somme des chiffres est
  divisible par $9$
  \item[$\star$] par $10$ si et seulement si le chiffre des unit�s est $0$.
\end{itemize}


\section*{Applications}

\subsection*{exercice 1:}
Donner tous les diviseurs de $18, 50, 35, 19$.

\subsection*{exercice 2:}
Dire si les affirmations suivantes sont vraies ou fausses. Justifier la r�ponse.
\begin{itemize}
  \item[$\star$]$4347$ divisible par $3$ et $9$
  \item[$\star$] $7482$ divisible par $2$ et $3$
  \item[$\star$] $789100$ divisible par $100$ et $9$
  \item[$\star$]si un nombre est divisible par $2$ et $4$ il l'est par $8$
  \item[$\star$]$210= 2*3*5*7$
  \item[$\star$]$28000= 2^{5}*5^{3}*7$
\end{itemize}

\subsection*{exercice 3:}
\begin{enumerate}
  \item Ecrire le nombre $60$ sous la forme d'un produit avec le
  maximum de facteurs entiers diff�rents de $1$.
  \item M�me question avec les nombres entiers: $48, 72$ et $525$.
  \item Ecrire les nombres entiers inf�rieurs � $10$ divisibles
  seulement par $1$.
\end{enumerate}
\end{document}
