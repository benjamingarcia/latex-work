\documentclass[12pt, twoside]{article}
\usepackage[francais]{babel}
\usepackage[T1]{fontenc}
\usepackage[latin1]{inputenc}
\usepackage[left=5mm, right=5mm, top=5mm, bottom=5mm]{geometry}
\usepackage{float}
\usepackage{graphicx}
\usepackage{array}
\usepackage{multirow}
\usepackage{amsmath,amssymb,mathrsfs}
\usepackage{soul}
\usepackage{textcomp}
\usepackage{eurosym}
 \usepackage{variations}
\usepackage{tabvar}


\pagestyle{empty}

\begin{document}


\section*{\center{Devoir maison 1}}


\bigskip





\fbox{

\begin{minipage}{18cm}
\textit{Devoir � rendre sur feuille grand format petits carreaux pour le
\textbf{mardi 5 octobre 2010}. Un r�sultat non justifi� donnera 0 point.}
\end{minipage}
}



\bigskip


\ul{Exercice 1 }: (\textit{4,5 points})

Chez un commer�ant, le prix de la ficelle est proportionnel � sa longueur. Un
cuisinier ach�te 3m de ficelle pour 2,25 \euro.

\begin{enumerate}
  \item Repr�senter graphiquement le prix de la ficelle en fonction de sa
  longueur. (Mettre en abscisse la longueur et en ordonn�e le prix � payer.)
  \item Lire sur le graphique:
  
  \begin{enumerate}
    \item le prix approximatif de 5m de ficelle,
    \item le prix approximatif de 7,50m de ficelle,
    \item la longueur approximative de ficelle achet�e pour 4,80 \euro.
  \end{enumerate}
  
  \item Retrouver les r�sulats de la question pr�c�dente par des calculs.
   Justifier votre r�ponse.
\end{enumerate}


\bigskip


\ul{Exercice 2}: (\textit{2,5 points})

\begin{enumerate}
  \item Antoine a construit une maquette de formule 1 � l'�chelle
  $\dfrac{1}{25}$. Sa maquette mesure 18,4cm. Quelle est la longueur r�elle de
  cette formule 1? Justifier votre r�ponse.
  \item Pour la m�me formule 1, Th�o a construit une maquette de 11,5cm. Quelle
  �chelle a-t-il choisi pour sa maquette?  Justifier votre r�ponse.
\end{enumerate}



\bigskip



\ul{Exercice 3}: (\textit{5 points})

Pour les jeux olympiques d'hiver, un magasin de sport affiche un article � 120
\euro.

\begin{enumerate}
  \item Le commer�ant propose une r�duction de 20 \% sur ce prix. Calculer le
  montant de la r�duction et le prix de vente de l'article apr�s r�duction.
  \item Se rendant compte qu'il vend � perte, il d�cide d'augmenter ce nouveau
  prix de 10 \%. Calculer le prix de vente de l'article apr�s cette
  augmentation.
  \item Quel est finalement le montant de la r�duction accord�e par le
  commer�ant par rapport au prix initial?
  \item Exprimer ce montant en pourcentage par rapport au prix initial.
\end{enumerate}


\bigskip

\ul{Exercice 4}: (\textit{5 points})


Un train relie deux villes A et C en passant par B. 

La vitesse moyenne d'un train entre A et B est de 190 km/h. Le temps n�cessaire
� ce parcours est de 2h30min. 

La distance entre les villes B et C est de 195 km. Le temps mis sur cette
portion du trajet est de 1h30min.

\begin{enumerate}
  \item Quelle distance s�pare les villes A et B?
  \item Quelle est la vitesse moyenne du train entre les villes B et C?
  \item Quelle est la distance totale parcourue par le train?
  \item Quelle est la vitesse moyenne du train sur l'ensemble du parcours?
\end{enumerate}


\bigskip

\ul{Exercice 5}: (\textit{3 points}) 


Un troupeau est constitu� de plusieurs esp�ces animales. Dans un premier
troupeau de 120 animaux, il y a 60 \% de moutons. Dans un autre troupeau de 180 animaux, il
y a 40 \% de moutons. Pour monter � l'alpage, les deux troupeaux sont
rassembl�s. Quel est le pourcentage de moutons dans les deux groupes r�unis?
\end{document}
