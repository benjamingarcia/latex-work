\documentclass[12pt, twoside]{article}
\usepackage[francais]{babel}
\usepackage[T1]{fontenc}
\usepackage[latin1]{inputenc}
\usepackage[left=5mm, right=5mm, top=5mm, bottom=5mm]{geometry}
\usepackage{float}
\usepackage{graphicx}
\usepackage{array}
\usepackage{multirow}
\usepackage{amsmath,amssymb,mathrsfs} 
\usepackage{soul}
\usepackage{textcomp}
\usepackage{eurosym}
\usepackage{lscape}
 \usepackage{variations}
\usepackage{tabvar}
 

\begin{document}


\begin{center}

\textbf{\ul{Correction exercice 23 p 122}}

\end{center}

\textbf{Triangle SRT}:

\begin{itemize}
  \item [$\bullet$] Tracer [ST] d'une longueur de 7cm.
  \item [$\bullet$] Construire l'angle $\widehat{S}$ mesurant 70�. Prolonger la
  demi-droite obtenue.
  \item [$\bullet$] Tracer [SR] d'une longueur de 7cm (le point R appartient �
  la demi-droite pr�c�dente).
  \item [$\bullet$] Tracer [RT].
  
\end{itemize}

\enskip

\textbf{Triangle RTU}:

\begin{itemize}
  \item [$\bullet$] Construire le point U � l'aide du compas en prenant comme
  �cartement la longueur RT.
  \item [$\bullet$] Tracer [RU] et [TU].
   \end{itemize}


\enskip


\textbf{Triangle VSR}:

\begin{itemize}
  \item [$\bullet$] Tracer la penpendiculaire � la droite (RS) passant par R.
  On la nomme ($d_1$).
  \item [$\bullet$] Tracer la perpendiculaire � la droite (ST) passant par S.
  On la nomme ($d_2$).
  \item [$\bullet$] Placer le point V qui est le point d'intersection de la
  droite ($d_1$) et de la droite ($d_2$).
  
\end{itemize}

\enskip


$\bullet$ \textbf{Mettre le codage.}


\bigskip


\bigskip



\begin{center}

\textbf{\ul{Correction exercice 23 p 122}}

\end{center}

\textbf{Triangle SRT}:

\begin{itemize}
  \item [$\bullet$] Tracer [ST] d'une longueur de 7cm.
  \item [$\bullet$] Construire l'angle $\widehat{S}$ mesurant 70�. Prolonger la
  demi-droite obtenue.
  \item [$\bullet$] Tracer [SR] d'une longueur de 7cm (le point R appartient �
  la demi-droite pr�c�dente).
  \item [$\bullet$] Tracer [RT].
  
\end{itemize}

\enskip

\textbf{Triangle RTU}:

\begin{itemize}
  \item [$\bullet$] Construire le point U � l'aide du compas en prenant comme
  �cartement la longueur RT.
  \item [$\bullet$] Tracer [RU] et [TU].
   \end{itemize}


\enskip


\textbf{Triangle VSR}:

\begin{itemize}
  \item [$\bullet$] Tracer la penpendiculaire � la droite (RS) passant par R.
  On la nomme ($d_1$).
  \item [$\bullet$] Tracer la perpendiculaire � la droite (ST) passant par S.
  On la nomme ($d_2$).
  \item [$\bullet$] Placer le point V qui est le point d'intersection de la
  droite ($d_1$) et de la droite ($d_2$).
  
\end{itemize}

\enskip


$\bullet$ \textbf{Mettre le codage.}


\end{document}
