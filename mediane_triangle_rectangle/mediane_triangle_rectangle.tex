%%This is a very basic article template.
%%There is just one section and two subsections.
\documentclass[12pt, twoside]{article}
\usepackage[francais]{babel}
\usepackage[T1]{fontenc}
\usepackage[latin1]{inputenc}
\usepackage[left=7mm, right=1cm, top=1cm, bottom=7mm]{geometry}
\usepackage{float}
\usepackage{graphicx}
\usepackage{array}
\usepackage{multirow}
\usepackage{amsmath,amssymb,mathrsfs}
\usepackage{soul}
\usepackage{textcomp}
\usepackage{eurosym}
\usepackage{variations}
\usepackage{tabvar}

\begin{document}

\ul{Activit�} : (D�monstration propri�t� 3)

 Soit $ABC$ un triangle et $D$ le milieu de $[AC]$.


On suppose que la m�diane issue de $B$ a m�me longueur que le segment $[AD]$.


\begin{enumerate}
  \item Faire un sch�ma et coder la figure.
  \item Quelle �galit� de longueurs a-t-on?
  \item Justifier que $A$ et $B$ appartiennent au cercle $\mathcal{C}$ de centre $D$ et de rayon $DC$. En d�duire que
  $\mathcal{C}$ est le cercle circonscrit au triangle $ABC$.
  \item Que peut-on dire du triangle $ABC$? Justifier.
\end{enumerate}

\bigskip

\ul{Activit�} : (D�monstration propri�t� 3)

 Soit $ABC$ un triangle et $D$ le milieu de $[AC]$.


On suppose que la m�diane issue de $B$ a m�me longueur que le segment $[AD]$.


\begin{enumerate}
  \item Faire un sch�ma et coder la figure.
  \item Quelle �galit� de longueurs a-t-on?
  \item Justifier que $A$ et $B$ appartiennent au cercle $\mathcal{C}$ de centre $D$ et de rayon $DC$. En d�duire que
  $\mathcal{C}$ est le cercle circonscrit au triangle $ABC$.
  \item Que peut-on dire du triangle $ABC$? Justifier.
\end{enumerate}

\bigskip

\ul{Activit�} : (D�monstration propri�t� 3)

 Soit $ABC$ un triangle et $D$ le milieu de $[AC]$.


On suppose que la m�diane issue de $B$ a m�me longueur que le segment $[AD]$.


\begin{enumerate}
  \item Faire un sch�ma et coder la figure.
  \item Quelle �galit� de longueurs a-t-on?
  \item Justifier que $A$ et $B$ appartiennent au cercle $\mathcal{C}$ de centre $D$ et de rayon $DC$. En d�duire que
  $\mathcal{C}$ est le cercle circonscrit au triangle $ABC$.
  \item Que peut-on dire du triangle $ABC$? Justifier.
\end{enumerate}

\bigskip

\ul{Activit�} : (D�monstration propri�t� 3)

 Soit $ABC$ un triangle et $D$ le milieu de $[AC]$.


On suppose que la m�diane issue de $B$ a m�me longueur que le segment $[AD]$.


\begin{enumerate}
  \item Faire un sch�ma et coder la figure.
  \item Quelle �galit� de longueurs a-t-on?
  \item Justifier que $A$ et $B$ appartiennent au cercle $\mathcal{C}$ de centre $D$ et de rayon $DC$. En d�duire que
  $\mathcal{C}$ est le cercle circonscrit au triangle $ABC$.
  \item Que peut-on dire du triangle $ABC$? Justifier.
\end{enumerate}

\bigskip

\ul{Activit�} : (D�monstration propri�t� 3)

 Soit $ABC$ un triangle et $D$ le milieu de $[AC]$.


On suppose que la m�diane issue de $B$ a m�me longueur que le segment $[AD]$.


\begin{enumerate}
  \item Faire un sch�ma et coder la figure.
  \item Quelle �galit� de longueurs a-t-on?
  \item Justifier que $A$ et $B$ appartiennent au cercle $\mathcal{C}$ de centre $D$ et de rayon $DC$. En d�duire que
  $\mathcal{C}$ est le cercle circonscrit au triangle $ABC$.
  \item Que peut-on dire du triangle $ABC$? Justifier.
\end{enumerate}

\end{document}
