\documentclass[12pt, twoside]{article}
\usepackage[francais]{babel}
\usepackage[T1]{fontenc}
\usepackage[latin1]{inputenc}
\usepackage[left=7mm, right=7mm, top=7mm, bottom=7mm]{geometry}
\usepackage{float}
\usepackage{graphicx}
\usepackage{array}
\usepackage{multirow}
\usepackage{amsmath,amssymb,mathrsfs}
\usepackage{soul}
\usepackage{textcomp}
\usepackage{eurosym}
 \usepackage{variations}
\usepackage{tabvar}

\pagestyle{empty}

\begin{document}

\begin{flushleft}
NOM PRENOM: \ldots \ldots \ldots \ldots \ldots \ldots \ldots \ldots \ldots
 
\bigskip

\end{flushleft}

\begin{center}
{\fbox{$4^{e}4$ \qquad \qquad \textbf{\Large{Mini-test 4 (sujet 1)}}
\qquad \qquad \ldots/01/2015}}
\end{center}



\bigskip 


\ul{Exercice 1}: Calculer en d�taillant les �tapes du calcul.

\enskip


\begin{tabular}{ccc}
$A=\dfrac{-7}{2}+\dfrac{5}{6}$ \qquad \qquad \qquad  \qquad \qquad  &
$B=\dfrac{11}{14}-\dfrac{17}{7}$ \qquad \qquad \qquad  \qquad \qquad  &
$C=\dfrac{-3}{4}+\dfrac{-4}{5}$ \\

\quad & \quad & \quad \\

\quad & \quad & \quad \\

\quad & \quad & \quad \\

\quad & \quad & \quad \\

\quad & \quad & \quad \\

\quad & \quad & \quad \\

\quad & \quad & \quad \\

\quad & \quad & \quad \\

\quad & \quad & \quad \\

\end{tabular}


\bigskip


\ul{Exercice 2}: Calculer les produits et donner le r�sultat
sous forme de fraction \textbf{irr�ductible} (par la m�thode de votre choix).

\enskip


\begin{tabular}{ccc}
$D=\dfrac{5}{-21} \times \dfrac{7}{-10}$ \qquad \qquad \qquad  \qquad \qquad  &
$E=12 \times \dfrac{-7}{18}$ \qquad \qquad \qquad  \qquad \qquad  &
$F=\dfrac{8}{3} \times \dfrac{-5}{4} \times \dfrac{6}{15}$ \\

\quad & \quad & \quad \\

\quad & \quad & \quad \\

\quad & \quad & \quad \\

\quad & \quad & \quad \\

\quad & \quad & \quad \\

\quad & \quad & \quad \\


\quad & \quad & \quad \\

\quad & \quad & \quad \\

\quad & \quad & \quad \\

\end{tabular}


\bigskip


\ul{Exercice 3}:

 Donner l'inverse de $\dfrac{-5}{6}$: \ldots \ldots \ldots
\ldots \ldots \qquad \qquad Donner l'inverse de $-3$: \ldots \ldots \ldots
\ldots \ldots


\bigskip

\bigskip

\ul{Exercice 4}: Calculer les quotients.

\enskip

\begin{tabular}{cc}
$G=\dfrac{4}{3} \div 5$ \qquad \qquad \qquad  \qquad \qquad  \qquad  \qquad
\qquad& $H=\dfrac{2}{7} \div \dfrac{6}{5}$ \qquad \qquad \qquad  \qquad \qquad  \\

\quad & \quad  \\

\quad & \quad  \\

\quad & \quad  \\

\quad & \quad  \\

\quad & \quad \\



\end{tabular}

\pagebreak


\begin{flushleft}
NOM PRENOM: \ldots \ldots \ldots \ldots \ldots \ldots \ldots \ldots \ldots
 
\bigskip

\end{flushleft}

\begin{center}
{\fbox{$4^{e}4$ \qquad \qquad \textbf{\Large{Mini-test 4 (sujet 2)}}
\qquad \qquad \ldots/01/2015}}
\end{center}



\bigskip 


\ul{Exercice 1}: Calculer en d�taillant les �tapes du calcul.

\enskip


\begin{tabular}{ccc}
$A=\dfrac{7}{4}+\dfrac{-5}{8}$ \qquad \qquad \qquad  \qquad \qquad  &
$B=\dfrac{13}{15}-\dfrac{7}{5}$ \qquad \qquad \qquad  \qquad \qquad  &
$C=\dfrac{-2}{3}+\dfrac{-4}{7}$ \\

\quad & \quad & \quad \\

\quad & \quad & \quad \\

\quad & \quad & \quad \\

\quad & \quad & \quad \\

\quad & \quad & \quad \\

\quad & \quad & \quad \\

\quad & \quad & \quad \\

\quad & \quad & \quad \\

\quad & \quad & \quad \\

\end{tabular}


\bigskip


\ul{Exercice 2}: Calculer les produits et donner le r�sultat
sous forme de fraction \textbf{irr�ductible} (par la m�thode de votre choix).

\enskip


\begin{tabular}{ccc}
$D=\dfrac{3}{-18} \times \dfrac{6}{-12}$ \qquad \qquad \qquad  \qquad \qquad  &
$E=15 \times \dfrac{-4}{20}$ \qquad \qquad \qquad  \qquad \qquad  &
$F=\dfrac{15}{16} \times \dfrac{4}{-6} \times \dfrac{9}{5}$ \\

\quad & \quad & \quad \\

\quad & \quad & \quad \\

\quad & \quad & \quad \\

\quad & \quad & \quad \\

\quad & \quad & \quad \\

\quad & \quad & \quad \\


\quad & \quad & \quad \\

\quad & \quad & \quad \\

\quad & \quad & \quad \\

\end{tabular}


\bigskip


\ul{Exercice 3}:

 Donner l'inverse de $\dfrac{-7}{3}$: \ldots \ldots \ldots
\ldots \ldots \qquad \qquad Donner l'inverse de $-4$: \ldots \ldots \ldots
\ldots \ldots


\bigskip

\bigskip

\ul{Exercice 4}: Calculer les quotients.

\enskip

\begin{tabular}{cc}
$G=\dfrac{2}{7} \div 3$ \qquad \qquad \qquad  \qquad \qquad  \qquad  \qquad
\qquad& $H=\dfrac{3}{5} \div \dfrac{6}{8}$ \qquad \qquad \qquad  \qquad \qquad 
\\

\quad & \quad  \\

\quad & \quad  \\

\quad & \quad  \\

\quad & \quad  \\

\quad & \quad \\



\end{tabular}






\pagebreak









\begin{flushleft}
NOM PRENOM: \ldots \ldots \ldots \ldots \ldots \ldots \ldots \ldots \ldots
 
\bigskip

\end{flushleft}

\begin{center}
{\fbox{$4^{e}3$ \qquad \qquad \textbf{\Large{Contr�le de cours 6 (sujet 1)}}
\qquad \qquad 09/04/2010}}
\end{center}



\bigskip 


\ul{Exercice 1}: Calculer en d�taillant les �tapes du calcul.

\enskip


\begin{tabular}{ccc}
$A=\dfrac{-7}{2}+\dfrac{5}{6}$ \qquad \qquad \qquad  \qquad \qquad  &
$B=\dfrac{11}{14}-\dfrac{17}{7}$ \qquad \qquad \qquad  \qquad \qquad  &
$C=\dfrac{-3}{4}+\dfrac{-4}{5}$ \\

\quad & \quad & \quad \\

\quad & \quad & \quad \\

\quad & \quad & \quad \\

\quad & \quad & \quad \\

\quad & \quad & \quad \\

\quad & \quad & \quad \\

\quad & \quad & \quad \\

\quad & \quad & \quad \\

\quad & \quad & \quad \\

\end{tabular}


\bigskip


\ul{Exercice 2}: Calculer les produits en simplifiant et donner le r�sultat
sous forme de fraction \textbf{irr�ductible}.

\enskip


\begin{tabular}{ccc}
$D=\dfrac{5}{-21} \times \dfrac{7}{-10}$ \qquad \qquad \qquad  \qquad \qquad  &
$E=12 \times \dfrac{-7}{18}$ \qquad \qquad \qquad  \qquad \qquad  &
$F=\dfrac{8}{3} \times \dfrac{-5}{4} \times \dfrac{6}{15}$ \\

\quad & \quad & \quad \\

\quad & \quad & \quad \\

\quad & \quad & \quad \\

\quad & \quad & \quad \\

\quad & \quad & \quad \\

\quad & \quad & \quad \\


\quad & \quad & \quad \\

\quad & \quad & \quad \\

\quad & \quad & \quad \\

\end{tabular}


\bigskip


\ul{Exercice 3}:

 Donner l'inverse de $\dfrac{-5}{6}$: \ldots \ldots \ldots
\ldots \ldots \qquad \qquad Donner l'inverse de $-3$: \ldots \ldots \ldots
\ldots \ldots


\bigskip

\bigskip

\ul{Exercice 4}: Calculer les quotients.

\enskip

\begin{tabular}{cc}
$G=\dfrac{4}{3} \div 5$ \qquad \qquad \qquad  \qquad \qquad  \qquad  \qquad
\qquad& $H=\dfrac{2}{7} \div \dfrac{6}{5}$ \qquad \qquad \qquad  \qquad \qquad  \\

\quad & \quad  \\

\quad & \quad  \\

\quad & \quad  \\

\quad & \quad  \\

\quad & \quad \\



\end{tabular}

\pagebreak


\begin{flushleft}
NOM PRENOM: \ldots \ldots \ldots \ldots \ldots \ldots \ldots \ldots \ldots
 
\bigskip

\end{flushleft}

\begin{center}
{\fbox{$4^{e}3$ \qquad \qquad \textbf{\Large{Contr�le de cours 6 (sujet 2)}}
\qquad \qquad 09/04/2010}}
\end{center}



\bigskip 


\ul{Exercice 1}: Calculer en d�taillant les �tapes du calcul.

\enskip


\begin{tabular}{ccc}
$A=\dfrac{7}{4}+\dfrac{-5}{8}$ \qquad \qquad \qquad  \qquad \qquad  &
$B=\dfrac{13}{15}-\dfrac{7}{5}$ \qquad \qquad \qquad  \qquad \qquad  &
$C=\dfrac{-2}{3}+\dfrac{-4}{7}$ \\

\quad & \quad & \quad \\

\quad & \quad & \quad \\

\quad & \quad & \quad \\

\quad & \quad & \quad \\

\quad & \quad & \quad \\

\quad & \quad & \quad \\

\quad & \quad & \quad \\

\quad & \quad & \quad \\

\quad & \quad & \quad \\

\end{tabular}


\bigskip


\ul{Exercice 2}: Calculer les produits en simplifiant et donner le r�sultat
sous forme de fraction \textbf{irr�ductible}.

\enskip


\begin{tabular}{ccc}
$D=\dfrac{3}{-18} \times \dfrac{6}{-12}$ \qquad \qquad \qquad  \qquad \qquad  &
$E=15 \times \dfrac{-4}{20}$ \qquad \qquad \qquad  \qquad \qquad  &
$F=\dfrac{15}{16} \times \dfrac{4}{-6} \times \dfrac{9}{5}$ \\

\quad & \quad & \quad \\

\quad & \quad & \quad \\

\quad & \quad & \quad \\

\quad & \quad & \quad \\

\quad & \quad & \quad \\

\quad & \quad & \quad \\


\quad & \quad & \quad \\

\quad & \quad & \quad \\

\quad & \quad & \quad \\

\end{tabular}


\bigskip


\ul{Exercice 3}:

 Donner l'inverse de $\dfrac{-7}{3}$: \ldots \ldots \ldots
\ldots \ldots \qquad \qquad Donner l'inverse de $-4$: \ldots \ldots \ldots
\ldots \ldots


\bigskip

\bigskip

\ul{Exercice 4}: Calculer les quotients.

\enskip

\begin{tabular}{cc}
$G=\dfrac{2}{7} \div 3$ \qquad \qquad \qquad  \qquad \qquad  \qquad  \qquad
\qquad& $H=\dfrac{3}{5} \div \dfrac{6}{8}$ \qquad \qquad \qquad  \qquad \qquad 
\\

\quad & \quad  \\

\quad & \quad  \\

\quad & \quad  \\

\quad & \quad  \\

\quad & \quad \\



\end{tabular}
\end{document}
