\documentclass[12pt, twoside]{article}
\usepackage[francais]{babel}
\usepackage[T1]{fontenc}
\usepackage[latin1]{inputenc}
\usepackage[left=5mm, right=5mm, top=3mm, bottom=3mm]{geometry}
\usepackage{float}
\usepackage{graphicx}
\usepackage{array}
\usepackage{multirow}
\usepackage{amsmath,amssymb,mathrsfs} 
\usepackage{soul}
\usepackage{textcomp}
\usepackage{eurosym}
 \usepackage{variations}
\usepackage{tabvar}

\begin{document}

\section*{\center{Correction devoir maison 4}}

\subsection*{Exercice 1}


\begin{enumerate}
  
  \item
  
 \ul{Donn�es}: [FD] est un diam�tre du cercle $\mathcal{L}$ et D est un point
   de ce cercle. 
  
  \ul{Propri�t�}: Si un triangle est inscrit dans un cercle de diam�tre l'un de
  ses c�t�s alors ce triangle est rectangle et admet ce diam�tre pour
  hypot�nuse.
 
  \ul{Conclusion}: Le triangle AFD est rectangle en A. 
  
  \medskip
  
  
  
  
  
  \item  
  
   \ul{Donn�es}: [AB] est un diam�tre du cercle $\mathcal{L}$ et C est un point
   de ce cercle.
  
  \ul{Propri�t�}: Si un triangle est inscrit dans un cercle de diam�tre l'un de
  ses c�t�s alors ce triangle est rectangle et admet ce diam�tre pour
  hypot�nuse.
  
  \ul{Conclusion}: Le triangle ABC est rectangle en C.  On en d�duit que les
  droites (CA) et (CB) sont perpendiculaires.
\end{enumerate}


\subsection*{Exercice 2}

\begin{enumerate}
  \item  
   \ul{Donn�es}: MAN est un triangle rectangle en B. [AD] est la m�diane issue
   de l'angle droit. AD=3cm. 
  
  \ul{Propri�t�}: Si un triangle est rectangle alors la m�diane issue de
  l'angle droit a pour longueur la moiti� de celle de l'hypot�nuse. 
   
  
  \ul{Conclusion}: MN=2 $\times$ AD=2 $\times$ 3=6cm
 
  
   \medskip
   
     
  \item  
  
   \ul{Donn�es}: MBN est un triangle rectangle en B. [BD] est la m�diane issue
   de l'angle droit. MN=6cm (MN est l'hypot�nuse).
  
  \ul{Propri�t�}: Si un triangle est rectangle alors la m�diane issue de
  l'angle droit a pour longueur la moiti� de celle de l'hypot�nuse.
  
  
  \ul{Conclusion}: BD=$\dfrac{1}{2} \times$ MN=$\dfrac{1}{2} \times$ 6=3cm  
\end{enumerate}


\subsection*{Exercice 4}

On construit le cercle de diam�tre [AB]. On place un point P sur ce cercle.

D'apr�s la propri�t�: ``si un triangle est inscrit dans un cercle de diam�tre
l'un de ses c�t�s alors ce triangle est rectangle et admet ce c�t� pour
hypot�nuse'', le triangle ABP est rectangle en P.

\enskip

De m�me, on construit le cercle de diam�tre [CD]. On place P sur le cercle. Le
triangle CDP est alors rectangle en P.


\enskip

Pour avoir ABP et CDP rectangles en P (un seul point), je dois donc placer P �
l'intersection de ces deux cercles.


\subsection*{Exercice 5}

\begin{tabular}{cc}


\begin{minipage}{5,5cm}
 \begin{tabular}{|c|c|c|}
  \hline
 temps (en min) & 45 &  60\\
  \hline
  distance (en km) & 9 &  x? \\ 
  \hline
  \end{tabular}
\end{minipage}
&
\begin{minipage}{13cm}

\quad

\quad


$x= 9 \times 60 \div 45 =12$
\quad Vitesse moyenne sur la partie A: 12 km/h.

\quad


\enskip


\end{minipage} \\





\bigskip






\begin{minipage}{5,5cm}
 \begin{tabular}{|c|c|c|}
  \hline
 temps (en min) & 90 &  60\\
  \hline
  distance (en km) & 15 &  x?\\ 
  \hline
  \end{tabular}
\end{minipage}
&
\begin{minipage}{13cm}
$x= 15 \times 60 \div 90 =10$ \quad
Vitesse moyenne sur la partie B : 10 km/h.
\end{minipage} \\

\bigskip

\begin{minipage}{5,5cm}
 \begin{tabular}{|c|c|c|}
  \hline
 temps (en min) & 135 &  60\\
  \hline
  distance (en km) & 24 &  x?\\ 
  \hline
  \end{tabular}
\end{minipage}
&
\begin{minipage}{13cm}

Il a couru 9+15=24 kilom�tres en 45+1h30min=2h15min=135 min.

$x= 24 \times 60 \div 135 \approx 10,67$

Vitesse moyenne sur l'ensemble du parcours: $\approx$ 10,67 km/h.
\end{minipage} \\

\end{tabular}
 


\end{document}
