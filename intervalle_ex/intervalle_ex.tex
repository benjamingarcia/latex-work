\documentclass[12pt, twoside]{article}
\usepackage[francais]{babel}
\usepackage[T1]{fontenc}
\usepackage[latin1]{inputenc}
\usepackage[left=1cm, right=1cm, top=8mm, bottom=8mm]{geometry}
\usepackage{float}
\usepackage{graphicx}
\usepackage{array}
\usepackage{multirow}
\usepackage{amsmath,amssymb,mathrsfs}
\pagestyle{empty}
\begin{document}

\textbf{Exercice 1:} On donne une s�rie de nombres: $-2,1 ; \ 3; \ \pi; \ -1; \
-\dfrac{10}{51}; \ \sqrt{2}; \ 10^{8}; \ 1,73$. \\
Indiquer si chacun des nombres ci-dessus appartiennent� l'intervalle
$I_{1}=[-2;2]$. \\
M�me question avec $I_{2}=]-1;3],\  I_{3}=]-\infty;\sqrt{3}]$ et
$I_{4}=]-\dfrac{22}{7}; +\infty[$.


\bigskip
\textbf{Exercice 2:}
\begin{enumerate}
  \item R�soudre l'in�quation $-7x+9 \geqslant 4$. Donner l'ensemble solution
  sous forme d'intervalle.
  \item R�soudre l'in�quation $2x-3,1 \geqslant 5,7$. Donner l'ensemble solution
  sous forme d'intervalle. 
  \item  R�soudre le syst�me d'in�quations suivant:
 \[  
 \left \{ 
  \begin{aligned}
  -7x+9 &\geqslant 4 \\
   2x-3,1 &\geqslant 5,7
  \end{aligned} 
\right. \]
Donner l'ensemble solution sous forme d'intervalle.
\end{enumerate}

\bigskip
\bigskip
\bigskip

\textbf{Exercice 1:} On donne une s�rie de nombres: $-2,1 ; \ 3; \ \pi; \ -1; \
-\dfrac{10}{51}; \ \sqrt{2}; \ 10^{8}; \ 1,73$. \\
Indiquer si chacun des nombres ci-dessus appartiennent� l'intervalle
$I_{1}=[-2;2]$. \\
M�me question avec $I_{2}=]-1;3],\ 
I_{3}=]-\infty;\sqrt{3}]$ et $I_{4}=]-\dfrac{22}{7}; +\infty[$.


\bigskip
\textbf{Exercice 2:}
\begin{enumerate}
  \item R�soudre l'in�quation $-7x+9 \geqslant 4$. Donner l'ensemble solution
  sous forme d'intervalle.
  \item R�soudre l'in�quation $2x-3,1 \geqslant 5,7$. Donner l'ensemble solution
  sous forme d'intervalle. 
  \item  R�soudre le syst�me d'in�quations suivant:
 \[  
 \left \{ 
  \begin{aligned}
  -7x+9 &\geqslant 4 \\
   2x-3,1 &\geqslant 5,7
  \end{aligned} 
\right. \]
Donner l'ensemble solution sous forme d'intervalle.
\end{enumerate}

\bigskip
\bigskip
\bigskip

\textbf{Exercice 1:} On donne une s�rie de nombres: $-2,1 ; \ 3; \ \pi; \ -1; \
-\dfrac{10}{51}; \ \sqrt{2}; \ 10^{8}; \ 1,73$. \\
Indiquer si chacun des nombres ci-dessus appartiennent� l'intervalle
$I_{1}=[-2;2]$. \\
M�me question avec $I_{2}=]-1;3],\ 
I_{3}=]-\infty;\sqrt{3}]$ et $I_{4}=]-\dfrac{22}{7}; +\infty[$.


\bigskip
\textbf{Exercice 2:}
\begin{enumerate}
  \item R�soudre l'in�quation $-7x+9 \geqslant 4$. Donner l'ensemble solution
  sous forme d'intervalle.
  \item R�soudre l'in�quation $2x-3,1 \geqslant 5,7$. Donner l'ensemble solution
  sous forme d'intervalle. 
  \item  R�soudre le syst�me d'in�quations suivant:
 \[  
 \left \{ 
  \begin{aligned}
  -7x+9 &\geqslant 4 \\
   2x-3,1 &\geqslant 5,7
  \end{aligned} 
\right. \]
Donner l'ensemble solution sous forme d'intervalle.
\end{enumerate}

\end{document}
