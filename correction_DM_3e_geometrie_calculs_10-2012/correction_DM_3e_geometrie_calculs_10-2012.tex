\documentclass[12pt, twoside]{article}
\usepackage[francais]{babel}
\usepackage[T1]{fontenc}
\usepackage[latin1]{inputenc}
\usepackage[left=5mm, right=5mm, top=5mm, bottom=5mm]{geometry}
\usepackage{float}
\usepackage{graphicx}
\usepackage{array}
\usepackage{multirow}
\usepackage{amsmath,amssymb,mathrsfs}
\usepackage{soul}
\usepackage{textcomp}
\usepackage{eurosym}
 \usepackage{variations}
\usepackage{tabvar}

\pagestyle{empty}
\begin{document}

\begin{center}
\fbox{Correction du devoir maison 1}
\end{center}

\enskip


\ul{Exercice 1}: $A=\dfrac{2}{3} \div \dfrac{5}{6}- \dfrac{2}{5}= \dfrac{2}{3}
\times
\dfrac{6}{5}-\dfrac{2}{5}=\dfrac{12}{15}-\dfrac{2}{5}=
\dfrac{12\div 3}{15 \div 3}-\dfrac{2}{5}=\dfrac{4}{5}-\dfrac{2}{5}=\dfrac{2}{5}$


\bigskip


\ul{Exercice 2}: 
Lorsqu'on agrandit une figure, tous les segments ont leurs dimensions
multipli�es par le m�me coefficient de proportionnalit� $\dfrac{11}{5}=2,2$.
Par contre, l'agrandissement conserve les angles.

 

\bigskip


\ul{Exercice 3}:


\begin{enumerate}
  \item [2.] ABD est inscrit dans le cercle ($\mathcal{C}$) de diam�tre [AD].
  D'apr�s la propri�t� ``si on joint les extr�mit�s d'un diam�tre d'un cercle �
  un point de ce cercle, alors le triangle ainsi form� est rectangle en ce
  point'', on en d�duit que le triangle ABD est rectangle en B.
  \item [3.] EB=EA=5,5 cm car ce sont des rayons du cercle. Le triangle AEB est
  isoc�le.
  \item [4.] EAB est isoc�le donc $\widehat{EAB}=\widehat{EBA}$. La
  somme des angles d'un triangle vaut 180�. 
  
  180-48=132 \quad $132 \div 2 = 66$ donc $\widehat{EAB}=\widehat{EBA}=66$�
  
  $\widehat{ADB}=180-90-66=24$�
  
  \item [5.] Le triangle ABD est rectangle en B. Donc
  $cos(\widehat{DAB})=\dfrac{AB}{AD}$ 
  
   $cos(66)=\dfrac{AB}{11}$ et $AB=cos(66) \times 11\approx 4,47$cm.
   
   \item [7.] Dans le triangle ADB: E appartient � [AD]; F appartient � [DB] et
   les droites (EF) et (AB) sont parall�les. Donc d'apr�s le th�or�eme de
   Thal�s, on a:
   
   $\dfrac{DE}{DA}=\dfrac{DF}{DB}=\dfrac{EF}{AB}$ \qquad  En rempla�ant par les
   valeurs excates: $\dfrac{5,5}{11}=\dfrac{EF}{cos(66) \times 11}$ donc
   $EF\approx 2,2$ cm.
   
   
\end{enumerate}

\bigskip


\ul{Exercice 4}:

\begin{enumerate}
  \item Le triangle RDU est rectangle en R. D'apr�s le th�or�me de Pythagore,
  on a: $DU^2= RD^2+ RU^2$.
  
  $DU^2=(15+3,75)^2+10^2=18,75^2+100^2=451,5625$
  
  La longueur DE est positive donc $DU=\sqrt{451,5625}=21,25$ dm.
  
  \item Le triangle RDU est rectangle en R donc on a:
  $cos(\widehat{RDU})=\dfrac{DR}{DU}=\dfrac{18,75}{21,25}=\dfrac{1875}{2125}=\dfrac{15}{17}$
  
  \enskip
  
  Le triangle REC est rectangle en R donc on a: 
 $cos(\widehat{REC})=\dfrac{RE}{RC}=\dfrac{15}{17}$
 
 
 $cos(\widehat{RDU})=cos(\widehat{REC})$ donc $\widehat{RDU}=\widehat{REC}$.
 
 \item $\widehat{RDU}$ et $\widehat{REC}$ sont correspondants et
 $\widehat{RDU}=\widehat{REC}$. D'apr�s la propri�t� ``si deux droites coup�es par
 une s�cante forment des angles correspondants de m�me mesure alors elles sont parall�les
 entre elles'', on peut afirmer que (EC) et (DU) sont parall�les.
 
\enskip

 \item 
 
 
 \begin{tabular}{|c|c|c|c|}
       \hline
       triangle RDU & 10 & 18,75 & 21,25 \\
       \hline
       triangle REC & 8 & 15 & 17 \\
       
       \hline
       \end{tabular}

Le coefficient de r�duction pour passer du triangle RDU au triangle REC est 0,8.



\enskip


\item $\mathcal{P}_{RDU}=18,75+10+21,25=50$dm \quad et \quad
$\mathcal{P}_{REC}=15+17+8=40$dm


$50 \times 0,8 =40$ \quad \fbox{Dans une r�duction de rapport 0,8, le
p�rim�tre d'une figure est multipli� par 0,8.}

\item $\mathcal{A}_{RDU}=\dfrac{RD \times RU}{2}=\dfrac{10 \times
18,75}{2}=93,75 dm^2$
 \quad et \quad
$\mathcal{A}_{REC}=\dfrac{RE \times RC}{2}=\dfrac{8\times
15}{2}=60 dm^2$


$93,75 \times 0,8^2=93,75 \times 0,64=60$ 

  \fbox{Dans une r�duction de
rapport 0,8, l'aire d'une figure est multipli�e par $0,8^2$.}

\end{enumerate}

\end{document}
