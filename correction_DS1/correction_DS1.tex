\documentclass[12pt, twoside]{article}
\usepackage[francais]{babel}
\usepackage[T1]{fontenc}
\usepackage[latin1]{inputenc}
\usepackage[left=1cm, right=1cm, top=8mm, bottom=8mm]{geometry}
\usepackage{float}
\usepackage{graphicx}
\usepackage{array}
\usepackage{multirow}
\usepackage{amsmath,amssymb,mathrsfs}
\pagestyle{empty}
\begin{document}

\begin{flushright}
$2^{de}5$
\end{flushright}
\section*{\center{Correction Devoir surveill� 1}}

\subsection*{Exercice 2}


\begin{enumerate}
  \item 
 
 
 \begin{align*}
        A=  & \dfrac{\dfrac{7}{15}-\dfrac{5}{12}}{\dfrac{7}{4} \times
        \dfrac{3}{5}}  =\dfrac{\dfrac{7 \times 4}{15 \times 4}- \dfrac{5 \times 5}{12
        \times 5}}{\dfrac{21}{20}} =\dfrac{\dfrac{28}{60}-\dfrac{25}{60}}{\dfrac{21}{20}} =\dfrac{3}{60} \times \dfrac{20}{21} =\dfrac{1}{21}
        \end{align*}

\bigskip
\begin{align*}
B= & \dfrac{2^{4} \times (-3)^{5} \times 5^{6}}{2^{7} \times 81 \times 125}=
\dfrac{-3^{5} \times 5^{6}}{2^{7} \times 3^{4} \times  5 \times 25}=
\dfrac{-3 \times 5^{3}}{2^{3}}= -\dfrac{375}{8}= -46,875
\end{align*}

\bigskip
\begin{align*}
C= & \dfrac{(3+ \sqrt{\pi})(3-\sqrt{\pi})}{7}-1,2=
\dfrac{9-\pi}{7}-\dfrac{1,2 \times 7}{7 }=
\dfrac{9-\pi-8,4}{7}=\dfrac{0,6-\pi}{7}\\
&= \dfrac{6-10 \pi}{70} = \dfrac{3-5 \pi}{35}
\end{align*}
 
\bigskip  
\item 
\enskip

\begin{center}
 $\begin{array}{|c|c|c|c|c|c|}
 \hline  & \mathbb{N} & \mathbb{Z} & \mathbb{D} & \mathbb{Q} & \mathbb{R} \\
 \hline  A & \not \in & \not \in & \not \in & \in  & \in \\
 \hline B & \not \in & \not \in & \in & \in & \in \\
 \hline C & \not \in & \not \in & \not \in & \not \in & \in\\
 \hline
 \end{array}$  
 \end{center}
\end{enumerate}



\subsection*{Exercice 3}

\begin{enumerate}
  \item cf cours
  \enskip
  \item 
  \enskip
  
    
  \begin{tabular}{cc}
   \begin{minipage}{5cm}
  \begin{tabular}{c|c}
  $113 \thinspace 300$ & \fbox{$2$} \\
  $56 \thinspace 650$ & $2$ \\
  $28 \thinspace 325$ & \fbox{$5$}\\
  $5 \thinspace 665$ & $5$ \\
  $1 \thinspace 133$ & \fbox{$11$} \\
  $103$ & $103$\\
  $1$ & \\
  \end{tabular}
  \end{minipage}
&
\begin{minipage}{8cm}
$2,3,5$ et $7$ ne divisent pas $103$. $\dfrac{103}{11} \approx 9,3$ et
$11>9,3$ donc $103$ est un nombre premier.
\end{minipage}
 \end{tabular}


donc $113 \thinspace 300= 2^{2} \times 5^{2} \times 11 \times 103$






\bigskip
 \begin{tabular}{cc}
   \begin{minipage}{5cm}
  \begin{tabular}{c|c}
  $20 \thinspace 790$ & \fbox{$2$} \\
  $10 \thinspace 395$ & $3$ \\
  $3 \thinspace 465$ & $3$\\
  $1 \thinspace 155$ & $3$ \\
  $385$ & \fbox{$5$} \\
  $77$ & $7$\\
  $11$ & \fbox{$11$}\\
  $1$ & \\
  \end{tabular}
  \end{minipage}
&
\begin{minipage}{8cm}
Donc $20 \thinspace 790 = 2 \times 3^{3} \times 5 \times 7 \times 11$. 
\end{minipage}
 \end{tabular} 


Le $pgcd$ de $113 \thinspace 300$ et de $20 \thinspace 790$ est $2 \times 5 \times
11=110$.

\item $60= 2^{2} \times 3 \times 5$ et $1155=5 \times 241= 3 \times 5 \times 7
\times 11$.
\medskip


 $\dfrac{17}{60}-\dfrac{24}{1155}= \dfrac{17 \times 7 \times 11}{60 \times 7
 \times 11}- \dfrac{24 \times 2^{2}}{1155 \times 2^{2}}=\dfrac{1213}{4620}$
\end{enumerate}

\subsection*{Exercice 4}


\begin{enumerate}
  \item $\mathcal{C}1$ est le cercle de centre $B$ et de rayon $5$. Le point
  $C$ est sur $\mathcal{C}1$, donc $BC=5$.
  \item Dans le cercle $\mathcal{C}2$, on a: $C$ est un point de $\mathcal{C}2$
  et $OB$ est un diam�tre de $\mathcal{C}2$ donc le triangle $OBC$ est un
  triangle rectangle en $C$.
  \item \begin{enumerate}
          \item Dans le triangle $OBC$ rectangle en $C$, nous utilisons le
          th�or�me de Pythagore:
         \begin{equation*}
          OB^{2} = BC^{2}+OC^{2}
          \end{equation*}
\begin{equation*}
          6^{2} = 5^{2}+OC^{2}
          \end{equation*}
          \begin{equation*}
          OC^{2} = 6^{2}-5^{2}=36-25=11
        \end{equation*}
Donc $OC= \sqrt{11}$.
\item $\mathcal{C}3$ est le cercle de centre $O$ et de rayon $OC$. Le point $D$
est sur $\mathcal{C}3$ donc $OD=OC=\sqrt{11}$. $x$ �tant l'abscisse du point
$D$, on trouve $x=\sqrt{11}$.
\item $x$ est un irrationnel.
\end{enumerate}
\item D'apr�s les prorpi�t�s des m�diatrices, $A$ est le milieu de $[OI]$. Or
$O$ est le point d'abscisse $0$, $I$ est le point d'abscisse $1$. On en d�duit
que l'abscisse de $A$ est $\dfrac{1}{2}=0,5$ qui est un nombre d�cimal.

\end{enumerate}

\begin{center}
	  \includegraphics[width=8cm]{images/corrig�DS1.png} 
\end{center}

\subsection*{Exercice 6}

Les diviseurs de $25$ sont: $1, \ 5, \ 25$ et $1+5 \not=25$.\\
Les diviseurs de $26$ sont: $1, \ 2, \ 13, \  26$ et $1+2+13 \not=26$.\\
Les diviseurs de $27$ sont: $1, \ 3, \ 9, \ 27$ et $1+3+9 \not=27$.\\
Les diviseurs de $28$ sont: $1, \ 2, \ 4,\ 7,\ 14,  \ 28$ et $1+2+4+7+14=28$\\
$28$ est le nombre parfait recherch�.
\end{document}
