\documentclass[12pt, twoside]{article}
\usepackage[francais]{babel}
\usepackage[T1]{fontenc}
\usepackage[latin1]{inputenc}
\usepackage[left=7mm, right=7mm, top=7mm, bottom=7mm]{geometry}
\usepackage{float}
\usepackage{graphicx}
\usepackage{array}
\usepackage{multirow}
\usepackage{amsmath,amssymb,mathrsfs}
\usepackage{soul}
\usepackage{textcomp}
\usepackage{eurosym}
 \usepackage{variations}
\usepackage{tabvar}

\pagestyle{empty}
\begin{document}


\begin{center}
\fbox{Correction du devoir maison 6}
\end{center}


\ul{Exercice 3}:

2) \ul{donn�es}: le triangle ABC est inscrit dans le cercle de diam�tre [AB].

\ul{propri�t�}: si un triangle est inscrit dans un cercle de diam�tre l'un de
ses c�t�s alors ce triangle est rectangle et admet ce c�t� pour hypot�nuse.


\ul{conclusion}: le triangle ABC est rectangle en C.


\bigskip

3) Le triangle ABC est rectangle en C. Donc d'apr�s le th�or�me de Pythagore,
on a:

$AB^2=AC^2+BC^2$

$7^2=AC^2+4,5^2$

$49=AC^2+20,25$

$AC^2=49-20,25$

$AC^2=28,75$


AC est une longueur positive donc $AC=\sqrt{28,75}\approx 5,4cm$ (valeur
arrondie au millim�tre pr�s par d�faut).

\bigskip


\ul{Exercice 4}:

\begin{enumerate}
  
  \item  sch�ma:
  
  \bigskip
  
  \bigskip
  
  \item  Le triangle ABC est rectangle en B. D'apr�s le th�or�me de Pythagore, on a:
  
  $AC^2=AB^2+BC^2$
  
  $AC^2=4^2+2,5^2$
  
  $AC^2=16+6,25$
  
  $AC^2=22,25$
  
  
  AC est une longueur positive
  donc $AC=\sqrt{22,25} \approx 4,7$.
  
  
  \medskip
  
 Le triangle DBC est rectangle en B. D'apr�s le th�or�me de Pythagore, on a:
  
  $DC^2=DB^2+BC^2$
  
  $DC^2=6^2+2,5^2$
  
  $DC^2=36+6,25$
  
  $DC^2=42,25$
  
  
  DC est une longueur positive
  donc $DC=\sqrt{42,25}=6,5$.
  
  
  \medskip
  
  La longueur totale de l'arbre est: AC+CD $\approx $ 4,7+6,5=11,2
  
\end{enumerate}

\bigskip


\ul{Exercice 5}:

Dans le triangle ABD, le plus long c�t� est [BD]. Donc on calcule s�par�ment
$BD^2$ et $AB^2+AD^2$:

\begin{center}
\begin{tabular}{c|c}
$BD^2=59^2$  &  $AB^2+AD^2=48^2+36^2$ \\
$BD^2=3481$  &  $AB^2+AD^2=2304+1296$\\
 \quad  &   $AB^2+AD^2=3600$\\   
\end{tabular}
  \end{center}


On constate que  $BD^2 \neq AB^2+AD^2$. Si le triangle �tait rectangle en A, on
aurait �galit� d'apr�s le th�or�me de Pythagore. Comme ce n'est pas le cas, ABD
n'est pas rectangle.  

\enskip


Donc la d�coupe d'Elodie n'est pas un rectangle.

\bigskip


\ul{Exercice 6}:

25 \% de 40: $\dfrac{25}{100} \times 40=10$ \quad La lampe a une r�duction de
10 euros donc elle co�te 30 \euro (40-10=30)


20 \% de 30: $\dfrac{20}{100}\times 30 =6$ \quad La lampe a une augmentation de
6 euros donc elle co�te 36 \euro (30+6=36)
\end{document}