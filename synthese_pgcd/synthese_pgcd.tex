\documentclass[12pt, twoside]{article}
\usepackage[francais]{babel}
\usepackage[T1]{fontenc}
\usepackage[latin1]{inputenc}
\usepackage[left=7mm, right=7mm, top=5mm, bottom=5mm]{geometry}
\usepackage{float}
\usepackage{graphicx}
\usepackage{array}
\usepackage{multirow}
\usepackage{amsmath,amssymb,mathrsfs}
\pagestyle{empty}
\begin{document}

\section*{\center{Bilan Plus Grand Diviseur Commun}}




\bigskip

\begin{center}
\begin{tabular}{|m{9cm}|m{9cm}|}
\hline


\textbf{Ce que je dois savoir} & \textbf{Ce que je dois savoir faire} \\

\hline

\enskip


\begin{itemize}
  \item [$\bullet$] Je connais le vocabulaire: division euclidienne, diviseur,
  dividende, quotient et reste.
  
  
  \item [$\bullet$] Je comprends le sens des expressions: �tre un multiple de,
  �tre divisible par, �tre un diviseur de.
  
  \item [$\bullet$] Je connais la d�finition du PGCD.
  
  \item [$\bullet$] Je connais les crit�res de divisibilit� par 2; 3; 4; 5; 9
  et 10.
  
  \item[$\bullet$] Je connais la d�finition de nombres premiers entre eux.
 
 
 \end{itemize}

&


\begin{itemize}
  \item[$\bullet$] Je sais effectuer une division euclidienne � la main ET � la
  calculatrice.


\item[$\bullet$] Je sais �num�rer les diviseurs communs de deux entiers et
trouver leur PGCD.

\item[$\bullet$] Je sais r�duire une fraction et la rendre irr�ductible.

\item[$\bullet$] Je sais appliquer l'algorithme des soustractions successives.

\item[$\bullet$] Je sais appliquer l'algorithme d'Euclide.

  \item[$\bullet$] Je sais r�soudre des probl�mes.
 
\end{itemize} \\

\hline

\end{tabular}
\end{center} 





\section*{\center{Bilan Plus Grand Diviseur Commun}}




\bigskip

\begin{center}
\begin{tabular}{|m{9cm}|m{9cm}|}
\hline


\textbf{Ce que je dois savoir} & \textbf{Ce que je dois savoir faire} \\

\hline

\enskip


\begin{itemize}
  \item [$\bullet$] Je connais le vocabulaire: division euclidienne, diviseur,
  dividende, quotient et reste.
  
  
  \item [$\bullet$] Je comprends le sens des expressions: �tre un multiple de,
  �tre divisible par, �tre un diviseur de.
  
  \item [$\bullet$] Je connais la d�finition du PGCD.
  
  \item [$\bullet$] Je connais les crit�res de divisibilit� par 2; 3; 4; 5; 9
  et 10.
  
  \item[$\bullet$] Je connais la d�finition de nombres premiers entre eux.
 
 
 \end{itemize}

&


\begin{itemize}
  \item[$\bullet$] Je sais effectuer une division euclidienne � la main ET � la
  calculatrice.


\item[$\bullet$] Je sais �num�rer les diviseurs communs de deux entiers et
trouver leur PGCD.

\item[$\bullet$] Je sais r�duire une fraction et la rendre irr�ductible.

\item[$\bullet$] Je sais appliquer l'algorithme des soustractions successives.

\item[$\bullet$] Je sais appliquer l'algorithme d'Euclide.

  \item[$\bullet$] Je sais r�soudre des probl�mes.
 
\end{itemize} \\

\hline

\end{tabular}
\end{center} 


\bigskip

\section*{\center{Bilan Plus Grand Diviseur Commun}}




\bigskip

\begin{center}
\begin{tabular}{|m{9cm}|m{9cm}|}
\hline


\textbf{Ce que je dois savoir} & \textbf{Ce que je dois savoir faire} \\

\hline

\enskip


\begin{itemize}
  \item [$\bullet$] Je connais le vocabulaire: division euclidienne, diviseur,
  dividende, quotient et reste.
  
  
  \item [$\bullet$] Je comprends le sens des expressions: �tre un multiple de,
  �tre divisible par, �tre un diviseur de.
  
  \item [$\bullet$] Je connais la d�finition du PGCD.
  
  \item [$\bullet$] Je connais les crit�res de divisibilit� par 2; 3; 4; 5; 9
  et 10.
  
  \item[$\bullet$] Je connais la d�finition de nombres premiers entre eux.
 
 
 \end{itemize}

&


\begin{itemize}
  \item[$\bullet$] Je sais effectuer une division euclidienne � la main ET � la
  calculatrice.


\item[$\bullet$] Je sais �num�rer les diviseurs communs de deux entiers et
trouver leur PGCD.

\item[$\bullet$] Je sais r�duire une fraction et la rendre irr�ductible.

\item[$\bullet$] Je sais appliquer l'algorithme des soustractions successives.

\item[$\bullet$] Je sais appliquer l'algorithme d'Euclide.

  \item[$\bullet$] Je sais r�soudre des probl�mes.
 
\end{itemize} \\

\hline

\end{tabular}
\end{center} 
\end{document}
