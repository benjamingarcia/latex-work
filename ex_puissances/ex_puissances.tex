\documentclass[12pt, twoside]{article}
\usepackage[francais]{babel}
\usepackage[T1]{fontenc}
\usepackage[latin1]{inputenc}
\usepackage[left=7mm, right=7mm, top=7mm, bottom=7mm]{geometry}
\usepackage{float}
\usepackage{graphicx}
\usepackage{array}
\usepackage{multirow}
\usepackage{amsmath,amssymb,mathrsfs}
\usepackage{soul}
\usepackage{textcomp}
\usepackage{eurosym}
 \usepackage{variations}
\usepackage{tabvar}


\pagestyle{empty}

\begin{document}


\section*{\center{Exercices sur les puissances}}

\subsection*{Exercice 1}

Parmi les nombres suivants, reconna�tre les nombres n�gatifs:
$(-6)^4$; \quad $6^8$; \quad $-132^{51}$; \quad $(-12)^{15}$; \quad $-(-35)^7$.

\subsection*{Exercice 2}

Calculer: 

$4^0=\ldots \ldots$ \qquad $(-6)^0=\ldots \ldots$ \qquad
$(-1,8)^1=\ldots \ldots$ \qquad $0,5^1=\ldots \ldots$ \qquad $1,2^1=\ldots
\ldots$ \qquad $-7^0=\ldots \ldots$.

\subsection*{Exercice 3}

Compl�ter:

$12^{-5}= \dfrac{1}{12^{\ldots \ldots}}$ \quad \quad
$\dfrac{1}{9^{\ldots\ldots}}=9^{-23}$ \quad \quad $7^{\ldots
\ldots}=\dfrac{1}{7^5}$ \quad \quad $8^{-6}=\dfrac{1}{8^{\ldots \ldots}}$
\quad \quad $(-7)^3=\dfrac{1}{(-7)^{\ldots \ldots}}$ \quad
\quad $\dfrac{1}{21^{\ldots \ldots}}=21^{15}$


\subsection*{Exercice 4}

Ecrire sous la forme d'un produit:

\begin{enumerate}
  \item d'une puissance de 2 et de 5: 
  
    $A=2 \times 2 \times 5 \times 5 \times 5
  \times 2 \times 2 \times 5 \times 5$ \qquad \qquad $B=25 \times 10 \times 5
  \times 8$ \qquad \qquad $C=\dfrac{2 \times 2 \times 2}{5 \times 5 \times 5
  \times 5 \times 2}$
  \item d'une puissance de 2, de 3 et de 7: 
  
  $D=2 \times 2 \times 2 \times 3
  \times 7 \times 7$ \qquad \qquad $E=32 \times 21 \times 12$ \qquad
  \qquad $F=\dfrac{2 \times 3 \times 7}{3 \times 3 \times 7 \times 7}$
\end{enumerate}

\subsection*{Exercice 5}

Calculer sans calculatrice les expressions suivantes:

 $A=3 \times 2^4+5 \times
4^3$ \qquad \qquad  $B=1-3^2 \times (-5)^2$ \qquad \qquad $C=2^3 \times
(-9)+3^3-(5^2+2^{-1})$.


\subsection*{Exercice 6}

Ecrire sous la forme d'une puissance:

\enskip

\begin{tabular}{llll} 
a) $3^4 \times 3^2$ \qquad & \qquad c) $(-5)^{-4} \times (-5)^{-3}$ \qquad &
\qquad e) $(7^2)^3$ \qquad & \qquad g) $7^5 \times 2^5$\\

b) $4^3 \times 4^{-5}$ \qquad & \qquad d) $\dfrac{2^4}{2^5}$ \qquad & \qquad f)
$(4^{-2})^3$ \qquad & \qquad h) $3^{-4} \times 5^{-4}$
\end{tabular}

\subsection*{Exercice 7}

Calculer astucieusement:

$A=2^4 \times 0,026 \times 5^4$ \qquad $B=5^{-2} \times 2^{-2} \times 84$
\qquad $C=2^{-3} \times 5^{-3} \times 2500$ \qquad $D=2^6 \times 36 \times 5^5$


\pagebreak


\section*{\center{Exercices sur les puissances de 10}}

\subsection*{Exercice 1}

Donner l'�criture d�cimale: \qquad $10^4$ \qquad $10^5$ \qquad $-10^{-3}$ \qquad
$10^{-1}$ \qquad $10^6$ \qquad $(-10)^1$


\subsection*{Exercice 2}

Ecrire avec une puissance de 10: \qquad 10000 \qquad 1000000  \qquad 1000 \qquad
$\dfrac{1}{10000}$ \qquad $\dfrac{1}{1000}$ \qquad 0,01

\subsection*{Exercice 3}

Exprimer sous la forme d'une puissance de 10:

\begin{center}
\begin{tabular}{lllll}
a) $(10^9)^4$ \qquad & \qquad  b) $ \dfrac {10^{-4}}{10^9}$ \qquad & \qquad c)
$10^12 \times 10^{-8} \times 10^5$ \qquad & \qquad d) $\dfrac{10^{-6}}{10^6}$
\qquad & \qquad e) $\dfrac{10^{41}\times 10^7}{10^{-39}}$ \\


\quad & \quad & \quad & \quad \\
f) $10^{-9} \times 10^{12}$ \qquad & \qquad g) $\dfrac{10^{-7}}{10^8}$ \qquad &
\qquad h) $(10^{-3)^{-6}}$ \qquad & \qquad i) $\dfrac{10^10}{10^{-5}}$ \qquad &
\qquad j) $\dfrac{10^{21}}{10^{-4} \times 10^{-18}}$\\
\end{tabular} 
\end{center}


\subsection*{Exercice 4}

Parmi les nombres suivants, quels sont ceux �crits en notation scientifique?

\enskip

a) $5,23 \times 10^{12}$ \quad \quad b) $72,43 \times 10^{-8}$ \quad \quad c)
$2,45 \times 100^{-9}$ \quad \quad d) $-1,47 \times 10^6$ \quad \quad e) $0,251
\times 10^3$ \quad \quad f) -7,6


\subsection*{Exercice 5}

Ecrire les nombres suivants en notation scientifique:


\enskip

\begin{center}
\begin{tabular}{lll}

a) 7283 \qquad & \qquad b) 25000 \qquad & \qquad c) 654,98 \\

\quad & \quad & \quad \\

d) 12,47 \qquad & \qquad e) 0,0058 \qquad &  \qquad f) 0,000149 \\

\quad & \quad & \quad \\

g) $0,67 \times 10^2$ \qquad & \qquad h) $159 \times 10^{-5}$ \qquad & \qquad
i) $0,009 \times 10^{-7}$\\


\end{tabular}
\end{center}



\end{document}
