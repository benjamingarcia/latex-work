\documentclass[12pt, twoside]{article}
\usepackage[francais]{babel}
\usepackage[T1]{fontenc}
\usepackage[latin1]{inputenc}
\usepackage[left=7mm, right=1cm, top=1cm, bottom=7mm]{geometry}
\usepackage{float}
\usepackage{graphicx}
\usepackage{array}
\usepackage{multirow}
\usepackage{amsmath,amssymb,mathrsfs}
\usepackage{soul}
\usepackage{textcomp}
\usepackage{eurosym}
 \usepackage{variations}
\usepackage{tabvar}
 
\pagestyle{empty}
\begin{document}
\begin{flushleft}
Coralie LOPEZ

Classe de seconde
\end{flushleft}
\section*{\center{Utilisation des logiciels de g�om�trie dynamique}}

\subsection*{Calcul et fonctions}


\begin{itemize}
  \item [$\bullet$] Fonction bas�e sur la g�om�trie (exemples: recherche d'une
  aire minimale, volume en fonction d'une hauteur, comparaisons
  d'aires/p�rim�tres) (dynamique)
  
  
  \enskip
  
 \item [$\bullet$] D�finition des fonctions trigonom�triques: enroulement de
 la droite sur le cercle. (dynamique)
 
 
 
 \enskip
 
 \item [$\bullet$] R�solution d'�quations ou d'in�quations (pour des
 probl�mes d'aires par exemple) (dynamique)
\end{itemize}

\subsection*{G�om�trie}


\begin{itemize}
  \item [$\bullet$] Probl�mes d'aires (dynamique)
  
  \enskip
  
  
  \item [$\bullet$] G�om�trie dans le plan (pas obligatoirement dynamique)
  
 
 \enskip
 
 \item [$\bullet$] Calcul vectoriel (pas obligatoirement dynamique)
 
  
   \enskip
   
     \item [$\bullet$] G�om�trie dans l'espace (dynamique)
 
  \enskip
   
     \item [$\bullet$] Triangles isom�triques et semblables (pas obligatoirement
  dynamique)
  
  \enskip
  
  
  \item [$\bullet$] Equations de droites (dynamique)

\end{itemize}

\bigskip

\fbox{
\begin{minipage}{18cm}
En conclusion, pratiquement tout le programme de seconde hormis la partie
statistique peut-�tre illustr� par des logiciels de g�om�trie dynamique ou non.
\end{minipage}
}
\end{document}
