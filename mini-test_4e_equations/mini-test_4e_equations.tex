\documentclass[12pt, twoside]{article}
\usepackage[francais]{babel}
\usepackage[T1]{fontenc}
\usepackage[latin1]{inputenc}
\usepackage[left=7mm, right=7mm, top=5mm, bottom=5mm]{geometry}
\usepackage{float}
\usepackage{graphicx}
\usepackage{array}
\usepackage{multirow}
\usepackage{amsmath,amssymb,mathrsfs}
\usepackage{soul}
\usepackage{textcomp}
\usepackage{eurosym}
 \usepackage{variations}
\usepackage{tabvar}

\pagestyle{empty}

\begin{document}

\begin{flushleft}
NOM PRENOM: \ldots \ldots \ldots \ldots \ldots \ldots \ldots \ldots \ldots
 
\bigskip

\end{flushleft}

\begin{center}
{\fbox{$4^{e}3$ \qquad \qquad \textbf{\Large{Contr�le de cours 9 (sujet 1)}}
\qquad \qquad 05/06/2013}}
\end{center}



\bigskip 


\ul{Exercice 1}: 

\begin{enumerate}
  \item Le nombre $-3$ est-il solution de l'�quation $t^2+2t+6=0$? Justifier.
  
  \ldots \ldots \ldots \ldots \ldots \ldots \ldots \ldots \ldots \ldots \ldots
  \ldots \ldots \ldots \ldots \ldots \ldots \ldots \ldots \ldots \ldots \ldots
  \ldots \ldots \ldots \ldots \ldots \ldots \ldots \ldots \ldots \ldots \ldots
  \ldots 
  \item Le nombre $2$ est-il solution de l'�quation $9-5y=6y-13$? Justifier.
  
 \ldots \ldots \ldots \ldots \ldots \ldots \ldots \ldots \ldots \ldots \ldots
  \ldots \ldots \ldots \ldots \ldots \ldots \ldots \ldots \ldots \ldots \ldots
  \ldots \ldots \ldots \ldots \ldots \ldots \ldots \ldots \ldots \ldots \ldots
  \ldots    
\end{enumerate}

\bigskip

\ul{Exercice 2:} R�soudre les �quations suivantes.


\enskip

\begin{tabular}{l|l|l|l}

$-3t=45$ \quad \quad \quad & \quad \quad \quad $\dfrac{y}{-4}=-9,1$ \quad  \quad
\quad \quad &\quad  \quad \quad $x+4,3=-12,2$ \quad \quad \quad & \quad \quad
\quad $a-\dfrac{7}{3}=-\dfrac{5}{6}$ \\

\quad & \quad & \quad & \quad \\

\quad & \quad & \quad & \quad \\

\quad & \quad & \quad & \quad \\



\end{tabular}

\bigskip

\ul{Exercice 3:} R�soudre les �quations suivantes.

\enskip

\begin{tabular}{c|c}
\qquad \qquad $5u-9=-12$ \qquad \qquad \qquad \qquad & \qquad \qquad
\qquad \qquad $-2t+4=-5+5t$ \\

\quad & \quad \\

\quad & \quad \\

\quad & \quad \\

\quad & \quad \\

\quad & \quad \\


\end{tabular}

\enskip



\begin{flushleft}
NOM PRENOM: \ldots \ldots \ldots \ldots \ldots \ldots \ldots \ldots \ldots
 
\bigskip

\end{flushleft}

\begin{center}
{\fbox{$4^{e}3$ \qquad \qquad \textbf{\Large{Contr�le de cours 9 (sujet 2)}}
\qquad \qquad 05/06/2013}}
\end{center}



\bigskip 


\ul{Exercice 1}: 

\begin{enumerate}
  \item Le nombre $-2$ est-il solution de l'�quation $u^2+3u+2=0$? Justifier.
  
  \ldots \ldots \ldots \ldots \ldots \ldots \ldots \ldots \ldots \ldots \ldots
  \ldots \ldots \ldots \ldots \ldots \ldots \ldots \ldots \ldots \ldots \ldots
  \ldots \ldots \ldots \ldots \ldots \ldots \ldots \ldots \ldots \ldots \ldots
  \ldots 
  \item Le nombre $3$ est-il solution de l'�quation $10-4w=6w-11$? Justifier.
  
 \ldots \ldots \ldots \ldots \ldots \ldots \ldots \ldots \ldots \ldots \ldots
  \ldots \ldots \ldots \ldots \ldots \ldots \ldots \ldots \ldots \ldots \ldots
  \ldots \ldots \ldots \ldots \ldots \ldots \ldots \ldots \ldots \ldots \ldots
  \ldots    
\end{enumerate}

\bigskip

\ul{Exercice 2:} R�soudre les �quations suivantes.


\enskip

\begin{tabular}{l|l|l|l}

$-4u=48$ \quad \quad \quad & \quad \quad \quad $\dfrac{y}{-3}=-8,2$ \quad  \quad
\quad \quad &\quad  \quad \quad $t+5,4=-7,5$ \quad \quad \quad & \quad \quad
\quad $x-\dfrac{3}{4}=-\dfrac{5}{8}$ \\

\quad & \quad & \quad & \quad \\

\quad & \quad & \quad & \quad \\

\quad & \quad & \quad & \quad \\



\end{tabular}

\bigskip

\ul{Exercice 3:} R�soudre les �quations suivantes.

\enskip

\begin{tabular}{c|c}
\qquad \qquad $4u-7=-16$ \qquad \qquad \qquad \qquad & \qquad \qquad
\qquad \qquad $-3y+5=-6+5y$ \\

\quad & \quad \\

\quad & \quad \\

\quad & \quad \\

\quad & \quad \\

\quad & \quad \\

\end{tabular}
\end{document}
