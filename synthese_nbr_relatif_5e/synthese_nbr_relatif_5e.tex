\documentclass[12pt, twoside]{article}
\usepackage[francais]{babel}
\usepackage[T1]{fontenc}
\usepackage[latin1]{inputenc}
\usepackage[left=8mm, right=8mm, top=8mm, bottom=8mm]{geometry}
\usepackage{float}
\usepackage{graphicx}
\usepackage{array}
\usepackage{multirow}
\usepackage{amsmath,amssymb,mathrsfs}
\pagestyle{empty}
\begin{document}

\center{\textbf{\Large{Bilan nombres relatifs}}}




\bigskip

\begin{center}
\begin{tabular}{|m{9cm}|m{10cm}|}
\hline
\textbf{Ce que je dois savoir} & \textbf{Ce que je dois savoir faire} \\
\hline

\enskip


\begin{itemize}
  \item[$\bullet$] Je sais reconna�tre et d�finir un nombre relatif, un nombre
  positif et un nombre n�gatif.
  
  \enskip
  
  \item [$\bullet$] Je connais la d�finition de ``nombres oppos�s'', ``distance
  � z�ro'', ``droite gradu�e'' et ``abscisse d'un point''.
  
  \enskip
  
  \item [$\bullet$]Je connais la d�finition de ``rep�re'', ``ordonn�e'' et
  ``abscisse'' et ``coordonn�e''.
  
  \enskip
  
  \item [$\bullet$] Je connais les r�gles de calcul pour additionner et
  soustraire deux nombres relatifs.
  \item [$\bullet$] Je connais les conventions pour simplifier 
  
  l'�criture d'un
  calcul.
  
  \enskip
  
  \item [$\bullet$] Je connais la m�thode pour calculer la distance entre deux
  points. 
\end{itemize}
&

\enskip
\begin{itemize}
  \item[$\bullet$] Je sais reconna�tre deux nombres oppos�s 
  
  graphiquement.
  \item[$\bullet$] Je sais placer un point sur une droite gradu�e.
  \item[$\bullet$] Je sais lire l'abscisse d'un point.
 \item[$\bullet$] Je sais comparer des nombres relatifs.
  \item[$\bullet$] Je sais lire les coordonn�es d'un point dans un rep�re.
   \item[$\bullet$] Je sais placer un point dans un rep�re.
  \item[$\bullet$] Je sais additionner deux nombres relatifs de m�me signe ou
  de signes contraires.
  \item[$\bullet$] Je sais soustraire deux nombres relatifs.
  \item[$\bullet$] Je sais effectuer une suite de calculs de nombres 
  
  relatifs.
  \item[$\bullet$] Je sais simplifier une �criture.
  \item[$\bullet$] Je sais effectuer un calcul en �criture simplifi�e.
  \item[$\bullet$] je sais calculer la distance entre deux points.
  
    \end{itemize} \\
\hline 

\end{tabular}
\end{center}

\bigskip

\bigskip







\center{\textbf{\Large{Bilan nombres relatifs}}}




\bigskip





\begin{center}
\begin{tabular}{|m{9cm}|m{10cm}|}
\hline
\textbf{Ce que je dois savoir} & \textbf{Ce que je dois savoir faire} \\
\hline

\enskip


\begin{itemize}
  \item[$\bullet$] Je sais reconna�tre et d�finir un nombre relatif, un nombre
  positif et un nombre n�gatif.
  
  \enskip
  
  \item [$\bullet$] Je connais la d�finition de ``nombres oppos�s'', ``distance
  � z�ro'', ``droite gradu�e'' et ``abscisse d'un point''.
  
  \enskip
  
  \item [$\bullet$]Je connais la d�finition de ``rep�re'', ``ordonn�e'' et
  ``abscisse'' et ``coordonn�e''.
  
  \enskip
  
  \item [$\bullet$] Je connais les r�gles de calcul pour additionner et
  soustraire deux nombres relatifs.
  \item [$\bullet$] Je connais les conventions pour simplifier 
  
  l'�criture d'un
  calcul.
  
  \enskip
  
  \item [$\bullet$] Je connais la m�thode pour calculer la distance entre deux
  points. 
\end{itemize}
&

\enskip
\begin{itemize}
  \item[$\bullet$] Je sais reconna�tre deux nombres oppos�s 
  
  graphiquement.
  \item[$\bullet$] Je sais placer un point sur une droite gradu�e.
  \item[$\bullet$] Je sais lire l'abscisse d'un point.
 \item[$\bullet$] Je sais comparer des nombres relatifs.
  \item[$\bullet$] Je sais lire les coordonn�es d'un point dans un rep�re.
   \item[$\bullet$] Je sais placer un point dans un rep�re.
  \item[$\bullet$] Je sais additionner deux nombres relatifs de m�me signe ou
  de signes contraires.
  \item[$\bullet$] Je sais soustraire deux nombres relatifs.
  \item[$\bullet$] Je sais effectuer une suite de calculs de nombres 
  
  relatifs.
  \item[$\bullet$] Je sais simplifier une �criture.
  \item[$\bullet$] Je sais effectuer un calcul en �criture simplifi�e.
  \item[$\bullet$] je sais calculer la distance entre deux points.
  
    \end{itemize} \\
\hline 

\end{tabular}
\end{center}
\end{document}
