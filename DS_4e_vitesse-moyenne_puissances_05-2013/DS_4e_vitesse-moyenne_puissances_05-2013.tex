\documentclass[12pt, twoside]{article}
\usepackage[francais]{babel}
\usepackage[T1]{fontenc}
\usepackage[latin1]{inputenc}
\usepackage[left=5mm, right=5mm, top=5mm, bottom=5mm]{geometry}
\usepackage{float}
\usepackage{graphicx}
\usepackage{array}
\usepackage{multirow}
\usepackage{amsmath,amssymb,mathrsfs}
\usepackage{textcomp}
\pagestyle{empty}
\usepackage{soul}

\begin{document} 

NOM PRENOM: \ldots\ldots\ldots\ldots\ldots


\bigskip

\begin{center}
{\fbox{$4^{e}3$ \qquad \qquad \textbf{\Large{Devoir surveill� 8 (sujet 1) }}
\qquad \qquad 22/05/2013}}
\end{center}


\enskip

\textit{
Les exercices 1 � 4 se font sur la photocopie, les autres sur votre feuille. Calculatrice non autoris�e pour les exercices 1 � 4.}


\bigskip

\ul{Exercice 1:} \textit{(3 points)} \quad Ecrire sous la forme d'une seule
puissance.

\begin{center}
$4 \times 4 \times 4 \times 4 \times 4 \times 4 = \ldots ^{\ldots}$ \qquad
\qquad  \qquad \qquad  $\dfrac{1}{12^{-7}}=\ldots ^{\ldots}$ \qquad \qquad
\qquad \qquad $(2^4)^3=\ldots ^{\ldots}$ 
\end{center}


\begin{center}
 $10^7 \times 10^{-4}=\ldots ^{\ldots}$ \qquad \qquad \qquad \qquad 
$0,0001=\ldots ^{\ldots}$ \qquad \qquad \qquad \qquad
 $\dfrac{10^{12}}{10^{-1}}=\ldots ^{\ldots}$
\end{center}


\bigskip

\ul{Exercice 2:} \textit{(2,5 points)} \quad Donner l'�criture d�cimale des
nombres suivants.


\begin{center}
$(-5)^0= \ldots \ldots$ \qquad \quad  $7,17^1= \ldots \ldots$ \qquad \quad
$2^{-2}=\ldots \ldots$ \qquad \quad $-89,36 \times 10^5 = \ldots \ldots$
\qquad \quad $14,5 \times 10^{-2}=\ldots \ldots$
\end{center}

\bigskip

\ul{Exercice 3:} \textit{(3 points)} \quad Donner l'�criture scientifique des
nombres suivants.

\enskip

$-617,4=\ldots \ldots \ldots$

\enskip

$0,000 $ $002$ $86= \ldots \ldots \ldots$

\enskip

$-74,9 \times 10^7 = \ldots \ldots \ldots$



\bigskip


\ul{Exercice 4:} \textit{(2 points)} \quad 
$A=\dfrac{5 \times 10^{-7} \times 36 \times 10^4}{12 \times 10^{-5}}$

\enskip

Calculer le
nombre A en d�taillant les �tapes de calcul et donner son r�sultat sous la forme d'un nombre d�cimal.


\ldots \ldots \ldots \ldots \ldots \ldots \ldots \ldots \ldots \ldots \ldots
\ldots \ldots \ldots \ldots \ldots \ldots \ldots \ldots \ldots \ldots \ldots
\ldots \ldots \ldots \ldots \ldots \ldots \ldots \ldots \ldots \ldots \ldots
\ldots \ldots \ldots

\ldots \ldots \ldots \ldots \ldots \ldots \ldots \ldots \ldots \ldots \ldots
\ldots \ldots \ldots \ldots \ldots \ldots \ldots \ldots \ldots \ldots \ldots
\ldots \ldots \ldots \ldots \ldots \ldots \ldots \ldots \ldots \ldots \ldots
\ldots \ldots \ldots

\ldots \ldots \ldots \ldots \ldots \ldots \ldots \ldots \ldots \ldots \ldots
\ldots \ldots \ldots \ldots \ldots \ldots \ldots \ldots \ldots \ldots \ldots
\ldots \ldots \ldots \ldots \ldots \ldots \ldots \ldots \ldots \ldots \ldots
\ldots \ldots \ldots

\ldots \ldots \ldots \ldots \ldots \ldots \ldots \ldots \ldots \ldots \ldots
\ldots \ldots \ldots \ldots \ldots \ldots \ldots \ldots \ldots \ldots \ldots
\ldots \ldots \ldots \ldots \ldots \ldots \ldots \ldots \ldots \ldots \ldots
\ldots \ldots \ldots

\ldots \ldots \ldots \ldots \ldots \ldots \ldots \ldots \ldots \ldots \ldots
\ldots \ldots \ldots \ldots \ldots \ldots \ldots \ldots \ldots \ldots \ldots
\ldots \ldots \ldots \ldots \ldots \ldots \ldots \ldots \ldots \ldots \ldots
\ldots \ldots \ldots

\ldots \ldots \ldots \ldots \ldots \ldots \ldots \ldots \ldots \ldots \ldots
\ldots \ldots \ldots \ldots \ldots \ldots \ldots \ldots \ldots \ldots \ldots
\ldots \ldots \ldots \ldots \ldots \ldots \ldots \ldots \ldots \ldots \ldots
\ldots \ldots \ldots


\ldots \ldots \ldots \ldots \ldots \ldots \ldots \ldots \ldots \ldots \ldots
\ldots \ldots \ldots \ldots \ldots \ldots \ldots \ldots \ldots \ldots \ldots
\ldots \ldots \ldots \ldots \ldots \ldots \ldots \ldots \ldots \ldots \ldots
\ldots \ldots \ldots

\ldots \ldots \ldots \ldots \ldots \ldots \ldots \ldots \ldots \ldots \ldots
\ldots \ldots \ldots \ldots \ldots \ldots \ldots \ldots \ldots \ldots \ldots
\ldots \ldots \ldots \ldots \ldots \ldots \ldots \ldots \ldots \ldots \ldots
\ldots \ldots \ldots


\ldots \ldots \ldots \ldots \ldots \ldots \ldots \ldots \ldots \ldots \ldots
\ldots \ldots \ldots \ldots \ldots \ldots \ldots \ldots \ldots \ldots \ldots
\ldots \ldots \ldots \ldots \ldots \ldots \ldots \ldots \ldots \ldots \ldots
\ldots \ldots \ldots

\ldots \ldots \ldots \ldots \ldots \ldots \ldots \ldots \ldots \ldots \ldots
\ldots \ldots \ldots \ldots \ldots \ldots \ldots \ldots \ldots \ldots \ldots
\ldots \ldots \ldots \ldots \ldots \ldots \ldots \ldots \ldots \ldots \ldots
\ldots \ldots \ldots

\bigskip


\ul{Exercice 5:} \textit{(4,5 points)}


\enskip


Calculer chaque expression en d�taillant chaque �tape de calcul. Un r�sultat
non expliqu� ne sera pas compt�.



\begin{center}


$B=\dfrac{-3(3+2^4)}{2^2 \times 5 -14}$ \qquad \qquad \qquad $C=7^2-(3^2 \times
5 + 2^2)$ \qquad \qquad \qquad $D=(6-3\times4)(3^3+7 \times 9)$

\end{center}


\bigskip





\ul{Exercice 6:} \textit{(5 points)}

\begin{enumerate}
  \item Un cycliste parcourt 48 km en une heure et demie. Quelle est alors sa
  vitesse moyenne?
  \item Au retour, sur le m�me trajet, il roule � une vitesse de 38,4 km/h.
  Combien de temps met-il pour effectuer le chemin du retour?
  \item Quel est sa vitesse moyenne sur l'ensemble de son parcours? On
  arrondira � l'unit�.
  \item Exprimer 38,4 km/h en m/s.
\end{enumerate}


\pagebreak

NOM PRENOM: \ldots\ldots\ldots\ldots\ldots


\bigskip

\begin{center}
{\fbox{$4^{e}3$ \qquad \qquad \textbf{\Large{Devoir surveill� 8 (sujet 2) }}
\qquad \qquad 22/05/2013}}
\end{center}


\enskip

\textit{
Les exercices 1 � 4 se font sur la photocopie, les autres sur votre feuille. Calculatrice non autoris�e pour les exercices 1 � 4.}


\bigskip

\ul{Exercice 1:} \textit{(3 points)} \quad Ecrire sous la forme d'une seule
puissance.

\begin{center}
$7 \times 7 \times 7 \times 7 \times 7 = \ldots ^{\ldots}$ \qquad
\qquad  \qquad \qquad $\dfrac{10^{16}}{10^{-2}}=\ldots ^{\ldots}$
 \qquad \qquad \qquad \qquad $0,00001=\ldots ^{\ldots}$ 
\end{center}


\begin{center}
 $10^8 \times 10^{-2}=\ldots ^{\ldots}$ \qquad \qquad \qquad \qquad 
 $(3^2)^4=\ldots ^{\ldots}$ \qquad \qquad \qquad \qquad
 $\dfrac{1}{15^{-9}}=\ldots ^{\ldots}$
 
\end{center}


\bigskip

\ul{Exercice 2:} \textit{(2,5 points)} \quad Donner l'�criture d�cimale des
nombres suivants.


\begin{center}
$(-6,4)^1= \ldots \ldots$ \qquad \quad  $2^{-1}= \ldots \ldots$ \qquad \quad
$4^0=\ldots \ldots$ \qquad \quad $-71,7 \times 10^{-3} = \ldots \ldots$
\qquad \quad $45,76 \times 10^4=\ldots \ldots$
\end{center}

\bigskip

\ul{Exercice 3:} \textit{(3 points)} \quad Donner l'�criture scientifique des
nombres suivants.

\enskip

$-0,000$ $005$ $19=\ldots \ldots \ldots$

\enskip

$3246,5= \ldots \ldots \ldots$

\enskip

$-475,1 \times 10^{-4} = \ldots \ldots \ldots$



\bigskip


\ul{Exercice 4:} \textit{(2 points)} \quad 
$A=\dfrac{4 \times 10^{-3} \times 28 \times 10^5}{14 \times 10^{-2}}$

\enskip

Calculer le
nombre A en d�taillant les �tapes de calcul et donner son r�sultat sous la forme d'un nombre d�cimal.


\ldots \ldots \ldots \ldots \ldots \ldots \ldots \ldots \ldots \ldots \ldots
\ldots \ldots \ldots \ldots \ldots \ldots \ldots \ldots \ldots \ldots \ldots
\ldots \ldots \ldots \ldots \ldots \ldots \ldots \ldots \ldots \ldots \ldots
\ldots \ldots \ldots

\ldots \ldots \ldots \ldots \ldots \ldots \ldots \ldots \ldots \ldots \ldots
\ldots \ldots \ldots \ldots \ldots \ldots \ldots \ldots \ldots \ldots \ldots
\ldots \ldots \ldots \ldots \ldots \ldots \ldots \ldots \ldots \ldots \ldots
\ldots \ldots \ldots

\ldots \ldots \ldots \ldots \ldots \ldots \ldots \ldots \ldots \ldots \ldots
\ldots \ldots \ldots \ldots \ldots \ldots \ldots \ldots \ldots \ldots \ldots
\ldots \ldots \ldots \ldots \ldots \ldots \ldots \ldots \ldots \ldots \ldots
\ldots \ldots \ldots

\ldots \ldots \ldots \ldots \ldots \ldots \ldots \ldots \ldots \ldots \ldots
\ldots \ldots \ldots \ldots \ldots \ldots \ldots \ldots \ldots \ldots \ldots
\ldots \ldots \ldots \ldots \ldots \ldots \ldots \ldots \ldots \ldots \ldots
\ldots \ldots \ldots

\ldots \ldots \ldots \ldots \ldots \ldots \ldots \ldots \ldots \ldots \ldots
\ldots \ldots \ldots \ldots \ldots \ldots \ldots \ldots \ldots \ldots \ldots
\ldots \ldots \ldots \ldots \ldots \ldots \ldots \ldots \ldots \ldots \ldots
\ldots \ldots \ldots

\ldots \ldots \ldots \ldots \ldots \ldots \ldots \ldots \ldots \ldots \ldots
\ldots \ldots \ldots \ldots \ldots \ldots \ldots \ldots \ldots \ldots \ldots
\ldots \ldots \ldots \ldots \ldots \ldots \ldots \ldots \ldots \ldots \ldots
\ldots \ldots \ldots


\ldots \ldots \ldots \ldots \ldots \ldots \ldots \ldots \ldots \ldots \ldots
\ldots \ldots \ldots \ldots \ldots \ldots \ldots \ldots \ldots \ldots \ldots
\ldots \ldots \ldots \ldots \ldots \ldots \ldots \ldots \ldots \ldots \ldots
\ldots \ldots \ldots

\ldots \ldots \ldots \ldots \ldots \ldots \ldots \ldots \ldots \ldots \ldots
\ldots \ldots \ldots \ldots \ldots \ldots \ldots \ldots \ldots \ldots \ldots
\ldots \ldots \ldots \ldots \ldots \ldots \ldots \ldots \ldots \ldots \ldots
\ldots \ldots \ldots


\ldots \ldots \ldots \ldots \ldots \ldots \ldots \ldots \ldots \ldots \ldots
\ldots \ldots \ldots \ldots \ldots \ldots \ldots \ldots \ldots \ldots \ldots
\ldots \ldots \ldots \ldots \ldots \ldots \ldots \ldots \ldots \ldots \ldots
\ldots \ldots \ldots

\ldots \ldots \ldots \ldots \ldots \ldots \ldots \ldots \ldots \ldots \ldots
\ldots \ldots \ldots \ldots \ldots \ldots \ldots \ldots \ldots \ldots \ldots
\ldots \ldots \ldots \ldots \ldots \ldots \ldots \ldots \ldots \ldots \ldots
\ldots \ldots \ldots

\bigskip


\ul{Exercice 5:} \textit{(4,5 points)}


\enskip


Calculer chaque expression en d�taillant chaque �tape de calcul. Un r�sultat
non expliqu� ne sera pas compt�.



\begin{center}


$B=\dfrac{-5(2+4^3)}{3 \times 7 -2^4}$ \qquad \qquad \qquad $C=8^2-(3^2 \times
4 - 5^2)$ \qquad \qquad \qquad $D=(10-3\times4)(6 \times 8 - 6^2)$

\end{center}


\bigskip





\ul{Exercice 6:} \textit{(5 points)}

\begin{enumerate}
  \item Un cycliste parcourt 48 km en une heure et demie. Quelle est alors sa
  vitesse moyenne?
  \item Au retour, sur le m�me trajet, il roule � une vitesse de 38,4 km/h.
  Combien de temps met-il pour effectuer le chemin du retour?
  \item Quel est sa vitesse moyenne sur l'ensemble de son parcours? On
  arrondira � l'unit�.
  \item Exprimer 38,4 km/h en m/s.
\end{enumerate}

\end{document}
