%%This is a very basic article template.
%%There is just one section and two subsections.
\documentclass[10pt, twoside]{article}
\usepackage[francais]{babel}
\usepackage[T1]{fontenc}
\usepackage[latin1]{inputenc}
\usepackage[left=7mm, right=7mm, top=2mm, bottom=2mm]{geometry}
\usepackage{float}
\usepackage{graphicx}
\usepackage{array}
\usepackage{multirow}
\usepackage{soul}
\usepackage{fancyhdr}

\usepackage{amsmath,amssymb,mathrsfs}
\pagestyle{empty}

\begin{document}



Pour comparer deux nombres, on peut : 
\begin{enumerate}
  \item \textbf{Comparer leurs �critures d�cimales}\\
  ex: $0,4<0,48<0,488$
  \bigskip
  
  \item \textbf{Utiliser les propri�t�s}\\
  ex: $\sqrt{8}<\sqrt{10}$ (car $0<8<10$) et $3<\pi$ donc
  $\sqrt{8}+3<\sqrt{10}+\pi$
  \bigskip
  
  
  \item \textbf{Comparer leurs carr�s}\\
  ex: Comparons $\sqrt{4+2\sqrt{3}}$ et $\sqrt{3}+1$. 
\enskip


  $\sqrt{4+2\sqrt{3}}$ et $\sqrt{3}+1$
  sont des nombres positifs, $(\sqrt{4+2\sqrt{3}})^{2}=4+2\sqrt{3}$ \thinspace
  et $(\sqrt{3}+1)^{2}=3+2\sqrt{3}+1=4+2\sqrt{3}$, donc $\sqrt{4+2\sqrt{3}}=\sqrt{3}+1$
  \bigskip
   
  
  \item \textbf{Etudier le signe de la diff�rence}\\
  ex: Comparons   $\dfrac{\sqrt{7}-\sqrt{5}}{\sqrt{2}}$  et 
  $\dfrac{\sqrt{2}}{\sqrt{5}+\sqrt{7}}$
  \medskip
  
 
$\dfrac{\sqrt{7}-\sqrt{5}}{\sqrt{2}}-\dfrac{\sqrt{2}}{\sqrt{5}+\sqrt{7}}=\dfrac{(\sqrt{7}-\sqrt{5})\times(\sqrt{7}+\sqrt{5})}{\sqrt{2}
\times (\sqrt{7}+\sqrt{5})}-\dfrac{\sqrt{2}\times
\sqrt{2}}{(\sqrt{5}+\sqrt{7}) \times
\sqrt{2}}=\dfrac{7-5-2}{\sqrt{2}(\sqrt{5}+\sqrt{7})}=0$ \thinspace
donc $\dfrac{\sqrt{7}-\sqrt{5}}{\sqrt{2}}=\dfrac{\sqrt{2}}{\sqrt{5}+\sqrt{7}}$

  \bigskip
  
  
  \item \textbf{Les comparer � un troisi�me nombre} (Soit $a,b,c$ trois r�els,
  si $a<b$ et $b<c$ alors $a<c$):\\
    ex: $\bullet \dfrac{1}{\sqrt{3}}<1$ et $1<\sqrt{5}$ donc
  $\dfrac{1}{\sqrt{3}}<\sqrt{5}$  

  \quad  \quad $\bullet-\sqrt{2}<0$ et $0<5,8$ donc $-\sqrt{2}<5,8$
  \bigskip
  
  
  \item \textbf{Etudier le quotient et comparer � $1$}\\
  ex: Comparons $\dfrac{2}{1+\sqrt{5}}$ et $\dfrac{\sqrt{5}-1}{2}$
  \enskip
  
  
  $\dfrac{\dfrac{2}{1+\sqrt{5}}}{\dfrac{\sqrt{5}-1}{2}}=\dfrac{2}{1+\sqrt{5}}
  \times \dfrac{\sqrt{5}-1}{2}=\dfrac{4}{5-1}=1$ donc
  $\dfrac{2}{1+\sqrt{5}}=\dfrac{\sqrt{5}-1}{2}$
\end{enumerate}
\rule{\linewidth}{0.4pt}

Pour comparer deux nombres, on peut : 
\begin{enumerate}
  \item \textbf{Comparer leurs �critures d�cimales}\\
  ex: $0,4<0,48<0,488$
  \bigskip
  
  \item \textbf{Utiliser les propri�t�s}\\
  ex: $\sqrt{8}<\sqrt{10}$ (car $0<8<10$) et $3<\pi$ donc
  $\sqrt{8}+3<\sqrt{10}+\pi$
  \bigskip
  
  
  \item \textbf{Comparer leurs carr�s}\\
  ex: Comparons $\sqrt{4+2\sqrt{3}}$ et $\sqrt{3}+1$. 
\enskip


  $\sqrt{4+2\sqrt{3}}$ et $\sqrt{3}+1$
  sont des nombres positifs, $(\sqrt{4+2\sqrt{3}})^{2}=4+2\sqrt{3}$ \thinspace
  et $(\sqrt{3}+1)^{2}=3+2\sqrt{3}+1=4+2\sqrt{3}$, donc $\sqrt{4+2\sqrt{3}}=\sqrt{3}+1$
  \bigskip
   
  
  \item \textbf{Etudier le signe de la diff�rence}\\
  ex: Comparons   $\dfrac{\sqrt{7}-\sqrt{5}}{\sqrt{2}}$  et 
  $\dfrac{\sqrt{2}}{\sqrt{5}+\sqrt{7}}$
  \medskip
  
 
$\dfrac{\sqrt{7}-\sqrt{5}}{\sqrt{2}}-\dfrac{\sqrt{2}}{\sqrt{5}+\sqrt{7}}=\dfrac{(\sqrt{7}-\sqrt{5})\times(\sqrt{7}+\sqrt{5})}{\sqrt{2}
\times (\sqrt{7}+\sqrt{5})}-\dfrac{\sqrt{2}\times
\sqrt{2}}{(\sqrt{5}+\sqrt{7}) \times
\sqrt{2}}=\dfrac{7-5-2}{\sqrt{2}(\sqrt{5}+\sqrt{7})}=0$ \thinspace
donc $\dfrac{\sqrt{7}-\sqrt{5}}{\sqrt{2}}=\dfrac{\sqrt{2}}{\sqrt{5}+\sqrt{7}}$

  \bigskip
  
  
  \item \textbf{Les comparer � un troisi�me nombre} (Soit $a,b,c$ trois r�els,
  si $a<b$ et $b<c$ alors $a<c$):\\
    ex: $\bullet \dfrac{1}{\sqrt{3}}<1$ et $1<\sqrt{5}$ donc
  $\dfrac{1}{\sqrt{3}}<\sqrt{5}$  

  \quad  \quad $\bullet-\sqrt{2}<0$ et $0<5,8$ donc $-\sqrt{2}<5,8$
  \bigskip
  
  
  \item \textbf{Etudier le quotient et comparer � $1$}\\
  ex: Comparons $\dfrac{2}{1+\sqrt{5}}$ et $\dfrac{\sqrt{5}-1}{2}$
  \enskip
  
  
  $\dfrac{\dfrac{2}{1+\sqrt{5}}}{\dfrac{\sqrt{5}-1}{2}}=\dfrac{2}{1+\sqrt{5}}
  \times \dfrac{\sqrt{5}-1}{2}=\dfrac{4}{5-1}=1$ donc
  $\dfrac{2}{1+\sqrt{5}}=\dfrac{\sqrt{5}-1}{2}$
\end{enumerate}

\end{document}
