\documentclass[12pt, twoside]{article}
\usepackage[francais]{babel}
\usepackage[T1]{fontenc}
\usepackage[latin1]{inputenc}
\usepackage[left=5mm, right=5mm, top=7mm, bottom=7mm]{geometry}
\usepackage{float}
\usepackage{graphicx}
\usepackage{array}
\usepackage{multirow}
\usepackage{amsmath,amssymb,mathrsfs}
\pagestyle{empty}
\begin{document}

\section*{\center{Bilan statistiques}}




\bigskip

\begin{center}
\begin{tabular}{|m{95mm}|m{95mm}|}
\hline


\textbf{Ce que je dois savoir} & \textbf{Ce que je dois savoir faire} \\

\hline

\enskip


\begin{itemize}
  \item [$\bullet$] Je connais le vocabulaire: effectif, effectif total,
  fr�quence, diagramme circulaire (et semi-circulaire) et diagramme en b�tons.
  
 
  
  \item [$\bullet$] Je sais calculer un effectif total.
  
  
  \item [$\bullet$] Je connais les formules pour calculer une moyenne.
  
  \item [$\bullet$] Je connais la propri�t� de proportionnalit� des 
  
  diagrammes.
  
  
  \item [$\bullet$] Je connais la propri�t� des fr�quences (nombre compris entre 0 et 1; la somme des fr�quences vaut 1).

 \end{itemize}

&


\begin{itemize}
  
  
    \item [$\bullet$] Je sais compl�ter un tableau d'effectifs.
    
  \item[$\bullet$] Je sais calculer une moyenne.


  \item[$\bullet$] Je sais calculer une moyenne pond�r�e.

  
      
  \item[$\bullet$] Je sais calculer des pourcentages.
  
  \item[$\bullet$] Je sais calculer des fr�quences.
  

  
  \item[$\bullet$] Je sais construire des diagrammes en b�tons et des
  diagrammes circulaires (et semi-circulaires).
  
    \item[$\bullet$] Je sais interpr�ter les r�sultats.

\end{itemize} \\

\hline

\end{tabular}
\end{center}


\bigskip



\section*{\center{Bilan statistiques}}


\bigskip

\bigskip

\bigskip

\bigskip

\begin{center}
\begin{tabular}{|m{95mm}|m{95mm}|}
\hline


\textbf{Ce que je dois savoir} & \textbf{Ce que je dois savoir faire} \\

\hline

\enskip


\begin{itemize}
  \item [$\bullet$] Je connais le vocabulaire: effectif, effectif total,
  fr�quence, diagramme circulaire (et semi-circulaire) et diagramme en b�tons.
  
 
  
  \item [$\bullet$] Je sais calculer un effectif total.
  
  
  \item [$\bullet$] Je connais les formules pour calculer une moyenne.
  
  \item [$\bullet$] Je connais la propri�t� de proportionnalit� des 
  
  diagrammes.
  
  
  \item [$\bullet$] Je connais la propri�t� des fr�quences (nombre compris entre 0 et 1; la somme des fr�quences vaut 1).

 \end{itemize}

&


\begin{itemize}
  
  
    \item [$\bullet$] Je sais compl�ter un tableau d'effectifs.
    
  \item[$\bullet$] Je sais calculer une moyenne.


  \item[$\bullet$] Je sais calculer une moyenne pond�r�e.

  
      
  \item[$\bullet$] Je sais calculer des pourcentages.
  
  \item[$\bullet$] Je sais calculer des fr�quences.
  

  
  \item[$\bullet$] Je sais construire des diagrammes en b�tons et des
  diagrammes circulaires (et semi-circulaires).
  
    \item[$\bullet$] Je sais interpr�ter les r�sultats.

\end{itemize} \\

\hline

\end{tabular}
\end{center}





\end{document}
