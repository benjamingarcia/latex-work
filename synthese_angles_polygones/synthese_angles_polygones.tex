\documentclass[12pt, twoside]{article}
\usepackage[francais]{babel}
\usepackage[T1]{fontenc}
\usepackage[latin1]{inputenc}
\usepackage[left=8mm, right=8mm, top=3mm, bottom=3mm]{geometry}
\usepackage{float}
\usepackage{graphicx}
\usepackage{array}
\usepackage{multirow}
\usepackage{amsmath,amssymb,mathrsfs}
\pagestyle{empty}
\begin{document}



\section*{\center{Bilan angles inscrits, polygones r�guliers}}




\bigskip

\begin{center}
\begin{tabular}{|m{9cm}|m{9cm}|}
\hline


\textbf{Ce que je dois savoir} & \textbf{Ce que je dois savoir faire} \\

\hline

\enskip


\begin{itemize}
  
  
  \item [$\bullet$] Je connais le vocabulaire: angle inscrit, angle au centre,
  intercepter un arc.
  
  \item [$\bullet$] Je connais les deux propri�t�s des angles.
  
  \item [$\bullet$] Je connais la d�finition de polygone r�gulier.
  
  \item [$\bullet$] Je connais la propri�t� d'un cercle.
  
  \item [$\bullet$] Je connais les principaux polygones r�guliers.


 \end{itemize}

&


\begin{itemize}
  
  
  \item [$\bullet$] Je sais reconnaitre des angles inscrits et des angles au
  centre interceptant le m�me arc.
  
  \item[$\bullet$] Je sais r�soudre des probl�mes d'angle.
  
  \item[$\bullet$] Je sais construire un polygone r�gulier.
  
  \item[$\bullet$] Je sais calculer des longueurs.
  
  \item[$\bullet$] Je sais calculer des angles.
  
  \item[$\bullet$] Je sais r�soudre des probl�mes.





 
\end{itemize} \\

\hline

\end{tabular}
\end{center}

\bigskip

\section*{\center{Bilan angles inscrits, polygones r�guliers}}




\bigskip

\begin{center}
\begin{tabular}{|m{9cm}|m{9cm}|}
\hline


\textbf{Ce que je dois savoir} & \textbf{Ce que je dois savoir faire} \\

\hline

\enskip


\begin{itemize}
  
  
  \item [$\bullet$] Je connais le vocabulaire: angle inscrit, angle au centre,
  intercepter un arc.
  
  \item [$\bullet$] Je connais les deux propri�t�s des angles.
  
  \item [$\bullet$] Je connais la d�finition de polygone r�gulier.
  
  \item [$\bullet$] Je connais la propri�t� d'un cercle.
  
  \item [$\bullet$] Je connais les principaux polygones r�guliers.


 \end{itemize}

&


\begin{itemize}
  
  
  \item [$\bullet$] Je sais reconnaitre des angles inscrits et des angles au
  centre interceptant le m�me arc.
  
  \item[$\bullet$] Je sais r�soudre des probl�mes d'angle.
  
  \item[$\bullet$] Je sais construire un polygone r�gulier.
  
  \item[$\bullet$] Je sais calculer des longueurs.
  
  \item[$\bullet$] Je sais calculer des angles.
  
  \item[$\bullet$] Je sais r�soudre des probl�mes.





 
\end{itemize} \\

\hline

\end{tabular}
\end{center}

\bigskip


\section*{\center{Bilan angles inscrits, polygones r�guliers}}




\bigskip

\begin{center}
\begin{tabular}{|m{9cm}|m{9cm}|}
\hline


\textbf{Ce que je dois savoir} & \textbf{Ce que je dois savoir faire} \\

\hline

\enskip


\begin{itemize}
  
  
  \item [$\bullet$] Je connais le vocabulaire: angle inscrit, angle au centre,
  intercepter un arc.
  
  \item [$\bullet$] Je connais les deux propri�t�s des angles.
  
  \item [$\bullet$] Je connais la d�finition de polygone r�gulier.
  
  \item [$\bullet$] Je connais la propri�t� d'un cercle.
  
  \item [$\bullet$] Je connais les principaux polygones r�guliers.


 \end{itemize}

&


\begin{itemize}
  
  
  \item [$\bullet$] Je sais reconnaitre des angles inscrits et des angles au
  centre interceptant le m�me arc.
  
  \item[$\bullet$] Je sais r�soudre des probl�mes d'angle.
  
  \item[$\bullet$] Je sais construire un polygone r�gulier.
  
  \item[$\bullet$] Je sais calculer des longueurs.
  
  \item[$\bullet$] Je sais calculer des angles.
  
  \item[$\bullet$] Je sais r�soudre des probl�mes.





 
\end{itemize} \\

\hline

\end{tabular}
\end{center}
\end{document}
