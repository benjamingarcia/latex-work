\documentclass[12pt, twoside]{article}
\usepackage[francais]{babel}
\usepackage[T1]{fontenc}
\usepackage[latin1]{inputenc}
\usepackage[left=5mm, right=5mm, top=3mm, bottom=3mm]{geometry}
\usepackage{float}
\usepackage{graphicx}
\usepackage{array}
\usepackage{multirow}
\usepackage{amsmath,amssymb,mathrsfs}
\usepackage{soul}
\usepackage{textcomp}
\usepackage{eurosym}
 \usepackage{variations}
\usepackage{tabvar}
\usepackage{lscape}


\pagestyle{empty}

\begin{document}




\section*{\center{Correction devoir surveill� 7}}

\subsection*{Exercice 1}

\begin{enumerate}
  \item $\dfrac{41}{100}=41 \div 100=0,41$ \qquad \qquad $\dfrac{45}{3}=45 \div
  3 =15$ \qquad \qquad $\dfrac{7}{5}=7 \div 5 =1,4$
  \item En utilisant la propri�t� du cours ``$\dfrac{a}{b} \times b =a$'', on
  trouve: \qquad 
$3 \times \dfrac{2}{3}=2$ \qquad $\dfrac{8}{11} \times 11=8$ 
\qquad $5 \times \dfrac{13}{5}=13$
\end{enumerate}

\subsection*{Exercice 2}


$\dfrac{3}{5}=\dfrac{9}{15}$ \qquad \qquad $\dfrac{36}{24}=\dfrac{6}{4}$ \qquad
\qquad  $2=\dfrac{18}{9}$ car $18 \div 9 = 2$ \qquad \qquad 
$\dfrac{8}{18}=\dfrac{4}{9}=\dfrac{16}{36}=\dfrac{0,16}{0,36}$


\subsection*{Exercice 3}

\begin{enumerate}
  \item $1^{ere}$ m�thode: $\dfrac{ 4}{3} \times 21=(4 \times
  21) \div 3=84 \div 3=28$ \qquad \qquad  $2^{eme}$ m�thode: $\dfrac{ 4}{3}
  \times 21=4 \times (21 \div 3 )=4 \times 7=28$
  \item $1^{ere}$ m�thode: $\dfrac{8}{20} \times 5=(8 \times 5) \div 20=40 \div
  20 =2$ \qquad \qquad  $2^{eme}$ m�thode: $\dfrac{8}{20} \times 5=(8 \div 20)
  \times 5=0,4 \times 5=2$
  
$3^{eme}$ m�thode: $\dfrac{8}{20} \times 5=  8 \times (5 \div 20)=8 \times
0,25=2$
\end{enumerate}

\subsection*{Exercice 4}

Une unit� est divis� en 5 part �gales donc une graduation repr�sente
$\dfrac{1}{5}$. 

A $\big ( \dfrac{1}{5} \big)$ et C $\big (
\dfrac{11}{5} \big )$ , on peut �crire aussi C $ \big ( 2+\dfrac{1}{5} \big )$.


\subsection*{Exercice 5}

Jean a gagn� 1600 \euro. Il donne $\dfrac{3}{8}$ de 1600 \euro  � Paul. Il faut
donc calculer $\dfrac{3}{8} \times 1600$ (plusieurs m�thodes de calcul
possibles): \qquad

$\dfrac{3}{8} \times 1600=3 \times (1600 \div 8)=3 \times 200=600$ \qquad 
Paul a 600 \euro.

\enskip

Anne obtient $\dfrac{2}{5}$ de 1600 \euro, il faut donc calculer $\dfrac{2}{5}
\times 1600$ (plusieurs m�thodes de calcul
possibles): 

 
$\dfrac{2}{5}\times 1600=2\times (1600 \div 5)=2 \times 320=640$
\qquad
Anne a 640 \euro. 

\enskip

Donc Jean garde 360 \euro (car 600+640=1240 et 1600-1240=360).

\subsection*{Exercice 6}

\begin{enumerate}
  \item Le point E semble �tre le sym�trique du point K par rapport � la droite
  (d). Le point C semble �tre le sym�trique du point H par rapport � la droite
  (d). 
  \item 
  
  a) $B_1$ n'est pas le sym�trique du point A par rapport � (d) car [$AB_1$]
  et (d) ne sont pas perpendiculaires.
  
  b) (A$B_2$) est perpendiculaire � (d) mais les points A et $B_2$ ne sont pas �
  la m�me distance de la droite (d). Donc ces deux points ne sont pas
  sym�triques.
  
  c) $B_3$ et A sont sym�triques car la droite (d) est la m�diatrice du segment
  [A$B_3$].
  
\end{enumerate}


\subsection*{Exercice 7}

Le tr�fle a un seul axe de sym�trie. Le triangle �quilat�ral poss�de 3 axes de
sym�trie. Le parall�logramme n'a pas d'axe de sym�trie. Le pentagone (figure 4)
poss�de 5 axes de sym�trie. La derni�re figure a 2 axes de sym�trie.
\end{document}
