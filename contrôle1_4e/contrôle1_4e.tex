\documentclass[12pt, twoside]{article}
\usepackage[francais]{babel}
\usepackage[T1]{fontenc}
\usepackage[latin1]{inputenc}
\usepackage[left=5mm, right=5mm, top=5mm, bottom=35mm]{geometry}
\usepackage{float}
\usepackage{graphicx}
\usepackage{array}
\usepackage{multirow}
\usepackage{amsmath,amssymb,mathrsfs}
\usepackage{textcomp}
\pagestyle{empty}
\usepackage{soul}

\begin{document} 


\begin{flushleft}
NOM PRENOM: \ldots \ldots \ldots \ldots \ldots \ldots \ldots \ldots \ldots
 \end{flushleft}


\begin{center}
{\fbox{$4^{e}3$ \qquad \qquad \textbf{\Large{Devoir surveill� 1}}
\qquad \qquad 02/10/2009}}
\end{center}



\textit{Remarques: Tout le devoir se fait sur les feuilles photocopi�es.}
CALCULATRICE INTERDITE

 
\bigskip

\ul{\textbf{Exercice 1:}} 

\begin{enumerate}
  \item Ecrire la r�gle pour additionner deux nombres relatifs de signes
  contraires: 
  
   \rule{18cm}{0,5pt}
   
    \rule{18cm}{0,5pt}
    
     \rule{18cm}{0,5pt}
     
      \rule{18cm}{0,5pt}
      

      
      
   \item   Ecrire la r�gle pour diviser deux nombres relatifs:
   
    \rule{18cm}{0,5pt}
    
     \rule{18cm}{0,5pt}
     
      \rule{18cm}{0,5pt}
      
       \rule{18cm}{0,5pt}
       
     
       
       
   \item Qu'appelle-t'on m�diane dans un triangle?
   
     \rule{18cm}{0,5pt}
     
      \rule{18cm}{0,5pt}
    
     \rule{18cm}{0,5pt}
  
  
\end{enumerate}


\bigskip



\ul{\textbf{Exercice 2:}} Calculer mentalement:

\enskip

\begin{tabular}{ccc}
\begin{minipage}{6cm}
\begin{enumerate}
\item [a.] $-2 \times 6= \ldots$



\item [b.] $(-8)+(-2)=\ldots$


\item [c.] $14 \div(-7)=\ldots$
\end{enumerate}
\end{minipage}
&
\begin{minipage}{6cm}
\begin{enumerate}
\item[d.] $(-7)-(+2)=\ldots$



\item [e.] $-5+12=\ldots$



\item [f.] $-5 \times (-3)=\ldots$
\end{enumerate}
\end{minipage}
&
\begin{minipage}{6cm}
\begin{enumerate}
\item[g.] $-8-5=\ldots$


\item[h.] $(-25) \div (-5)=\ldots$
\end{enumerate}
\end{minipage}
\end{tabular} 

\bigskip



\ul{\textbf{Exercice 3:}} D�terminer le signe des produits suivants (on ne
demande pas de les calculer):
 

\begin{enumerate}
  \item[a.]  $(-2) \times 3,78 \times (-1) \times (+5,2) \times (-2)$ 
  \qquad \quad \quad \quad signe: \ldots \ldots \ldots
  \item[b.] $1 \times (-5) \times (-10,2) \times (-3) \times (-6) \times (+2)$
   \quad \qquad  signe: \ldots \ldots \ldots
   \item[c.] $2 \times (+3) \times (+4,5) $ \qquad \qquad  \qquad \qquad
  \qquad \quad \quad \quad signe: \ldots \ldots \ldots
  
\end{enumerate} 

\bigskip

\medskip

\ul{\textbf{Exercice 4:}} Effectuer le calcul suivant en soulignant � chaque
�tape le calcul en cours:

\begin{center}
$A=-6 -4 \times (+5)$ \qquad \qquad \qquad \qquad \qquad \qquad \qquad
\qquad $B=7 \times 2 - 3 \times 4 -8$

\end{center}




 
\pagebreak 

\begin{flushleft}
NOM PRENOM: \ldots \ldots \ldots \ldots \ldots \ldots \ldots \ldots \ldots
 \end{flushleft}

\medskip


\begin{center}
$C= 5 \times (-7+10)-12$ \qquad \qquad \qquad \qquad  \qquad \qquad \qquad
\qquad  $D=\dfrac{2-7 \times (-6) +4}{-5-1+(-3) \times (-5)}$

\end{center}

\bigskip

\bigskip

\bigskip

\bigskip

\bigskip

\bigskip

\bigskip

\bigskip

\bigskip

\bigskip

 

\ul{\textbf{Exercice 5:}} Traduire les phrases suivantes par un calcul:

\begin{enumerate}
  \item La diff�rence du produit de -4 par -5 et de 6.
  
  \bigskip
  

  
  \item Le quotient de la somme de 3 et -2 par le produit de 7 par -1.
\end{enumerate}

\bigskip




\bigskip

\ul{\textbf{Exercice 6:}} Traduire les expressions math�matiques suivantes par
des phrases:
 
\begin{enumerate}
  \item $(-6)+ 5 \times (-2)$
  
  \rule{18cm}{0,5pt}
  
  \rule{18cm}{0,5pt}
  
  \item $(-7-2) \times (-4)$
  
  \rule{18cm}{0,5pt}
  
  \rule{18cm}{0,5pt}
\end{enumerate}
\end{document}
