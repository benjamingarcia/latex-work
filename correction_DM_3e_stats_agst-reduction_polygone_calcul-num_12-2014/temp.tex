\begin{enumerate}
  \item ABCDEF est un hexagone r�gulier donc $\widehat{COD}=\dfrac{360}{6}=60$�.
  D'o� $\widehat{COE}=2 \times \widehat{COD} = 2 \times 60= 120$�.

  \item $\widehat{CAE}$ est un angle au centre qui intercepte l'arc CE.
  $\widehat{COE}$ est un angle au centre interceptant le m�me angle. 
  
  D'apr�s la propri�t�: ``Dans un cercle, la mesure d'un angle inscrit est �gale
  � la moiti� de celle d'un angle au centre interceptant le m�me arc'', on
  conclut que $\widehat{CAE}=\widehat{COE} \div 2=120 \div 2= 60$�.
  
  \item OAB est un triangle isoc�le (car OA=OB). L'angle $\widehat{AOB}$ mesure
  60� donc OAB est un triangle �quilat�ral. On a donc OA=AB=BO.
  
  P�rim�tre de l'hexagone: 192 m. Donc OA=AB=$192 \div 6=32$m.
  
  \item Comme OAB est un triangle �quilat�ral, on en d�duit que la hauteur [AH]
  est aussi une bissectrice, une m�diatrice et une m�diane de ce triangle. Donc
  $BH=AB \div 2= 16$m.
  
  Le triangle OHB est rectangle en H. OH=32 m et BH=16 m. D'apr�s le th�or�me de
  Pythagore, on a: $OB^2=HO^2+HB^2$
  
  $32^2=HO^2+16^2$ 
  
  $1024=HO^2+256$ \qquad donc $HO^2=1024-256=768$
  
  OH est une mesure positive donc $OH=\sqrt{768}\approx 27,71$m.
  
  \item $Aire_{OBA}=\dfrac{OH \times AB}{2}=\dfrac{27,71 \times 32}{2}=443,36
  m^2$
  
  \item $Aire_{ABCDEF}= 6 \times Aire_{OBA}= 6 \times 443,36=2660,16 m^2$
\end{enumerate}