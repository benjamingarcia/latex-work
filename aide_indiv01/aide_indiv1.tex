\documentclass[12pt, twoside]{article}
\usepackage[francais]{babel}
\usepackage[T1]{fontenc}
\usepackage[latin1]{inputenc}
\usepackage[left=1cm, right=1cm, top=1cm, bottom=1cm]{geometry}
\usepackage{float}
\usepackage{graphicx}
\usepackage{array}
\usepackage{multirow}
\usepackage{amsmath,amssymb,mathrsfs}


\begin{document}

\begin{flushright}
$2^{de}5$
\end{flushright}
\begin{flushleft}
02/10/2008
\end{flushleft}
\section*{\center{Aide individualis�e}}

\subsection*{Les fractions}
On rappelle que le \textbf{d�nominateur} doit �tre \textbf{non nul} pour que la
fraction ait du sens (on n'a pas le droit de diviser par $0$).
\begin{enumerate}
  \item R�gles des signes:
 $\dfrac{a}{-b}=\dfrac{-a}{b}=-\dfrac{a}{b}$ \quad et \quad 
 $\dfrac{-a}{-b}=\dfrac{a}{b}$.
 \medskip
 \item pour simplifier ou r�duire au m�me d�nominateur:
 $\dfrac{ka}{kb}=\dfrac{a}{b}$ \quad et \quad
  $\dfrac{a:k}{b:k}=\dfrac{a}{b}$.
  \medskip
  \item Addition de fractions ayant m�me d�nominateur (sinon r�duire au m�me
  d�nominateur): 
  $\dfrac{a}{b}+\dfrac{c}{b}=\dfrac{a+c}{b}$
  \medskip
  \item Multiplication: 
  $k \times \dfrac{a}{b}=\dfrac{ka}{b}=\dfrac{a}{b} \times k$ \quad et \quad
  $\dfrac{a}{b} \times \dfrac{c}{d}=\dfrac{a \times c}{b \times d}$
  \medskip
  \item Division: \quad
  $\dfrac{a}{b}:\dfrac{c}{d}=\dfrac{\dfrac{a}{b}}{\dfrac{c}{d}}=\dfrac{a}{b}
  \times \dfrac{d}{c}= \dfrac{ad}{bc}$
\end{enumerate}


\subsection*{Les racines car�es}
\textbf{D�finition:} $a$ �tant un nombre \textbf{positif} (ou nul), $\sqrt{a}$
est le nombre positif (ou nul) qui �lev� au carr� donne $a$:  \quad $\sqrt{a}
\times \sqrt{a}=(\sqrt{a})^2=a$ avec $a\geqslant 0$.
 
 \bigskip
 \textbf{R�gles:}
 \begin{enumerate}
   \item \fbox{$\sqrt{a}$ n'est d�finie que si $a$ est un nombre
   \textbf{positif} (ou nul).}
  \medskip 
\item $a$ �tant un nombre positif, il existe $2$ nombres, $\sqrt{a}$ et
$-\sqrt{a}$ qui �lev�s au carr� donnent $a$.
\item $a$ et $b$ �tant des nombres positifs et $b \not =0$, on a:\\
$\sqrt{a} \times \sqrt{b}=\sqrt{ab}$\\ 
$\sqrt{a^{2}}=a$ car $a \geqslant 0$\\ 
$\dfrac{\sqrt{a}}{\sqrt{b}}=\sqrt{\dfrac{a}{b}}$.
 \end{enumerate}
 
 \textit{ATTENTION:} il faut calculer le nombre sous le radical $\sqrt{\  }$
 \textbf{avant} de calculer la racine carr�e!\\
  \begin{center}
 \fbox{$\sqrt{a+b} \not =
 \sqrt{a}+\sqrt{b}$} 
  \end{center}
 
 
 \newpage
  \subsection*{Applications}
  
  \textbf{Exercice 1:} 
  \begin{enumerate}
  \item  Effectuer les calculs ci-dessous (sans la calculatrice). Pour chacun
  d'eux figure le r�sultat. Si votre r�sultat est diff�rent, refaire le calcul et m'appeler si le r�sultat est encore diff�rent.\\
  
  
 \begin{center}
 \begin{tabular}{|c|c|}
   \hline
   nombres & r�sultats \\[5mm]
   \hline 
   $A=4 \times (\dfrac{1}{4}-\dfrac{1}{3})$ & $-\dfrac{1}{3}$ \\[5mm]
   \hline
   $B=\dfrac{2}{3}+\dfrac{3}{2}-\dfrac{1}{6}$ & $2$ \\[6mm]
   \hline
   $C=\dfrac{\dfrac{2}{3}}{\dfrac{5}{6}}-2$ & $-\dfrac{6}{5}$\\[10mm]
   \hline
   \end{tabular}
 \end{center}

\medskip
  \item Trouver la nature de $A,B$ et $C$.\\
  \bigskip
  
  
  \textbf{M�thode:}
  \begin{tabular}{|l|}
  \hline Trouver la nature d'un nombre c'est trouver le plus
  petit ensemble ( $\mathbb{N}, \mathbb{Z}, \mathbb{D}, \mathbb{Q}$ ou
  $\mathbb{R}$) \\
   auquel il appartient. Tr�s souvent, on simplifie ce nombre �
  l'aide des r�gles de calculs \\ mais en gardant toujours la \textbf{valeur
  exacte}.\\
  \hline
  \end{tabular}
  \end{enumerate}
  
\bigskip
\bigskip
\textbf{Exercice 2:} Soit $D=2^{2}\times 3 \times 5 \times 7$, $E=2 \times 5
\times 11$ et $F=2 \times 3^{2} \times 7 \times 17$.
\begin{enumerate}
  \item Donner le $pgcd$ de $D$ et $E$?
  \item Donner le $pgcd$ de $F$ et $E$?
  \item Donner le $pgcd$ de $D$ et $F$?
\end{enumerate}


\bigskip
\bigskip
\textbf{Exercice 3:} Tous  les calculs suivants sont \fbox{faux}. Trouver
pourquoi et les corriger.
\begin{enumerate}
  \item $(3\sqrt{2})^{2}=6$
  \item $\sqrt{16+9}=4+3$
  \item $(5\sqrt{2})^{2}=10$
  \enskip
  \item $\dfrac{1+5}{5}=1$
  \enskip
  \item $\dfrac{\dfrac{7}{4}}{\dfrac{1}{2}}=\dfrac{7}{8}$
  \enskip
  \item $\dfrac{\sqrt{2}+5}{\sqrt{2}+10}=\dfrac{5}{10}$
\end{enumerate}
\end{document}
