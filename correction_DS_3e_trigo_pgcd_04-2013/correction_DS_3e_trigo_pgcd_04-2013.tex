\documentclass[12pt, twoside]{article}
\usepackage[francais]{babel}
\usepackage[T1]{fontenc}
\usepackage[latin1]{inputenc}
\usepackage[left=5mm, right=5mm, top=6mm, bottom=6mm]{geometry}
\usepackage{float}
\usepackage{graphicx}
\usepackage{array}
\usepackage{multirow}
\usepackage{amsmath,amssymb,mathrsfs}
\usepackage{soul}
\usepackage{textcomp}
\usepackage{eurosym}
 \usepackage{variations}
\usepackage{tabvar}

\pagestyle{empty}
\begin{document}

\begin{center}
\fbox{Correction du devoir surveill� 3}
\end{center}

\enskip


\ul{Exercice 1}:

\begin{enumerate}
  \item Calcul du PGCD de 210 et 165 � l'aide de l'agorithme d'Euclide:
  
  \begin{center}
  $210=165 \times 1 +45$ \qquad  $165=45 \times 3 + 30$ \qquad  $45=30
  \times 1 + 15$ \qquad  $30=15 \times 2+0$ \qquad  
  \end{center}

  Donc
  PGCD(210;165)=15
  
  
  \item Un mur a pour dimension 210 cm sur 165 cm. Pour avoir un nombre entier
  de carreaux, il faut trouver un diviseur commun � 210 et 165. Pour avoir un
  carreau de taille maximale, il faut trouver le plus grand diviseur commun �
  165 et 210.  La taille maximale est donc 15 cm.
  
  
  \item $210 \div 15=14$ \qquad \quad $165\div 15 = 11$ 
  \qquad \quad $11 \times 14 = 154$ \qquad \quad Il y aura 154 carreaux.

\end{enumerate}


\bigskip

\ul{Exercice 2}:

\begin{tabular}{cc}
\begin{minipage}{16cm}
\begin{enumerate}
  \item Le triangle ABC est rectangle en C. On a:
  $tan(\widehat{ABC})=\dfrac{AC}{BC}$.
   Donc $tan(50)=\dfrac{AC}{4}$ 
   
     et  $AC=tan(50) \times 4 \approx
  4,8$ cm.
  \item Le triangle ABC est rectangle en C. On a:
  $cos(\widehat{ABC})=\dfrac{BC}{AB}$.
   Donc $cos(\widehat{ABC})=\dfrac{4}{7}$ 

 et $\widehat{ABC}=ar
 ccos(4 \div 7) \approx 55$�.  
  
  \item 
  Le triangle ABC est rectangle en C. On a:
  $sin(\widehat{ABC})=\dfrac{AC}{7}$.
  Donc $sin(67)=\dfrac{AC}{4}$ 
  
  et $AC=sin(67) \times 7 \approx
  6,4$ cm.   
\end{enumerate}
\end{minipage}
&
\begin{minipage}{3cm}

\end{minipage}
\end{tabular}



\bigskip


\ul{Exercice 3:}

\begin{enumerate}
  \item [2.] Le triangle RTI est rectangle en R. On a:
  $tan(\widehat{RTI})=\dfrac{RI}{TR}$. Donc $tan(\widehat{RTI})=\dfrac{18}{7,5}$


et  $\widehat{RTI}=arctan(18 \div 7,5) \approx 67$�.
  
  \enskip
  
  \item [3.] Le triangle LAI est rectangle en L. On a:
  $sin(\widehat{LAI})=\dfrac{LI}{AI}$. Donc $sin(\widehat{LAI})=\dfrac{12}{13}$

 et $\widehat{LAI}=arcsin(12 \div 13) \approx 67$�.
  
  \enskip
  
  \ul{Autre m�thode:} Dans le triangle TRI: $\widehat{TIR}=180-90-67=33$�. 
  
  Dans le triangle LAI: $\widehat{LAI}=180-90-33=67$�.
\end{enumerate}


\bigskip

\ul{Exercice 4:}
Le triangle ADH est inscrit dans le cercle de diam�tre [AH]. D'apr�s la
prorpi�t� ``si un triangle est inscrit dans un cercle de diam�tre l'un de ses
c�t�s alors ce triangle est rectangle et admet ce c�t� pour hypot�nuse'', on en
d�duit que ADH est rectangle en D.

\enskip

Le triangle ADH est rectangle en A. On a:
  $sin(\widehat{DAH})=\dfrac{DH}{AH}$. 
  
  Donc $sin(60)=\dfrac{DH}{12}$
  \quad et \quad $DH=sin(60) \times 12 \approx 10,4$cm.
  
  
  
\end{document}
