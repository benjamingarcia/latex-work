\documentclass[12pt, twoside]{article}
\usepackage[francais]{babel}
\usepackage[T1]{fontenc}
\usepackage[latin1]{inputenc}
\usepackage[left=7mm, right=7mm, top=7mm, bottom=7mm]{geometry}
\usepackage{float}
\usepackage{graphicx}
\usepackage{array}
\usepackage{multirow}
\usepackage{amsmath,amssymb,mathrsfs}
\usepackage{soul}
\usepackage{textcomp}
\usepackage{eurosym}
 \usepackage{variations}
\usepackage{tabvar}

\pagestyle{empty}
\begin{document}


\begin{center}
\textbf{Correction devoir surveill� 5}
\end{center}


\medskip



\begin{tabular}{c|c}
\begin{minipage}{9cm}

\qquad \quad \textbf{Exercice 1}

\enskip

\begin{enumerate}
  \item L'angle cod� sur la figure peut se nommer: 
  
  $\widehat{DAC}$ ou
  $\widehat{CAD}$ ou $\widehat{yAD}$ ou $\widehat{DAy}$.
  \item [3.]  $\widehat{xAz}$ est obtus,  $\widehat{ADC}$ 
  est
   aigu et $\widehat{DCu}$ est plat.
  
  \bigskip
  
  
\textbf{Exercice 2}

\enskip

L'angle $\widehat{xOy}$ mesure 120�. 


L'angle $\widehat{tAv}$ mesure 35�.  
\end{enumerate}

\end{minipage}
&
\begin{minipage}{9cm}
\qquad \textbf{Exercice 3}

\enskip

\qquad M�thode pour tracer le triangle POF:

\begin{enumerate}
  \item Je trace le segment [PF] de mesure 8cm.
  \item Je trace l'angle $\widehat{PFO}=55$� (c'est un angle aigu).
  \item Je trace l'angle $\widehat{OPF}=102$� (c'est un angle obtus).
  \item Je prolonge les demi-droites obtenues: je note O le point
  d'intersection.
\end{enumerate}
\end{minipage}
\end{tabular}



\bigskip




\textbf{Exercice 5}

\enskip

\begin{enumerate}
  \item D'apr�s les codages, $\widehat{xOt}=90$� et $\widehat{tOy}=45$�. Donc
  $\widehat{xOy}=\widehat{xOt}+\widehat{tOy}=90+45=135$�.
  \item $\widehat{uOv}$ est un angle plat donc $\widehat{uOv}=180$�. 
  
  $\widehat{uOv}=\widehat{uOx}+\widehat{xOy}+\widehat{yOv}$
  
  180=71+ $\widehat{xOy}$+32 d'o� $\widehat{xOy}$=180-(71+32)=180-103=77�.
  \item D'apr�s les codages, $\widehat{GHO}=\widehat{KHI}=52$� et
  $\widehat{OHK}=76$�. Montrer que les points G, H et I sont align�s revient �
  montrer que l'angle $\widehat{GHI}$ est plat (c'est-�-dire de mesure 180�).
  
  $\widehat{GHI}=\widehat{GHO}+\widehat{OHK}+\widehat{KHI}$=52+76+52=180�.
  
   Donc les points G, H et K sont align�s.
\end{enumerate}

\bigskip

\begin{center}
\textbf{Correction devoir surveill� 5}
\end{center}


\medskip



\begin{tabular}{c|c}
\begin{minipage}{9cm}

\qquad \quad \textbf{Exercice 1}

\enskip

\begin{enumerate}
  \item L'angle cod� sur la figure peut se nommer: 
  
  $\widehat{DAC}$ ou
  $\widehat{CAD}$ ou $\widehat{yAD}$ ou $\widehat{DAy}$.
  \item [3.]  $\widehat{xAz}$ est obtus,  $\widehat{ADC}$ 
  est
   aigu et $\widehat{DCu}$ est plat.
  
  \bigskip
  
  
\textbf{Exercice 2}

\enskip

L'angle $\widehat{xOy}$ mesure 120�. 


L'angle $\widehat{tAv}$ mesure 35�.  
\end{enumerate}

\end{minipage}
&
\begin{minipage}{9cm}
\qquad \textbf{Exercice 3}

\enskip

\qquad M�thode pour tracer le triangle POF:

\begin{enumerate}
  \item Je trace le segment [PF] de mesure 8cm.
  \item Je trace l'angle $\widehat{PFO}=55$� (c'est un angle aigu).
  \item Je trace l'angle $\widehat{OPF}=102$� (c'est un angle obtus).
  \item Je prolonge les demi-droites obtenues: je note O le point
  d'intersection.
\end{enumerate}
\end{minipage}
\end{tabular}



\bigskip



\textbf{Exercice 5}

\enskip

\begin{enumerate}
  \item D'apr�s les codages, $\widehat{xOt}=90$� et $\widehat{tOy}=45$�. Donc
  $\widehat{xOy}=\widehat{xOt}+\widehat{tOy}=90+45=135$�.
  \item $\widehat{uOv}$ est un angle plat donc $\widehat{uOv}=180$�. 
  
  $\widehat{uOv}=\widehat{uOx}+\widehat{xOy}+\widehat{yOv}$
  
  180=71+ $\widehat{xOy}$+32 d'o� $\widehat{xOy}$=180-(71+32)=180-103=77�.
  \item D'apr�s les codages, $\widehat{GHO}=\widehat{KHI}=52$� et
  $\widehat{OHK}=76$�. Montrer que les points G, H et I sont align�s revient �
  montrer que l'angle $\widehat{GHI}$ est plat (c'est-�-dire de mesure 180�).
  
  $\widehat{GHI}=\widehat{GHO}+\widehat{OHK}+\widehat{KHI}$=52+76+52=180�.
  
  Donc les points G, H et K sont align�s.
\end{enumerate}
\end{document}
