\documentclass[12pt, twoside]{article}
\usepackage[francais]{babel}
\usepackage[T1]{fontenc}
\usepackage[latin1]{inputenc}
\usepackage[left=5mm, right=5mm, top=5mm, bottom=5mm]{geometry}
\usepackage{float}
\usepackage{graphicx}
\usepackage{array}
\usepackage{multirow}
\usepackage{amsmath,amssymb,mathrsfs}
\usepackage{soul}
\usepackage{textcomp}
\usepackage{eurosym}
 \usepackage{variations}
\usepackage{tabvar}


\pagestyle{empty}

\begin{document}




\section*{\center{Devoir maison 6}}


\bigskip





\fbox{

\begin{minipage}{18cm}
\textit{Devoir � rendre pour le \textbf{mercredi 26 mars 2014}. Pour chaque
probl�me, on �crira le calcul effectu�: une r�ponse sans explication donnera 0
point.}
\end{minipage}
}


\bigskip


\ul{Exercice 1}: (\textit{3 points})

\begin{enumerate}
  \item Poser et effectuer les divisions euclidiennes: \qquad a) 3952 $\div$ 5
  \qquad b) 581 $\div$ 6
  \item Pour chaque division, �crire l'�galit� de la division euclidienne.
\end{enumerate}


\bigskip

\ul{Exercice 2}: (\textit{3 points})

\enskip

Poser et effectuer les divisions d�cimales: \qquad a) 47,6 $\div$ 7
  \qquad b) 507,6 $\div$ 9
  
  
  \bigskip
  
  
  \ul{Exercice 3}: (\textit{2 points})

\enskip

Armand commande 100 CD vierges sur internet. Avec les frais de livraison qui
sont de 12 \euro, il paye finalement 

60 \euro. Quel est le prix d'un CD?

\bigskip


\ul{Exercice 4}: (\textit{2 points})

\enskip

Un fleuriste dispose de 200 roses. Il souhaite composer des bouquets de 12
roses.

\begin{enumerate}
  \item Combien peut-il faire de bouquets?
  \item Combien reste-t-il de roses?
\end{enumerate}


\bigskip


\ul{Exercice 5}: (\textit{1,5 points})

\enskip

6 amis d�cident de se cotiser pour offrir un cadeau � Yacine. Le cadeau co�te
49,80 \euro. Quel est le montant de 

la participation de chacun?



\bigskip


\ul{Exercice 6}: (\textit{3 points})

\enskip

Trouver toutes les valeurs possibles des deux chiffres manquants (d�sign�s par
\quad et \quad ) sachant que le nombre 

5 \quad 4 \quad est
divisible � la fois par 3 et par 10. 
Expliquer votre r�ponse.


\bigskip


\ul{Exercice 7}: (\textit{2,5 points})


\enskip

Marion a pass� 17 589 secondes sur internet.

\begin{enumerate}
  \item Exprimer cette dur�e en minutes et secondes.
  \item Exprimer cette dur�e en heures, minutes et secondes.
\end{enumerate}


\bigskip


\ul{Exercice 8}: (\textit{3 points})

\enskip

Val�rie a achet� 8 bouteilles de 75 cl de jus d'orange avec un billet de 20
\euro. La caissi�re lui a rendu 8,60 \euro.


\begin{enumerate}
  \item  Quel �tait le prix d'une bouteille? Donner la valeur approch�e par
  exc�s au centime pr�s.
  \item Quel �tait le prix d'un litre de jus d'orange?
  \end{enumerate}
\end{document}
