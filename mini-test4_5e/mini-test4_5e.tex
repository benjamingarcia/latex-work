\documentclass[12pt, twoside]{article}
\usepackage[francais]{babel}
\usepackage[T1]{fontenc}
\usepackage[latin1]{inputenc}
\usepackage[left=7mm, right=7mm, top=7mm, bottom=7mm]{geometry}
\usepackage{float}
\usepackage{graphicx}
\usepackage{array}
\usepackage{multirow}
\usepackage{amsmath,amssymb,mathrsfs}
\usepackage{soul}
\usepackage{textcomp}
\usepackage{eurosym}
 \usepackage{variations}
\usepackage{tabvar}

\pagestyle{empty}

\begin{document}

\begin{flushleft}
NOM PRENOM: \ldots \ldots \ldots \ldots \ldots \ldots \ldots \ldots \ldots
 
\bigskip

\end{flushleft}

\begin{center}
{\fbox{$5^{e}1$ \qquad \qquad \textbf{\Large{Contr�le de cours 4 (sujet 1)}}
\qquad \qquad 04/01/2010}}
\end{center}

\bigskip
\ul{Exercice 1} : Ecrire la r�gle permettant de calculer la somme de deux nombres relatifs de m�me signe.
 \begin{center}
     \rule{15cm}{0,5pt}
     \rule{15cm}{0,5pt}
     \rule{15cm}{0,5pt}
 \end{center}

\ul{Exercice 2}: Calculer les expressions suivantes (on pourra �ventuellement
transformer l'�criture).
\begin{center}
	\begin{tabular}{|cc|cc|cc|c}
	$A=(+7)+(+3)$ & \qquad \qquad  & $B=(+3)+(-2)$ & \qquad \qquad  & $C=(-2)-(-6)$ & \qquad \qquad  & $D=(+16)+(-3,6)$ \\
	\qquad & \qquad & \qquad & \qquad & \qquad & \qquad & \qquad\\
	\qquad & \qquad & \qquad & \qquad & \qquad & \qquad & \qquad\\
	\qquad & \qquad & \qquad & \qquad & \qquad & \qquad & \qquad\\
	\qquad & \qquad & \qquad & \qquad & \qquad & \qquad & \qquad\\
	\qquad & \qquad & \qquad & \qquad & \qquad & \qquad & \qquad\\
	\end{tabular}
\end{center} 

\begin{center}
	\begin{tabular}{|cc|cc|cc|c}
	$E=(-2,1)-(+2,3)$ & \qquad \qquad  & $F=-3+1,2$ & \qquad \qquad  & $G=8-(+2)$ & \qquad \qquad  & $H=2-6$ \\
	\qquad & \qquad & \qquad & \qquad & \qquad & \qquad & \qquad\\
	\qquad & \qquad & \qquad & \qquad & \qquad & \qquad & \qquad\\
	\qquad & \qquad & \qquad & \qquad & \qquad & \qquad & \qquad\\
	\qquad & \qquad & \qquad & \qquad & \qquad & \qquad & \qquad\\
	\qquad & \qquad & \qquad & \qquad & \qquad & \qquad & \qquad\\
	\end{tabular}
\end{center} 

\bigskip

\begin{flushleft}
NOM PRENOM: \ldots \ldots \ldots \ldots \ldots \ldots \ldots \ldots \ldots
 
\bigskip

\end{flushleft}

\begin{center}
{\fbox{$5^{e}1$ \qquad \qquad \textbf{\Large{Contr�le de cours 4 (sujet 2)}}
\qquad \qquad 04/01/2010}}
\end{center}

\bigskip
\ul{Exercice 1} : Ecrire la r�gle permettant de calculer la somme de deux nombres relatifs de signes contraires.
 \begin{center}
     \rule{15cm}{0,5pt}
     \rule{15cm}{0,5pt}
     \rule{15cm}{0,5pt}
 \end{center}

\ul{Exercice 2}: Calculer les expressions suivantes (on pourra �ventuellement
transformer l'�criture).
\begin{center}
	\begin{tabular}{|cc|cc|cc|c}
	$I=(+4)+(-5)$ & \qquad \qquad  & $J=(-3)-(-7)$ & \qquad \qquad  & $K=(+7)+(+2)$ & \qquad \qquad  & $L=(-1,4)+(+2,4)$ \\
	\qquad & \qquad & \qquad & \qquad & \qquad & \qquad & \qquad\\
	\qquad & \qquad & \qquad & \qquad & \qquad & \qquad & \qquad\\
	\qquad & \qquad & \qquad & \qquad & \qquad & \qquad & \qquad\\
	\qquad & \qquad & \qquad & \qquad & \qquad & \qquad & \qquad\\
	\qquad & \qquad & \qquad & \qquad & \qquad & \qquad & \qquad\\
	\end{tabular}
\end{center} 

\begin{center}
	\begin{tabular}{|cc|cc|cc|c}
	$M=-5+1,7$ & \qquad \qquad  & $N=4-9$ & \qquad \qquad  & $O=10-(+3)$ & \qquad \qquad  & $P=(-5,1)-(+6,3)$ \\
	\qquad & \qquad & \qquad & \qquad & \qquad & \qquad & \qquad\\
	\qquad & \qquad & \qquad & \qquad & \qquad & \qquad & \qquad\\
	\qquad & \qquad & \qquad & \qquad & \qquad & \qquad & \qquad\\
	\qquad & \qquad & \qquad & \qquad & \qquad & \qquad & \qquad\\
	\qquad & \qquad & \qquad & \qquad & \qquad & \qquad & \qquad\\
	\end{tabular}
\end{center} 

\pagebreak

\begin{flushleft}
NOM PRENOM: \ldots \ldots \ldots \ldots \ldots \ldots \ldots \ldots \ldots
 
\bigskip

\end{flushleft}

\begin{center}
{\fbox{$5^{e}1$ \qquad \qquad \textbf{\Large{Contr�le de cours 4 (sujet 3)}}
\qquad \qquad 04/01/2010}}
\end{center}

\bigskip
\ul{Exercice 1} : Ecrire la r�gle permettant de calculer la somme de deux nombres relatifs de m�me signe.
 \begin{center}
     \rule{15cm}{0,5pt}
     \rule{15cm}{0,5pt}
     \rule{15cm}{0,5pt}
 \end{center}

\ul{Exercice 2}: Calculer les expressions suivantes (on pourra �ventuellement
transformer l'�criture).
\begin{center}
	\begin{tabular}{|cc|cc|cc|c}
	$Q=(+2)+(+14)$ & \qquad \qquad  & $R=(+5)+(-1)$ & \qquad \qquad  & $S=(-2)-(-8)$ & \qquad \qquad  & $T=(+2,4)+(-6,4)$
	\\ \qquad & \qquad & \qquad & \qquad & \qquad & \qquad & \qquad\\
	\qquad & \qquad & \qquad & \qquad & \qquad & \qquad & \qquad\\
	\qquad & \qquad & \qquad & \qquad & \qquad & \qquad & \qquad\\
	\qquad & \qquad & \qquad & \qquad & \qquad & \qquad & \qquad\\
	\qquad & \qquad & \qquad & \qquad & \qquad & \qquad & \qquad\\
	\end{tabular}
\end{center} 

\begin{center}
	\begin{tabular}{|cc|cc|cc|c}
	$U=(-3,3)-(+3,5)$ & \qquad \qquad  & $V=4-12$ & \qquad \qquad  & $W=-5,1+2,3$ & \qquad \qquad  & $X=7-(+3)$ \\
	\qquad & \qquad & \qquad & \qquad & \qquad & \qquad & \qquad\\
	\qquad & \qquad & \qquad & \qquad & \qquad & \qquad & \qquad\\
	\qquad & \qquad & \qquad & \qquad & \qquad & \qquad & \qquad\\
	\qquad & \qquad & \qquad & \qquad & \qquad & \qquad & \qquad\\
	\qquad & \qquad & \qquad & \qquad & \qquad & \qquad & \qquad\\
	\end{tabular}
\end{center} 

\bigskip

\begin{flushleft}
NOM PRENOM: \ldots \ldots \ldots \ldots \ldots \ldots \ldots \ldots \ldots
 
\bigskip

\end{flushleft}

\begin{center}
{\fbox{$5^{e}1$ \qquad \qquad \textbf{\Large{Contr�le de cours 4 (sujet 4)}}
\qquad \qquad 04/01/2010}}
\end{center}

\bigskip
\ul{Exercice 1} : Ecrire la r�gle permettant de calculer la somme de deux nombres relatifs de signes contraires.
 \begin{center}
     \rule{15cm}{0,5pt}
     \rule{15cm}{0,5pt}
     \rule{15cm}{0,5pt}
 \end{center}

\ul{Exercice 2}: Calculer les expressions suivantes (on pourra �ventuellement
transformer l'�criture).

\begin{center}
	\begin{tabular}{|cc|cc|cc|c}
	$E=(+3)+(+8)$ & \qquad \qquad  & $F=(+6)+(-2)$ & \qquad \qquad  & $G=(-4)-(-9)$ & \qquad \qquad  & $H=(+3,7)+(-8,7)$ \\
	\qquad & \qquad & \qquad & \qquad & \qquad & \qquad & \qquad\\
	\qquad & \qquad & \qquad & \qquad & \qquad & \qquad & \qquad\\
	\qquad & \qquad & \qquad & \qquad & \qquad & \qquad & \qquad\\
	\qquad & \qquad & \qquad & \qquad & \qquad & \qquad & \qquad\\
	\qquad & \qquad & \qquad & \qquad & \qquad & \qquad & \qquad\\
	\end{tabular}
\end{center} 

\begin{center}
	\begin{tabular}{|cc|cc|cc|c}
	$I=(-5,6)-(+5,1)$ & \qquad \qquad  & $J=7-16$ & \qquad \qquad  & $K=-7,3+2,4$ & \qquad \qquad  & $L=8-(+4)$ \\
	\qquad & \qquad & \qquad & \qquad & \qquad & \qquad & \qquad\\
	\qquad & \qquad & \qquad & \qquad & \qquad & \qquad & \qquad\\
	\qquad & \qquad & \qquad & \qquad & \qquad & \qquad & \qquad\\
	\qquad & \qquad & \qquad & \qquad & \qquad & \qquad & \qquad\\
	\qquad & \qquad & \qquad & \qquad & \qquad & \qquad & \qquad\\
	\end{tabular}
\end{center} 

\pagebreak


\begin{flushleft}
NOM PRENOM: \ldots \ldots \ldots \ldots \ldots \ldots \ldots \ldots \ldots
 
\bigskip

\end{flushleft}

\begin{center}
{\fbox{$5^{e}2$ \qquad \qquad \textbf{\Large{Contr�le de cours 4 (sujet 1)}}
\qquad \qquad 04/01/2010}}
\end{center}

\bigskip
\ul{Exercice 1} : Ecrire la r�gle permettant de calculer la somme de deux nombres relatifs de m�me signe.
 \begin{center}
     \rule{15cm}{0,5pt}
     \rule{15cm}{0,5pt}
     \rule{15cm}{0,5pt}
 \end{center}

\ul{Exercice 2}: Calculer les expressions suivantes (on pourra �ventuellement
transformer l'�criture).
\begin{center}
	\begin{tabular}{|cc|cc|cc|c}
	$A=(+7)+(+3)$ & \qquad \qquad  & $B=(+3)+(-2)$ & \qquad \qquad  & $C=(-2)-(-6)$ & \qquad \qquad  & $D=(+16)+(-3,6)$ \\
	\qquad & \qquad & \qquad & \qquad & \qquad & \qquad & \qquad\\
	\qquad & \qquad & \qquad & \qquad & \qquad & \qquad & \qquad\\
	\qquad & \qquad & \qquad & \qquad & \qquad & \qquad & \qquad\\
	\qquad & \qquad & \qquad & \qquad & \qquad & \qquad & \qquad\\
	\qquad & \qquad & \qquad & \qquad & \qquad & \qquad & \qquad\\
	\end{tabular}
\end{center} 

\begin{center}
	\begin{tabular}{|cc|cc|cc|c}
	$E=(-2,1)-(+2,3)$ & \qquad \qquad  & $F=-3+1,2$ & \qquad \qquad  & $G=8-(+2)$ & \qquad \qquad  & $H=2-6$ \\
	\qquad & \qquad & \qquad & \qquad & \qquad & \qquad & \qquad\\
	\qquad & \qquad & \qquad & \qquad & \qquad & \qquad & \qquad\\
	\qquad & \qquad & \qquad & \qquad & \qquad & \qquad & \qquad\\
	\qquad & \qquad & \qquad & \qquad & \qquad & \qquad & \qquad\\
	\qquad & \qquad & \qquad & \qquad & \qquad & \qquad & \qquad\\
	\end{tabular}
\end{center} 

\bigskip

\begin{flushleft}
NOM PRENOM: \ldots \ldots \ldots \ldots \ldots \ldots \ldots \ldots \ldots
 
\bigskip

\end{flushleft}

\begin{center}
{\fbox{$5^{e}2$ \qquad \qquad \textbf{\Large{Contr�le de cours 4 (sujet 2)}}
\qquad \qquad 04/01/2010}}
\end{center}

\bigskip
\ul{Exercice 1} : Ecrire la r�gle permettant de calculer la somme de deux nombres relatifs de signes contraires.
 \begin{center}
     \rule{15cm}{0,5pt}
     \rule{15cm}{0,5pt}
     \rule{15cm}{0,5pt}
 \end{center}

\ul{Exercice 2}: Calculer les expressions suivantes (on pourra �ventuellement
transformer l'�criture).
\begin{center}
	\begin{tabular}{|cc|cc|cc|c}
	$I=(+4)+(-5)$ & \qquad \qquad  & $J=(-3)-(-7)$ & \qquad \qquad  & $K=(+7)+(+2)$ & \qquad \qquad  & $L=(-1,4)+(+2,4)$ \\
	\qquad & \qquad & \qquad & \qquad & \qquad & \qquad & \qquad\\
	\qquad & \qquad & \qquad & \qquad & \qquad & \qquad & \qquad\\
	\qquad & \qquad & \qquad & \qquad & \qquad & \qquad & \qquad\\
	\qquad & \qquad & \qquad & \qquad & \qquad & \qquad & \qquad\\
	\qquad & \qquad & \qquad & \qquad & \qquad & \qquad & \qquad\\
	\end{tabular}
\end{center} 

\begin{center}
	\begin{tabular}{|cc|cc|cc|c}
	$M=-5+1,7$ & \qquad \qquad  & $N=4-9$ & \qquad \qquad  & $O=10-(+3)$ & \qquad \qquad  & $P=(-5,1)-(+6,3)$ \\
	\qquad & \qquad & \qquad & \qquad & \qquad & \qquad & \qquad\\
	\qquad & \qquad & \qquad & \qquad & \qquad & \qquad & \qquad\\
	\qquad & \qquad & \qquad & \qquad & \qquad & \qquad & \qquad\\
	\qquad & \qquad & \qquad & \qquad & \qquad & \qquad & \qquad\\
	\qquad & \qquad & \qquad & \qquad & \qquad & \qquad & \qquad\\
	\end{tabular}
\end{center} 

\pagebreak

\begin{flushleft}
NOM PRENOM: \ldots \ldots \ldots \ldots \ldots \ldots \ldots \ldots \ldots
 
\bigskip

\end{flushleft}

\begin{center}
{\fbox{$5^{e}2$ \qquad \qquad \textbf{\Large{Contr�le de cours 4 (sujet 3)}}
\qquad \qquad 04/01/2010}}
\end{center}

\bigskip
\ul{Exercice 1} : Ecrire la r�gle permettant de calculer la somme de deux nombres relatifs de m�me signe.
 \begin{center}
     \rule{15cm}{0,5pt}
     \rule{15cm}{0,5pt}
     \rule{15cm}{0,5pt}
 \end{center}

\ul{Exercice 2}: Calculer les expressions suivantes (on pourra �ventuellement
transformer l'�criture).
\begin{center}
	\begin{tabular}{|cc|cc|cc|c}
	$Q=(+2)+(+14)$ & \qquad \qquad  & $R=(+5)+(-1)$ & \qquad \qquad  & $S=(-2)-(-8)$ & \qquad \qquad  & $T=(+2,4)+(-6,4)$
	\\ \qquad & \qquad & \qquad & \qquad & \qquad & \qquad & \qquad\\
	\qquad & \qquad & \qquad & \qquad & \qquad & \qquad & \qquad\\
	\qquad & \qquad & \qquad & \qquad & \qquad & \qquad & \qquad\\
	\qquad & \qquad & \qquad & \qquad & \qquad & \qquad & \qquad\\
	\qquad & \qquad & \qquad & \qquad & \qquad & \qquad & \qquad\\
	\end{tabular}
\end{center} 

\begin{center}
	\begin{tabular}{|cc|cc|cc|c}
	$U=(-3,3)-(+3,5)$ & \qquad \qquad  & $V=4-12$ & \qquad \qquad  & $W=-5,1+2,3$ & \qquad \qquad  & $X=7-(+3)$ \\
	\qquad & \qquad & \qquad & \qquad & \qquad & \qquad & \qquad\\
	\qquad & \qquad & \qquad & \qquad & \qquad & \qquad & \qquad\\
	\qquad & \qquad & \qquad & \qquad & \qquad & \qquad & \qquad\\
	\qquad & \qquad & \qquad & \qquad & \qquad & \qquad & \qquad\\
	\qquad & \qquad & \qquad & \qquad & \qquad & \qquad & \qquad\\
	\end{tabular}
\end{center} 

\bigskip

\begin{flushleft}
NOM PRENOM: \ldots \ldots \ldots \ldots \ldots \ldots \ldots \ldots \ldots
 
\bigskip

\end{flushleft}

\begin{center}
{\fbox{$5^{e}2$ \qquad \qquad \textbf{\Large{Contr�le de cours 4 (sujet 4)}}
\qquad \qquad 04/01/2010}}
\end{center}

\bigskip
\ul{Exercice 1} : Ecrire la r�gle permettant de calculer la somme de deux nombres relatifs de signes contraires.
 \begin{center}
     \rule{15cm}{0,5pt}
     \rule{15cm}{0,5pt}
     \rule{15cm}{0,5pt}
 \end{center}

\ul{Exercice 2}: Calculer les expressions suivantes (on pourra �ventuellement
transformer l'�criture).
\begin{center}
	\begin{tabular}{|cc|cc|cc|c}
	$E=(+3)+(+8)$ & \qquad \qquad  & $F=(+6)+(-2)$ & \qquad \qquad  & $G=(-4)-(-9)$ & \qquad \qquad  & $H=(+3,7)+(-8,7)$ \\
	\qquad & \qquad & \qquad & \qquad & \qquad & \qquad & \qquad\\
	\qquad & \qquad & \qquad & \qquad & \qquad & \qquad & \qquad\\
	\qquad & \qquad & \qquad & \qquad & \qquad & \qquad & \qquad\\
	\qquad & \qquad & \qquad & \qquad & \qquad & \qquad & \qquad\\
	\qquad & \qquad & \qquad & \qquad & \qquad & \qquad & \qquad\\
	\end{tabular}
\end{center} 

\begin{center}
	\begin{tabular}{|cc|cc|cc|c}
	$I=(-5,6)-(+5,1)$ & \qquad \qquad  & $J=7-16$ & \qquad \qquad  & $K=-7,3+2,4$ & \qquad \qquad  & $L=8-(+4)$ \\
	\qquad & \qquad & \qquad & \qquad & \qquad & \qquad & \qquad\\
	\qquad & \qquad & \qquad & \qquad & \qquad & \qquad & \qquad\\
	\qquad & \qquad & \qquad & \qquad & \qquad & \qquad & \qquad\\
	\qquad & \qquad & \qquad & \qquad & \qquad & \qquad & \qquad\\
	\qquad & \qquad & \qquad & \qquad & \qquad & \qquad & \qquad\\
	\end{tabular}
\end{center} 
\end{document}
