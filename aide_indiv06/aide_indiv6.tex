%%This is a very basic article template.
%%There is just one section and two subsections.
\documentclass[12pt, twoside]{article}
\usepackage[francais]{babel}
\usepackage[T1]{fontenc}
\usepackage[latin1]{inputenc}
\usepackage[left=7mm, right=7mm, top=5mm, bottom=20mm]{geometry}
\usepackage{float}
\usepackage{graphicx}
\usepackage{array}
\usepackage{multirow}
\usepackage{amsmath,amssymb,mathrsfs}
\usepackage{soul}
\pagestyle{empty}


\begin{document}

\section*{\center{Aide individualis�e : �quation-produits et �quation-quotients}}

\subsection*{Equation-produits}
L'�quation $A(x) \times B(x)=0$ est �quivalente � $A(x)=0$ ou
$B(x)=0$. 


Un produit de facteurs est nul si et seulement si l'un ou l'autre des
facteurs est nul.

\enskip

Cette �quation est appel�e \textbf{�quation-produit}.

\enskip

\textit{Remarque}: D�s qu'une �quation n'est pas d'un type simple ``$ax+b=0$'' ,
on tente de se ramener � une �quation produit.

\ul{Exemple} : $(x-2)(x-4)=0$\\
alors 
$
\begin{array}{cccc} 
 & x-2=0 &ou& x-4=0\\
& x=2 &ou& x=4\\
\end{array}
$


\subsection*{Equation-quotients}
 L'�quation $\dfrac{A(x)}{B(x)}=0$ est �quivalente � $A(x)=0$ et $B(x)\neq0$.\\
 
 \ul{Exemples} : 
 \begin{itemize}
  \item  [$\bullet$]  $ \dfrac{4x-1}{x+3}=0$\bigskip\\
  \ul{M�thode} : \begin{enumerate}
  \item on r�sout $4x-1=0$ donc \ul{$x=1/4$}
  \item on r�sout $x+3=0$ donc \ul{$x=-3$}
  \item on v�rifie que $-3\neq1/4$ (car $-3$ est une valeur interdite : on ne peut pas diviser par $0$).
  \item on conclut : la solution est $1/4$
\end{enumerate}
\bigskip
\item [$\bullet$] $\dfrac{x+1}{2x+2}=0$\bigskip\\ 
\ul{M�thode} : \begin{enumerate}
                                                   \item on r�sout $x+1=0$ donc \ul{$x=-1$}
                                                   \item on r�sout $2x+2=0$ donc $2x=-2$ d'o� \ul{$x=-1$}
                                                   \item $-1=-1$
                                                   \item on conclut : $-1$ est une valeur interdite (car $-1+1=0$ et
                                                   on ne peut das diviser par 0).  Donc il n'y a pas de solution.
                                                 \end{enumerate}
\end{itemize}

\pagebreak
\subsection*{Exercice 1}
R�soudre les �quations suivantes : 
\begin{enumerate}
  \item $(2x+1)(x-3)=0$
  \item $\dfrac{5x+3}{7x-2}=0$
\end{enumerate}


\subsection*{Exercice 2}
R�soudre les �quations suivantes : 
\begin{enumerate}
  \item  $(4-3x)(2+x)=0$
  \item  $x�-9=0$
  \item  $x�-4=0$
  \item  $(x+1)(5+7x)-3\times(x+1)=0$
\end{enumerate}


\subsection*{Exercice 3}
R�soudre les �quations suivantes : 
\begin{enumerate}
  \item $\dfrac{4x-1}{x}=0$
  \item $\dfrac{(2x-1)(x-3)}{x-2}=0$
  \item $\dfrac{(x+6)(3x-5)}{2x+12}=0$
  \item $\dfrac{1}{x+4}=-\dfrac{2}{x+1}$
\end{enumerate}


\end{document}
